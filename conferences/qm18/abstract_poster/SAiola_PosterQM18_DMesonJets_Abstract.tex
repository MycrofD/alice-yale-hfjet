\documentclass[12pt]{article}
%
\pagestyle{empty} \setlength{\topmargin}{-0.2in}
\setlength{\textheight}{9.9in} \setlength{\oddsidemargin}{-0.1in}
\setlength{\textwidth}{6.5in}
\parindent=20pt \parskip=0pt
%
\begin{document}

\centerline{\bf 
Measurement of the $p_{\rm T}$-differential cross section and
}

\centerline{\bf 
fragmentation function of D$^0$-tagged jets in pp collisions with ALICE
}

\vspace{12pt}

\centerline{ {\bf Salvatore Aiola}, on behalf of the ALICE Collaboration }

\vspace{12pt}

\centerline{Physics Department, Yale University,
}\centerline{New Haven, CT 06511, USA, {\it salvatore.aiola@yale.edu}}

\vspace{12pt}


\vspace{12pt} \vspace{12pt}

%Up to now heavy-quark jets have eluded a precise quantitative description in the framework of perturbative Quantum Chromo-Dynamics (pQCD).
The production of heavy quarks is described in the framework of perturbative Quantum Chromo-Dynamics (pQCD) via the leading order process ${\rm gg\, (q\bar{q})} \rightarrow {\rm Q\bar{Q}}$.
Among higher-order processes, gluon splitting ${\rm gg\, (q\bar{q})} \rightarrow {\rm gg} \rightarrow \rm gc\bar{c}$
is known to account for a large fraction of the charm produced at the LHC.
Charmed hadrons coming from the fragmentation of charm quarks
produced in the leading order process
are expected to carry a larger fraction of the total jet momentum,
as compared to those coming from gluon splitting.
Therefore the measurement of the charm jet fragmentation functions (FFs) 
can be used to improve our understanding of the charm production mechanisms.

Heavy-flavor jets can also provide important insights into the Quark-Gluon Plasma (QGP)
produced in ultra-relativistic heavy-ion collisions, as heavy quarks are predicted
to interact with the QGP constituents differently compared to light quarks and gluons. 
Measuring charm jets gives a better estimate of the initial parton energy
when compared to single-hadron measurements.
A better handle on the initial parton energy opens the possibility of more precise tests of 
heavy-quark transport models of the QGP in the low momentum region,
where the effect of the quark mass is expected to be more prominent.
In addition, the internal structure of the jet can also provide important information
about how the lost energy is dissipated in the medium.

We present the measurement of the production of charm jets tagged with fully reconstructed D$^0$ mesons in minimum bias pp collisions at $\sqrt{s} = 7$~TeV with the ALICE detector. 
The production of charm jets is investigated both in the $p_{\rm T}$-differential cross section and in the distribution of the jet momentum fraction ($z$) carried by the D$^0$ meson.
D$^0$-meson candidates are identified in their hadronic decay channel D$^0\rightarrow$K$^{-}\pi^{+}$ and combined with the other tracks reconstructed in the central barrel using the anti-$k_{\rm T}$ jet-finding algorithm.
\end{document}