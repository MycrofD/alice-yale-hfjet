\documentclass[12pt]{article}
%
\pagestyle{empty} \setlength{\topmargin}{-0.2in}
\setlength{\textheight}{9.9in} \setlength{\oddsidemargin}{-0.1in}
\setlength{\textwidth}{6.5in}
\parindent=20pt \parskip=0pt
%
\begin{document}

\centerline{\bf 
Measurement of the $p_{\rm T}$-differential cross section and
}

\centerline{\bf 
fragmentation function of D$^0$-tagged jets in pp collisions with ALICE
}

\vspace{12pt}

\centerline{ {\bf Salvatore Aiola}, on behalf of the ALICE Collaboration }

\vspace{12pt}

\centerline{Physics Department, Yale University,
}\centerline{New Haven, CT 06511, USA, {\it salvatore.aiola@yale.edu}}

\vspace{12pt}


\vspace{12pt} \vspace{12pt}

Jets are a fundamental feature of high-energy particle interactions. 
They result from the fragmentation of hard-scattered partons, 
a key process of Quantum Chromodynamics (QCD). 
The study of the charm content of jets is interesting because up to now
it has eluded a precise quantitative understanding in the framework of perturbative
QCD (pQCD). This is in contrast with other hard processes that are successfully described
by pQCD, such as top and bottom production and the inclusive jet cross section.
The charm content of jets is known to arise both from prompt production in the process ${\rm gg\, (q\bar{q})} \rightarrow {\rm c\bar{c}}$, and
from the parton shower of gluons and light-flavor quarks.
The relative contribution of these two competing mechanisms is understood only qualitatively
and it is known to depend on the center-of-mass energy of the two colliding protons.
Furthermore charm hadrons coming from the fragmentation of prompt charm jets 
are expected to carry a larger fraction of the total jet momentum,
as compared to those where the charm content arises later in the
parton shower. Therefore the measurement of charm jet fragmentation functions (FFs) 
can be used to estimate the relative strength of the two mechanisms.

Heavy-flavor jets can also provide important insights into the Quark-Gluon Plasma (QGP)
produced in ultra-relativistic heavy-ion collisions, as heavy quarks are predicted
to interact with the QGP differently compared to light quarks and gluons. 
However, their production mechanisms must first be studied in the vacuum, 
in order to provide a baseline for the observation of possible modifications induced by the presence of the QGP. 

We present the current status of the measurement of jets that contain a D meson (D-tagged jets) using the ALICE detector.
The aim of the analysis is to extract both the $p_{\rm T}$ spectrum of the D-tagged jets and the jet-momentum fraction of the D mesons. 
We identify D-meson candidates via their hadronic decay channels using topological selections and particle identification.
These D-meson candidates are combined with the other charged tracks reconstructed by the central tracking system, 
using the anti-$k_{\rm T}$ jet-finding algorithm.
We extract the yield of D-tagged jets through an invariant mass analysis of the D-meson candidates associated with each jet, 
in bins of jet $p_{\rm T}$ and momentum fraction carried by the D meson. 
For this analysis we use data collected
by ALICE with minimum bias triggers in pp collisions at 7 TeV. We will discuss also
the perspectives for the same measurement in pp collisions at 8 and 13 TeV using events triggered
by the electromagnetic calorimeters.

\end{document}