\documentclass[12pt]{article}
%
\pagestyle{empty} \setlength{\topmargin}{-0.2in}
\setlength{\textheight}{9.9in} \setlength{\oddsidemargin}{-0.1in}
\setlength{\textwidth}{6.5in}
\parindent=20pt \parskip=0pt
%
\begin{document}

\centerline{\bf 
Measurement of the $p_{\rm T}$-differential cross section and
}

\centerline{\bf 
fragmentation function of D$^0$-tagged jets in pp collisions with ALICE
}

\vspace{12pt}

\centerline{ {\bf Salvatore Aiola}, on behalf of the ALICE Collaboration }

\vspace{12pt}

\centerline{Physics Department, Yale University,
}\centerline{New Haven, CT 06511, USA, {\it salvatore.aiola@yale.edu}}

\vspace{12pt}


\vspace{12pt} \vspace{12pt}

The study of heavy-quark jets is interesting because up to now
it has eluded a precise quantitative description in the framework of perturbative
Quantum Chromo-Dynamics (pQCD).
Heavy quarks are produced in the leading order process ${\rm gg\, (q\bar{q})} \rightarrow {\rm Q\bar{Q}}$.
Among higher-order processes, gluon splitting ${\rm gg\, (q\bar{q})} \rightarrow {\rm gg} \rightarrow \rm gc\bar{c}$
is known to account for a large fraction of the charm produced at the LHC.
Charmed hadrons coming from the fragmentation of charm quarks
produced in the leading order process
are expected to carry a larger fraction of the total jet momentum,
as compared to those coming from gluon splitting or other higher order processes. 
Therefore the measurement of the charm jet fragmentation functions (FFs) 
can be used to improve our understanding of the charm production mechanisms.

Heavy-flavor jets can also provide important insights into the Quark-Gluon Plasma (QGP)
produced in ultra-relativistic heavy-ion collisions, as heavy quarks are predicted
to interact with the QGP differently compared to light quarks and gluons. 
However, the differences between the heavy-quark and light-quark dynamics in the medium
are expected to be negligible, except for momenta comparable to the mass of the charm quark $\sim 1.3$~GeV/$c^2$.
Measuring charm jets gives a better estimate of the initial parton energy,
thus allowing a more precise test of heavy-quark transport models of the QGP in the low momentum region.

We present the measurement of the $p_{\rm T}$-differential cross section and FF of charm jets tagged with fully reconstructed D$^0$ mesons (D$^0$-tagged jets) using the ALICE detector.
We identify D$^0$-meson candidates in their hadronic decay channel D$^0\rightarrow \pi^{-}$K$^{+}$, using topological selections and particle identification to reduce the combinatorial background.
Jets are reconstructed out of the D$^0$ candidates and the other charged particles measured by the central tracking system, 
using the anti-$k_{\rm T}$ jet-finding algorithm.
We extract the yield of D$^0$-tagged jets through an invariant mass analysis of the D-meson candidates associated with each jet, 
to obtain the yield as a function of jet $p_{\rm T}$ and as a function of the momentum fraction carried by the D$^0$ meson. 
For this analysis we use data collected
by ALICE with minimum bias triggers in pp collisions at 7 TeV. 
\end{document}