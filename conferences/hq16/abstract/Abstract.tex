\documentclass[12pt]{article}
%
\pagestyle{empty} \setlength{\topmargin}{-0.2in}
\setlength{\textheight}{9.9in} \setlength{\oddsidemargin}{-0.1in}
\setlength{\textwidth}{6.5in}
\parindent=0pt \parskip=0pt
%
\begin{document}

\centerline{\bf 
Measurement of D-meson tagged jets
}

\centerline{\bf 
in pp collisions at 7 TeV with ALICE
}

\vspace{12pt}

\centerline{ {\bf Salvatore Aiola}$^{\rm a}$ on behalf of the ALICE Collaboration }

\vspace{12pt}

\centerline{$^{\rm a}$Physics Department, Yale University,
}\centerline{New Haven, CT 06511, USA, {\it salvatore.aiola@yale.edu}}

\vspace{12pt}


\vspace{12pt} \vspace{12pt}

Jets are a fundamental feature of high-energy particle interactions. 
They result from the fragmentation of hard-scattered partons, 
a key process of Quantum Chromodynamics (QCD). 
In particular, the production and the internal properties of heavy-flavor jets 
in pp collisions are not yet satisfactorily described by neither analytical nor 
phenomenological approaches to QCD. Measurements are needed
in order to provide important constraints to models inspired by perturbative QCD 
and Monte-Carlo generators, such as PYTHIA and POWHEG, widely used in high-energy particle physics.

Heavy-flavor jets can also provide important insights into the Quark-Gluon Plasma (QGP)
produced in ultra-relativistic heavy-ion collisions, as heavy quarks are predicted
to interact with the QGP differently compared to light quarks and gluons. 
However, their production mechanisms must first be studied in the vacuum, 
in order to provide a baseline for the observation of possible modifications induced by the presence of the QGP. 

We present the current status of the measurement of jets that contain a D meson (D-tagged jet) with \mbox{ALICE}.
The aim of the analysis is to extract both the $p_{\rm T}$ spectrum of the D-tagged jets and the jet-momentum fraction of the D mesons. 
We identify D-meson candidates via their hadronic decay channels using topological selections and particle identification.
These D-meson candidates are combined with the other charged tracks reconstructed by the central tracking system, 
using the anti-$k_{\rm T}$ jet-finding algorithm.
We extract the yield of D-tagged jets through an invariant mass analysis of the D-meson candidates associated with each jet, 
in bins of jet $p_{\rm T}$ and momentum fraction carried by the D meson. Finally we use a standard unfolding procedure 
to correct the jet $p_{\rm T}$ spectrum for detector inefficiencies and momentum resolution. For this analysis we use data collected
by ALICE with minimum bias triggers in pp collisions at 7 TeV. We will discuss also
the perspectives for the same measurement in Pb--Pb and p--Pb collisions.

\end{document}