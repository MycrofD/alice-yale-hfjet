% $Header: /Users/joseph/Documents/LaTeX/beamer/solutions/conference-talks/conference-ornate-20min.en.tex,v 90e850259b8b 2007/01/28 20:48:30 tantau $

\documentclass[xcolor={usenames,dvipsnames}]{beamer}

% This file is a solution template for:

% - Talk at a conference/colloquium.
% - Talk length is about 20min.
% - Style is ornate.



% Copyright 2004 by Till Tantau <tantau@users.sourceforge.net>.
%
% In principle, this file can be redistributed and/or modified under
% the terms of the GNU Public License, version 2.
%
% However, this file is supposed to be a template to be modified
% for your own needs. For this reason, if you use this file as a
% template and not specifically distribute it as part of a another
% package/program, I grant the extra permission to freely copy and
% modify this file as you see fit and even to delete this copyright
% notice. 


\mode<presentation>
{
  \usetheme{Copenhagen}
  \usecolortheme{dolphin}
}
  % or ...

  \setbeamercovered{transparent}
  % or whatever (possibly just delete it)


\usepackage[percent]{overpic}
\usepackage[english]{babel}
\usepackage{setspace}
\usepackage{comment}
% or whatever

\usepackage[latin1]{inputenc}
% or whatever

\usepackage[at]{easylist}
\usepackage{times}
\usepackage[T1]{fontenc}
\usepackage{xcolor}
% Or whatever. Note that the encoding and the font should match. If T1
% does not look nice, try deleting the line with the fontenc.

\definecolor{darkestblue}{RGB}{1,8,100}
\definecolor{darkerblue}{RGB}{3,17,150}
\definecolor{darkblue}{RGB}{7,26,200}
\definecolor{lightred}{RGB}{202,103,104}
\definecolor{lightgreen}{RGB}{106,202,107}

%particles
\newcommand{\jpsi}{\rm J/$\psi$}
\newcommand{\psip}{$\psi^\prime$}
\newcommand{\jpsiDY}{\rm J/$\psi$\,/\,DY}
\newcommand{\chic}{$\chi_{\rm c}$}
\newcommand{\pip}{$\pi^{+}$}
\newcommand{\pim}{$\pi^{-}$}
\newcommand{\pizero}{$\pi^{0}$}
\newcommand{\kap}{K$^{+}$}
\newcommand{\kam}{K$^{-}$}
\newcommand{\pbar}{$\rm\overline{p}$}
\newcommand{\ccbar}{\ensuremath{\mathrm{c\overline{c}}}}
\newcommand{\bbbar}{\ensuremath{\mathrm{b\overline{b}}}}
\newcommand{\Dzero}{\ensuremath{\mathrm{D^{0}}}}
\newcommand{\Dzerobar}{\ensuremath{\mathrm{\overline{D}^{0}}}}
\newcommand{\Dpm}{\ensuremath{\mathrm{D^{\pm}}}}
\newcommand{\Ds}{\ensuremath{\mathrm{D_{s}^{\pm}}}}
\newcommand{\Dstar}{\ensuremath{\mathrm{D^{*\pm}}}}

%collision systems
\newcommand{\pp}{pp}
\newcommand{\pPb}{p--Pb}
\newcommand{\PbPb}{Pb--Pb}

%detectors
\newcommand{\ezdc}{$E_{\rm ZDC}$}

%units
\newcommand{\GeVc}{GeV/$c$}
\newcommand{\GeVcsq}{GeV/$c^2$}

%others
\newcommand{\degree}{$^{\rm o}$}
\newcommand{\s}{\ensuremath{\sqrt{s}}}
\newcommand{\snn}{\ensuremath{\sqrt{s_{\rm NN}}}}
\newcommand{\y}{\ensuremath{y}}
\newcommand{\pt}{\ensuremath{p_{\rm T}}}
\newcommand{\dedx}{d$E$/d$x$}
\newcommand{\dndy}{d$N$/d$y$}
\newcommand{\dndydpt}{${\rm d}^2N/({\rm d}y {\rm d}p_{\rm t})$}
\newcommand{\zpar}{\ensuremath{z_{||}}}
\newcommand{\zpargen}{\ensuremath{z_{||}^{\mathrm{part}}}}
\newcommand{\zpardet}{\ensuremath{z_{||}^{\mathrm{det}}}}
\newcommand{\ptchjet}{\ensuremath{p_{\mathrm{T,ch\, jet}}}}
\newcommand{\ptjet}{\ensuremath{p_{\mathrm{T,jet}}}}
\newcommand{\ptchjetgen}{\ensuremath{p_{\mathrm{T,ch\,jet}}^{\mathrm{truth}}}}
\newcommand{\ptchjetdet}{\ensuremath{p_{\mathrm{T,ch\,jet}}^{\mathrm{reco}}}}
\newcommand{\ptd}{\ensuremath{p_{\mathrm{T,D}}}}
\newcommand{\ptdgen}{\ensuremath{p_{\mathrm{T,D}}^{\mathrm{truth}}}}
\newcommand{\ptddet}{\ensuremath{p_{\mathrm{T,D}}^{\mathrm{reco}}}}
\newcommand{\antikt}{anti-\ensuremath{k_{\mathrm{T}}}}
\newcommand{\kt}{\ensuremath{k_{\mathrm{T}}}}
\newcommand{\pthard}{\ensuremath{p_{\mathrm{T,hard}}}}

\begin{comment}
\AtBeginSection[]{
  \begin{frame}
  \vfill
  \centering
  \begin{beamercolorbox}[sep=8pt,center,shadow=true,rounded=true]{title}
    \usebeamerfont{title}\insertsectionhead\par%
  \end{beamercolorbox}
  \vfill
  \end{frame}
}
\end{comment}

\setbeamersize{text margin left=15pt,text margin right=15pt} 

\title[Jets and heavy flavor in heavy-ion collisions with ALICE] % (optional, use only with long paper titles)
{New results on jets and heavy flavor in heavy-ion collisions with ALICE}

\author[Salvatore Aiola (Yale University)]% (optional, use only with lots of authors)
{Salvatore Aiola, \\
on behalf of the ALICE Collaboration}
% - Give the names in the same order as the appear in the paper.
% - Use the \inst{?} command only if the authors have different
%   affiliation.

\institute[Yale University] % (optional, but mostly needed)
{Yale University}

\date[May 18th, 2017] % (optional, should be abbreviation of conference name)
{LHCP Conference \\
Shanghai Jiao Tong University, China\\
May 18th, 2017}
% - Either use conference name or its abbreviation.
% - Not really informative to the audience, more for people (including
%   yourself) who are reading the slides online

\newcommand*\oldmacro{}%
\let\oldmacro\insertshorttitle%
\renewcommand*\insertshorttitle{%
   \oldmacro\hfill%
   \insertframenumber\,/\,\inserttotalframenumber}

\subject{High-Energy Physics}
% This is only inserted into the PDF information catalog. Can be left
% out. 



% If you have a file called "university-logo-filename.xxx", where xxx
% is a graphic format that can be processed by latex or pdflatex,
% resp., then you can add a logo as follows:

% \pgfdeclareimage[height=0.5cm]{university-logo}{university-logo-filename}
% \logo{\pgfuseimage{university-logo}}


% If you wish to uncover everything in a step-wise fashion, uncomment
% the following command: 

%\beamerdefaultoverlayspecification{<+->}


\begin{document}

\begin{frame}
  \titlepage
\end{frame}

\begin{frame}{Outline}
   \tableofcontents
\end{frame}


% Structuring a talk is a difficult task and the following structure
% may not be suitable. Here are some rules that apply for this
% solution: 

% - Exactly two or three sections (other than the summary).
% - At *most* three subsections per section.
% - Talk about 30s to 2min per frame. So there should be between about
%   15 and 30 frames, all told.

% - A conference audience is likely to know very little of what you
%   are going to talk about. So *simplify*!
% - In a 20min talk, getting the main ideas across is hard
%   enough. Leave out details, even if it means being less precise than
%   you think necessary.
% - If you omit details that are vital to the proof/implementation,
%   just say so once. Everybody will be happy with that.


%\begin{overpic}[width=\textwidth, trim=0 0 0 0, clip]{img/823_D0_Charged_R040_JetPtBins_DPt_30}
%\end{overpic}

%\begin{columns}
%\column{0.5\textwidth}
%\column{0.5\textwidth}
%\end{columns}

\section{Introduction}

\begin{frame}[fragile]{Probing the Quark-Gluon Plasma}
\begin{columns}
\column{0.7\textwidth}
\small
\begin{easylist}[itemize]
@ Droplet of Quark-Gluon Plasma (QGP) produced in ultra-relativistic heavy-ion collisions
@@ Extremely hot (... MeV), dense (), small () and short-lived ()
@ Self-produced hard probes: energetic partons moving through the QGP
@ Factorization theorem: 
@@ $Q^2\gtrsim1\,\mathrm{GeV}^{2}$: produced before the QGP formation time
@@ Evolution through the QGP
@@ Fragmentation
\end{easylist}
\column{0.3\textwidth}
\begin{overpic}[width=\textwidth, trim=0 0 0 0, clip]{img/jetquenching}
\end{overpic}
\\
\begin{overpic}[width=\textwidth, trim=0 0 0 0, clip]{img/bullet}
\end{overpic}
\end{columns}
\end{frame}

\subsection{Jets}
\begin{frame}{Jet Yield Suppression}
\begin{columns}
\column{0.5\textwidth}
\begin{overpic}[width=\textwidth, trim=0 0 0 0, clip]{img/jets/2016-Sep-22-Jet_RAAComp0010}
\end{overpic}
\column{0.5\textwidth}
\small
\begin{itemize}
\item Nuclear modification factor:\\
\vspace{4pt}
$R_{\rm AA} = \frac{\mathrm{d}N_{\rm jets}^{\rm AA} / \mathrm{d}\pt}{T_{\rm AA}\mathrm{d}\sigma_{\rm jets}^{\rm\pp} / \mathrm{d}\pt}$,\\
\vspace{4pt}
where $T_{\rm AA}$ is the nuclear ``thickness'' of the Pb nuclei
\item Compilation of jet $R_{\rm AA}$ results from Run-1 ($\snn=2.76$~TeV) and Run-2 ($\snn=5.02$~TeV)
\item Strong suppression, slowly decreasing with \pt\
\end{itemize}
\end{columns}
\begin{itemize}
\item Where does the energy go?
\item Is the internal structure of the jet modified?
\item What is the energy loss mechanism? Path length dependence?
\end{itemize}
\end{frame}

\subsection{Heavy-Flavor}

\begin{frame}[fragile]{Heavy-Flavor $R_{\rm AA}$}
\small
\begin{itemize}
\item Charm and beauty produced in high-$Q^{2}$ processes
\item Negligible thermal production in the QGP ($m_{\rm c,b}\gg T_{\rm QGP}$)
\item Heavy-flavor ``puzzle'': early expecations $R_{\rm AA}^{\rm B} > R_{\rm AA}^{\rm D} > R_{\rm AA}^{\rm \pi}$
\end{itemize}
\begin{columns}
\column{0.36\textwidth}
\begin{overpic}[width=1.15\textwidth, trim=0 0 0 0, clip]{img/hf/2015-Sep-24-AverageDmesonRaavspT_Pions_Nch_010}
\end{overpic}
\column{0.64\textwidth}
\hspace{-50pt}
\small
\begin{easylist}[itemize]
@ Advancements in the energy loss models
@@ Radiative vs. collisional energy loss%~\cite{Djordjevic:2014,Blagojevic:2015,Alberico:2011}
@@ Fragmentation and hadronic phase%~\cite{Djordjevic:2014}
@@ Temperature dependence%~\cite{Das:2015}
@ Models still not very well constrained
@@ Multiple theoretical approaches consistent with the measured $R_{\rm AA}^{\rm D}$ (and some also with $v_{2}^{\rm D}$)
\end{easylist}
\end{columns}
\end{frame}

\section{\pPb\ collisions}

\begin{frame}[fragile]{QGP in small systems?}
\begin{easylist}[itemize]
@ QGP-like effects observed in the soft sector
@@ Elliptic flow
@@ Strangeness enhancement
@ What about hard probes?
@@ Jet quenching not observed (yet?)
@@ Difficult to characterize the event activity classes $\approx$ centrality classes in \PbPb\
\end{easylist}
\end{frame}

\subsection{Jets}

\begin{frame}{Semi-Inclusive Hadron-Jet Production}
\begin{columns}
\column{0.5\textwidth}
\begin{overpic}[width=.9\textwidth, trim=0 0 0 0, clip]{img/jets/2017-Feb-01-prelim_spectraDrecoilZNA_R04}
\end{overpic}
\column{0.5\textwidth}
\begin{overpic}[width=.8\textwidth, trim=0 0 0 0, clip]{img/jets/2017-Feb-01-ppb5_RCP_pap_AKT04_ZNA_split2}
\end{overpic}
\end{columns}
\small
\begin{itemize}
\item Jets recoiled from high-\pt\ hadron
\item $\Delta_{\rm recoil} =$~TT\{12,50\}$-$TT\{6,7\} $\rightarrow$ suppress combinatorial background from underlying event
\item \alert{No modification of the jet yield observed in high-multiplicity \pPb}
\end{itemize}
\end{frame}

\begin{frame}{Jet Hard Substructure}
\begin{overpic}[width=\textwidth, trim=0 0 0 0, clip]{img/jets/2017-Feb-01-zg_unfolded_20GeV_ALL}
\end{overpic}
\vspace{-10pt}
\small
\begin{itemize}
\item ---Insert formula for zg ---
\item Look for modification of the jet hard substructure
\item \alert{No modification observed in minimum-bias \pPb\ compared to PYTHIA}
\item Next: redo the analysis in multiplicity classes, measure a \pp\ baseline
\end{itemize}
\end{frame}

\subsection{Heavy Flavor}

\begin{frame}{D mesons production vs. multiplicity}
\begin{columns}
\column{0.5\textwidth}
\begin{overpic}[width=.8\textwidth, trim=0 0 0 0, clip]{img/hf/2016-May-24-AverageDmesQpPb_ZNA_AllCentr_FDcomb_paper}
\end{overpic}
\footnotesize
\begin{itemize}
\item No suppression observed for high-\pt\ D mesons
\item \alert{No ordering w.r.t. multiplicity classes}
\end{itemize}
\column{0.5\textwidth}
\begin{overpic}[width=.8\textwidth, trim=0 0 0 0, clip]{img/hf/2016-May-24-Daverage_vsPt_pp1_pPb1_Nch_xeqyline}
\end{overpic}
\footnotesize
\begin{itemize}
\item Hint of a deviation from \pp\ at high multiplicity
\end{itemize}
\end{columns}
\end{frame}

\begin{frame}{Azimuthal D-h correlations}
\begin{columns}
\column{0.5\textwidth}
\begin{overpic}[width=\textwidth, trim=0 0 0 0, clip]{img/hf/plotComparison_WeightedAverage_pp_pPb_UniqueCanvas_Style1_18044}
\end{overpic}
\column{0.5\textwidth}
\small
\begin{itemize}
\item Powerful tool to study modification of the fragmentation of charm jets
\item \alert{No modification observed} in minimum-bias \pPb\ ($\snn=5$~TeV) compared to a \pp\ baseline ($\snn=7$~TeV)
\end{itemize}
\end{columns}
\end{frame}

\begin{frame}{Heavy-Flavor Muon Forward/Backward Ratio}
\begin{columns}
\column{0.65\textwidth}
\begin{overpic}[width=\textwidth, trim=0 0 0 0, clip]{img/hf/2017-Feb-05-Fig4}
\end{overpic}
\column{0.35\textwidth}
\begin{itemize}
\item ALICE muon arm: $\eta_{\rm lab} > xx$
\item Cold Nuclear Matter effects more pronounced at large rapidities
\item Measurement in agreement with NLO calculations with nuclear shadowing
\end{itemize}
\end{columns}
 \alert{Crucial to disentangle from QGP phenomelogy}
\end{frame}

\section{\PbPb\ collisions}

\subsection{Jets}

\begin{frame}{Jet-Hadron Correlations}
\begin{columns}
\column{0.5\textwidth}
\begin{overpic}[width=\textwidth, trim=0 0 0 0, clip]{img/jets/2017-Feb-02-NearSideYratioJ20-40_C30-50L_FINAL}
\end{overpic}
\column{0.5\textwidth}
\begin{overpic}[width=\textwidth, trim=0 0 0 0, clip]{img/jets/2017-Feb-02-AwaySideYratioJ20-40_C30-50L_FINAL}
\end{overpic}
\end{columns}
\footnotesize
\begin{itemize}
\item Partons expected to loose more energy when traversing more medium (out-of-plane) [add cartoon?]
\item Models predict that path-length is a subleading effect (compared to energy loss fluctuations and out-of-cone radiation) [cite jewel paper]
\item \alert{No difference observed between in- and out-of-plane jet-hadron yields} (in qualitative agreement with models)
\end{itemize}
\end{frame}

\begin{frame}{Jet Mass}
\begin{columns}
\column{0.75\textwidth}
\begin{overpic}[width=\textwidth, trim=0 0 0 0, clip]{img/jets/cMassPbPbNoKineCor_v2-22938}
\end{overpic} \\
\begin{overpic}[width=\textwidth, trim=0 0 0 0, clip]{img/jets/cRatioPbPbOpPbNoKineCor_v2-22932}
\end{overpic}
\column{0.25\textwidth}
\footnotesize
\begin{itemize}
\item Jet mass measures the ``virtuality'' of the jet
\item Expected to increase in jet-medium interactions (JEWEL)
\end{itemize}
\end{columns}
\small
\alert{Measurement shows no significant deviations from \pPb\ and from PYTHIA}
\end{frame}

\begin{frame}{Nsubjettiness}
\begin{columns}
\column{0.7\textwidth}
\begin{overpic}[width=\textwidth, trim=0 0 0 0, clip]{img/jets/2017-Feb-03-Tau2to1_40to60_Full_Results_0}
\end{overpic} 
\column{0.3\textwidth}
\begin{itemize}
\item $\tau_2 / \tau_1 = 1 \rightarrow$ the jet has two hard cores
\item \alert{No modification observed compared to PYTHIA}
\end{itemize}
\end{columns}
\end{frame}

\subsection{Heavy Flavor}

\begin{frame}{Nuclear Modification Factor}
\begin{columns}
\column{0.5\textwidth}
\begin{overpic}[width=.9\textwidth, trim=0 0 0 0, clip]{img/hf/2017-Apr-14-DmesonAverage_3050_2011comparison_14Apr2017}
\end{overpic} 
\column{0.5\textwidth}
\begin{overpic}[width=.9\textwidth, trim=0 0 0 0, clip]{img/hf/2017-Feb-01-RAA_Run1_2_0_10_new}
\end{overpic} 
\end{columns}
\footnotesize
\begin{itemize}
\item $R_{\rm AA}$ measured in Run-2 with much greater precision
\item No dependence on $\snn$ observed
\item Potential to be more constraining for the models
\end{itemize}
\end{frame}

\begin{frame}{D-meson Elliptic Flow}
\begin{columns}
\column{0.5\textwidth}
\begin{overpic}[width=.8\textwidth, trim=0 0 0 0, clip]{img/hf/2017-Feb-02-DmesonAveragev2_276_5_comparison}
\end{overpic} 
\column{0.5\textwidth}
\begin{overpic}[width=.8\textwidth, trim=0 0 0 0, clip]{img/hf/2017-Feb-02-D0DplusAveragev2_Comparison_with_pions_3040}
\end{overpic}
\end{columns}
\begin{itemize}
\item $v_2$ of D mesons measured in Run-2 with much greater precision
\item At low-\pt $v_2(D) < v_2(\pi)$
\end{itemize}
\end{frame}

\begin{frame}{e-h Correlations}
\begin{overpic}[width=.7\textwidth, trim=0 0 0 0, clip]{img/hf/2017-Jan-31-DphiHFEwSys4Bins}
\end{overpic} 
\end{frame}

\section{Conclusions and Outlook}

\begin{frame}{Summary}
\end{frame}

\begin{frame}{Future Plans}
\end{frame}

\section*{Extra Slides}

\subsection*{Jets in \pPb}

\begin{frame}{Jet Mass}
\begin{overpic}[width=\textwidth, trim=0 0 0 0, clip]{img/jets/cMasspPbNoKineCor_v2-22926}
\end{overpic}\\
\begin{overpic}[width=\textwidth, trim=0 0 0 0, clip]{img/jets/2016-Sep-22-cRatiopPbOPyNoKineCor}
\end{overpic}
\end{frame}

\end{document}
