\documentclass[12pt]{article}
%
\pagestyle{empty} \setlength{\topmargin}{-0.2in}
\setlength{\textheight}{9.9in} \setlength{\oddsidemargin}{-0.1in}
\setlength{\textwidth}{6.5in}
\parindent=20pt \parskip=0pt
%
\begin{document}

\centerline{\bf 
Study of the heavy-flavor production 
}

\centerline{\bf 
in pp collision with ALICE
}

\vspace{12pt}

\centerline{ {\bf Salvatore Aiola}, on behalf of the ALICE Collaboration }

\vspace{12pt}

\centerline{Physics Department, Yale University,
}\centerline{New Haven, CT 06511, USA, {\it salvatore.aiola@yale.edu}}

\vspace{12pt}


\vspace{12pt} \vspace{12pt}
The study of heavy-flavor production in ultra-relativistic heavy-ion collisions has already produced
a stunning amount of evidence of the Quark-Gluon Plasma (QGP) and of its transport properties. 
Heavy quarks are unique for several reasons: they are genuine hard probes, even at low momentum;
their interaction with the QGP can modify their phase-space distribution, but not their flavor, with very low
(and calculable) thermal destruction and creation rates in the hot nuclear medium being probed.
However, the precise interpretation of some recent results calls for a more comprehensive and deeper understanding
of the properties of in-vacuum heavy-flavor production.

The study of the charm content of jets is interesting because 
a precise quantitative understanding in the framework of perturbative
Quantum Chromodynamics (pQCD) is still lacking.
The charm content of jets is known to arise both from prompt production in the process ${\rm gg\, (q\bar{q})} \rightarrow {\rm c\bar{c}}$, and
from the parton shower of gluons and light-flavor quarks.
The relative contribution of these two competing mechanisms is understood only qualitatively
and it is known to depend on the center-of-mass energy of the two colliding protons.
The measurement of charm jet fragmentation functions is sensitive to the production mechanism and 
can be used to estimate their relative strengths.

At the same time the measurement of the charm production cross section down
to low momentum is essential to measure precisely the total cross section, a key ingredient in the calculation
of the thermal charm generation in the QGP. Furthermore important constraints on the low-$x$ part of the gluon parton distribution functions can come
as well from these measurements at different center-of-mass collisional energies.

We present a comprehensive overview of the recent \mbox{ALICE} heavy-flavor measurements in pp collisions from both the LHC Run-1 and Run-2 data,
including the first measurement of charm jet production cross section and fragmentation function.

\end{document}