\documentclass[12pt]{article}
%
\pagestyle{empty} \setlength{\topmargin}{-0.2in}
\setlength{\textheight}{9.9in} \setlength{\oddsidemargin}{-0.1in}
\setlength{\textwidth}{6.5in}
\parindent=20pt \parskip=0pt
\usepackage{lineno}
\linenumbers
%
\begin{document}

\centerline{\bf 
Exploring new differential observables
}

\centerline{\bf 
for heavy-flavor production in pp collisions with ALICE
}

\vspace{12pt}

\centerline{ {\bf Salvatore Aiola}, on behalf of the ALICE Collaboration }

\vspace{12pt}

\centerline{Physics Department, Yale University,
}\centerline{New Haven, CT 06511, USA, {\it salvatore.aiola@yale.edu}}

\vspace{12pt}


\vspace{12pt} \vspace{12pt}
The study of heavy-flavor production in ultra-relativistic heavy-ion collisions has already produced
a remarkable amount of evidence of the Quark-Gluon Plasma (QGP) formation and of its transport properties. 
Heavy quarks are unique probes for several reasons: they are genuine hard probes, even at low momentum;
their interaction with the QGP can modify their phase-space distribution, but not their flavor, with very low
(and calculable) thermal destruction and creation rates in the hot nuclear medium being probed.
However, a precise interpretation of some recent results calls for a more comprehensive and deeper understanding
of the properties of in-vacuum heavy-flavor production.

The study of the charm content of jets is interesting because 
a quantitative description in the framework of perturbative
Quantum Chromodynamics (pQCD) is still lacking.
The charm content of jets containing prompt D mesons is known to arise both from prompt production 
in the process ${\rm gg\, (q\bar{q})} \rightarrow {\rm c\bar{c}}$, and
from the parton shower of gluons and light-flavor quarks.
The relative contribution of these two competing mechanisms is understood only qualitatively
and it is known to depend on the center-of-mass energy of the two colliding protons.
The measurement of charm jet fragmentation functions is sensitive to the production mechanism and 
can be used to estimate their relative strengths.

The full reconstruction of jets with charm content is therefore complementary to other observables sensitive to the charm quark fragmentation, 
such as the shape and yield of angular correlations of D mesons with
charged particles. The charm production mechanism is also explored in high multiplicity pp events,
in order to constrain better the role of Multi-Parton Interactions (MPI).
Furthermore important constraints on the low-$x$ part of the gluon parton distribution functions can come
as well from heavy-flavor cross-section measurements at different center-of-mass collision energies.

We present an overview of new and more differential measurements of heavy-flavor observables in pp collisions 
from both the LHC Run-1 and Run-2 data collected by ALICE. In particular the first ALICE measurement of charm jet production cross section 
and fragmentation function in pp collisions at $\sqrt{s}=7$~TeV will be shown. For this measurement, fully reconstructed ${\rm D}^0$ mesons are
used to identify the charm content of the jets and measure their fragmentation down to low $z$ (fraction of jet momentum carried by the D meson).

\end{document}