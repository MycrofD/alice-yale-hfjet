% $Header: /Users/joseph/Documents/LaTeX/beamer/solutions/conference-talks/conference-ornate-20min.en.tex,v 90e850259b8b 2007/01/28 20:48:30 tantau $

\documentclass[xcolor={usenames,dvipsnames}]{beamer}

% This file is a solution template for:

% - Talk at a conference/colloquium.
% - Talk length is about 20min.
% - Style is ornate.



% Copyright 2004 by Till Tantau <tantau@users.sourceforge.net>.
%
% In principle, this file can be redistributed and/or modified under
% the terms of the GNU Public License, version 2.
%
% However, this file is supposed to be a template to be modified
% for your own needs. For this reason, if you use this file as a
% template and not specifically distribute it as part of a another
% package/program, I grant the extra permission to freely copy and
% modify this file as you see fit and even to delete this copyright
% notice. 


\mode<presentation>
{
  \usetheme{AnnArbor}
  % or ...

  \setbeamercovered{transparent}
  % or whatever (possibly just delete it)
 }

\usepackage[percent]{overpic}
\usepackage[english]{babel}
\usepackage{setspace}
% or whatever

\usepackage[latin1]{inputenc}
% or whatever
\usepackage{listings}
\usepackage[at]{easylist}
\usepackage{times}
\usepackage[T1]{fontenc}
\usepackage{xcolor}
\usepackage{setspace}
% Or whatever. Note that the encoding and the font should match. If T1
% does not look nice, try deleting the line with the fontenc.

\definecolor{darkestblue}{RGB}{1,8,100}
\definecolor{darkerblue}{RGB}{3,17,150}
\definecolor{darkblue}{RGB}{7,26,200}
\definecolor{lightred}{RGB}{202,103,104}
\definecolor{lightgreen}{RGB}{106,202,107}

\lstloadlanguages{{C++}}

\lstset{escapeinside={(*@}{@*)}}

\lstdefinestyle{base}{
  language=C++,
  basicstyle=\color{black},
  keywordstyle=\color{red},
  commentstyle=\color{ForestGreen},
}
%\lstdefinestyle{base}{
%  language=C++,
%  basicstyle=\color{black!20},
%  keywordstyle=\color{red!20},
%  commentstyle=\color{ForestGreen!20},
%  moredelim=**[is][\only<+>{\color{black}\lstset{style=highlight}}]{@}{@},
%}

\AtBeginSection[]{
  \begin{frame}
  \vfill
  \centering
  \begin{beamercolorbox}[sep=8pt,center,shadow=true,rounded=true]{title}
    \usebeamerfont{title}\insertsectionhead\par%
  \end{beamercolorbox}
  \vfill
  \end{frame}
}

\title[Memory management in C++] % (optional, use only with long paper titles)
{Manage memory efficiently in your C++ code with smart pointers}

\author[Salvatore Aiola]% (optional, use only with lots of authors)
{Salvatore Aiola}
% - Give the names in the same order as the appear in the paper.
% - Use the \inst{?} command only if the authors have different
%   affiliation.

\institute[Yale University] % (optional, but mostly needed)
{Yale University}

\date[Nov. 3rd, 2017] % (optional, should be abbreviation of conference name)
{ALICE Analysis Tutorial Week \\
CERN, November 3rd, 2017}
% - Either use conference name or its abbreviation.
% - Not really informative to the audience, more for people (including
%   yourself) who are reading the slides online

\subject{High-Energy Physics}
% This is only inserted into the PDF information catalog. Can be left
% out. 



% If you have a file called "university-logo-filename.xxx", where xxx
% is a graphic format that can be processed by latex or pdflatex,
% resp., then you can add a logo as follows:

% \pgfdeclareimage[height=0.5cm]{university-logo}{university-logo-filename}
% \logo{\pgfuseimage{university-logo}}


% If you wish to uncover everything in a step-wise fashion, uncomment
% the following command: 

%\beamerdefaultoverlayspecification{<+->}


\begin{document}

\begin{frame}
  \titlepage
\end{frame}

\begin{frame}{Outline}
   \tableofcontents
\end{frame}


% Structuring a talk is a difficult task and the following structure
% may not be suitable. Here are some rules that apply for this
% solution: 

% - Exactly two or three sections (other than the summary).
% - At *most* three subsections per section.
% - Talk about 30s to 2min per frame. So there should be between about
%   15 and 30 frames, all told.

% - A conference audience is likely to know very little of what you
%   are going to talk about. So *simplify*!
% - In a 20min talk, getting the main ideas across is hard
%   enough. Leave out details, even if it means being less precise than
%   you think necessary.
% - If you omit details that are vital to the proof/implementation,
%   just say so once. Everybody will be happy with that.


%\begin{overpic}[width=\textwidth, trim=0 0 0 0, clip]{img/823_D0_Charged_R040_JetPtBins_DPt_30}
%\end{overpic}

%\begin{columns}
%\column{0.5\textwidth}
%\column{0.5\textwidth}
%\end{columns}

\section{Introduction}

\begin{frame}[fragile]{What is a Pointer?}
\begin{columns}
\column{0.5\textwidth}
\begin{overpic}[width=\textwidth, trim=0 0 0 0, clip]{img/Pointers}
\put(70,22){\footnotesize \textcolor{NavyBlue}{\textbf{5}}}
\end{overpic}
\column{0.5\textwidth}
A pointer is an object whose value ``points to'' another value stored somewhere else in memory
\begin{itemize}
\item it contains a \textbf{memory address}
\item \textcolor{red}{\textbf{dereferencing}}: obtaining the \textcolor{NavyBlue}{\textbf{value}} stored at the pointed location
\item very flexible and powerful tool
\end{itemize}
\end{columns}
\end{frame}

\begin{frame}[fragile]{Using a Pointer}
\scriptsize
\begin{lstlisting}[style=base, gobble=4]
    /* Defining a pointer */
    int* a; // declares a pointer that can point to an integer value
    (*@ \onslide<2-> @*)//DANGER: the pointer points to a random memory portion!
    
    (*@ \onslide<3-> @*)int* b = nullptr; // OK, pointer is initialised to a null memory address

    (*@ \onslide<4-> @*)int* c = new int; // allocate memory for an integer value in the heap
    //and assign its memory address to this pointer
    
    (*@ \onslide<5-> @*)int** d = &a; // this pointer points to a pointer to an integer value
    
    (*@ \onslide<6-> @*)MyObject* e = new MyObject(); // allocate memory for MyObject
    // and assign its memory address to the pointer e
    
    (*@ \onslide<7-> @*)/* Using a pointer */
    int f = *c; // dereferencing a pointer and assigning the pointed
    // value to another integer variable
    
    (*@ \onslide<8-> @*)e->DoSomething(); // dereferencing a pointer and calling
    // the method DoSomething() of the instance of MyObject
    // pointed by e
\end{lstlisting}
\end{frame}

\section{Why a raw pointer is hard to love}

\begin{frame}[fragile]{Memory leak}
\scriptsize
\begin{lstlisting}[style=base, gobble=4]
    void MyAnalysisTask::UserExec()
    {
      TLorentzVector* v = nullptr;
      for (int i = 0; i < InputEvent()->GetNumberOfTracks(); i++) {
        AliVTrack* track = InputEvent()->GetTrack(i);
        if (!track) continue;
        v = new TLorentzVector(track->Px(), 
          track->Py(), track->Pz(), track->M());
    
        // my analysis here
        std::cout << v->Pt() << std::endl;
      }
  
      delete v;
    }
\end{lstlisting}
\small 
\begin{center}
What is the problem with this code?
\end{center}
\end{frame}

\begin{frame}[fragile]{Array or single value?}
\begin{itemize}
\item A pointer can point to a single value or to an array $\rightarrow$ no way to infer it from its declaration
\item Different syntax to destroy (= deallocate, free) the pointed object for arrays and single objects 
\end{itemize}
\scriptsize
\begin{lstlisting}[style=base, gobble=4]
    AliVTrack* FilterTracks();

    void UserExec()
    {
      TLorentzVector *vect = new TLorentzVector(0,0,0,0);
      double *trackPts = new double[100];
      AliVTrack *returnValue = FilterTracks();

      // here use the pointers

      delete vect;
      delete[] trackPts;
      delete returnValue; // or should I use delete[] ??
    }
\end{lstlisting}
\end{frame}

\begin{frame}[fragile]{Double deletes}
\begin{itemize}
\item Each memory allocation should match a corresponding deallocation
\item Difficult to keep track of all memory allocations/deallocations in a large project
\item \textbf{Ownership} of the pointed memory is ambiguous: multiple deletes of the same object may occur
\end{itemize}
\scriptsize
\begin{lstlisting}[style=base, gobble=4]
    AliVTrack* FilterTracks();
    void AnalyzeTracks(AliVTrack* tracks);

    void MyAnalysisTask::UserExec()
    {
      AliVTrack* tracks = FilterTracks();

      AnalyzeTracks(tracks);

      delete[] tracks; // should I actually delete it?? 
      //or was it already deleted by AnalyzeTracks?
    }
\end{lstlisting}
\end{frame}

\section{Smart Pointers}

\begin{frame}[fragile]{Smart Pointers}
\linespread{1.5}
\begin{itemize}
\item Clear (\emph{shared} or \emph{exclusive}) \textbf{ownership} of the pointed object
\pause
\item \alert{\textbf{Automatic garbage collection}}: memory is deallocated when the last pointer goes out of scope
\pause
\item Available since C++11
\end{itemize}
\end{frame}

\begin{frame}[fragile]{Exclusive-Ownership Pointers: \texttt{unique\_ptr}}
\linespread{1.5}
\begin{itemize}
\item Automatic garbage collection with \textbf{\alert{no additional CPU or memory overhead}} (i.e. it uses the same resources as a raw pointer)
\pause
\item \texttt{unique\_ptr} \textbf{owns} the object it points
\pause
\item Memory automatically released when \texttt{unique\_ptr} goes out of scope or when its \texttt{reset(T* ptr)} method is called
\pause
\item Only one \texttt{unique\_ptr} can point to the same memory address
\end{itemize}
\end{frame}

\begin{frame}[fragile]{\texttt{unique\_ptr} example / 1}
\scriptsize
\begin{lstlisting}[style=base, gobble=4]
    void MyFunction() {
      std::unique_ptr<TLorentzVector> vector(new TLorentzVector(0,0,0,0));
      std::unique_ptr<TLorentzVector> vector2(new TLorentzVector(0,0,0,0));
  
      // use vector and vector2
  
      // dereferencing unique_ptr works exactly as a raw pointer
      std::cout << vector->Pt() << std::endl;
  
      // the line below does not compile!
      // vector = vector2;
      // cannot assign the same address to two unique_ptr instances
  
      vector.swap(vector2); // however I can swap the memory addresses
  
      // this also releases the memory previously pointed by vector2
      vector2.reset(new TLorentzVector(0,0,0,0)); 
  
      // objects pointed by vector and vector2 are deleted here
    }
\end{lstlisting}
\end{frame}

\begin{frame}[fragile]{\texttt{unique\_ptr} example / 2}
\scriptsize
\begin{lstlisting}[style=base, gobble=4]
    void MyAnalysisTask::UserExec()
    {
      for (int i = 0; i < InputEvent()->GetNumberOfTracks(); i++) {
        AliVTrack* track = InputEvent()->GetTrack(i);
        if (!track) continue;
        std::unique_ptr<TLorentzVector> v(new TLorentzVector(track->Px(), 
          track->Py(), track->Pz(), track->M()));
    
        // my analysis here
        std::cout << v->Pt() << std::endl;
        // no need to delete
        // v is automatically deallocated after each for loop
      }
    }
\end{lstlisting}
\small 
\begin{center}
No memory leak here! :)
\end{center}
\end{frame}

\begin{frame}[fragile]{Shared-Ownership Pointers: \texttt{shared\_ptr}}
\linespread{1.2}
\begin{itemize}
\item Automatic garbage collection with some CPU and memory overhead
\pause
\item The pointed object is \emph{collectively owned} by one or more \texttt{shared\_ptr} instances
\pause
\item Memory automatically released the last \texttt{shared\_ptr} goes out of scope or when it is re-assigned
\end{itemize}
\vspace{-10pt}
\begin{center}
\begin{overpic}[width=.5\textwidth, trim=0 0 0 0, clip]{img/Sharedptr}
\end{overpic}
\end{center}
\end{frame}

\begin{frame}[fragile]{\texttt{shared\_ptr} example / 1}
\scriptsize
\begin{lstlisting}[style=base, gobble=4]
    void MyFunction() {
      std::shared_ptr<TLorentzVector> vector(new TLorentzVector(0,0,0,0));
      std::shared_ptr<TLorentzVector> vector2(new TLorentzVector(0,0,0,0));
  
      // dereferencing shared_ptr works exactly as a raw pointer
      std::cout << vector->Pt() << std::endl;
  
      // assignment is allowed between shared_ptr instances
      vector = vector2; 
      // the object previously pointed by vector is deleted!
      // vector and vector2 now share the ownership of the same object
  
      // object pointed by both vector and vector2 is deleted here
    }
\end{lstlisting}
\end{frame}

\begin{frame}[fragile]{\texttt{shared\_ptr} example / 2}
\scriptsize
\begin{lstlisting}[style=base, gobble=4]
    class MyClass {
      public:
        MyClass();
      private:
        void MyFunction();
        std::shared_ptr<TLorentzVector> fVector;
    };

    void MyClass::MyFunction() {
      std::shared_ptr<TLorentzVector> vector(new TLorentzVector(0,0,0,0));
  
      // assignment is allowed between shared_ptr instances
      fVector = vector;
      // the object previously pointed by fVector (if any) is deleted
      // vector and fVector now share the ownership of the same object

      // here vector goes out-of-scope
      // however fVector is a class member so the object is not deleted!
      // it will be deleted automatically when this instance of the class
      // is deleted (and therefore fVector goes out-of-scope) :)
    }
\end{lstlisting}
\end{frame}

\begin{frame}[fragile]{Some word of caution on \texttt{shared\_ptr}}
\scriptsize
\begin{lstlisting}[style=base, gobble=4]
    void MyClass::MyFunction() {
      auto ptr = new TLorentzVector(0,0,0,0);
  
      std::shared_ptr<TLorentzVector> v1 (ptr);
      std::shared_ptr<TLorentzVector> v2 (ptr);
  
      // a double delete occurs here!
    }
\end{lstlisting}
\begin{center}
\small
What is the problem with the code above?
\end{center}
\end{frame}

\begin{frame}[fragile]{Some word of caution on \texttt{shared\_ptr}}
\scriptsize
\begin{lstlisting}[style=base, gobble=4]
    void MyFunction() {
      auto ptr = new TLorentzVector(0,0,0,0);
  
      std::shared_ptr<TLorentzVector> v1 (ptr);
      std::shared_ptr<TLorentzVector> v2 (ptr);
  
      // a double delete occurs here!
    }
\end{lstlisting}
\small
\begin{itemize}
\item \texttt{v1} does not know about \texttt{v2} and viceversa!
\item Two control blocks have been created for the same pointed objects
\end{itemize}
\end{frame}

\begin{frame}[fragile]{Some word of caution on \texttt{shared\_ptr}}
\scriptsize
\begin{lstlisting}[style=base, gobble=4]
    void MyFunction() {
      std::shared_ptr<TLorentzVector> v1 (new TLorentzVector(0,0,0,0));
      std::shared_ptr<TLorentzVector> v2 (v1);
  
      // this is fine!
    }
\end{lstlisting}
\small
\begin{itemize}
\item Solution: use raw pointers only when absolutely needed (if at all)
\end{itemize}
\end{frame}

\begin{frame}[fragile]{Usage Notes for ALICE Software}
\begin{itemize}
\item \textcolor{ForestGreen}{Can be used in the implementation files of \textbf{AliPhysics} (*.cxx files)}
\item \textcolor{orange}{In the header files (*.h) need to hide them from CINT} (therefore cannot be used as non-transient class members) \\
{\scriptsize
\begin{lstlisting}[style=base, gobble=4]
    #if !(defined(__CINT__) || defined(__MAKECINT__))
    // your C++11 code goes here
    #endif
\end{lstlisting}
}
\item \textcolor{red}{Cannot be used anywhere in \textbf{AliRoot}}
\end{itemize}
\end{frame}

\section{Conclusions}

\begin{frame}[fragile]{Final remarks}
\begin{itemize}
\item When the extra-flexibility of a pointer is not needed, do not use it
\item Alternative to pointers: arguments by reference (not covered here)
\item Avoid raw pointers whenever possible!
\item Smart pointers (\texttt{unique\_ptr} and \texttt{shared\_ptr}) should cover most use cases and provide
a much more robust and safe memory management
\end{itemize}
References \\
Effective modern C++, Scott Meyers (O'Reilly 2015) \\
\href{http://en.cppreference.com/}{http://en.cppreference.com/}
\end{frame}

\end{document}
