% $Header: /Users/joseph/Documents/LaTeX/beamer/solutions/conference-talks/conference-ornate-20min.en.tex,v 90e850259b8b 2007/01/28 20:48:30 tantau $

\documentclass[xcolor={usenames,dvipsnames}]{beamer}

% This file is a solution template for:

% - Talk at a conference/colloquium.
% - Talk length is about 20min.
% - Style is ornate.



% Copyright 2004 by Till Tantau <tantau@users.sourceforge.net>.
%
% In principle, this file can be redistributed and/or modified under
% the terms of the GNU Public License, version 2.
%
% However, this file is supposed to be a template to be modified
% for your own needs. For this reason, if you use this file as a
% template and not specifically distribute it as part of a another
% package/program, I grant the extra permission to freely copy and
% modify this file as you see fit and even to delete this copyright
% notice. 


\mode<presentation>
{
  \usetheme{AnnArbor}
  % or ...

  \setbeamercovered{transparent}
  % or whatever (possibly just delete it)
 }

\usepackage[percent]{overpic}
\usepackage[english]{babel}
\usepackage{multirow}
% or whatever

\usepackage[latin1]{inputenc}
% or whatever

\usepackage{times}
\usepackage[T1]{fontenc}
% Or whatever. Note that the encoding and the font should match. If T1
% does not look nice, try deleting the line with the fontenc.
%particles
\newcommand{\jpsi}{\rm J/$\psi$}
\newcommand{\psip}{$\psi^\prime$}
\newcommand{\jpsiDY}{\rm J/$\psi$\,/\,DY}
\newcommand{\chic}{$\chi_{\rm c}$}
\newcommand{\pip}{$\pi^{+}$}
\newcommand{\pim}{$\pi^{-}$}
\newcommand{\pizero}{$\pi^{0}$}
\newcommand{\kap}{K$^{+}$}
\newcommand{\kam}{K$^{-}$}
\newcommand{\pbar}{$\rm\overline{p}$}
\newcommand{\ccbar}{\ensuremath{\mathrm{c\overline{c}}}}
\newcommand{\bbbar}{\ensuremath{\mathrm{b\overline{b}}}}
\newcommand{\Dzero}{\ensuremath{\mathrm{D^{0}}}}
\newcommand{\Dzerobar}{\ensuremath{\mathrm{\overline{D}^{0}}}}
\newcommand{\Dpm}{\ensuremath{\mathrm{D^{\pm}}}}
\newcommand{\Ds}{\ensuremath{\mathrm{D_{s}^{\pm}}}}
\newcommand{\Dstar}{\ensuremath{\mathrm{D^{*\pm}}}}

%collision systems
\newcommand{\pp}{pp}
\newcommand{\pPb}{p--Pb}
\newcommand{\PbPb}{Pb--Pb}

%detectors
\newcommand{\ezdc}{$E_{\rm ZDC}$}

%units
\newcommand{\GeVc}{GeV/$c$}
\newcommand{\GeVcsq}{GeV/$c^2$}

%others
\newcommand{\degree}{$^{\rm o}$}
\newcommand{\s}{\ensuremath{\sqrt{s}}}
\newcommand{\snn}{\ensuremath{\sqrt{s_{\rm NN}}}}
\newcommand{\y}{\ensuremath{y}}
\newcommand{\pt}{\ensuremath{p_{\rm T}}}
\newcommand{\dedx}{d$E$/d$x$}
\newcommand{\dndy}{d$N$/d$y$}
\newcommand{\dndydpt}{${\rm d}^2N/({\rm d}y {\rm d}p_{\rm t})$}
\newcommand{\zpar}{\ensuremath{z_{||}}}
\newcommand{\zpargen}{\ensuremath{z_{||}^{\mathrm{part}}}}
\newcommand{\zpardet}{\ensuremath{z_{||}^{\mathrm{det}}}}
\newcommand{\ptchjet}{\ensuremath{p_{\mathrm{T,ch\, jet}}}}
\newcommand{\ptjet}{\ensuremath{p_{\mathrm{T,jet}}}}
\newcommand{\ptchjetgen}{\ensuremath{p_{\mathrm{T,ch\,jet}}^{\mathrm{part}}}}
\newcommand{\ptchjetdet}{\ensuremath{p_{\mathrm{T,ch\,jet}}^{\mathrm{det}}}}
\newcommand{\ptd}{\ensuremath{p_{\mathrm{T,D}}}}
\newcommand{\ptdgen}{\ensuremath{p_{\mathrm{T,D}}^{\mathrm{part}}}}
\newcommand{\ptddet}{\ensuremath{p_{\mathrm{T,D}}^{\mathrm{det}}}}
\newcommand{\antikt}{anti-\ensuremath{k_{\mathrm{T}}}}
\newcommand{\Antikt}{Anti-\ensuremath{k_{\mathrm{T}}}}
\newcommand{\kt}{\ensuremath{k_{\mathrm{T}}}}
\newcommand{\pthard}{\ensuremath{p_{\mathrm{T,hard}}}}

\AtBeginSection[]{
  \begin{frame}
  \vfill
  \centering
  \begin{beamercolorbox}[sep=8pt,center,shadow=true,rounded=true]{title}
    \usebeamerfont{title}\insertsectionhead\par%
  \end{beamercolorbox}
  \vfill
  \end{frame}
}

\title[D-Tagged Jets in \pp] % (optional, use only with long paper titles)
{D-Tagged Jets in \pp\ Collisions}

\author[Salvatore Aiola]% (optional, use only with lots of authors)
{Salvatore Aiola}
% - Give the names in the same order as the appear in the paper.
% - Use the \inst{?} command only if the authors have different
%   affiliation.

\institute[Yale University] % (optional, but mostly needed)
{Yale University}

\date[PAG-HFCJ - Nov. 8th, 2017] % (optional, should be abbreviation of conference name)
{PAG-HFCJ \\
November 8th, 2017}
% - Either use conference name or its abbreviation.
% - Not really informative to the audience, more for people (including
%   yourself) who are reading the slides online

\subject{High-Energy Physics}
% This is only inserted into the PDF information catalog. Can be left
% out. 



% If you have a file called "university-logo-filename.xxx", where xxx
% is a graphic format that can be processed by latex or pdflatex,
% resp., then you can add a logo as follows:

% \pgfdeclareimage[height=0.5cm]{university-logo}{university-logo-filename}
% \logo{\pgfuseimage{university-logo}}


% If you wish to uncover everything in a step-wise fashion, uncomment
% the following command: 

%\beamerdefaultoverlayspecification{<+->}


\begin{document}

\begin{frame}
  \titlepage
\end{frame}

%\begin{frame}{Outline}
 %   \tableofcontents
 %\end{frame}


% Structuring a talk is a difficult task and the following structure
% may not be suitable. Here are some rules that apply for this
% solution: 

% - Exactly two or three sections (other than the summary).
% - At *most* three subsections per section.
% - Talk about 30s to 2min per frame. So there should be between about
%   15 and 30 frames, all told.

% - A conference audience is likely to know very little of what you
%   are going to talk about. So *simplify*!
% - In a 20min talk, getting the main ideas across is hard
%   enough. Leave out details, even if it means being less precise than
%   you think necessary.
% - If you omit details that are vital to the proof/implementation,
%   just say so once. Everybody will be happy with that.

%\begin{overpic}[width=.85\textwidth, trim=0 0 0 0, clip]{img/ReflectionTemplates_DPt_NoJet_DoubleGaus_1010}
%\put(0,61){{\tiny No jet requirement}}
%\put(60,61){{\tiny \textcolor{ForestGreen}{\textbf{Used for QM17 preliminary}}}}
%\end{overpic}

%\begin{columns}
%\column{0.5\textwidth}
%\column{0.5\textwidth}
%\end{columns}

\section{Strategy}

\begin{frame}{Sig/Bkg Efficiency vs. Cut Variations: $d_{0,\rm K}d_{0,\pi}$}
\begin{columns}
\column{0.7\textwidth}
\tiny $5 < \ptchjet < 15$~\GeVc\ and  $4 < \ptd < 6$~\GeVc\\
\begin{overpic}[width=1.1\textwidth, trim=0 0 0 0, clip]{img/topo_cuts/d0d0_CutEfficiency_copy_JetPt5_15_DPt4_6}
\end{overpic}
\column{0.3\textwidth}
\footnotesize
\textbf{Signal} and \textbf{\textcolor{NavyBlue}{Non-Prompt}} from LHC15i2\{b,c,d,e\} (charm-enhanced with \pt-hard bins)\\
\vspace{10pt}
\textbf{\textcolor{BrickRed}{Background}} from LHC14j4\{b,c,d,e\} (minimum-bias) \\
\vspace{10pt}
\textcolor{magenta}{Magenta line: cut for maximum significance} \\
\textcolor{ForestGreen}{Green line: cut for maximum S/B ratio} 
\end{columns}
\end{frame}

\begin{frame}{Significance and S/B vs. Cut Variations: $d_{0,\rm K}d_{0,\pi}$}
\tiny $5 < \ptchjet < 15$~\GeVc\ and  $4 < \ptd < 6$~\GeVc\\
\begin{columns}
\column{0.5\textwidth}
\begin{overpic}[width=\textwidth, trim=0 0 0 0, clip]{img/topo_cuts/d0d0_CutFraction_MB_Bkg_JetPt5_15_DPt4_6}
\end{overpic}
\column{0.5\textwidth}
\begin{overpic}[width=\textwidth, trim=0 0 0 0, clip]{img/topo_cuts/d0d0_CutSignificance_MB_Bkg_JetPt5_15_DPt4_6}
\end{overpic}
\end{columns}
\footnotesize
\textbf{Signal} and \textbf{\textcolor{NavyBlue}{Non-Prompt}} from LHC15i2\{b,c,d,e\} (charm-enhanced with \pt-hard bins)\\
\vspace{5pt}
\textbf{\textcolor{BrickRed}{Background}} from LHC14j4\{b,c,d,e\} (minimum-bias) \\
\vspace{5pt}
\textcolor{magenta}{Magenta line: cut for maximum significance} \\
\textcolor{ForestGreen}{Green line: cut for maximum S/B ratio} \\
\vspace{5pt}
Issue with absolute normalisation of signal (background should be ok)
\end{frame}

\begin{frame}{Significance and S/B vs. Cut Variations: $\cos(\theta^{*})$}
\tiny $15 < \ptchjet < 30$~\GeVc\ and  $6 < \ptd < 12$~\GeVc\\
\begin{columns}
\column{0.5\textwidth}
\begin{overpic}[width=\textwidth, trim=0 0 0 0, clip]{img/topo_cuts/CosThetaStar_CutFraction_MB_Bkg_JetPt15_30_DPt6_12}
\end{overpic}
\column{0.5\textwidth}
\begin{overpic}[width=\textwidth, trim=0 0 0 0, clip]{img/topo_cuts/CosThetaStar_CutSignificance_MB_Bkg_JetPt15_30_DPt6_12}
\end{overpic}
\end{columns}
\footnotesize
\textbf{Signal} and \textbf{\textcolor{NavyBlue}{Non-Prompt}} from LHC15i2\{b,c,d,e\} (charm-enhanced with \pt-hard bins)\\
\vspace{5pt}
\textbf{\textcolor{BrickRed}{Background}} from LHC14j4\{b,c,d,e\} (minimum-bias) \\
\vspace{5pt}
\textcolor{magenta}{Magenta line: cut for maximum significance} \\
\textcolor{ForestGreen}{Green line: cut for maximum S/B ratio} \\
\vspace{5pt}
Issue with absolute normalisation of signal (background should be ok)
\end{frame}

\begin{frame}{(Updated) Strategy}
\begin{itemize}
\item Use LHC15i2\{b,c,d,e\} 
\item Compromise between significance (fit stability) and S/B ratio (statistical precision in side-band subtraction) in each \ptchjet\ and \ptd\ bin
\item Wide \ptd\ bins: try also to keep a balance so that the full \ptd\ range in each bin is represented
\item Try to avoid enhancing too much the non-prompt fraction $\rightarrow$ topomatic cut kept at 2 sigma for all bins
\end{itemize}
\end{frame}

\section{Results}

\begin{frame}{New Cuts}
\footnotesize
\begin{table}
\begin{tabular}{llrrrr}
\ptchjet\ (\GeVc) & \ptd\ (\GeVc) & DCA ($\mu$m) & $\cos(\theta^{*})$ & $d_{\pi}d_{\rm K}$ ($\mu$m$^2$) & $\cos(\theta_{\rm p})$ \\
\hline \hline
\multirow{4}{*}{$5 < \ptchjet < 15$}		& $2 < \ptd < 4$ & 250 & 0.70 & -15000 & 0.84 \\
								& $4 < \ptd < 6$ & 250 & 0.70 & -10000 & 0.94 \\ 
								& $6 < \ptd < 9$ & 200 & 0.65 & -8000 & 0.97 \\ 
								& $9 < \ptd < 12$ & 150 & 0.60 & -5000 & 0.98 \\ 
\hline
\multirow{2}{*}{$15 < \ptchjet < 30$}	& $6 < \ptd < 12$ & 150 & 0.50 & -8000 & 0.90 \\
								& $12 < \ptd < 30$ & 150 & 0.60 & -2000 & 0.98 \\
\hline
\end{tabular}
\end{table}
\end{frame}

\begin{frame}{Inv.Mass Fits: $5 < \ptchjet < 15$~\GeVc\ / 1}
\begin{columns}
\column{0.5\textwidth}
\begin{center}
\begin{overpic}[width=\textwidth, trim=0 0 0 0, clip]{img/inv_mass/AnyINT_D0_D0toKpiCuts_loosest_pid_Charged_R040_DPtBins_JetPt_5_15_DPt1_CutMaxSig_bis_SideBand_D0_D0toKpiCuts_loosest_pid_Charged_R040_DPtSpectrum_JetPt_5_15_DPt1_SideBand_CutMaxSig_bis}
\end{overpic}
\end{center}
\column{0.5\textwidth}
\begin{center}
\begin{overpic}[width=\textwidth, trim=0 0 0 0, clip]{img/inv_mass/AnyINT_D0_D0toKpiCuts_loosest_pid_Charged_R040_DPtBins_JetPt_5_15_DPt2_CutMaxSig_bis_SideBand_D0_D0toKpiCuts_loosest_pid_Charged_R040_DPtSpectrum_JetPt_5_15_DPt2_SideBand_CutMaxSig_bis}
\end{overpic}
\end{center}
\end{columns}
\end{frame}

\begin{frame}{Inv.Mass Fits: $5 < \ptchjet < 15$~\GeVc\ / 2}
\begin{columns}
\column{0.5\textwidth}
\begin{center}
\begin{overpic}[width=\textwidth, trim=0 0 0 0, clip]{img/inv_mass/AnyINT_D0_D0toKpiCuts_loosest_pid_Charged_R040_DPtBins_JetPt_5_15_DPt3_CutMaxSig_bis_SideBand_D0_D0toKpiCuts_loosest_pid_Charged_R040_DPtSpectrum_JetPt_5_15_DPt3_SideBand_CutMaxSig_bis}
\end{overpic}
\end{center}
\column{0.5\textwidth}
\begin{center}
\begin{overpic}[width=\textwidth, trim=0 0 0 0, clip]{img/inv_mass/AnyINT_D0_D0toKpiCuts_loosest_pid_Charged_R040_DPtBins_JetPt_5_15_DPt4_CutMaxSig_bis_SideBand_D0_D0toKpiCuts_loosest_pid_Charged_R040_DPtSpectrum_JetPt_5_15_DPt4_SideBand_CutMaxSig_bis}
\end{overpic}
\end{center}
\end{columns}
\end{frame}

\begin{frame}{Inv.Mass Fits: $15 < \ptchjet < 30$~\GeVc}
\begin{columns}
\column{0.5\textwidth}
\begin{center}
\begin{overpic}[width=\textwidth, trim=0 0 0 0, clip]{img/inv_mass/AnyINT_D0_D0toKpiCuts_loosest_pid_Charged_R040_DPtBins_JetPt_15_30_DPt1_CutMaxSig_bis_SideBand_D0_D0toKpiCuts_loosest_pid_Charged_R040_DPtSpectrum_JetPt_15_30_DPt1_SideBand_CutMaxSig_bis}
\end{overpic}
\end{center}
\column{0.5\textwidth}
\begin{center}
\begin{overpic}[width=\textwidth, trim=0 0 0 0, clip]{img/inv_mass/AnyINT_D0_D0toKpiCuts_loosest_pid_Charged_R040_DPtBins_JetPt_15_30_DPt2_CutMaxSig_bis_SideBand_D0_D0toKpiCuts_loosest_pid_Charged_R040_DPtSpectrum_JetPt_15_30_DPt2_SideBand_CutMaxSig_bis}
\end{overpic}
\end{center}
\end{columns}
\end{frame}

\section{Conclusions}

\begin{frame}{Conclusions}
\begin{itemize}
\item Statistics are sufficient to measure fragmentation function in two jet \pt\ bins: [5, 15] and [15, 30]
\item Wide \ptd\ bins for the \zpar\ spectrum: [2, 4, 6, 9, 15] and [6, 12, 30]
\item Can use a similar approach for the jet \pt\ spectrum as well with \ptd\ bins [2, 4, 6, 9, 15, 30] (i.e. update the current preliminary)
\item New cuts are ready to be used
\item Next: calculate efficiency vs. \ptd\ and \ptchjet\ with the new cuts
\item Then: unfolding
\end{itemize}
\end{frame}

\section{Extra Slides}

\subsection{Efficiency vs. Topological Cuts}

\begin{frame}{Sig/Bkg Efficiency vs. Cut Variations: $\cos(\theta_{\rm p})$}
\textcolor{red}{From HFCJ November 2nd}
\begin{columns}
\column{0.7\textwidth}
\begin{overpic}[width=1.1\textwidth, trim=0 0 0 0, clip]{img/topo_cuts/CosPointing_CutEfficiency_copy_JetPt5_15_DPt3_6}
\put(12,10){\tiny $5 < \ptchjet < 15$~\GeVc\ and  $3 < \ptd < 6$~\GeVc}
\end{overpic}
\column{0.3\textwidth}
\footnotesize
\textbf{Signal} and \textbf{\textcolor{NavyBlue}{Non-Prompt}} from LHC15i2\{b,c,d,e\} (charm-enhanced with \pt-hard bins)\\
\vspace{10pt}
\textbf{\textcolor{BrickRed}{Background}} from LHC14j4\{b,c,d,e\} (minimum-bias) \\
\vspace{10pt}
\textbf{For each bin of \ptchjet\ and \ptd}
\end{columns}
\end{frame}

\begin{frame}{Sig/Bkg Efficiency vs. Cut Variations: $\cos(\theta^{*})$}
\textcolor{red}{From HFCJ November 2nd}
\begin{columns}
\column{0.7\textwidth}
\begin{overpic}[width=1.1\textwidth, trim=0 0 0 0, clip]{img/topo_cuts/CosThetaStar_CutEfficiency_copy_JetPt15_30_DPt6_9}
\put(12,10){\tiny $5 < \ptchjet < 15$~\GeVc\ and  $6 < \ptd < 9$~\GeVc}
\end{overpic}
\column{0.3\textwidth}
\footnotesize
\textbf{Signal} and \textbf{\textcolor{NavyBlue}{Non-Prompt}} from LHC15i2\{b,c,d,e\} (charm-enhanced with \pt-hard bins)\\
\vspace{10pt}
\textbf{\textcolor{BrickRed}{Background}} from LHC14j4\{b,c,d,e\} (minimum-bias) \\
\vspace{10pt}
\textbf{For each bin of \ptchjet\ and \ptd}
\end{columns}
\end{frame}

\end{document}
