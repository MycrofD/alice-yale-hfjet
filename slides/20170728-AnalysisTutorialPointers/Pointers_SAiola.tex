% $Header: /Users/joseph/Documents/LaTeX/beamer/solutions/conference-talks/conference-ornate-20min.en.tex,v 90e850259b8b 2007/01/28 20:48:30 tantau $

\documentclass[xcolor={usenames,dvipsnames}]{beamer}

% This file is a solution template for:

% - Talk at a conference/colloquium.
% - Talk length is about 20min.
% - Style is ornate.



% Copyright 2004 by Till Tantau <tantau@users.sourceforge.net>.
%
% In principle, this file can be redistributed and/or modified under
% the terms of the GNU Public License, version 2.
%
% However, this file is supposed to be a template to be modified
% for your own needs. For this reason, if you use this file as a
% template and not specifically distribute it as part of a another
% package/program, I grant the extra permission to freely copy and
% modify this file as you see fit and even to delete this copyright
% notice. 


\mode<presentation>
{
  \usetheme{AnnArbor}
  % or ...

  \setbeamercovered{transparent}
  % or whatever (possibly just delete it)
 }

\usepackage[percent]{overpic}
\usepackage[english]{babel}
\usepackage{setspace}
% or whatever

\usepackage[latin1]{inputenc}
% or whatever
\usepackage{listings}
\usepackage[at]{easylist}
\usepackage{times}
\usepackage[T1]{fontenc}
\usepackage{xcolor}
% Or whatever. Note that the encoding and the font should match. If T1
% does not look nice, try deleting the line with the fontenc.

\definecolor{darkestblue}{RGB}{1,8,100}
\definecolor{darkerblue}{RGB}{3,17,150}
\definecolor{darkblue}{RGB}{7,26,200}
\definecolor{lightred}{RGB}{202,103,104}
\definecolor{lightgreen}{RGB}{106,202,107}

\AtBeginSection[]{
  \begin{frame}
  \vfill
  \centering
  \begin{beamercolorbox}[sep=8pt,center,shadow=true,rounded=true]{title}
    \usebeamerfont{title}\insertsectionhead\par%
  \end{beamercolorbox}
  \vfill
  \end{frame}
}

\title[Some pointers on pointers] % (optional, use only with long paper titles)
{Some pointers on pointers}

\author[Salvatore Aiola]% (optional, use only with lots of authors)
{Salvatore Aiola}
% - Give the names in the same order as the appear in the paper.
% - Use the \inst{?} command only if the authors have different
%   affiliation.

\institute[Yale University] % (optional, but mostly needed)
{Yale University}

\date[July 28th, 2017] % (optional, should be abbreviation of conference name)
{Analysis Tutorial \\
ALICE Week \\
CERN, July 28th, 2017}
% - Either use conference name or its abbreviation.
% - Not really informative to the audience, more for people (including
%   yourself) who are reading the slides online

\subject{High-Energy Physics}
% This is only inserted into the PDF information catalog. Can be left
% out. 



% If you have a file called "university-logo-filename.xxx", where xxx
% is a graphic format that can be processed by latex or pdflatex,
% resp., then you can add a logo as follows:

% \pgfdeclareimage[height=0.5cm]{university-logo}{university-logo-filename}
% \logo{\pgfuseimage{university-logo}}


% If you wish to uncover everything in a step-wise fashion, uncomment
% the following command: 

%\beamerdefaultoverlayspecification{<+->}


\begin{document}

\begin{frame}
  \titlepage
\end{frame}

\begin{frame}{Outline}
   \tableofcontents
\end{frame}


% Structuring a talk is a difficult task and the following structure
% may not be suitable. Here are some rules that apply for this
% solution: 

% - Exactly two or three sections (other than the summary).
% - At *most* three subsections per section.
% - Talk about 30s to 2min per frame. So there should be between about
%   15 and 30 frames, all told.

% - A conference audience is likely to know very little of what you
%   are going to talk about. So *simplify*!
% - In a 20min talk, getting the main ideas across is hard
%   enough. Leave out details, even if it means being less precise than
%   you think necessary.
% - If you omit details that are vital to the proof/implementation,
%   just say so once. Everybody will be happy with that.


%\begin{overpic}[width=\textwidth, trim=0 0 0 0, clip]{img/823_D0_Charged_R040_JetPtBins_DPt_30}
%\end{overpic}

%\begin{columns}
%\column{0.5\textwidth}
%\column{0.5\textwidth}
%\end{columns}

\section{Introduction}

\begin{frame}[fragile]{What is a Pointer?}
\begin{columns}
\column{0.5\textwidth}
\begin{overpic}[width=\textwidth, trim=0 0 0 0, clip]{img/Pointers}
\end{overpic}
\column{0.5\textwidth}
A pointer is an object whose value ``points to'' another value stored somewhere else in memory
\begin{itemize}
\item Very powerful tool
\item Great power = great responsibility!
\item Extensive use of pointers in ROOT/AliRoot/AliPhysics
\end{itemize}
\end{columns}
\end{frame}

\begin{frame}[fragile]{Using a Pointer}
\scriptsize
\begin{lstlisting}[language=C++]
/* Defining a pointer */
int* a; // declares a pointer that can point to an integer value
//DANGER: the pointer points to a random memory portion!

int* b = nullptr; // OK, pointer is initialized to a null memory address

int* c = new int; // allocate memory for an integer value in the heap 
//and assign its memory address to this pointer

int** d = &a; // this pointer points to a pointer to an integer value

MyObject* e = new MyObject(); // allocate memory for MyObject
// and assign its memory address to this pointer

/* Using a pointer */
int f = *c; // dereferencing a pointer and assigning the pointed
// value to another integer variable

e->DoSomething(); // dereferencing a pointer and calling
// the method DoSomething() of the instance of MyObject
// pointed by e
\end{lstlisting}
\end{frame}

\section{Why a raw pointer is hard to love}

\begin{frame}[fragile]{Memory leak}
\scriptsize
\begin{lstlisting}[language=C++]
void UserExec()
{
  for (int i = 0; i < InputEvent()->GetNumberOfTracks(); i++) {
    AliVTrack* track = InputEvent()->GetTrack(i);
    if (!track) continue;
    TLorentzVector* v = new TLorentzVector(track->Px(), 
      track->Py(), track->Pz(), track->M());
    
    // my analysis here
    std::cout << v->Pt() << std::endl;
  }
  
  delete v;
}
\end{lstlisting}
\small 
\begin{center}
What is the problem with this code?
\end{center}
\end{frame}

\begin{frame}[fragile]{Array or single value?}
\begin{itemize}
\item A pointer can point to a single value or to an array, however its declaration does not indicate it
\item Different syntax to destroy (= deallocate, free) the pointed object for arrays and single objects 
\end{itemize}
\scriptsize
\begin{lstlisting}[language=C++]
AliVTrack* FilterTracks();

void UserExec()
{
  TLorentzVector *vect = new TLorentzVector(0,0,0,0);
  double *trackPts = new double[100];
  AliVTrack *returnValue = FilterTracks();

  // here use the pointers

  delete vect;
  delete[] trackPts;
  delete returnValue; // or should I use delete[] ??
}
\end{lstlisting}
\end{frame}

\begin{frame}[fragile]{Double deletes}
\begin{itemize}
\item Each memory allocation should match a corresponding deallocation
\item Difficult to keep track of all memory allocations in a large project
\item Ownership of the pointed memory is ambiguous: multiple deletes of the same object may occur
\end{itemize}
\scriptsize
\begin{lstlisting}[language=C++]
AliVTrack* FilterTracks();
void AnalyzeTracks(AliVTrack* tracks);

void UserExec()
{
  AliVTrack* tracks = FilterTracks();

  AnalyzeTracks(tracks);

  delete[] tracks; // should I actually delete it?? 
  //or was it already deleted by AnalyzeTracks?
}
\end{lstlisting}
\end{frame}

\section{Smart Pointers}

\begin{frame}[fragile]{Smart Pointers}
\begin{itemize}
\item Clear (shared or exclusive) ownership of the pointed object
\item Automatic garbage collection: memory is deallocated when the last pointer goes out of scope
\item Available since C++11
\item Can be used in the implementation files of AliPhysics (*.cxx files)
\item In the header files (*.h) need to hide them from CINT (therefore cannot be used as non-transient class members) \\
{\scriptsize
\begin{lstlisting}[language=C++]
#if !(defined(__CINT__) || defined(__MAKECINT__))
// your C++11 code goes here
#endif
\end{lstlisting}
}
\item Cannot be used anywhere in AliRoot
\end{itemize}
\end{frame}

\begin{frame}[fragile]{Exclusive-Ownership Pointers: unique\_ptr}
\begin{itemize}
\item Automatic garbage collection with no additional CPU or memory overhead (i.e. it uses the same resources as a raw pointer)
\item unique\_ptr owns the object it points
\item Memory is automatically released when unique\_ptr goes out of scope or when its reset method is called
\item Only one unique\_ptr can point to the same memory address
\end{itemize}
\end{frame}

\begin{frame}[fragile]{unique\_ptr example / 1}
\scriptsize
\begin{lstlisting}[language=C++]
void MyFunction() {
  std::unique_ptr<TLorentzVector> vector(new TLorentzVector(0,0,0,0));
  std::unique_ptr<TLorentzVector> vector2(new TLorentzVector(0,0,0,0));
  
  // use vector and vector2
  
  // dereferencing unique_ptr works exactly as a raw pointer
  std::cout << vector->Pt() << std::endl;
  
  // the line below does not compile!
  // vector = vector2;
  // cannot assign the same address to two unique_ptr instances
  
  vector.swap(vector2); // however I can swap the memory addresses
  
  // this also releases the memory previously pointed by vector2
  vector2.reset(new TLorentzVector(0,0,0,0)); 
  
  // objects pointed by vector and vector2 are deleted here
}
\end{lstlisting}
\end{frame}

\begin{frame}[fragile]{unique\_ptr example / 2}
\scriptsize
\begin{lstlisting}[language=C++]
void UserExec()
{
  for (int i = 0; i < InputEvent()->GetNumberOfTracks(); i++) {
    AliVTrack* track = InputEvent()->GetTrack(i);
    if (!track) continue;
    std::unique_ptr<TLorentzVector> v(new TLorentzVector(track->Px(), 
      track->Py(), track->Pz(), track->M()));
    
    // my analysis here
    std::cout << v->Pt() << std::endl;
  }
}
\end{lstlisting}
\small 
\begin{center}
No memory leak here! :)
\end{center}
\end{frame}

\begin{frame}[fragile]{Shared-Ownership Pointers: shared\_ptr}
\begin{itemize}
\item Automatic garbage collection with some CPU and memory overhead
\item The pointed object is collectively owned by one or more shared\_ptr instances
\item Memory is automatically released the last shared\_ptr goes out of scope or when it is re-assigned
\end{itemize}
\begin{center}
\begin{overpic}[width=.5\textwidth, trim=0 0 0 0, clip]{img/Sharedptr}
\end{overpic}
\end{center}
\end{frame}

\begin{frame}[fragile]{shared\_ptr example / 1}
\scriptsize
\begin{lstlisting}[language=C++]
void MyFunction() {
  std::shared_ptr<TLorentzVector> vector(new TLorentzVector(0,0,0,0));
  std::shared_ptr<TLorentzVector> vector2(new TLorentzVector(0,0,0,0));
  
  // dereferencing shared_ptr works exactly as a raw pointer
  std::cout << vector->Pt() << std::endl;
  
  // assignment is allowed between shared_ptr instances
  vector = vector2; 
  // the object previously pointed by vector is deleted!
  // vector and vector2 now share the ownership of the same object
  
  // object pointed by both vector and vector2 is deleted here
}
\end{lstlisting}
\end{frame}

\begin{frame}[fragile]{shared\_ptr example / 2}
\scriptsize
\begin{lstlisting}[language=C++]
class MyClass {
  public:
    MyClass();
  private:
    void MyFunction();
    std::shared_ptr<TLorentzVector> fVector;
};

void MyClass::MyFunction() {
  std::shared_ptr<TLorentzVector> vector(new TLorentzVector(0,0,0,0));
  
  // assignment is allowed between shared_ptr instances
  fVector = vector;
  // the object previously pointed by fVector (if any) is deleted
  // vector and fVector now share the ownership of the same object

  // here vector goes out-of-scope
  // however fVector is a class member so the object is not deleted!
  // it will be deleted automatically when this instance of the class
  // is deleted (and therefore fVector goes out-of-scope) :)
}
\end{lstlisting}
\end{frame}

\begin{frame}[fragile]{Some word of caution on shared\_ptr}
\scriptsize
\begin{lstlisting}[language=C++]
void MyClass::MyFunction() {
  auto ptr = new TLorentzVector(0,0,0,0);
  
  std::shared_ptr<TLorentzVector> v1 (ptr);
  std::shared_ptr<TLorentzVector> v2 (ptr);
  
  // a double delete occurs here!
}
\end{lstlisting}
\begin{center}
\small
What is the problem with the code above?
\end{center}
\end{frame}

\begin{frame}[fragile]{Some word of caution on shared\_ptr}
\scriptsize
\begin{lstlisting}[language=C++]
void MyFunction() {
  auto ptr = new TLorentzVector(0,0,0,0);
  
  std::shared_ptr<TLorentzVector> v1 (ptr);
  std::shared_ptr<TLorentzVector> v2 (ptr);
  
  // a double delete occurs here!
}
\end{lstlisting}
\small
\begin{itemize}
\item v1 does not know about v2 and viceversa!
\item Two control blocks have been created for the same pointed objects
\end{itemize}
\end{frame}

\begin{frame}[fragile]{Some word of caution on shared\_ptr}
\scriptsize
\begin{lstlisting}[language=C++]
void MyFunction() {
  std::shared_ptr<TLorentzVector> v1 (new TLorentzVector(0,0,0,0));
  std::shared_ptr<TLorentzVector> v2 (v1);
  
  // this is fine!
}
\end{lstlisting}
\small
\begin{itemize}
\item Solution: use raw pointers only when absolutely needed (if at all)
\end{itemize}
\end{frame}

\section{Conclusions}

\begin{frame}[fragile]{Final remarks}
\begin{itemize}
\item When the extra-flexibility of a pointer is not needed, do not use it
\item Alternative to pointers: arguments by reference (not covered here)
\item Avoid raw pointers whenever possible!
\item Smart pointers (unique\_ptr and shared\_ptr) should cover most use cases and provide
a much more robust and safe memory management
\end{itemize}
References \\
Effective modern C++, Scott Meyers (O'Reilly 2015) \\
\href{http://en.cppreference.com/}{http://en.cppreference.com/}
\end{frame}

\end{document}
