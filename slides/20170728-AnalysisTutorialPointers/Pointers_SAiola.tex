% $Header: /Users/joseph/Documents/LaTeX/beamer/solutions/conference-talks/conference-ornate-20min.en.tex,v 90e850259b8b 2007/01/28 20:48:30 tantau $

\documentclass[xcolor={usenames,dvipsnames}]{beamer}

% This file is a solution template for:

% - Talk at a conference/colloquium.
% - Talk length is about 20min.
% - Style is ornate.



% Copyright 2004 by Till Tantau <tantau@users.sourceforge.net>.
%
% In principle, this file can be redistributed and/or modified under
% the terms of the GNU Public License, version 2.
%
% However, this file is supposed to be a template to be modified
% for your own needs. For this reason, if you use this file as a
% template and not specifically distribute it as part of a another
% package/program, I grant the extra permission to freely copy and
% modify this file as you see fit and even to delete this copyright
% notice. 


\mode<presentation>
{
  \usetheme{AnnArbor}
  % or ...

  \setbeamercovered{transparent}
  % or whatever (possibly just delete it)
 }

\usepackage[percent]{overpic}
\usepackage[english]{babel}
\usepackage{setspace}
% or whatever

\usepackage[latin1]{inputenc}
% or whatever
\usepackage{listings}
\usepackage[at]{easylist}
\usepackage{times}
\usepackage[T1]{fontenc}
\usepackage{xcolor}
% Or whatever. Note that the encoding and the font should match. If T1
% does not look nice, try deleting the line with the fontenc.

\definecolor{darkestblue}{RGB}{1,8,100}
\definecolor{darkerblue}{RGB}{3,17,150}
\definecolor{darkblue}{RGB}{7,26,200}
\definecolor{lightred}{RGB}{202,103,104}
\definecolor{lightgreen}{RGB}{106,202,107}

\AtBeginSection[]{
  \begin{frame}
  \vfill
  \centering
  \begin{beamercolorbox}[sep=8pt,center,shadow=true,rounded=true]{title}
    \usebeamerfont{title}\insertsectionhead\par%
  \end{beamercolorbox}
  \vfill
  \end{frame}
}

\title[Some pointers on pointers] % (optional, use only with long paper titles)
{Some pointers on pointers}

\author[Salvatore Aiola]% (optional, use only with lots of authors)
{Salvatore Aiola}
% - Give the names in the same order as the appear in the paper.
% - Use the \inst{?} command only if the authors have different
%   affiliation.

\institute[Yale University] % (optional, but mostly needed)
{Yale University}

\date[July 28th, 2017] % (optional, should be abbreviation of conference name)
{Analysis Tutorial \\
ALICE Week \\
CERN, July 28th, 2017}
% - Either use conference name or its abbreviation.
% - Not really informative to the audience, more for people (including
%   yourself) who are reading the slides online

\subject{High-Energy Physics}
% This is only inserted into the PDF information catalog. Can be left
% out. 



% If you have a file called "university-logo-filename.xxx", where xxx
% is a graphic format that can be processed by latex or pdflatex,
% resp., then you can add a logo as follows:

% \pgfdeclareimage[height=0.5cm]{university-logo}{university-logo-filename}
% \logo{\pgfuseimage{university-logo}}


% If you wish to uncover everything in a step-wise fashion, uncomment
% the following command: 

%\beamerdefaultoverlayspecification{<+->}


\begin{document}

\begin{frame}
  \titlepage
\end{frame}

\begin{frame}{Outline}
   \tableofcontents
\end{frame}


% Structuring a talk is a difficult task and the following structure
% may not be suitable. Here are some rules that apply for this
% solution: 

% - Exactly two or three sections (other than the summary).
% - At *most* three subsections per section.
% - Talk about 30s to 2min per frame. So there should be between about
%   15 and 30 frames, all told.

% - A conference audience is likely to know very little of what you
%   are going to talk about. So *simplify*!
% - In a 20min talk, getting the main ideas across is hard
%   enough. Leave out details, even if it means being less precise than
%   you think necessary.
% - If you omit details that are vital to the proof/implementation,
%   just say so once. Everybody will be happy with that.


%\begin{overpic}[width=\textwidth, trim=0 0 0 0, clip]{img/823_D0_Charged_R040_JetPtBins_DPt_30}
%\end{overpic}

%\begin{columns}
%\column{0.5\textwidth}
%\column{0.5\textwidth}
%\end{columns}

\section{Introduction}

\subsection{Disclaimer}

\begin{frame}[fragile]{Disclaimer}
\begin{easylist}[itemize]
@ Some parts will be too boring for some people, some other parts will be too advanced for others
@@ On the other hand, most people might find some parts interesting
@ Not a comprehensive lecture on memory management in C++
@@ Not even close!
@@ Rather some suggestions to write efficient, maintainable and robust code
@ Not an IT expert! 
@@ Experience from writing analysis code in ALICE + some limited readings
\end{easylist}
\end{frame}


\subsection{Pointers and Memory Management}

\begin{frame}[fragile]{What is a Pointer?}
\begin{columns}
\column{0.5\textwidth}
\begin{overpic}[width=\textwidth, trim=0 0 0 0, clip]{img/Pointers}
\end{overpic}
\column{0.5\textwidth}
A pointer is an object whose value ``points to'' another value stored somewhere else in memory
\begin{itemize}
\item Very powerful tool
\item Great power = great responsibility!
\item Extensive use of pointers in ROOT/AliRoot/AliPhysics
\end{itemize}
\end{columns}
\end{frame}

\begin{frame}[fragile]{Using a Pointer}
\scriptsize
\begin{lstlisting}[language=C++]
/* Defining a pointer */
int* a; // declares a pointer that can point to an integer value
//DANGER: the pointer points to a random memory portion!

int* b = nullptr; // OK, pointer is initialized to a null memory address

int* c = new int; // allocate memory for an integer value in the heap 
//and assign its memory address to this pointer

int** d = &a; // this pointer points to a pointer to an integer value

MyObject* e = new MyObject(); // allocate memory for MyObject
// and assign its memory address to this pointer

/* Using a pointer */
int f = *c; // dereferencing a pointer and assigning the pointed
// value to another integer variable

e->DoSomething(); // dereferencing a pointer and calling
// the method DoSomething() of the instance of MyObject
// pointed by e
\end{lstlisting}
\end{frame}

\section{Why a raw pointer is hard to love}

\begin{frame}[fragile]{Array or single value?}
\begin{itemize}
\item A pointer can point to a single value or to an array, however its declaration does not indicate it
\item Different syntax to destroy (= deallocate, free) the pointed object for arrays and single objects 
\end{itemize}
\scriptsize
\begin{lstlisting}[language=C++]
void UserExec()
{
  MyTrack *track = new MyTrack(0,0,0,0);
  double *trackPts = new double[100];
  double *returnValue = AnalyzeTracks(trackPts);

  // here use the pointers

  delete track;
  delete[] trackPts;
  delete returnValue; // or should I use delete[] ??
}
\end{lstlisting}
\end{frame}

\begin{frame}[fragile]{Memory leaks and double deletes}
\begin{itemize}
\item Each memory allocation should match a corresponding deallocation
\item Difficult to keep track of all memory allocations in a large project
\item Ownership of the pointed memory is ambiguous: multiple deletes of the same object may occur
\end{itemize}
\scriptsize
\begin{lstlisting}[language=C++]
void UserExec()
{
  AliVTracks* tracks = FilterTracks();

  AnalyzeTracks(tracks);

  delete[] tracks; // should I actually delete it?? 
  //or was it already deleted by AnalyzeTracks?
}
\end{lstlisting}
\end{frame}

\section{Smart Pointers}

\section{Conclusions}

\begin{frame}{Final remarks}
\begin{itemize}
\item When the extra-flexibility of a pointer is not needed, do not use it
\item Alternative example: arguments  by reference (not covered here)
\item Avoid raw pointers whenever possible!
\item Smart pointers (unique\_ptr and shared\_ptr) should cover most use cases and provide
a much more robust and safe memory management
\end{itemize}
\end{frame}

\end{document}
