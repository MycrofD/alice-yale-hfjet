% $Header: /Users/joseph/Documents/LaTeX/beamer/solutions/conference-talks/conference-ornate-20min.en.tex,v 90e850259b8b 2007/01/28 20:48:30 tantau $

\documentclass[xcolor={usenames,dvipsnames}]{beamer}

% This file is a solution template for:

% - Talk at a conference/colloquium.
% - Talk length is about 20min.
% - Style is ornate.



% Copyright 2004 by Till Tantau <tantau@users.sourceforge.net>.
%
% In principle, this file can be redistributed and/or modified under
% the terms of the GNU Public License, version 2.
%
% However, this file is supposed to be a template to be modified
% for your own needs. For this reason, if you use this file as a
% template and not specifically distribute it as part of a another
% package/program, I grant the extra permission to freely copy and
% modify this file as you see fit and even to delete this copyright
% notice. 


\mode<presentation>
{
  \usetheme{AnnArbor}
  % or ...

  \setbeamercovered{transparent}
  % or whatever (possibly just delete it)
 }

\usepackage[percent]{overpic}
\usepackage[english]{babel}
\usepackage{bm}
% or whatever

\usepackage[latin1]{inputenc}
% or whatever

\usepackage{times}
\usepackage[T1]{fontenc}
% Or whatever. Note that the encoding and the font should match. If T1
% does not look nice, try deleting the line with the fontenc.
%particles
\newcommand{\jpsi}{\rm J/$\psi$}
\newcommand{\psip}{$\psi^\prime$}
\newcommand{\jpsiDY}{\rm J/$\psi$\,/\,DY}
\newcommand{\chic}{$\chi_{\rm c}$}
\newcommand{\pip}{$\pi^{+}$}
\newcommand{\pim}{$\pi^{-}$}
\newcommand{\pizero}{$\pi^{0}$}
\newcommand{\kap}{K$^{+}$}
\newcommand{\kam}{K$^{-}$}
\newcommand{\pbar}{$\rm\overline{p}$}
\newcommand{\ccbar}{\ensuremath{\mathrm{c\overline{c}}}}
\newcommand{\bbbar}{\ensuremath{\mathrm{b\overline{b}}}}
\newcommand{\Dzero}{\ensuremath{\mathrm{D^{0}}}}
\newcommand{\Dzerobar}{\ensuremath{\mathrm{\overline{D}^{0}}}}
\newcommand{\Dpm}{\ensuremath{\mathrm{D^{\pm}}}}
\newcommand{\Ds}{\ensuremath{\mathrm{D_{s}^{\pm}}}}
\newcommand{\Dstar}{\ensuremath{\mathrm{D^{*\pm}}}}

%collision systems
\newcommand{\pp}{pp}
\newcommand{\pPb}{p--Pb}
\newcommand{\PbPb}{Pb--Pb}

%detectors
\newcommand{\ezdc}{$E_{\rm ZDC}$}

%units
\newcommand{\GeVc}{GeV/$c$}
\newcommand{\GeVcsq}{GeV/$c^2$}

%others
\newcommand{\degree}{$^{\rm o}$}
\newcommand{\s}{\ensuremath{\sqrt{s}}}
\newcommand{\snn}{\ensuremath{\sqrt{s_{\rm NN}}}}
\newcommand{\y}{\ensuremath{y}}
\newcommand{\pt}{\ensuremath{p_{\rm T}}}
\newcommand{\dedx}{d$E$/d$x$}
\newcommand{\dndy}{d$N$/d$y$}
\newcommand{\dndydpt}{${\rm d}^2N/({\rm d}y {\rm d}p_{\rm t})$}
\newcommand{\zpar}{\ensuremath{z_{||}}}
\newcommand{\zpargen}{\ensuremath{z_{||}^{\mathrm{part}}}}
\newcommand{\zpardet}{\ensuremath{z_{||}^{\mathrm{det}}}}
\newcommand{\ptchjet}{\ensuremath{p_{\mathrm{T,ch\, jet}}}}
\newcommand{\ptjet}{\ensuremath{p_{\mathrm{T,jet}}}}
\newcommand{\ptchjetgen}{\ensuremath{p_{\mathrm{T,ch\,jet}}^{\mathrm{part}}}}
\newcommand{\ptchjetdet}{\ensuremath{p_{\mathrm{T,ch\,jet}}^{\mathrm{det}}}}
\newcommand{\ptd}{\ensuremath{p_{\mathrm{T,D}}}}
\newcommand{\ptdgen}{\ensuremath{p_{\mathrm{T,D}}^{\mathrm{part}}}}
\newcommand{\ptddet}{\ensuremath{p_{\mathrm{T,D}}^{\mathrm{det}}}}
\newcommand{\antikt}{anti-\ensuremath{k_{\mathrm{T}}}}
\newcommand{\Antikt}{Anti-\ensuremath{k_{\mathrm{T}}}}
\newcommand{\kt}{\ensuremath{k_{\mathrm{T}}}}
\newcommand{\pthard}{\ensuremath{p_{\mathrm{T,hard}}}}

\AtBeginSection[]{
  \begin{frame}
  \vfill
  \centering
  \begin{beamercolorbox}[sep=8pt,center,shadow=true,rounded=true]{title}
    \usebeamerfont{title}\insertsectionhead\par%
  \end{beamercolorbox}
  \vfill
  \end{frame}
}

\title[D-tagged jets in \pp\ and \pPb\ collisions] % (optional, use only with long paper titles)
{D-tagged jets in \pp\ and \pPb\ collisions}

\author[Salvatore Aiola]% (optional, use only with lots of authors)
{Salvatore Aiola$^{1}$ \\
Ant\^onio C.O. da Silva$^{2,3}$ \\
Barbara A. Trzeciak$^{3}$ \\ 
\bigskip
on behalf of PWG-HF}
% - Give the names in the same order as the appear in the paper.
% - Use the \inst{?} command only if the authors have different
%   affiliation.

\institute[Yale University] % (optional, but mostly needed)
{$^{1}$Yale University\\
$^{2}$University of S\~ao Paulo \\
$^{3}$Utrecht University}

\date[PF - Jan. 24th, 2017] % (optional, should be abbreviation of conference name)
{ALICE Physics Forum \\
January 24th, 2017}
% - Either use conference name or its abbreviation.
% - Not really informative to the audience, more for people (including
%   yourself) who are reading the slides online

\subject{High-Energy Physics}
% This is only inserted into the PDF information catalog. Can be left
% out. 



% If you have a file called "university-logo-filename.xxx", where xxx
% is a graphic format that can be processed by latex or pdflatex,
% resp., then you can add a logo as follows:

% \pgfdeclareimage[height=0.5cm]{university-logo}{university-logo-filename}
% \logo{\pgfuseimage{university-logo}}


% If you wish to uncover everything in a step-wise fashion, uncomment
% the following command: 

%\beamerdefaultoverlayspecification{<+->}


\begin{document}

\begin{frame}
  \titlepage
\end{frame}

\begin{frame}{Outline}
    \tableofcontents
 \end{frame}


% Structuring a talk is a difficult task and the following structure
% may not be suitable. Here are some rules that apply for this
% solution: 

% - Exactly two or three sections (other than the summary).
% - At *most* three subsections per section.
% - Talk about 30s to 2min per frame. So there should be between about
%   15 and 30 frames, all told.

% - A conference audience is likely to know very little of what you
%   are going to talk about. So *simplify*!
% - In a 20min talk, getting the main ideas across is hard
%   enough. Leave out details, even if it means being less precise than
%   you think necessary.
% - If you omit details that are vital to the proof/implementation,
%   just say so once. Everybody will be happy with that.

\section{Introduction}

\begin{frame}{Analysis Overview}
\begin{itemize}
\item \alert{Select D meson candidates using PID and topological cuts}
\begin{itemize}
\item as in D meson spectra analysis
\end{itemize}
\item \alert{For each candidate reconstruct the jet using the \antikt\ algorithm}
\begin{itemize}
\item replace D-meson daughters with D-meson 4-momentum
\end{itemize}
\item Use \kt\ algorithm to estimate the average background (\pPb)
\item Invariant mass analysis to extract the signal
\begin{itemize}
\item \alert{Side-Band (SB) \ptjet-spectra subtraction (in bins of \ptd)}, or
\item invariant mass fits in bins of \ptjet\ (internal cross-check)
\item efficiency correction applied at this step as a \ptd-dependent weight
\end{itemize}
\item \alert{B feed-down subtraction} using a POWHEG+PYTHIA simulation
\item \alert{Unfolding for jet momentum resolution} with detector response from PYTHIA6+GEANT3 simulation
\item Datasets:
\begin{itemize}
\item \pp\ at $\s=7$~TeV: LHC10b,c,d,e pass4 (312 M events)
\item \pPb\ at $\s=5.02$~TeV: LHC13b,c pass2 (98 M events)
\end{itemize}
\end{itemize}
\end{frame}

\section{\pp\ Analysis}

\subsection{Raw Yield Extraction}

\begin{frame}{Side-Band Subtraction}
\begin{center}
\begin{overpic}[width=.8\textwidth, trim=0 0 0 0, clip]{img/D0_Charged_R040_DPtBins_JetPt_5_30_SideBand_D0_Charged_R040_JetPtSpectrum_DPt_30_SideBand}
\end{overpic}
\end{center}
\vspace{-20pt}
\footnotesize
\begin{itemize}
\item Invariant mass fits (in bins of \ptd) determine position and width of the mass peak
\item Background normalization is $B_{\rm fit}/B_{\rm SB}$: $B_{\rm fit}$ = bkg. under the peak ($2\sigma_{\rm fit}$) from fit and
$B_{\rm SB}$ = integral of the SB ($4\sigma_{\rm fit}<|m-m_{\rm fit}|<8\sigma_{\rm fit}$)
\end{itemize}
\end{frame}

\begin{frame}{Side-Band Spectra}
\begin{columns}
\column{.28\textwidth}
\begin{overpic}[width=\textwidth, trim=0 0 380 0, clip]{img/D0_Charged_R040_JetPtSpectrum_DPt_30_SideBand_BkgVsSig}
\end{overpic}
\column{.28\textwidth}
\begin{overpic}[width=\textwidth, trim=380 0 0 0, clip]{img/D0_Charged_R040_JetPtSpectrum_DPt_30_SideBand_BkgVsSig}
\end{overpic}
\column{.44\textwidth}
\begin{center}
\vspace{-10pt}
\tiny \sffamily
$3.0 < \ptd < 30.0$~\GeVc

\begin{overpic}[width=.8\textwidth, trim=0 0 0 0, clip]{img/D0_Charged_R040_JetPtSpectrum_DPt_30_SideBand_TotalBkgVsSig}
\end{overpic}
\end{center}
\footnotesize
\vspace{-20pt}
\begin{itemize}
\item \ptjet\ spectra scaled by a \ptd-dependent efficiency correction factor
\item Sum together all the \ptd\ bins
\end{itemize}
\end{columns}
\begin{itemize}
\item \textcolor{BrickRed}{SB} jet spectra ($4\sigma_{\rm fit}<|m-m_{\rm fit}|<8\sigma_{\rm fit}$) \textcolor{ForestGreen}{subtracted} from \textcolor{NavyBlue}{peak region} ($|m-m_{\rm fit}|<2\sigma_{\rm fit}$)
\end{itemize}
\end{frame}

\subsection{B feed-down correction}

\begin{frame}{B feed-down correction}

\begin{columns}
\column{.45\textwidth}
\begin{overpic}[width=\textwidth, trim=0 0 0 0, clip]{img/BFeedDown_InternalPlot}
\end{overpic}
\column{.50\textwidth}
\small
\begin{itemize}
\item B Feed-Down (FD) spectrum is estimated with a POWHEG+PYTHIA6 simulation (scaled by the luminosity)
\item Weighted by the ratio of the prompt and non-prompt \Dzero\ reconstruction efficiencies and smeared using detector response
\item Subtracted from the measured yield before unfolding
\end{itemize}
\vspace{-5pt}
\end{columns}
{\tiny
$\textcolor{ForestGreen}{N^{\rm c\rightarrow\Dzero}_{\rm det}(\ptchjet)} = \textcolor{NavyBlue}{N^{\rm c,b\rightarrow\Dzero}_{\rm raw}(\ptchjet)} - 
\textcolor{BrickRed}{R_{\rm det}^{\rm b\rightarrow\Dzero}(\ptchjet) \otimes \sum_{\ptd} \frac{\epsilon^{\rm b\rightarrow\Dzero}(\ptd)}{\epsilon^{\rm c\rightarrow\Dzero}(\ptd)} N^{\rm b\rightarrow\Dzero}_{\rm POWHEG}(\ptd,\ptchjet)}$
}

{\tiny
Note: no normalization is applied and not divided by the bin width, these are counts weighted by the reconstruction efficiency
}

\end{frame}

\subsection{Unfolding}

\begin{frame}{Unfolding}
\begin{columns}
\column{.50\textwidth}
\begin{overpic}[width=\textwidth, trim=0 0 0 0, clip]{img/SideBand_DPt_30_UnfoldingSummary_Bayes}
\end{overpic}
\column{.50\textwidth}
\begin{overpic}[width=\textwidth, trim=0 0 0 0, clip]{img/SideBand_DPt_30_UnfoldingSummary_Bayes_UnfoldedOverMeasured}
\end{overpic}
\end{columns}
\begin{itemize}
\item Very small correction (5-10\%), much smaller than the statistical uncertainty!
\item Correction is small because: good momentum resolution (track-only, D-meson requirement) and large bins
\end{itemize}
\end{frame}

\subsection{Updates}

\begin{frame}{Updates since the preview}

{\small
\begin{itemize}
\item Reflection templates (i.e. wrong mass hypothesis) added in the fit procedure $\rightarrow$ central points shifted by about 5\% down
\item Cut on D-meson \pt\ changed from $\ptd>2$~\GeVc\ to $\ptd>3$~\GeVc\
\item Fixed event counting to include events w/o reconstructed vertex 
\item Systematic uncertainties finalized (raw yield extr., B feed-down)
\item Requests and open points from the Physics Forum Preview
\begin{itemize}
\item Validate POWHEG simulation with data (A. Dainese)
\item Show response matrix with expanded color scale (A. Morsch)
\item Ratios of yield extraction multi-trials for each \ptd\ bin (E. Bruna)
\item Internal plot of D-meson momentum fraction (A. Morsch)
\item Plot D-meson \pt\ spectrum on top of jet \pt\ spectrum
\item Effect of the \ptd\ cut in POWHEG (M. van Leeuwen)
\item Number of tracks in the jet (A. Dainese)
\item Check statistical uncertainty of the efficiency
\end{itemize}
\end{itemize}
}
\end{frame}

\begin{frame}{Reflection Templates}
\begin{columns}
\column{.50\textwidth}
\begin{overpic}[width=\textwidth, trim=0 0 0 0, clip]{img/ReflectionVariationComparison}
\end{overpic}
\column{.50\textwidth}
\begin{overpic}[width=\textwidth, trim=0 0 0 0, clip]{img/ReflectionVariationComparison_Ratio}
\end{overpic}
\end{columns}
\begin{itemize}
\item Reflection (= wrong mass hypothesis) templates (from MC) have been added in the fit procedure
\item 5\% effect on the central points (purple dash-dotted line)
\item Templates are fit with several functions (double Gaussian, Gaussian, pol3, pol6) $\rightarrow$ negligible effect
\item The ratio reflection/signal ($\sim 0.3$) is varied by $\pm50$~\%: up to 10\% effect (red dashed line, taken as systematic uncertainty)
\end{itemize}
\end{frame}

\subsubsection{Raw Yield Extraction}

\begin{frame}{Raw Yield Extraction Uncertainty}
\begin{columns}
\column{.50\textwidth}
\begin{overpic}[width=\textwidth, trim=0 0 0 0, clip]{img/CompareRawYieldUncVariations_AfterDbinSum_Ratio}
\end{overpic}
\column{.50\textwidth}
The following variations are considered:
\begin{itemize}
\item Free, fixed sigma and mean (fixed sigma varied $\pm 15$~\% of MC value)
\item Background functions: exponential, linear, pol2
\item Binning of the invariant mass histogram (0.006, 0.012 \GeVcsq) and fit range limits
\end{itemize}
\end{columns}
\begin{itemize}
\item In total over 300 trials!
\item 30 of them picked randomly: uncertainty is calculated as RMS of the trials with respect to the average
\end{itemize}
\end{frame}


\subsubsection{B Feed-Down}

\begin{frame}{B Feed-Down Uncertainty}
\begin{columns}
\column{.50\textwidth}
\begin{overpic}[width=\textwidth, trim=0 0 0 0, clip]{img/BFeedDown_JetPtSpectrum_DPt_30_Unfolded_c_JetPtSpectrum_bEfficiencyMultiply_cEfficiencyDivide_Ratio}
\end{overpic}
\column{.50\textwidth}
The following variations are considered:
\begin{itemize}
\item mass of the b quark
\item factorization and renormalization scales
\item PDF
\end{itemize}
\end{columns}
\begin{itemize}
\item The largest variation in each bin is symmetrized and taken as systematic uncertainty
\end{itemize}
\end{frame}

\subsubsection{Requests from PF}

\begin{frame}{POWHEG vs. FONLL (beauty)}
\begin{center}
\begin{overpic}[width=.75\textwidth, trim=80 500 70 50, clip]{img/1205_6344v1_p11}
\end{overpic}
\end{center}
{\small
\begin{itemize}
\item \href{https://doi.org/10.1007/JHEP10(2012)137}{Cacciari et al, Theoretical predictions for charm and bottom production at the LHC, JHEP 10 (2012) 137}
\item Good agreement between POWHEG+PYTHIA and FONLL (left)
\item FONLL in agreement with data (right)
\end{itemize}
}
\end{frame}

\begin{frame}{D-Meson Momentum Fraction}
\begin{columns}
\column{.50\textwidth}
\begin{overpic}[width=\textwidth, trim=0 0 0 0, clip]{img/TheoryComparison_ZSpectraNorm_powheg_Charged_R040}
\end{overpic}
\column{.50\textwidth}
\begin{overpic}[width=\textwidth, trim=0 0 0 0, clip]{img/TheoryComparison_ZSpectraNorm_powheg_Charged_R040_Ratio}
\end{overpic}
\end{columns}
\begin{itemize}
\item Momentum fraction calculated as \zpar$=\frac{\bm{p}_{\rm jet}\cdot{\bm{p}}_{\rm D^0}}{\bm{p}_{\rm jet}^2}$
\item Full 2D (\ptchjet, \zpar) side-band analysis
\item Fragmentation compatible with POWHEG+PYTHIA6
\end{itemize}
\end{frame}

\begin{frame}{D-Meson \pt\ vs. Jet \pt}
\begin{columns}
\column{.50\textwidth}
\begin{overpic}[width=\textwidth, trim=0 0 0 0, clip]{img/Comparison_DPt_JetPt_Spectra}
\end{overpic}
\column{.50\textwidth}
\begin{overpic}[width=\textwidth, trim=0 0 0 0, clip]{img/Comparison_DPt_JetPt_Spectra_Ratio}
\end{overpic}
\end{columns}
\begin{itemize}
\item Not of immediate interpretation
\item Data and POWHEG show a similar trend, although different shape at low \pt
\end{itemize}
\end{frame}

\begin{frame}{Effect of \ptd\ cut}
\begin{columns}
\column{.50\textwidth}
\begin{overpic}[width=\textwidth, trim=0 0 0 0, clip]{img/D0_MCTruth_Charged_R040_jet_pt_SpectraComparison}
\end{overpic}
\column{.50\textwidth}
\begin{overpic}[width=\textwidth, trim=0 0 0 0, clip]{img/D0_MCTruth_Charged_R040_jet_pt_SpectraComparison_Ratio}
\end{overpic}
\end{columns}
%\begin{itemize}
%\end{itemize}
\end{frame}

\begin{frame}{Number of jet constituents}
\begin{overpic}[width=\textwidth, trim=0 0 0 0, clip]{img/NJetConstituents_LHC10_Train847_efficiency_LHC14j4_Train1018_efficiency}
\end{overpic}
\begin{itemize}
\item Obtained with a full 2D (\ptchjet, num. of jet const.) side-band analysis
\item Good agreement with MC (PYTHIA6 Perugia-2011)
\item Distribution peaked at 2-3 constituents per jet (including the \Dzero\ meson)
\end{itemize}
\end{frame}

\begin{frame}{Statistical precision of reconstruction efficiency}
\begin{columns}
\column{.50\textwidth}
\begin{overpic}[width=\textwidth, trim=0 0 0 0, clip]{img/D0_Jet_AKTChargedR040_pt_scheme_JetPtDPtSpectrum_PartialEfficiency}
\end{overpic}
\column{.50\textwidth}
\begin{overpic}[width=\textwidth, trim=0 0 0 0, clip]{img/RecoEfficiencyRelStatUnc}
\end{overpic}
\end{columns}
%\begin{itemize}
%\end{itemize}
\end{frame}

\subsection{Final Uncertainties}

\begin{frame}{Final Uncertainties}
\begin{columns}
\column{.50\textwidth}
\begin{overpic}[width=\textwidth, trim=0 0 0 0, clip]{img/CompareVariations_DataSystematics_LHC10_Ratio}
\end{overpic}
\column{.50\textwidth}
\begin{overpic}[width=\textwidth, trim=0 0 0 0, clip]{img/CompareUncertainties_DataSystematics_LHC10}
\end{overpic}
\end{columns}
\begin{itemize}
\item Systematic uncertainty dominated by Reflection/Signal ratio (10\%)
\item Overall uncertainty dominated by statistics
\item Figure on the right: request for approval as Technical Preliminary
\end{itemize}
\end{frame}

\begin{frame}{\pt-independent Systematic Uncertainties}
\begin{center}
    \begin{tabular}{lr}
    \hline
Source & Uncertainty (\%) \\ \hline
Unfolding & 5.0 \\
Selection Cuts & 5.0 \\
Tracking Efficiency (D-Meson) & 4.0 \\
Branching Ratio & 1.0 \\
Luminosity & 3.5 \\
\hline
Total & 8.9 \\
\hline
    \end{tabular}
    \end{center}
\end{frame}

\begin{frame}{\pt-dependent Systematic Uncertainties}
\begin{center}
    \begin{tabular}{lrrrrrr}
    \hline
Source & \multicolumn{6}{c}{Uncertainty (\%)} \\ \hline
\ptchjet (\GeVc) & 5 - 6 & 6 - 8 & 8 - 10 & 10 - 14 & 14 - 20 & 20 - 30\\ \hline
Tracking Eff. & 1 & 2 & 3 & 3 & 3 & 4\\
Raw Yield Extr. & 3 & 3 & 3 & 7 & 8 & 16\\
Refl/Sign & 0 & 2 & 3 & 8 & 12 & 12\\
B Feed-Down & 4 & 4 & 5 & 10 & 9 & 10\\
\hline
\pt-indep & \multicolumn{6}{c}{8.9} \\
\hline
Total Syst. & 10 & 11 & 12 & 17 & 20 & 25\\
\hline
Stat. Unc. & 10.7 & 10.0 & 14.8 & 28.2 & 31.2 & 48.4\\
\hline
Total Unc. & 15 & 15 & 19 & 33 & 37 & 54\\
\hline
    \end{tabular}
    \end{center}
\end{frame}

\subsection{Figures for Approval}

\begin{frame}{Phys.Prel: \Dzero-jet \pt-differential cross section in \pp\ collisions at $\s=7$~TeV}
\begin{overpic}[width=.5\textwidth, trim=0 0 0 0, clip]{img/D0JetCrossSection_pp7TeV}
\end{overpic}
%\begin{itemize}
%\end{itemize}
\end{frame}

\begin{frame}{Tech.Prel.: Side-Band Signal Extraction}
\begin{overpic}[width=\textwidth, trim=0 0 0 0, clip]{img/SideBandInvMass_QM17}
\end{overpic}
%\begin{itemize}
%\end{itemize}
\end{frame}

\begin{frame}{Tech.Prel.: B Feed-Down Correction}
\begin{overpic}[width=\textwidth, trim=0 0 0 0, clip]{img/BFeedDown_QM17}
\end{overpic}
%\begin{itemize}
%\end{itemize}
\end{frame}

\begin{frame}{Tech.Prel.: Relative Uncertainties}
\begin{overpic}[width=\textwidth, trim=0 0 0 0, clip]{img/Uncertainties_QM17}
\end{overpic}
%\begin{itemize}
%\end{itemize}
\end{frame}

\begin{frame}{Sim.: Reconstruction Efficiency}
\begin{center}
\begin{overpic}[width=.7\textwidth, trim=0 0 0 0, clip]{img/Efficiency_QM17}
\end{overpic}
\end{center}
\vspace{-20pt}
\begin{itemize}
\item Efficiency is higher for b~$\rightarrow$~\Dzero\ because of the longer decay length of B mesons (topological cuts)
\end{itemize}
\end{frame}


\section{\pPb\ Analysis}

\section{Summary}

\begin{frame}{Conclusions}
\end{frame}

\section*{Extra}

\begin{frame}{Extra Slides}
\huge
\begin{center}
Extra Slides
\end{center}
\end{frame}

\subsection*{Raw Yield Extraction}

\begin{frame}{Invariant Mass Fits in \ptjet\ bins}
\begin{center}
\begin{overpic}[width=.85\textwidth, trim=0 0 0 0, clip]{img/D0_Charged_R040_JetPtBins_DPt_30}
\end{overpic}
\end{center}
\vspace{-20pt}
\small
\begin{itemize}
\item Efficiency correction applied as a \ptd-dependent weight when filling the invariant mass plots
\end{itemize}
\end{frame}

\begin{frame}{Comparison of the two methods}
\begin{columns}
\column{.50\textwidth}
\begin{overpic}[width=\textwidth, trim=0 0 0 0, clip]{img/D0_Charged_R040_jet_pt_SpectraComparison}
\end{overpic}
\column{.50\textwidth}
\begin{overpic}[width=\textwidth, trim=0 0 0 0, clip]{img/D0_Charged_R040_jet_pt_SpectraComparison_Ratio}
\end{overpic}
\end{columns}
\begin{itemize}
\item The two methods give consistent results
\item Discrepancies are well within statistical precision -- albeit fluctuations are partially correlated
\end{itemize}
\end{frame}

\begin{frame}{Raw Yield Variations in \ptd\ bins (SB method) / 1}
\begin{columns}
\column{.50\textwidth}
$3 < \ptd < 4$~\GeVc\
\begin{overpic}[width=.8\textwidth, trim=0 0 0 0, clip]{img/CompareRawYieldUncVariations_0_Ratio}
\end{overpic}
$4 < \ptd < 5$~\GeVc\
\begin{overpic}[width=.8\textwidth, trim=0 0 0 0, clip]{img/CompareRawYieldUncVariations_1_Ratio}
\end{overpic}
\column{.50\textwidth}
$5 < \ptd < 6$~\GeVc\
\begin{overpic}[width=.8\textwidth, trim=0 0 0 0, clip]{img/CompareRawYieldUncVariations_2_Ratio}
\end{overpic}
$6 < \ptd < 7$~\GeVc\
\begin{overpic}[width=.8\textwidth, trim=0 0 0 0, clip]{img/CompareRawYieldUncVariations_3_Ratio}
\end{overpic}
\end{columns}
\end{frame}

\begin{frame}{Raw Yield Variations in \ptd\ bins (SB method) / 2}
\begin{columns}
\column{.50\textwidth}
$7 < \ptd < 8$~\GeVc\
\begin{overpic}[width=.8\textwidth, trim=0 0 0 0, clip]{img/CompareRawYieldUncVariations_4_Ratio}
\end{overpic}
$8 < \ptd < 10$~\GeVc\
\begin{overpic}[width=.8\textwidth, trim=0 0 0 0, clip]{img/CompareRawYieldUncVariations_5_Ratio}
\end{overpic}
\column{.50\textwidth}
$10 < \ptd < 12$~\GeVc\
\begin{overpic}[width=.8\textwidth, trim=0 0 0 0, clip]{img/CompareRawYieldUncVariations_6_Ratio}
\end{overpic}
$12 < \ptd < 16$~\GeVc\
\begin{overpic}[width=.8\textwidth, trim=0 0 0 0, clip]{img/CompareRawYieldUncVariations_7_Ratio}
\end{overpic}
\end{columns}
\end{frame}

\begin{frame}{Raw Yield Variations in \ptd\ bins (SB method) / 3}
\begin{columns}
\column{.50\textwidth}
$16 < \ptd < 30$~\GeVc\
\begin{overpic}[width=.8\textwidth, trim=0 0 0 0, clip]{img/CompareRawYieldUncVariations_8_Ratio}
\end{overpic}
\column{.50\textwidth}
$3 < \ptd < 30$~\GeVc\
\begin{overpic}[width=.8\textwidth, trim=0 0 0 0, clip]{img/CompareRawYieldUncVariations_AfterDbinSum_Ratio}
\end{overpic}
\end{columns}
\end{frame}

\begin{frame}{Raw Yield Systematic Uncertainty}
\begin{columns}
\column{.50\textwidth}
\begin{overpic}[width=\textwidth, trim=0 0 0 0, clip]{img/AverageRawYieldVsDefault}
\end{overpic}
\column{.50\textwidth}
\begin{overpic}[width=\textwidth, trim=0 0 0 0, clip]{img/AverageRawYieldVsDefault_Ratio}
\end{overpic}
\end{columns}
\end{frame}

\begin{frame}{Side-Band Spectra}
\begin{center}
\begin{overpic}[width=.85\textwidth, trim=0 0 0 0, clip]{img/D0_Charged_R040_JetPtSpectrum_DPt_30_SideBand_BkgVsSig}
\end{overpic}
\end{center}
\begin{itemize}
\item \textcolor{BrickRed}{SB} jet spectra ($4\sigma_{\rm fit}<|m-m_{\rm fit}|<8\sigma_{\rm fit}$) \textcolor{ForestGreen}{subtracted} from \textcolor{NavyBlue}{peak region} ($|m-m_{\rm fit}|<2\sigma_{\rm fit}$)
\end{itemize}
\end{frame}

\begin{frame}{MC Closure Test}
\begin{columns}
\column{.50\textwidth}
\begin{overpic}[width=\textwidth, trim=0 0 0 0, clip]{img/MC_Charged_R040_JetPtSpectrum_DPt_30_SpectraComparison}
\end{overpic}
\column{.50\textwidth}
\begin{overpic}[width=\textwidth, trim=0 0 0 0, clip]{img/MC_Charged_R040_JetPtSpectrum_DPt_30_SpectraComparison_Ratio}
\end{overpic}
\end{columns}
\begin{itemize}
\item Spectra obtained using the inv. mass fit and SB methods compared with MC truth ($\ptd > 2$~\GeVc)
\end{itemize}
\end{frame}

\subsection*{B Feed-Down}

\begin{frame}{FD vs. \ptchjet\ and \ptd}
\begin{overpic}[width=.8\textwidth, trim=0 0 0 0, clip]{img/BFeedDownVsPtJet_powheg_Full_R040_1478868679_1478869008_Ratio}
\end{overpic}
\begin{itemize}
\item POWHEG+PYTHIA: Feed-Down fraction monotonically increases as a function of \ptchjet\ (left)
\item PYTHIA+GEANT3: Reconstruction efficiency largely independent of \ptchjet, for $5<\ptchjet<30$~\GeVc\ (right)
\end{itemize}
\end{frame}

\begin{frame}{Prompt vs. Non-Prompt Detector Response}
\begin{center}
\begin{overpic}[width=.8\textwidth, trim=0 0 0 0, clip]{img/DetectorJetPtResolutionComparison}
\end{overpic}
\end{center}
\vspace{-20pt}
\small
\begin{itemize}
\item Momentum resolution as a function of \ptchjet
\item Significantly different response for b~$\rightarrow$~\Dzero (blue) compared c~$\rightarrow$~\Dzero (red)
\end{itemize}
\end{frame}

\begin{frame}{Smear w/ b~$\rightarrow$~\Dzero, unfold w/ c~$\rightarrow$~\Dzero}
\begin{columns}
\column{.50\textwidth}
\begin{overpic}[width=\textwidth, trim=0 0 0 0, clip]{img/FD_FoldUnfold_Comparison}
\end{overpic}
\column{.50\textwidth}
\begin{overpic}[width=\textwidth, trim=0 0 0 0, clip]{img/FD_FoldUnfold_Comparison_Ratio}
\end{overpic}
\end{columns}
\begin{itemize}
\item Feed-Down spectrum smeared using b~$\rightarrow$~\Dzero\ response (blue)
\item Unfolded using c~$\rightarrow$~\Dzero\ response (red)
\item Sanity check: unfold using b~$\rightarrow$~\Dzero\ response (green)
\end{itemize}
\end{frame}

\begin{frame}{POWHEG vs. FONLL (charm)}
\begin{center}
\begin{overpic}[width=.75\textwidth, trim=80 500 70 50, clip]{img/1205_6344v1_p9}
\end{overpic}
\end{center}
{\small
\begin{itemize}
\item \href{https://doi.org/10.1007/JHEP10(2012)137}{Cacciari et al, Theoretical predictions for charm and bottom production at the LHC, JHEP 10 (2012) 137}
\item D meson spectra are in the upper band of FONLL calculations
\item Similarly as observed in \Dzero-jet compared with POWHEG+PYTHIA
\end{itemize}
}
\end{frame}

\subsection*{Detector Performance}

\begin{frame}{Efficiency vs. \ptchjet\ and \ptd}
\begin{columns}
\column{.50\textwidth}
\begin{overpic}[width=\textwidth, trim=0 0 0 0, clip]{img/D0_Jet_AKTChargedR040_pt_scheme_JetPtDPtSpectrum_PartialEfficiency}
\end{overpic}
\column{.50\textwidth}
\begin{overpic}[width=\textwidth, trim=0 0 0 0, clip]{img/D0_Jet_AKTChargedR040_pt_scheme_JetPtDPtSpectrum_PartialEfficiencyRatios}
\end{overpic}
\end{columns}
\begin{itemize}
\item Reconstruction efficiency largely independent of \ptchjet, for $5<\ptchjet<30$~\GeVc
\end{itemize}
\end{frame}

\begin{frame}{Detector Response \ptchjet\ Matrix}
\begin{center}
\begin{overpic}[width=.65\textwidth, trim=0 0 0 0, clip]{img/D0_Jet_AKTChargedR040_pt_scheme_JetPtSpectrum_DPt_30_DetectorResponse_canvas}
\end{overpic}
\end{center}
\end{frame}

\subsection*{Unfolding}

\begin{frame}{Unfolding Systematics}
\begin{columns}
\column{.50\textwidth}
\begin{center}
\tiny
Method Comparison
\begin{overpic}[width=.8\textwidth, trim=0 0 0 0, clip]{img/SideBand_DPt_30_UnfoldingMethod_Ratio}
\end{overpic}\\
Bayesian Regularization
\begin{overpic}[width=.8\textwidth, trim=0 0 0 0, clip]{img/SideBand_DPt_30_UnfoldingRegularization_Bayes_PriorResponseTruth_Ratio}
\end{overpic}
\end{center}
\column{.50\textwidth}
\begin{center}
\tiny
Prior Choice \\
\begin{overpic}[width=.8\textwidth, trim=0 0 0 0, clip]{img/SideBand_DPt_30_UnfoldingPrior_Bayes_Ratio}
\end{overpic}
\end{center}
\vspace{-20pt}
\begin{itemize}
\scriptsize
\item Baseline: Bayesian
\begin{itemize}
\tiny
\item fast convergence
\item stable after 3 iterations
\item then $< 1$\% variations
\end{itemize}
\item Methods: bin-by-bin correction, SVD
\begin{itemize}
\tiny
\item equivalent results within few \%
\end{itemize}
\item Priors: PYTHIA and $\ptjet^{-a}$ with $a=3, 7$
\begin{itemize}
\tiny
\item no effect on the unfolding
\end{itemize}
\end{itemize}
\end{columns}
\end{frame}

\begin{frame}{Unfolding: Summary}
\begin{itemize}
\item Very small correction
\item Very robust against:
\begin{itemize}
\item method variation (Bayesian, SVD, Bin-By-Bin)
\item priors choice
\item regularization strength
\end{itemize}
\item Systematic uncertainty much smaller than statistical $\approx 20-50$~\%
\item No uncertainty quoted, or few \% based on MC closure tests
\end{itemize}
\end{frame}

\begin{frame}{Bin-By-Bin Correction Factors}
\begin{center}
\begin{overpic}[width=.65\textwidth, trim=0 5 50 10, clip]{img/SideBand_DPt_30_BinByBinCorrectionFactors}
\end{overpic}
\end{center}
\vspace{-20pt}
\footnotesize
\begin{itemize}
\item Bin-By-Bin correction factors: ratio between particle-level and detector-level
\item Some dependence on the prior (= particle-level spectrum)
\item 5-15\% correction
\end{itemize}
\end{frame}

\begin{frame}{Method Comparison}
\begin{columns}
\column{.50\textwidth}
\begin{overpic}[width=\textwidth, trim=0 0 0 0, clip]{img/SideBand_DPt_30_UnfoldingMethod}
\end{overpic}
\column{.50\textwidth}
\begin{overpic}[width=\textwidth, trim=0 0 0 0, clip]{img/SideBand_DPt_30_UnfoldingMethod_Ratio}
\end{overpic}
\end{columns}
\begin{itemize}
\item Baseline: Bayesian
\item Compared with: bin-by-bin correction, regularized SVD
\item All methods give equivalent results within few \% (large statistical uncertainty!)
\end{itemize}
\end{frame}

\begin{frame}{Prior Choice (Bayesian)}
\begin{columns}
\column{.50\textwidth}
\begin{overpic}[width=\textwidth, trim=0 0 0 0, clip]{img/SideBand_DPt_30_UnfoldingPrior_Bayes}
\end{overpic}
\column{.50\textwidth}
\begin{overpic}[width=\textwidth, trim=0 0 0 0, clip]{img/SideBand_DPt_30_UnfoldingPrior_Bayes_Ratio}
\end{overpic}
\end{columns}
\begin{itemize}
\item Priors: PYTHIA spectrum, power laws index 3 and 7 (quite large variation)
\item $\rightarrow$ no effect on the unfolding
\end{itemize}
\end{frame}

\begin{frame}{Bayesian Regularization (iterations)}
\begin{columns}
\column{.50\textwidth}
\begin{overpic}[width=\textwidth, trim=0 0 0 0, clip]{img/SideBand_DPt_30_UnfoldingRegularization_Bayes_PriorResponseTruth}
\end{overpic}
\column{.50\textwidth}
\begin{overpic}[width=\textwidth, trim=0 0 0 0, clip]{img/SideBand_DPt_30_UnfoldingRegularization_Bayes_PriorResponseTruth_Ratio}
\end{overpic}
\end{columns}
\begin{itemize}
\item Very fast convergence
\item Stable after 3 iterations, then $< 1$\% variations
\end{itemize}
\end{frame}

\begin{frame}{Priors}
\begin{columns}
\column{.50\textwidth}
\begin{overpic}[width=\textwidth, trim=0 0 0 0, clip]{img/SideBand_DPt_30_Priors}
\end{overpic}
\column{.50\textwidth}
\begin{overpic}[width=\textwidth, trim=0 0 0 0, clip]{img/SideBand_DPt_30_Priors_Ratio}
\end{overpic}
\end{columns}
\begin{itemize}
\item PYTHIA6 spectrum
\item Power-law spectra: $\ptjet^{-a}$, with $a=3, 7$
\end{itemize}
\end{frame}

\begin{frame}{MC Unfolding Closure Test: Signal-Only Spectra}
\begin{columns}
\column{.50\textwidth}
\begin{overpic}[width=\textwidth, trim=0 0 0 0, clip]{img/MC_SignalOnly_DPt_30_UnfoldingRegularization_Bayes_PriorResponseTruth_Ratio}
\end{overpic}
\column{.50\textwidth}
\begin{overpic}[width=\textwidth, trim=0 0 0 0, clip]{img/MC_SignalOnly_DPt_30_UnfoldingMethod_Ratio}
\end{overpic}
\end{columns}
\begin{itemize}
\item Detector-level spectra obtained (signal-only via MC truth info) are unfolded and compared with the MC generated spectra
\end{itemize}
\end{frame}

\begin{frame}{MC Unfolding Closure Test: Inv.Mass Fit Spectra}
\begin{columns}
\column{.50\textwidth}
\begin{overpic}[width=\textwidth, trim=0 0 0 0, clip]{img/MC_SideBand_DPt_30_UnfoldingRegularization_Bayes_PriorResponseTruth_Ratio}
\end{overpic}
\column{.50\textwidth}
\begin{overpic}[width=\textwidth, trim=0 0 0 0, clip]{img/MC_SideBand_DPt_30_UnfoldingMethod_Ratio}
\end{overpic}
\end{columns}
\begin{itemize}
\item Spectra obtained using the inv. mass fit are unfolded and compared with the MC generated spectra
\end{itemize}
\end{frame}

\begin{frame}{\pt-dependent Systematic Uncertainties (no smoothening)}
\begin{center}
    \begin{tabular}{lrrrrrr}
    \hline
Source & \multicolumn{6}{c}{Uncertainty (\%)} \\ \hline
\ptchjet (\GeVc) & 5 - 6 & 6 - 8 & 8 - 10 & 10 - 14 & 14 - 20 & 20 - 30\\ \hline
Tracking Eff. & 1.1 & 1.5 & 3.8 & 2.6 & 2.9 & 4.3\\
Raw Yield Extr. & 3.2 & 4.9 & 2.7 & 6.6 & 7.6 & 16.8\\
Refl/Sign & 0.2 & 1.8 & 2.6 & 8.0 & 13.1 & 11.1\\
B Feed-Down & 4.1 & 4.4 & 4.9 & 10.0 & 8.6 & 10.3\\
\hline
\pt-indep & \multicolumn{6}{c}{8.9} \\
\hline
Total Syst. & 10.4 & 11.3 & 11.5 & 17.2 & 19.8 & 24.6\\
\hline
Stat. Unc. & 10.7 & 10.0 & 14.8 & 28.2 & 31.2 & 48.4\\
\hline
Total Unc. & 14.9 & 15.1 & 18.7 & 33.0 & 37.0 & 54.3\\
\hline
    \end{tabular}
    \end{center}
\end{frame}

\end{document}
