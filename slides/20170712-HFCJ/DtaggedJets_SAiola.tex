% $Header: /Users/joseph/Documents/LaTeX/beamer/solutions/conference-talks/conference-ornate-20min.en.tex,v 90e850259b8b 2007/01/28 20:48:30 tantau $

\documentclass[xcolor={usenames,dvipsnames}]{beamer}

% This file is a solution template for:

% - Talk at a conference/colloquium.
% - Talk length is about 20min.
% - Style is ornate.



% Copyright 2004 by Till Tantau <tantau@users.sourceforge.net>.
%
% In principle, this file can be redistributed and/or modified under
% the terms of the GNU Public License, version 2.
%
% However, this file is supposed to be a template to be modified
% for your own needs. For this reason, if you use this file as a
% template and not specifically distribute it as part of a another
% package/program, I grant the extra permission to freely copy and
% modify this file as you see fit and even to delete this copyright
% notice. 


\mode<presentation>
{
  \usetheme{AnnArbor}
  % or ...

  \setbeamercovered{transparent}
  % or whatever (possibly just delete it)
 }

\usepackage[percent]{overpic}
\usepackage[english]{babel}
% or whatever

\usepackage[latin1]{inputenc}
% or whatever

\usepackage{times}
\usepackage[T1]{fontenc}
% Or whatever. Note that the encoding and the font should match. If T1
% does not look nice, try deleting the line with the fontenc.
%particles
\newcommand{\jpsi}{\rm J/$\psi$}
\newcommand{\psip}{$\psi^\prime$}
\newcommand{\jpsiDY}{\rm J/$\psi$\,/\,DY}
\newcommand{\chic}{$\chi_{\rm c}$}
\newcommand{\pip}{$\pi^{+}$}
\newcommand{\pim}{$\pi^{-}$}
\newcommand{\pizero}{$\pi^{0}$}
\newcommand{\kap}{K$^{+}$}
\newcommand{\kam}{K$^{-}$}
\newcommand{\pbar}{$\rm\overline{p}$}
\newcommand{\ccbar}{\ensuremath{\mathrm{c\overline{c}}}}
\newcommand{\bbbar}{\ensuremath{\mathrm{b\overline{b}}}}
\newcommand{\Dzero}{\ensuremath{\mathrm{D^{0}}}}
\newcommand{\Dzerobar}{\ensuremath{\mathrm{\overline{D}^{0}}}}
\newcommand{\Dpm}{\ensuremath{\mathrm{D^{\pm}}}}
\newcommand{\Ds}{\ensuremath{\mathrm{D_{s}^{\pm}}}}
\newcommand{\Dstar}{\ensuremath{\mathrm{D^{*\pm}}}}

%collision systems
\newcommand{\pp}{pp}
\newcommand{\pPb}{p--Pb}
\newcommand{\PbPb}{Pb--Pb}

%detectors
\newcommand{\ezdc}{$E_{\rm ZDC}$}

%units
\newcommand{\GeVc}{GeV/$c$}
\newcommand{\GeVcsq}{GeV/$c^2$}

%others
\newcommand{\degree}{$^{\rm o}$}
\newcommand{\s}{\ensuremath{\sqrt{s}}}
\newcommand{\snn}{\ensuremath{\sqrt{s_{\rm NN}}}}
\newcommand{\y}{\ensuremath{y}}
\newcommand{\pt}{\ensuremath{p_{\rm T}}}
\newcommand{\dedx}{d$E$/d$x$}
\newcommand{\dndy}{d$N$/d$y$}
\newcommand{\dndydpt}{${\rm d}^2N/({\rm d}y {\rm d}p_{\rm t})$}
\newcommand{\zpar}{\ensuremath{z_{||}}}
\newcommand{\zpargen}{\ensuremath{z_{||}^{\mathrm{part}}}}
\newcommand{\zpardet}{\ensuremath{z_{||}^{\mathrm{det}}}}
\newcommand{\ptchjet}{\ensuremath{p_{\mathrm{T,ch\, jet}}}}
\newcommand{\ptjet}{\ensuremath{p_{\mathrm{T,jet}}}}
\newcommand{\ptchjetgen}{\ensuremath{p_{\mathrm{T,ch\,jet}}^{\mathrm{part}}}}
\newcommand{\ptchjetdet}{\ensuremath{p_{\mathrm{T,ch\,jet}}^{\mathrm{det}}}}
\newcommand{\ptd}{\ensuremath{p_{\mathrm{T,D}}}}
\newcommand{\ptdgen}{\ensuremath{p_{\mathrm{T,D}}^{\mathrm{part}}}}
\newcommand{\ptddet}{\ensuremath{p_{\mathrm{T,D}}^{\mathrm{det}}}}
\newcommand{\antikt}{anti-\ensuremath{k_{\mathrm{T}}}}
\newcommand{\Antikt}{Anti-\ensuremath{k_{\mathrm{T}}}}
\newcommand{\kt}{\ensuremath{k_{\mathrm{T}}}}
\newcommand{\pthard}{\ensuremath{p_{\mathrm{T,hard}}}}

\AtBeginSection[]{
  \begin{frame}
  \vfill
  \centering
  \begin{beamercolorbox}[sep=8pt,center,shadow=true,rounded=true]{title}
    \usebeamerfont{title}\insertsectionhead\par%
  \end{beamercolorbox}
  \vfill
  \end{frame}
}

\title[Underlying event in \pp] % (optional, use only with long paper titles)
{Underlying event in \pp\ collisions}

\author[Salvatore Aiola]% (optional, use only with lots of authors)
{Salvatore Aiola}
% - Give the names in the same order as the appear in the paper.
% - Use the \inst{?} command only if the authors have different
%   affiliation.

\institute[Yale University] % (optional, but mostly needed)
{Yale University}

\date[PAG-HFCJ - July 12th, 2017] % (optional, should be abbreviation of conference name)
{PAG-HFCJ \\
July 12th, 2017}
% - Either use conference name or its abbreviation.
% - Not really informative to the audience, more for people (including
%   yourself) who are reading the slides online

\subject{High-Energy Physics}
% This is only inserted into the PDF information catalog. Can be left
% out. 



% If you have a file called "university-logo-filename.xxx", where xxx
% is a graphic format that can be processed by latex or pdflatex,
% resp., then you can add a logo as follows:

% \pgfdeclareimage[height=0.5cm]{university-logo}{university-logo-filename}
% \logo{\pgfuseimage{university-logo}}


% If you wish to uncover everything in a step-wise fashion, uncomment
% the following command: 

%\beamerdefaultoverlayspecification{<+->}


\begin{document}

\begin{frame}
  \titlepage
\end{frame}

%\begin{frame}{Outline}
 %   \tableofcontents
 %\end{frame}


% Structuring a talk is a difficult task and the following structure
% may not be suitable. Here are some rules that apply for this
% solution: 

% - Exactly two or three sections (other than the summary).
% - At *most* three subsections per section.
% - Talk about 30s to 2min per frame. So there should be between about
%   15 and 30 frames, all told.

% - A conference audience is likely to know very little of what you
%   are going to talk about. So *simplify*!
% - In a 20min talk, getting the main ideas across is hard
%   enough. Leave out details, even if it means being less precise than
%   you think necessary.
% - If you omit details that are vital to the proof/implementation,
%   just say so once. Everybody will be happy with that.

%\begin{overpic}[width=.85\textwidth, trim=0 0 0 0, clip]{img/ReflectionTemplates_DPt_NoJet_DoubleGaus_1010}
%\put(0,61){{\tiny No jet requirement}}
%\put(60,61){{\tiny \textcolor{ForestGreen}{\textbf{Used for QM17 preliminary}}}}
%\end{overpic}

\section{Average background density}

\begin{frame}{Distribution of the background density $\rho$}
\begin{columns}
\column{0.6\textwidth}
\begin{overpic}[width=1.15\textwidth, trim=0 0 0 0, clip]{img/RhoDistributionDetLev}
\put(35,39){{\scriptsize 300M minimum-bias events}}
\put(35,35){{\scriptsize \pp\ at $\s=7$~TeV (2010, pass4)}}
\end{overpic}
The transverse plane method has a slightly smaller $\left<\rho\right>$, but longer tail \\(3-jet events? misidentified leading jet?)
\column{0.4\textwidth}
\begin{itemize}
\item \small \textbf{CMS method}: median of \kt\ cluster momentum density with event occupancy correction factor\\
{{\tiny \href{https://doi.org/10.1007/JHEP08(2012)130}{JHEP08(2012)130}, \href{https://doi.org/10.1007/JHEP04(2010)065}{JHEP04(2010)065}}} \\
\tiny$\rho_{\rm CMS} = \underset{j\in{\rm physical jets}}{\rm median}\left\{\frac{p_{{\rm T}j}}{A_j}\right\}\cdot C$ \\
\tiny$\left<C\right>\approx0.3$
\item \small \textcolor{BrickRed}{\textbf{Trans plane}}: momentum density in the $\phi$ directions perpendicular to the leading jet \\
{\tiny$\rho_{\rm trans} = \frac{1}{Acc} \underset{\rm perp.tracks}{\sum}{p_{\rm T,track}}$,}\\
{\tiny where perp. tracks are such that $67.5^{\circ} < \phi_{\rm track} - \phi_{\rm lead.jet}<112.5^{\circ}$}\\
{\tiny $Acc = (\eta_{\rm max} - \eta_{\rm min}) \frac{\pi}{2}$}
\end{itemize}
\end{columns}
\end{frame}

\begin{frame}{Average background density $\rho$ vs. multiplicity}
\begin{center}
\vspace{-15pt}
\begin{overpic}[width=.80\textwidth, trim=10 0 0 35, clip]{img/MeanRhoVsCentDetLev}
\put(27,43){{\scriptsize 300M minimum-bias events}}
\put(27,39){{\scriptsize \pp\ at $\s=7$~TeV (2010, pass4)}}
\put(27,35){{\scriptsize Multiplicty classes calculated using V0 detectors}}
\end{overpic}
\end{center}
\vspace{-15pt}
\footnotesize
The average background density is quite flat vs. event activity classes (around 400 MeV/c), except for the 30\% of the events with largest event activity, where it can go up to 1 GeV/c.
(These numbers must be multiplied by the jet area, $A\approx0.5$ for $R=0.4$).
\end{frame}

\begin{frame}{Average background density $\rho$ vs. leading jet/track \pt}
\begin{columns}
\column{0.5\textwidth}
\begin{overpic}[width=\textwidth, trim=10 0 0 35, clip]{img/MeanRhoVsLeadJetPtDetLev}
\put(27,43){{\scriptsize $\left<\rho\right>$ vs. $p_{\rm T,lead.jet}$}}
\end{overpic}
\column{0.5\textwidth}
\begin{overpic}[width=\textwidth, trim=10 0 0 35, clip]{img/MeanRhoVsLeadTrackPtDetLev}
\put(27,43){{\scriptsize $\left<\rho\right>$ vs. $p_{\rm T,lead.track}$}}
\end{overpic}
\end{columns}
\footnotesize
\begin{itemize}
\item It is generally assumed that the underlying event is independent of the hard process $\rightarrow$ only true up to a certain point
\item Events with a jet $\pt>10$~\GeVc\ have $\left<\rho\right> > 1$~\GeVc\ (compared to $\left<\rho\right> \approx 0.45$~\GeVc\ in minimum-bias events)
\item For leading track/jet $\pt>10$~\GeVc\ the dependence of the UE on the hard process scale is weak
\end{itemize}
\end{frame}

\begin{frame}{Comparison with PYTHIA6}
\begin{columns}
\column{0.5\textwidth}
\begin{overpic}[width=\textwidth, trim=10 0 0 35, clip]{img/LHC10_Train981_LHC14j4_Train1167RhoDistr_Mean_RhoDev_Rho_Signal}
\put(27,43){{\scriptsize CMS Method}}
\end{overpic}
\column{0.5\textwidth}
\begin{overpic}[width=\textwidth, trim=10 0 0 35, clip]{img/LHC10_Train981_LHC14j4_Train1167RhoDistr_Mean_RhoTransDev_RhoTrans_Signal}
\put(27,43){{\scriptsize Trans Plane}}
\end{overpic}
\end{columns}
\footnotesize
\begin{itemize}
\item Very good agreement between data and \textcolor{NavyBlue}{PYTHIA6+GEANT3 (detector level)}
\item At \textcolor{BrickRed}{PYTHIA6 particle level} $\rho_{\rm CMS}$ (left plot) just slightly larger than detector level (particle level includes all charged particles down to zero momentum)
\item Not understood why with the trans plane method $\rho_{\rm trans}$ (right plot) is significantly larger at particle level
\end{itemize}
\end{frame}

\section{Background fluctuations}

\begin{frame}{Random Cones}
\begin{columns}
\column{0.45\textwidth}
\begin{overpic}[width=1.1\textwidth, trim=10 0 0 35, clip]{img/RCDeltaPt_RhoDev_Rho_Signal}
\put(17,35){{\scriptsize CMS Method}}
\end{overpic}
\begin{overpic}[width=1.1\textwidth, trim=10 0 0 35, clip]{img/RCDeltaPt_RhoDev_RhoExclLeadJets_Signal}
\put(17,35){{\scriptsize CMS Method,}}
\put(17,31){{\scriptsize excl. lead. jet}}
\end{overpic}
\column{0.55\textwidth}
\begin{center}
\begin{overpic}[width=0.9\textwidth, trim=10 0 25 35, clip]{img/RCDeltaPt_RhoTransDev_RhoTrans_Signal}
\put(17,35){{\scriptsize Trans Plane}}
\end{overpic}
\end{center}
\vspace{-10pt}
\footnotesize
\begin{itemize}
\item The standard deviation is $0.4-0.5$~\GeVc, \textbf{same magnitude as $\left<\rho\right>$}
\item Similarly to $\left<\rho\right>$, dependence on hard process (not shown here, fluctuations are larger if a jet with $\pt>10$~\GeVc\ is required)
\end{itemize}
\end{columns}
\end{frame}

\section{\Dzero-Jets}

\begin{frame}{Background Subtraction (using the CMS method)}
\begin{columns}
\column{0.45\textwidth}
\begin{overpic}[width=1.1\textwidth, trim=0 0 0 35, clip]{img/AnyINT_D0_Charged_R040_JetPtSpectrum_DPt_30_SideBand_Normalized_canvas}
\put(17,15){{\scriptsize Unsubtracted}}
\end{overpic}
\begin{overpic}[width=1.1\textwidth, trim=0 0 0 35, clip]{img/AnyINT_D0_Charged_R040_JetCorrPtSpectrum_DPt_30_SideBand_Normalized_canvas}
\put(17,19){{\scriptsize Subtracted}}
\put(17,15){{\scriptsize CMS method}}
\end{overpic}
\column{0.55\textwidth}
\begin{center}
\begin{overpic}[width=0.6\textwidth, trim=10 0 25 35, clip]{img/RatioCorrOverUncorr}
\put(17,19){{\scriptsize Ratio $\approx0.75$}}
\end{overpic}\\
\vspace{5pt}
\tiny Shown below is the distribution of the \pt\ subtracted $=\ptchjet^{\rm raw}-\ptchjet^{\rm sub}$
\begin{overpic}[width=0.8\textwidth, trim=0 0 0 35, clip]{img/AnyINT_D0_Charged_R040_JetBkgPtSpectrum_DPt_30_SideBand_Normalized_canvas}
\end{overpic}
\end{center}
\end{columns}
\end{frame}

\begin{frame}{Summary}
\begin{itemize}
\item The underlying event in \pp\ collisions has been characterised using the traditional transverse plane method (see e.g. [1], [2]) and the modern jet-area-based method proposed by Cacciari et al. [3] and improved by the CMS collaboration for sparse environments [4]
\item In minimum-bias \pp\ collisions at $\s=7$~TeV, $\left<\rho\right>\approx0.4-0.5$~\GeVc, largely independent of the technique used to estimate it
\item Events that contain a hard process have $\left<\rho\right>\approx1.5$~\GeVc, weakly depending on the hard scale and with some variations depending on the technique used to estimate it
(different techniques have different biases from the presence of the hard process)
\end{itemize}
{\tiny
{[1]} \href{https://aliceinfo.cern.ch/Notes/sites/aliceinfo.cern.ch.Notes/files/notes/analysis/rma/2013-Mar-29-analysis_note-ppJet_note.pdf}{Analysis note: Inclusive jet spectrum in \pp\ collisions at $\s=2.76$~TeV with ALICE} \\
{[2]} \href{https://doi.org/10.1016/j.physletb.2013.04.026}{Phys. Lett. B 722 (2013) 262-272} \\
{[3]} \href{https://doi.org/10.1007/JHEP04(2010)065}{JHEP04(2010)065}\\
{[4]} \href{https://doi.org/10.1007/JHEP08(2012)130}{JHEP08(2012)130}
}
\end{frame}

\begin{frame}{Summary (cont'd)}
\begin{itemize}
\item The HEP community \textbf{never subtracts} the UE in jet measurements (to my knowledge) $\rightarrow$ the UE must be added in the theoretical models
\item Usually this is done by default when comparing to Monte Carlo generators
\item pQCD calculations need to explicitly include it (but in any case pQCD calculations need also external inputs for hadronisation)
\item In the HI community we want to compare \pp\ results with \PbPb, where UE subtraction is obviously unavoidable
\item My proposal would be to publish both subtracted and unsubtracted
\item For the subtracted case I would propose to do an average subtraction at particle level (\textbf{not event-by-event}) 
because fluctuations are as large as the amount of the correction itself (so a jet-by-jet subtraction is quite meaningless)
\item This approach has been used in [1], [2]
\end{itemize}
\end{frame}


\section{Extra Slides}

\begin{frame}{Average background density $\rho$ vs. leading D-jet candidate}
\begin{columns}
\column{0.5\textwidth}
\begin{overpic}[width=\textwidth, trim=10 0 0 35, clip]{img/DmesonJets_AnyINT_histosDmesonJets_AnyINT_D0_Jet_AKTChargedR040_pt_scheme_RhoVsLeadJetPt_Profile}
\put(27,23){{\scriptsize $\left<\rho\right>$ vs. $p_{\rm T,cand.D-jet}^{\rm lead}$}}
\end{overpic}
\column{0.5\textwidth}
\begin{overpic}[width=\textwidth, trim=10 0 0 35, clip]{img/DmesonJets_AnyINT_histosDmesonJets_AnyINT_D0_Jet_AKTChargedR040_pt_scheme_RhoVsLeadDPt_Profile}
\put(27,23){{\scriptsize $\left<\rho\right>$ vs. $p_{\rm T,cand.D}^{\rm lead}$}}
\end{overpic}
\end{columns}
\end{frame}

\end{document}
