% $Header: /Users/joseph/Documents/LaTeX/beamer/solutions/conference-talks/conference-ornate-20min.en.tex,v 90e850259b8b 2007/01/28 20:48:30 tantau $

\documentclass[xcolor={usenames,dvipsnames}]{beamer}

% This file is a solution template for:

% - Talk at a conference/colloquium.
% - Talk length is about 20min.
% - Style is ornate.



% Copyright 2004 by Till Tantau <tantau@users.sourceforge.net>.
%
% In principle, this file can be redistributed and/or modified under
% the terms of the GNU Public License, version 2.
%
% However, this file is supposed to be a template to be modified
% for your own needs. For this reason, if you use this file as a
% template and not specifically distribute it as part of a another
% package/program, I grant the extra permission to freely copy and
% modify this file as you see fit and even to delete this copyright
% notice. 


\mode<presentation>
{
  \usetheme{AnnArbor}
  % or ...

  \setbeamercovered{transparent}
  % or whatever (possibly just delete it)
 }

\usepackage[percent]{overpic}
\usepackage[english]{babel}
\usepackage{setspace}
\usepackage{appendixnumberbeamer}
% or whatever

\usepackage[latin1]{inputenc}
% or whatever

\usepackage{times}
\usepackage[T1]{fontenc}
% Or whatever. Note that the encoding and the font should match. If T1
% does not look nice, try deleting the line with the fontenc.
%particles
\newcommand{\jpsi}{\rm J/$\psi$}
\newcommand{\psip}{$\psi^\prime$}
\newcommand{\jpsiDY}{\rm J/$\psi$\,/\,DY}
\newcommand{\chic}{$\chi_{\rm c}$}
\newcommand{\pip}{$\pi^{+}$}
\newcommand{\pim}{$\pi^{-}$}
\newcommand{\pizero}{$\pi^{0}$}
\newcommand{\kap}{K$^{+}$}
\newcommand{\kam}{K$^{-}$}
\newcommand{\pbar}{$\rm\overline{p}$}
\newcommand{\ccbar}{\ensuremath{\mathrm{c\overline{c}}}}
\newcommand{\bbbar}{\ensuremath{\mathrm{b\overline{b}}}}
\newcommand{\Dzero}{\ensuremath{\mathrm{D^{0}}}}
\newcommand{\Dzerobar}{\ensuremath{\mathrm{\overline{D}^{0}}}}
\newcommand{\Dpm}{\ensuremath{\mathrm{D^{\pm}}}}
\newcommand{\Ds}{\ensuremath{\mathrm{D_{s}^{\pm}}}}
\newcommand{\Dstar}{\ensuremath{\mathrm{D^{*\pm}}}}

%collision systems
\newcommand{\pp}{pp}
\newcommand{\pPb}{p--Pb}
\newcommand{\PbPb}{Pb--Pb}

%detectors
\newcommand{\ezdc}{$E_{\rm ZDC}$}

%units
\newcommand{\GeVc}{GeV/$c$}
\newcommand{\GeVcsq}{GeV/$c^2$}

%others
\newcommand{\degree}{$^{\rm o}$}
\newcommand{\s}{\ensuremath{\sqrt{s}}}
\newcommand{\snn}{\ensuremath{\sqrt{s_{\rm NN}}}}
\newcommand{\y}{\ensuremath{y}}
\newcommand{\pt}{\ensuremath{p_{\rm T}}}
\newcommand{\dedx}{d$E$/d$x$}
\newcommand{\dndy}{d$N$/d$y$}
\newcommand{\dndydpt}{${\rm d}^2N/({\rm d}y {\rm d}p_{\rm t})$}
\newcommand{\zpar}{\ensuremath{z_{||}}}
\newcommand{\zpargen}{\ensuremath{z_{||}^{\mathrm{part}}}}
\newcommand{\zpardet}{\ensuremath{z_{||}^{\mathrm{det}}}}
\newcommand{\ptchjet}{\ensuremath{p_{\mathrm{T,ch\, jet}}}}
\newcommand{\ptjet}{\ensuremath{p_{\mathrm{T,jet}}}}
\newcommand{\ptchjetgen}{\ensuremath{p_{\mathrm{T,ch\,jet}}^{\mathrm{part}}}}
\newcommand{\ptchjetdet}{\ensuremath{p_{\mathrm{T,ch\,jet}}^{\mathrm{det}}}}
\newcommand{\ptd}{\ensuremath{p_{\mathrm{T,D}}}}
\newcommand{\ptdgen}{\ensuremath{p_{\mathrm{T,D}}^{\mathrm{part}}}}
\newcommand{\ptddet}{\ensuremath{p_{\mathrm{T,D}}^{\mathrm{det}}}}
\newcommand{\antikt}{anti-\ensuremath{k_{\mathrm{T}}}}
\newcommand{\Antikt}{Anti-\ensuremath{k_{\mathrm{T}}}}
\newcommand{\kt}{\ensuremath{k_{\mathrm{T}}}}
\newcommand{\pthard}{\ensuremath{p_{\mathrm{T,hard}}}}

\AtBeginSection[]{
  \begin{frame}
  \vfill
  \centering
  \begin{beamercolorbox}[sep=8pt,center,shadow=true,rounded=true]{title}
    \usebeamerfont{title}\insertsectionhead\par%
  \end{beamercolorbox}
  \vfill
  \end{frame}
}

\title[D-Tagged Jets in \pp\ and \PbPb] % (optional, use only with long paper titles)
{D-Tagged Jets in \pp\ and \PbPb\ Collisions}

\author[Salvatore Aiola]% (optional, use only with lots of authors)
{Salvatore Aiola \\ for PAG-HFCJ}
% - Give the names in the same order as the appear in the paper.
% - Use the \inst{?} command only if the authors have different
%   affiliation.

\institute[Yale University] % (optional, but mostly needed)
{Yale University}

\date[PWG-HF - Sept. 26th, 2017] % (optional, should be abbreviation of conference name)
{PWG-HF \\
September 26th, 2017}
% - Either use conference name or its abbreviation.
% - Not really informative to the audience, more for people (including
%   yourself) who are reading the slides online

\subject{High-Energy Physics}
% This is only inserted into the PDF information catalog. Can be left
% out. 



% If you have a file called "university-logo-filename.xxx", where xxx
% is a graphic format that can be processed by latex or pdflatex,
% resp., then you can add a logo as follows:

% \pgfdeclareimage[height=0.5cm]{university-logo}{university-logo-filename}
% \logo{\pgfuseimage{university-logo}}


% If you wish to uncover everything in a step-wise fashion, uncomment
% the following command: 

%\beamerdefaultoverlayspecification{<+->}


\begin{document}

\begin{frame}
  \titlepage
\end{frame}

\begin{frame}{Outline}
   \tableofcontents
\end{frame}


% Structuring a talk is a difficult task and the following structure
% may not be suitable. Here are some rules that apply for this
% solution: 

% - Exactly two or three sections (other than the summary).
% - At *most* three subsections per section.
% - Talk about 30s to 2min per frame. So there should be between about
%   15 and 30 frames, all told.

% - A conference audience is likely to know very little of what you
%   are going to talk about. So *simplify*!
% - In a 20min talk, getting the main ideas across is hard
%   enough. Leave out details, even if it means being less precise than
%   you think necessary.
% - If you omit details that are vital to the proof/implementation,
%   just say so once. Everybody will be happy with that.

%\begin{overpic}[width=.85\textwidth, trim=0 0 0 0, clip]{img/ReflectionTemplates_DPt_NoJet_DoubleGaus_1010}
%\put(0,61){{\tiny No jet requirement}}
%\put(60,61){{\tiny \textcolor{ForestGreen}{\textbf{Used for QM17 preliminary}}}}
%\end{overpic}

%\begin{columns}
%\column{0.5\textwidth}
%\column{0.5\textwidth}
%\end{columns}

\section{\Dzero-Jet \pt-differential cross section in \pp\ @ $\s=7$~TeV}

\subsection{Reminder: preliminary at QM17}
\begin{frame}{QM17 Preliminary}
\begin{columns}
\column{0.45\textwidth}
\begin{center}
\begin{overpic}[width=1.1\textwidth, trim=0 0 0 0, clip]{img/2017-Jun-12-D0JetCrossSection_pp7TeV}
\end{overpic}
\end{center}
\vspace{-5pt}
\scriptsize
\column{0.55\textwidth}
\small
\begin{itemize}
\item \Dzero\ candidates identified in the ${\rm K}\pi$ decay channel
\item Jets reconstructed out of the \Dzero-candidate 4-momentum + all remaining tracks
\item Invariant mass analysis: in jet \pt\ bins or in D \pt\ bins (side band method)
\item MB \pp\ @ 7 TeV, 2010 (LHC10b,c,d,e)
\item Preliminary shown at QM17 (\href{https://indico.cern.ch/event/433345/contributions/2358064/}{poster})
\item Good agreement with POWHEG within uncertainties
\end{itemize}

\end{columns}
\end{frame}

\subsection{Reflections}

\begin{frame}{Issue with reflection templates: fixed}
\begin{center}
\begin{overpic}[width=.6\textwidth, trim=0 0 0 0, clip]{img/fixed_reflections}
\end{overpic}\\
Fixed small issue in \textbf{reflections}: PID response task was configured for pass2 instead of pass4\\
1-7\% effect on the final spectrum (well within uncertainties)
\end{center}
\end{frame}

\subsection{pass2 vs. pass4 cuts}

\begin{frame}{Comparison of prompt/non-prompt efficiencies}
\begin{columns}
\column{0.58\textwidth}
\centering
\begin{overpic}[width=\textwidth, trim=0 0 0 0, clip]{img/CompareEfficiencyPass4vsPass2cuts}
\end{overpic}
{
\tiny
\begin{tabular}{ll}
\textcolor{black}{LHC15i2\_Train961\_cresponse} & prompt, pass2 cuts\\
\textcolor{NavyBlue}{LHC15i2\_Train1073\_cresponse} & prompt, pass4 cuts\\
\textcolor{BrickRed}{LHC15i2\_Train973\_bresponse} & non-prompt, pass2 cuts\\
\textcolor{ForestGreen}{LHC15i2\_Train1081\_bresponse} & non-prompt, pass4 cuts
\end{tabular}
}
\begin{spacing}{1.5}

\end{spacing}
\raggedright
{\tiny
\begin{spacing}{0.5}
\textbf{Note}: the efficiencies are shown also for $\ptd<3$~\GeVc\ but only D mesons with $\ptd>3$~\GeVc\ are used in the analysis.
In this kinematic region the ratios of the prompt efficiencies of pass4 over pass2 is roughly unity or higher by $10$\%;
the non-prompt efficiency is about 20\% lower for the pass4 cuts compared to the pass2 cuts.
\end{spacing}
}
\column{0.42\textwidth}
\begin{overpic}[width=\textwidth, trim=0 0 0 0, clip]{img/ComparePromptEfficiencyPass4vsPass2cuts_Ratio}
\put(30,10){\footnotesize Prompt}
\put(18,7){\color{red}\line(0,1){53}}
\end{overpic}
\begin{overpic}[width=\textwidth, trim=0 0 0 0, clip]{img/CompareNonPromptEfficiencyPass4vsPass2cuts_Ratio}
\put(30,10){\footnotesize Non-Prompt}
\put(18,7){\color{red}\line(0,1){53}}
\end{overpic}
\end{columns}
\end{frame}

\begin{frame}{Comparison of efficiency-corrected raw yields}
\begin{columns}
\column{0.5\textwidth}
\begin{overpic}[width=\textwidth, trim=0 0 0 0, clip]{img/ComparePass4vsPass2cutsSideBandEfficiency}
\end{overpic}
\column{0.5\textwidth}
\begin{overpic}[width=\textwidth, trim=0 0 0 0, clip]{img/ComparePass4vsPass2cutsSideBandEfficiency_Ratio}
\end{overpic}
\end{columns}
{\footnotesize
\begin{description}
\item[\textcolor{black}{LHC10\_Train823\_with1010}] pass2 cuts
\item[\textcolor{NavyBlue}{LHC10\_Train883}] pass4 cuts [PRC 94, 054908 (2016)]
\end{description}}
\end{frame}

\begin{frame}{Comparison of raw yields with FD subtraction}
\begin{columns}
\column{0.5\textwidth}
\begin{overpic}[width=\textwidth, trim=0 0 0 0, clip]{img/ComparePass4vsPass2cutsSideBandEfficiencyFDCorrected}
\end{overpic}
\column{0.5\textwidth}
\begin{overpic}[width=\textwidth, trim=0 0 0 0, clip]{img/ComparePass4vsPass2cutsSideBandEfficiencyFDCorrected_Ratio}
\end{overpic}
\end{columns}
{\footnotesize
\begin{description}
\item[\textcolor{black}{LHC10\_Train823\_with1010}] pass2 cuts
\item[\textcolor{NavyBlue}{LHC10\_Train883}] pass4 cuts [PRC 94, 054908 (2016)]
\end{description}}
\end{frame}

\section{Underlying Event in \pp}

\subsection{Average background density}

\begin{frame}{Distribution of the background density $\rho$}
\begin{columns}
\column{0.6\textwidth}
\begin{overpic}[width=1.15\textwidth, trim=0 0 0 0, clip]{img/RhoDistributionDetLev}
\put(35,39){{\scriptsize 300M minimum-bias events}}
\put(35,35){{\scriptsize \pp\ at $\s=7$~TeV (2010, pass4)}}
\end{overpic}
The transverse plane method has a slightly smaller $\left<\rho\right>$, but longer tail \\ (3-jet events? misidentified leading jet?)
\column{0.4\textwidth}
\begin{itemize}
\item \small \textcolor{BrickRed}{\textbf{Trans plane}}: momentum density in the $\phi$ directions perpendicular to the leading jet \\
{\tiny$\rho_{\rm trans} = \frac{1}{Acc} \underset{\rm perp.tracks}{\sum}{p_{\rm T,track}}$,}\\
{\tiny where perp. tracks are such that $67.5^{\circ} < \phi_{\rm track} - \phi_{\rm lead.jet}<112.5^{\circ}$}\\
{\tiny $Acc = (\eta_{\rm max} - \eta_{\rm min}) \frac{\pi}{2}$}
\item \small \textbf{CMS method}: median of \kt\ cluster momentum density with event occupancy correction factor\\
{{\tiny \href{https://doi.org/10.1007/JHEP08(2012)130}{JHEP08(2012)130}, \href{https://doi.org/10.1007/JHEP04(2010)065}{JHEP04(2010)065}}} \\
\tiny$\rho_{\rm CMS} = \underset{j\in{\rm physical jets}}{\rm median}\left\{\frac{p_{{\rm T}j}}{A_j}\right\}\cdot C$ \\
\tiny$\left<C\right>\approx0.3$
\end{itemize}
\end{columns}
\end{frame}

\begin{frame}{Average background density $\rho$ vs. multiplicity}
\begin{center}
\vspace{-15pt}
\begin{overpic}[width=.80\textwidth, trim=10 0 0 35, clip]{img/MeanRhoVsCentDetLev}
\put(27,43){{\scriptsize 300M minimum-bias events}}
\put(27,39){{\scriptsize \pp\ at $\s=7$~TeV (2010, pass4)}}
\put(27,35){{\scriptsize Multiplicty classes calculated using V0 detectors}}
\end{overpic}
\end{center}
\vspace{-15pt}
\footnotesize
The average background density is quite flat vs. event activity classes (around 400 MeV/c), except for the 30\% of the events with largest event activity, where it can go up to 1 GeV/c.
(These numbers must be multiplied by the jet area, $A\approx0.5$ for $R=0.4$).
\end{frame}

\begin{frame}{Average background density $\rho$ vs. leading jet/track \pt}
\begin{columns}
\column{0.5\textwidth}
\begin{overpic}[width=\textwidth, trim=10 0 0 35, clip]{img/MeanRhoVsLeadJetPtDetLev}
\put(27,43){{\scriptsize $\left<\rho\right>$ vs. $p_{\rm T,lead.jet}$}}
\end{overpic}
\column{0.5\textwidth}
\begin{overpic}[width=\textwidth, trim=10 0 0 35, clip]{img/MeanRhoVsLeadTrackPtDetLev}
\put(27,43){{\scriptsize $\left<\rho\right>$ vs. $p_{\rm T,lead.track}$}}
\end{overpic}
\end{columns}
\footnotesize
\begin{itemize}
\item It is generally assumed that the underlying event is independent of the hard process $\rightarrow$ only true up to a certain point
\item Events with a jet $\pt>10$~\GeVc\ have $\left<\rho\right> > 1$~\GeVc\ (compared to $\left<\rho\right> \approx 0.45$~\GeVc\ in minimum-bias events)
\item For leading track/jet $\pt>10$~\GeVc\ the dependence of the UE on the hard process scale is weak
\end{itemize}
\end{frame}

\subsection{Background vs. \Dzero-Jets}

\begin{frame}{Average background density $\rho$ vs. leading D-jet candidate}
\begin{columns}
\column{0.5\textwidth}
\begin{overpic}[width=\textwidth, trim=10 0 0 35, clip]{img/DmesonJets_AnyINT_histosDmesonJets_AnyINT_D0_Jet_AKTChargedR040_pt_scheme_RhoVsLeadJetPt_Profile}
\put(27,23){{\scriptsize $\left<\rho\right>$ vs. $p_{\rm T,cand.D-jet}^{\rm lead}$}}
\end{overpic}
\column{0.5\textwidth}
\begin{overpic}[width=\textwidth, trim=10 0 0 35, clip]{img/DmesonJets_AnyINT_histosDmesonJets_AnyINT_D0_Jet_AKTChargedR040_pt_scheme_RhoVsLeadDPt_Profile}
\put(27,23){{\scriptsize $\left<\rho\right>$ vs. $p_{\rm T,cand.D}^{\rm lead}$}}
\end{overpic}
\end{columns}
\end{frame}


\section{\Dzero-Jet Fragmentation Function in \pp}

\subsection{Raw Yield Extraction}

\begin{frame}{Invariant Mass Fits in \zpar\ bins}
\begin{center}
\begin{overpic}[width=.8\textwidth, trim=0 0 0 0, clip]{img/AnyINT_D0_Charged_R040_JetZBins_DPt_30_JetPt_5_30}
\put(70,20){\tiny $\ptd > 3$~\GeVc}
\put(70,16){\tiny $5 < \ptchjet < 30$~\GeVc}
\end{overpic}
\end{center}
\vspace{-5pt}
\small
\begin{itemize}
\item No signal in $0 < \zpar < 0.2$ (as expected due to the kinematic cuts)
\item Good S/B in all bins
\end{itemize}
\end{frame}

\begin{frame}{Side Band Method}
\begin{center}
\begin{overpic}[width=.8\textwidth, trim=0 0 0 0, clip]{img/AnyINT_D0_Charged_R040_DPtBins_JetPt_5_30_SideBand_D0_Charged_R040_JetPtSpectrum_DPt_30_SideBand}
\end{overpic}
\end{center}
\vspace{-5pt}
\small
\begin{itemize}
\item Same as for the jet \pt\ spectrum
\item $\ptd > 3$~\GeVc\ and $5 < \ptchjet < 30$~\GeVc
\end{itemize}
\end{frame}

\begin{frame}{Side Band Method in \zpar\ bins}
\begin{columns}
\column{0.65\textwidth}
\begin{overpic}[width=\textwidth, trim=0 0 0 0, clip]{img/AnyINT_D0_Charged_R040_JetZSpectrum_DPt_30_JetPt_5_30_SideBand_BkgVsSig}
\end{overpic}
\column{0.35\textwidth}
\begin{overpic}[width=\textwidth, trim=0 0 0 0, clip]{img/AnyINT_D0_Charged_R040_JetZSpectrum_DPt_30_JetPt_5_30_SideBand_TotalBkgVsSig}
\end{overpic}
\vspace{-5pt}
\begin{itemize}
\item Subtraction of the \zpar\ spectra in bins of \ptd
\end{itemize}
\end{columns}
\end{frame}

\begin{frame}{Method Comparison}
\begin{columns}
\column{0.5\textwidth}
\begin{overpic}[width=\textwidth, trim=0 0 0 0, clip]{img/AnyINT_D0_Charged_R040_d_z_SpectraComparison}
\end{overpic}
\column{0.5\textwidth}
\begin{overpic}[width=\textwidth, trim=0 0 0 0, clip]{img/AnyINT_D0_Charged_R040_d_z_SpectraComparison_Ratio}
\end{overpic}
\end{columns}
\vspace{-5pt}
\begin{itemize}
\item The two methods agree quite well except in the bin $0.4<\zpar<0.6$
\item Invariant mass for the bin $0.4<\zpar<0.6$ has a larger width (see backup slide)
\end{itemize}
\end{frame}

\begin{frame}{Optimisation of the kinematic cuts}
\begin{columns}
\column{0.6\textwidth}
\begin{overpic}[width=\textwidth, trim=0 0 0 0, clip]{img/z_powheg_charged}
\end{overpic}
\column{0.4\textwidth}
\small
\begin{center}
The kinematic cuts $\ptd>3$~\GeVc\ and $5<\ptchjet<30$~\GeVc\ alter the shape of the \zpar\ distribution
\end{center}
\end{columns}
\small
\begin{itemize}
\item Unbiased FF down to $\zpar=0.2$ for jets down to $\ptchjet=5$~\GeVc\ needs $\ptd>1$~\GeVc
\item We could also try to measure the FF for two bins of \ptchjet: $5-15$ and $15-30$~\GeVc
\end{itemize}
\end{frame}

\begin{frame}{$\ptd>1$~\GeVc\ and $5<\ptchjet<15$~\GeVc}
\begin{center}
\begin{overpic}[width=.7\textwidth, trim=0 0 0 0, clip]{img/AnyINT_D0_Charged_R040_DPtBins_JetPt_5_15_SideBand_D0_Charged_R040_JetZSpectrum_DPt_10_JetPt_5_15_SideBand}
\end{overpic}
\end{center}
\vspace{-10pt}
\small
The bin $2-3$~\GeVc\ could be included, also in the jet \pt\ spectrum \\
The bin $1-2$~\GeVc\ is probably out of reach with the present techniques, but could try different approach as in PRC 94, 054908 (2016)
\end{frame}

\begin{frame}{$\ptd>3$~\GeVc\ and $15<\ptchjet<30$~\GeVc}
\begin{columns}
\column{0.65\textwidth}
\begin{overpic}[width=\textwidth, trim=0 0 0 0, clip]{img/AnyINT_D0_Charged_R040_DPtBins_JetPt_15_30_SideBand_D0_Charged_R040_JetZSpectrum_DPt_30_JetPt_15_30_SideBand}
\end{overpic}
\column{0.35\textwidth}
\begin{itemize}
\item Not looking promising, but could try to \alert{optimise topological cuts} to gain statistics
\item \alert{Presence of a jet} may decrease the combinatorial background (even with weaker topological cuts)
\end{itemize}
\end{columns}
\end{frame}

\subsection{Systematic Uncertainty: raw yield and B feed-down}

\begin{frame}{Raw Yield Extraction}
\begin{columns}
\column{0.5\textwidth}
\begin{overpic}[width=\textwidth, trim=0 0 0 0, clip]{img/CompareRawYieldUncVariations_AfterDbinSum}
\end{overpic}
\column{0.5\textwidth}
\begin{overpic}[width=\textwidth, trim=0 0 0 0, clip]{img/CompareRawYieldUncVariations_AfterDbinSum_Ratio}
\end{overpic}
\end{columns}
\vspace{-5pt}
\begin{itemize}
\item Code updated to work with \zpar
\item Same strategy used for the jet \pt\ spectrum
\end{itemize}
\end{frame}

\begin{frame}{Reflection Template Variations}
\begin{columns}
\column{0.5\textwidth}
\begin{overpic}[width=\textwidth, trim=0 0 0 0, clip]{img/SideBandReflectionVariationComparison}
\end{overpic}
\column{0.5\textwidth}
\begin{overpic}[width=\textwidth, trim=0 0 0 0, clip]{img/SideBandReflectionVariationComparison_Ratio}
\end{overpic}
\end{columns}
\vspace{-5pt}
\begin{itemize}
\item Code updated to work with \zpar
\item Same strategy used for the jet \pt\ spectrum
\end{itemize}
\end{frame}

\begin{frame}{Simulation Variations (Generator Level)}
\begin{columns}
\column{0.5\textwidth}
\begin{overpic}[width=\textwidth, trim=0 0 0 0, clip]{img/BFeedDown_1505317519_1242_JetZSpectrum_DPt_30_JetPt_5_30_GeneratorLevel_JetZSpectrum}
\end{overpic}
\column{0.5\textwidth}
\begin{overpic}[width=\textwidth, trim=0 0 0 0, clip]{img/BFeedDown_1505317519_1242_JetZSpectrum_DPt_30_JetPt_5_30_GeneratorLevel_JetZSpectrum_Ratio}
\end{overpic}
\end{columns}
\vspace{-5pt}
\begin{itemize}
\item Code updated to work with \zpar
\item Same strategy used for the jet \pt\ spectrum
\end{itemize}
\end{frame}

\begin{frame}{Simulation Variations (Detector Level)}
\begin{columns}
\column{0.5\textwidth}
\begin{overpic}[width=\textwidth, trim=0 0 0 0, clip]{img/BFeedDown_1505317519_1242_JetZSpectrum_DPt_30_JetPt_5_30_DetectorLevel_JetZSpectrum_bEfficiencyMultiply_cEfficiencyDivide}
\end{overpic}
\column{0.5\textwidth}
\begin{overpic}[width=\textwidth, trim=0 0 0 0, clip]{img/BFeedDown_1505317519_1242_JetZSpectrum_DPt_30_JetPt_5_30_DetectorLevel_JetZSpectrum_bEfficiencyMultiply_cEfficiencyDivide_Ratio}
\end{overpic}
\end{columns}
\vspace{-5pt}
\begin{itemize}
\item Code updated to work with \zpar
\item Same strategy used for the jet \pt\ spectrum
\item It includes ratio of prompt/non-prompt efficiency and detector resolution (non-prompt)
\end{itemize}
\end{frame}

\begin{frame}{Feed-Down Cross-Section with Systematics}
\begin{columns}
\column{0.5\textwidth}
\begin{overpic}[width=\textwidth, trim=0 0 0 0, clip]{img/BFeedDown_1505317519_1242_JetZSpectrum_DPt_30_JetPt_5_30_GeneratorLevel_JetZSpectrum_canvas}
\end{overpic}
\column{0.5\textwidth}
\begin{overpic}[width=\textwidth, trim=0 0 0 0, clip]{img/BFeedDown_1505317519_1242_JetZSpectrum_DPt_30_JetPt_5_30_DetectorLevel_JetZSpectrum_bEfficiencyMultiply_cEfficiencyDivide_canvas}
\end{overpic}
\end{columns}
\vspace{-5pt}
\begin{itemize}
\item Left: generator level
\item Right: detector level (divided by prompt efficiency)
\end{itemize}
\end{frame}

\section{\PbPb\ @ $\snn=5.02$~TeV: status}

\begin{frame}{Efficiency-corrected yields}
\small
\begin{itemize}
\item Much better significance in Run-2 compared to Run-1
\item Larger luminosity, larger cross section, better ITS performance
\end{itemize}
\vspace{-5pt}
\begin{columns}
\column{0.4\textwidth}
\begin{center}
\textbf{Run-1}\\
\begin{overpic}[width=.8\textwidth, trim=0 0 0 0, clip]{img/pbpb_invmass_run1}
\end{overpic}
\end{center}
\column{0.6\textwidth}
\begin{center}
\textbf{Run-2}\\
\begin{overpic}[width=.8\textwidth, trim=0 0 0 0, clip]{img/pbpb_invmass_run2}
\end{overpic}
\end{center}
\end{columns}
\end{frame}

\begin{frame}{Efficiency-corrected yields}
\small
\begin{itemize}
\item Direct extraction in jet \pt\ bins
\item Significant improvement in statistics compared to Run-1
\end{itemize}
\vspace{-5pt}
\begin{columns}
\column{0.5\textwidth}
\small
\begin{overpic}[width=\textwidth, trim=0 0 0 0, clip]{img/pbpb_efficiency_corrected_yields}
\end{overpic}
\column{0.5\textwidth}
\begin{overpic}[width=\textwidth, trim=0 0 0 0, clip]{img/pbpb_statistical_unc}
\end{overpic}
\end{columns}
\end{frame}

\begin{frame}{Current status}
\begin{columns}
\column{0.6\textwidth}
\begin{overpic}[width=\textwidth, trim=0 0 0 0, clip]{img/pbpb_unfolded_spectrum}
\end{overpic}
\column{0.4\textwidth}
\begin{itemize}
\item Analysis chain fully developed up to unfolding
\item \textbf{\textcolor{red}{Work in progress!}}
\item Just a snapshot of the current status
\item Still need to check the stability of the corrections (especially unfolding, tracking and D-jet efficiencies)
\end{itemize}
\end{columns}
\end{frame}

\section{Conclusions}

\begin{frame}{Conclusions}
\begin{itemize}
\item Updates since QM17 on \pt-differential cross section in \pp: \textbf{fixed issue with reflections} + use \textbf{new topological cuts} from 2016 PRC paper
\item Underlying event in \pp: after some discussion in PAG-HFCJ and PWG-JE the consensus was \textbf{not to apply} any correction in the \pp\ result (as it is common practice in the HEP community)
\item \textbf{Fragmentation function}: the full analysis chain has been adapted to work with \zpar; now working on optimisation of kinematic/topological cuts
\item \textbf{Paper proposal} in the fall for a paper about \Dzero-jets in \pp\ collisions @ 7 TeV (cross section and FF)
\item \textbf{\PbPb\ collisions}: full analysis chain has been developed, now working on stability checks, systematics
\end{itemize}
\end{frame}

\appendix
\section*{Extra Slides}

\begin{frame}{Average background: data vs. MC}
\begin{columns}
\column{0.5\textwidth}
\begin{overpic}[width=\textwidth, trim=10 0 0 35, clip]{img/LHC10_Train993_LHC14j4_Train1175RhoDistribution_Mean_RhoDev_Rho_Signal}
\put(27,43){{\scriptsize CMS Method}}
\end{overpic}
\column{0.5\textwidth}
\begin{overpic}[width=\textwidth, trim=10 0 0 35, clip]{img/LHC10_Train993_LHC14j4_Train1175RhoDistribution_Mean_RhoTransDev_RhoTrans_Signal}
\put(27,43){{\scriptsize Trans Plane}}
\end{overpic}
\end{columns}
\footnotesize
\begin{itemize}
\item Very good agreement between data and \textcolor{NavyBlue}{PYTHIA6+GEANT3 (detector level)}
\item At \textcolor{BrickRed}{PYTHIA6 particle level} $\rho_{\rm CMS}$ (left plot) just slightly larger than detector level (particle level includes all charged particles down to zero momentum)
\item Not understood why with the trans plane method $\rho_{\rm trans}$ (right plot) is significantly larger at particle level
\end{itemize}
\end{frame}

\begin{frame}{Background fluctuations: data vs. MC}
\begin{columns}
\column{0.5\textwidth}
\begin{overpic}[width=\textwidth, trim=10 0 0 35, clip]{img/LHC10_Train993_LHC14j4_Train1175RCDeltaPtDistribution_Mean_RhoDev_Rho_Signal}
\put(27,43){{\scriptsize RC}}
\end{overpic}
\column{0.5\textwidth}
\begin{overpic}[width=\textwidth, trim=10 0 0 35, clip]{img/LHC10_Train993_LHC14j4_Train1175RCExclLeadJetDeltaPtDistribution_Mean_RhoDev_RhoExclLeadJets_Signal}
\put(27,43){{\scriptsize RC, excl. lead. jets}}
\end{overpic}
\end{columns}
\footnotesize
\begin{itemize}
\item Very good agreement between data and \textcolor{NavyBlue}{PYTHIA6+GEANT3 (detector level)}
\end{itemize}
\end{frame}

\begin{frame}{Background Subtraction (using the CMS method)}
\begin{columns}
\column{0.45\textwidth}
\begin{overpic}[width=1.1\textwidth, trim=0 0 0 35, clip]{img/AnyINT_D0_Charged_R040_JetPtSpectrum_DPt_30_SideBand_Normalized_canvas}
\put(17,15){{\scriptsize Unsubtracted}}
\end{overpic}
\begin{overpic}[width=1.1\textwidth, trim=0 0 0 35, clip]{img/AnyINT_D0_Charged_R040_JetCorrPtSpectrum_DPt_30_SideBand_Normalized_canvas}
\put(17,19){{\scriptsize Subtracted}}
\put(17,15){{\scriptsize CMS method}}
\end{overpic}
\column{0.55\textwidth}
\begin{center}
\begin{overpic}[width=0.45\textwidth, trim=10 0 25 35, clip]{img/RatioCorrOverUncorr}
\put(17,25){{\tiny CMS method}}
\put(17,17){{\tiny ratio $\approx0.75$}}
\end{overpic}\quad
\begin{overpic}[width=0.49\textwidth, trim=10 0 25 35, clip]{img/RatioCorrOverUncorr_exclead}
\put(17,25){{\tiny CMS method, excl. lead. jet}}
\put(17,17){{\tiny ratio $\approx0.85$}}
\end{overpic}
\vspace{5pt}
\tiny Shown below is the distribution of the \pt\ subtracted $=\ptchjet^{\rm raw}-\ptchjet^{\rm sub}$
\begin{overpic}[width=0.8\textwidth, trim=0 0 0 35, clip]{img/AnyINT_D0_Charged_R040_JetBkgPtSpectrum_DPt_30_SideBand_Normalized_canvas}
\end{overpic}
\end{center}
\end{columns}
\end{frame}

\begin{frame}{Occupancy correction factor (CMS method)}
\begin{columns}
\column{0.5\textwidth}
\begin{overpic}[width=\textwidth, trim=0 0 0 0, clip]{img/OccCorrFactDistribution_RhoDev_Rho_Signal}
\put(15,39){{\scriptsize 300M minimum-bias events}}
\put(15,35){{\scriptsize \pp\ at $\s=7$~TeV (2010, pass4)}}
\put(15,31){{\scriptsize All event activity classes}}
\end{overpic} 
\column{0.5\textwidth}
\begin{overpic}[width=\textwidth, trim=0 0 0 0, clip]{img/RhoDev_Rho_OccCorrvsCent_Profile}
\end{overpic}
\end{columns}
\begin{columns}
\column{0.4\textwidth}
\begin{overpic}[width=\textwidth, trim=0 0 0 0, clip]{img/RhoDev_Rho_OccCorrvsCent}
\end{overpic}
\column{0.6\textwidth}
\begin{itemize}
\item Strong correlation of the occupancy factor vs. event activity class (as expected)
\end{itemize}
\end{columns}
\end{frame}

\begin{frame}{Random Cones}
\begin{columns}
\column{0.45\textwidth}
\begin{overpic}[width=1.1\textwidth, trim=10 0 0 35, clip]{img/RCDeltaPt_RhoDev_Rho_Signal}
\put(17,35){{\scriptsize CMS Method}}
\end{overpic}
\begin{overpic}[width=1.1\textwidth, trim=10 0 0 35, clip]{img/RCDeltaPt_RhoDev_RhoExclLeadJets_Signal}
\put(17,35){{\scriptsize CMS Method,}}
\put(17,31){{\scriptsize excl. lead. jet}}
\end{overpic}
\column{0.55\textwidth}
\begin{center}
\begin{overpic}[width=0.9\textwidth, trim=10 0 25 35, clip]{img/RCDeltaPt_RhoTransDev_RhoTrans_Signal}
\put(17,35){{\scriptsize Trans Plane}}
\end{overpic}
\end{center}
\end{columns}
\end{frame}

\begin{frame}{Standard deviation of the background fluctuations}
\begin{columns}
\column{0.45\textwidth}
\begin{overpic}[width=1.1\textwidth, trim=10 0 0 35, clip]{img/StdDevDeltaPtVsCentRhoDev_RhoExclLeadJets_Signal}
\end{overpic}
\begin{overpic}[width=1.1\textwidth, trim=10 0 0 35, clip]{img/StdDevDeltaPtVsLeadJetPtRhoDev_RhoExclLeadJets_Signal}
\end{overpic}
\column{0.55\textwidth}
\begin{center}
\begin{overpic}[width=0.9\textwidth, trim=10 0 25 35, clip]{img/MeanRhoVsCentDetLev}
\end{overpic}
\end{center}
\vspace{-10pt}
\footnotesize
\begin{itemize}
\item The standard deviation is $0.4-0.5$~\GeVc, \textbf{same magnitude as $\left<\rho\right>$}
\item Similarly to $\left<\rho\right>$, dependence on hard process (fluctuations are larger if a jet with $\pt>10$~\GeVc\ is required)
\end{itemize}
\end{columns}
\end{frame}

\begin{frame}{Comparison of B feed-down fractions}
\begin{columns}
\column{0.5\textwidth}
\begin{overpic}[width=\textwidth, trim=0 0 0 0, clip]{img/CompareBFeedDownPass4vsPass2cuts}
\end{overpic}
\column{0.5\textwidth}
\begin{overpic}[width=\textwidth, trim=0 0 0 0, clip]{img/CompareBFeedDownPass4vsPass2cuts_Ratio}
\end{overpic}
\end{columns}
{\footnotesize
\begin{description}
\item[\textcolor{black}{LHC10\_Train823\_with1010}] pass2 cuts
\item[\textcolor{NavyBlue}{LHC10\_Train883}] pass4 cuts [PRC 94, 054908 (2016)]
\end{description}}
\end{frame}

\begin{frame}{Invariant Mass Fit Width vs. \zpar}
\begin{overpic}[width=.8\textwidth, trim=0 0 0 0, clip]{img/AnyINT_D0_Charged_R040_JetZSpectrum_DPt_30_JetPt_5_30_InvMassFit_MassWidth_canvas}
\end{overpic}
\end{frame}


\end{document}
