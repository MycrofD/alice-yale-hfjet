% $Header: /Users/joseph/Documents/LaTeX/beamer/solutions/conference-talks/conference-ornate-20min.en.tex,v 90e850259b8b 2007/01/28 20:48:30 tantau $

\documentclass[xcolor={usenames,dvipsnames}]{beamer}

% This file is a solution template for:

% - Talk at a conference/colloquium.
% - Talk length is about 20min.
% - Style is ornate.



% Copyright 2004 by Till Tantau <tantau@users.sourceforge.net>.
%
% In principle, this file can be redistributed and/or modified under
% the terms of the GNU Public License, version 2.
%
% However, this file is supposed to be a template to be modified
% for your own needs. For this reason, if you use this file as a
% template and not specifically distribute it as part of a another
% package/program, I grant the extra permission to freely copy and
% modify this file as you see fit and even to delete this copyright
% notice. 


\mode<presentation>
{
  \usetheme{AnnArbor}
  % or ...

  \setbeamercovered{transparent}
  % or whatever (possibly just delete it)
 }

\usepackage[percent]{overpic}
\usepackage[english]{babel}
\usepackage{multirow}
% or whatever

\usepackage[latin1]{inputenc}
% or whatever

\usepackage{times}
\usepackage[T1]{fontenc}
% Or whatever. Note that the encoding and the font should match. If T1
% does not look nice, try deleting the line with the fontenc.
%particles
\newcommand{\jpsi}{\rm J/$\psi$}
\newcommand{\psip}{$\psi^\prime$}
\newcommand{\jpsiDY}{\rm J/$\psi$\,/\,DY}
\newcommand{\chic}{$\chi_{\rm c}$}
\newcommand{\pip}{$\pi^{+}$}
\newcommand{\pim}{$\pi^{-}$}
\newcommand{\pizero}{$\pi^{0}$}
\newcommand{\kap}{K$^{+}$}
\newcommand{\kam}{K$^{-}$}
\newcommand{\pbar}{$\rm\overline{p}$}
\newcommand{\ccbar}{\ensuremath{\mathrm{c\overline{c}}}}
\newcommand{\bbbar}{\ensuremath{\mathrm{b\overline{b}}}}
\newcommand{\Dzero}{\ensuremath{\mathrm{D^{0}}}}
\newcommand{\Dzerobar}{\ensuremath{\mathrm{\overline{D}^{0}}}}
\newcommand{\Dpm}{\ensuremath{\mathrm{D^{\pm}}}}
\newcommand{\Ds}{\ensuremath{\mathrm{D_{s}^{\pm}}}}
\newcommand{\Dstar}{\ensuremath{\mathrm{D^{*\pm}}}}

%collision systems
\newcommand{\pp}{pp}
\newcommand{\pPb}{p--Pb}
\newcommand{\PbPb}{Pb--Pb}

%detectors
\newcommand{\ezdc}{$E_{\rm ZDC}$}

%units
\newcommand{\GeVc}{GeV/$c$}
\newcommand{\GeVcsq}{GeV/$c^2$}

%others
\newcommand{\degree}{$^{\rm o}$}
\newcommand{\s}{\ensuremath{\sqrt{s}}}
\newcommand{\snn}{\ensuremath{\sqrt{s_{\rm NN}}}}
\newcommand{\y}{\ensuremath{y}}
\newcommand{\pt}{\ensuremath{p_{\rm T}}}
\newcommand{\dedx}{d$E$/d$x$}
\newcommand{\dndy}{d$N$/d$y$}
\newcommand{\dndydpt}{${\rm d}^2N/({\rm d}y {\rm d}p_{\rm t})$}
\newcommand{\zpar}{\ensuremath{z_{||}}}
\newcommand{\zpargen}{\ensuremath{z_{||}^{\mathrm{part}}}}
\newcommand{\zpardet}{\ensuremath{z_{||}^{\mathrm{det}}}}
\newcommand{\ptchjet}{\ensuremath{p_{\mathrm{T,ch\, jet}}}}
\newcommand{\ptjet}{\ensuremath{p_{\mathrm{T,jet}}}}
\newcommand{\ptchjetgen}{\ensuremath{p_{\mathrm{T,ch\,jet}}^{\mathrm{truth}}}}
\newcommand{\ptchjetdet}{\ensuremath{p_{\mathrm{T,ch\,jet}}^{\mathrm{reco}}}}
\newcommand{\ptd}{\ensuremath{p_{\mathrm{T,D}}}}
\newcommand{\ptdgen}{\ensuremath{p_{\mathrm{T,D}}^{\mathrm{truth}}}}
\newcommand{\ptddet}{\ensuremath{p_{\mathrm{T,D}}^{\mathrm{reco}}}}
\newcommand{\antikt}{anti-\ensuremath{k_{\mathrm{T}}}}
\newcommand{\kt}{\ensuremath{k_{\mathrm{T}}}}
\newcommand{\pthard}{\ensuremath{p_{\mathrm{T,hard}}}}

\AtBeginSection[]{
  \begin{frame}
  \vfill
  \centering
  \begin{beamercolorbox}[sep=8pt,center,shadow=true,rounded=true]{title}
    \usebeamerfont{title}\insertsectionhead\par%
  \end{beamercolorbox}
  \vfill
  \end{frame}
}

\title[D-Tagged Jets in \pp] % (optional, use only with long paper titles)
{D-Tagged Jets in \pp\ Collisions}

\author[Salvatore Aiola]% (optional, use only with lots of authors)
{Salvatore Aiola}
% - Give the names in the same order as the appear in the paper.
% - Use the \inst{?} command only if the authors have different
%   affiliation.

\institute[Yale University] % (optional, but mostly needed)
{Yale University}

\date[PAG-HFCJ - Nov. 22nd, 2017] % (optional, should be abbreviation of conference name)
{PAG-HFCJ \\
November 22nd, 2017}
% - Either use conference name or its abbreviation.
% - Not really informative to the audience, more for people (including
%   yourself) who are reading the slides online

\subject{High-Energy Physics}
% This is only inserted into the PDF information catalog. Can be left
% out. 



% If you have a file called "university-logo-filename.xxx", where xxx
% is a graphic format that can be processed by latex or pdflatex,
% resp., then you can add a logo as follows:

% \pgfdeclareimage[height=0.5cm]{university-logo}{university-logo-filename}
% \logo{\pgfuseimage{university-logo}}


% If you wish to uncover everything in a step-wise fashion, uncomment
% the following command: 

%\beamerdefaultoverlayspecification{<+->}


\begin{document}

\begin{frame}
  \titlepage
\end{frame}

%\begin{frame}{Outline}
 %   \tableofcontents
 %\end{frame}


% Structuring a talk is a difficult task and the following structure
% may not be suitable. Here are some rules that apply for this
% solution: 

% - Exactly two or three sections (other than the summary).
% - At *most* three subsections per section.
% - Talk about 30s to 2min per frame. So there should be between about
%   15 and 30 frames, all told.

% - A conference audience is likely to know very little of what you
%   are going to talk about. So *simplify*!
% - In a 20min talk, getting the main ideas across is hard
%   enough. Leave out details, even if it means being less precise than
%   you think necessary.
% - If you omit details that are vital to the proof/implementation,
%   just say so once. Everybody will be happy with that.

%\begin{overpic}[width=.85\textwidth, trim=0 0 0 0, clip]{img/ReflectionTemplates_DPt_NoJet_DoubleGaus_1010}
%\put(0,61){{\tiny No jet requirement}}
%\put(60,61){{\tiny \textcolor{ForestGreen}{\textbf{Used for QM17 preliminary}}}}
%\end{overpic}

%\begin{columns}
%\column{0.5\textwidth}
%\column{0.5\textwidth}
%\end{columns}

\section{Efficiency}

\begin{frame}{Prompt Efficiency}
\begin{columns}
\column{0.5\textwidth}
\begin{overpic}[width=\textwidth, trim=0 0 0 0, clip]{img/efficiency/Comparison_Prompt_D0toKpiCuts_D0toKpiCuts_D0JetOptimLowJetPtv4_D0toKpiCuts_D0JetOptimHighJetPtv4}
\end{overpic}
\column{0.5\textwidth}
\begin{overpic}[width=\textwidth, trim=0 0 0 0, clip]{img/efficiency/Comparison_Prompt_D0toKpiCuts_D0toKpiCuts_D0JetOptimLowJetPtv4_D0toKpiCuts_D0JetOptimHighJetPtv4_Ratio}
\end{overpic}
\end{columns}
\footnotesize
\begin{itemize}
\item Efficiency lower by up to 50\%
\end{itemize}
\end{frame}

\begin{frame}{Non-Prompt Efficiency}
\begin{columns}
\column{0.5\textwidth}
\begin{overpic}[width=\textwidth, trim=0 0 0 0, clip]{img/efficiency/Comparison_NonPrompt_D0toKpiCuts_D0toKpiCuts_D0JetOptimLowJetPtv4_D0toKpiCuts_D0JetOptimHighJetPtv4}
\end{overpic}
\column{0.5\textwidth}
\begin{overpic}[width=\textwidth, trim=0 0 0 0, clip]{img/efficiency/Comparison_NonPrompt_D0toKpiCuts_D0toKpiCuts_D0JetOptimLowJetPtv4_D0toKpiCuts_D0JetOptimHighJetPtv4_Ratio}
\end{overpic}
\end{columns}
\footnotesize
\begin{itemize}
\item Efficiency lower by up to 60\%
\item Almost flat efficiency for $\ptd>3$~\GeVc\
\end{itemize}
\end{frame}

\section{Raw Yields}

\begin{frame}{Jet \pt\ spectrum, $\ptd>2$~\GeVc}
\begin{columns}
\column{0.5\textwidth}
\begin{overpic}[width=\textwidth, trim=0 0 0 0, clip]{img/raw_yields/Charged_R040_JetPtSpectrum_DPt_20_SideBand_SpectraComparison}
\end{overpic}
\column{0.5\textwidth}
\begin{overpic}[width=\textwidth, trim=0 0 0 0, clip]{img/raw_yields/Charged_R040_JetPtSpectrum_DPt_20_SideBand_SpectraComparison_Ratio}
\end{overpic}
\end{columns}
\end{frame}

\begin{frame}{FF, $\ptd>2$~\GeVc\ and $5<\ptchjet<15$~\GeVc}
\begin{columns}
\column{0.5\textwidth}
\begin{overpic}[width=\textwidth, trim=0 0 0 0, clip]{img/raw_yields/Charged_R040_JetZSpectrum_DPt_20_JetPt_5_15_SideBand_SpectraComparison}
\end{overpic}
\column{0.5\textwidth}
\begin{overpic}[width=\textwidth, trim=0 0 0 0, clip]{img/raw_yields/Charged_R040_JetZSpectrum_DPt_20_JetPt_5_15_SideBand_SpectraComparison_Ratio}
\end{overpic}
\end{columns}
\end{frame}

\begin{frame}{FF, $\ptd>6$~\GeVc\ and $15<\ptchjet<30$~\GeVc}
\begin{columns}
\column{0.5\textwidth}
\begin{overpic}[width=\textwidth, trim=0 0 0 0, clip]{img/raw_yields/Charged_R040_JetZSpectrum_DPt_60_JetPt_15_30_SideBand_SpectraComparison}
\end{overpic}
\column{0.5\textwidth}
\begin{overpic}[width=\textwidth, trim=0 0 0 0, clip]{img/raw_yields/Charged_R040_JetZSpectrum_DPt_60_JetPt_15_30_SideBand_SpectraComparison_Ratio}
\end{overpic}
\end{columns}
\end{frame}

\section{Statistical Uncertainties}

\begin{frame}{Jet \pt\ spectrum, $\ptd>2$~\GeVc}
\centering
Statistical Uncertainties, no efficiency correction
\vspace{25pt}
\begin{columns}
\column{0.5\textwidth}
\centering
Side-Band in \ptd\ bins\\
\begin{overpic}[width=\textwidth, trim=0 0 0 0, clip]{img/stat_unc/Charged_R040_JetPtSpectrum_DPt_20_SideBand_SpectraUncertaintyComparison}
\end{overpic}
\column{0.5\textwidth}
\centering
Inv.Mass Fit in \ptchjet\ bins\\
\begin{overpic}[width=\textwidth, trim=0 0 0 0, clip]{img/stat_unc/Charged_R040_JetPtSpectrum_DPt_20_InvMassFit_SpectraUncertaintyComparison}
\end{overpic}
\end{columns}
\end{frame}

\begin{frame}{FF, $\ptd>2$~\GeVc\ and $5<\ptchjet<15$~\GeVc}
\centering
Statistical Uncertainties, no efficiency correction
\vspace{25pt}
\begin{columns}
\column{0.5\textwidth}
\centering
Side-Band in \ptd\ bins \\
\begin{overpic}[width=\textwidth, trim=0 0 0 0, clip]{img/stat_unc/Charged_R040_JetZSpectrum_DPt_20_JetPt_5_15_SideBand_SpectraUncertaintyComparison}
\end{overpic}
\column{0.5\textwidth}
\centering
Inv.Mass Fit in \zpar\ bins\\
\begin{overpic}[width=\textwidth, trim=0 0 0 0, clip]{img/stat_unc/Charged_R040_JetZSpectrum_DPt_20_JetPt_5_15_InvMassFit_SpectraUncertaintyComparison}
\end{overpic}
\end{columns}
\end{frame}

\begin{frame}{FF, $\ptd>6$~\GeVc\ and $15<\ptchjet<30$~\GeVc}
\centering
Statistical Uncertainties, no efficiency correction
\vspace{25pt}
\begin{columns}
\column{0.5\textwidth}
\centering
Side-Band in \ptd\ bins \\
\begin{overpic}[width=\textwidth, trim=0 0 0 0, clip]{img/stat_unc/Charged_R040_JetZSpectrum_DPt_60_JetPt_15_30_SideBand_SpectraUncertaintyComparison}
\end{overpic}
\column{0.5\textwidth}
\centering
Inv.Mass Fit in \zpar\ bins\\
\begin{overpic}[width=\textwidth, trim=0 0 0 0, clip]{img/stat_unc/Charged_R040_JetZSpectrum_DPt_60_JetPt_15_30_InvMassFit_SpectraUncertaintyComparison}
\end{overpic}
\end{columns}
\end{frame}

\begin{frame}{Statistical Uncertainties w/ eff. corr. (SB method)}
\centering
\footnotesize
Jet \pt\ spectrum, $\ptd>2$~\GeVc \\
\begin{overpic}[width=.4\textwidth, trim=0 0 0 0, clip]{img/stat_unc_eff/Charged_R040_JetPtSpectrum_DPt_20_SideBand_SpectraUncertaintyComparison}
\end{overpic}\\
\footnotesize
Fragmentation Function
\begin{columns}
\column{0.5\textwidth}
\centering
\tiny
$\ptd>2$~\GeVc\ and $5<\ptchjet<15$~\GeVc\\
\begin{overpic}[width=.8\textwidth, trim=0 0 0 0, clip]{img/stat_unc_eff/Charged_R040_JetZSpectrum_DPt_20_JetPt_5_15_SideBand_SpectraUncertaintyComparison}
\end{overpic}
\column{0.5\textwidth}
\centering
\tiny
$\ptd>6$~\GeVc\ and $15<\ptchjet<30$~\GeVc\\
\begin{overpic}[width=.8\textwidth, trim=0 0 0 0, clip]{img/stat_unc_eff/Charged_R040_JetZSpectrum_DPt_60_JetPt_15_30_SideBand_SpectraUncertaintyComparison}
\end{overpic}
\end{columns}
\end{frame}

\section{B Feed-Down}

\begin{frame}{Prompt vs. Non-Prompt}
\begin{columns}
\column{0.5\textwidth}
\begin{overpic}[width=\textwidth, trim=0 0 0 0, clip]{img/efficiency/Comparison_Prompt_NonPrompt}
\end{overpic}
\column{0.5\textwidth}
\begin{overpic}[width=\textwidth, trim=0 0 0 0, clip]{img/efficiency/Comparison_Prompt_NonPrompt_Ratio}
\end{overpic}
\end{columns}
\footnotesize
\begin{itemize}
\item Slightly better ratio prompt/non-prompt
\end{itemize}
\end{frame}

\begin{frame}{B Feed-Down w/ non-prompt efficiency}
\begin{columns}
\column{0.5\textwidth}
\begin{overpic}[width=\textwidth, trim=0 0 0 0, clip]{img/feed_down/Comparison_JetPtSpectrum_DPt_20_DetectorLevel_JetPtSpectrum_bEfficiencyMultiply_BFeedDown_1505317519_1399_BFeedDown_1505317519_1399_oldcuts}
\end{overpic}
\column{0.5\textwidth}
\begin{overpic}[width=\textwidth, trim=0 0 0 0, clip]{img/feed_down/Comparison_JetPtSpectrum_DPt_20_DetectorLevel_JetPtSpectrum_bEfficiencyMultiply_BFeedDown_1505317519_1399_BFeedDown_1505317519_1399_oldcuts_Ratio}
\end{overpic}
\end{columns}
\footnotesize
\begin{itemize}
\item Smaller non-prompt with the new cuts...
\end{itemize}
\end{frame}

\begin{frame}{B Feed-Down w/ non-prompt/prompt}
\begin{columns}
\column{0.5\textwidth}
\begin{overpic}[width=\textwidth, trim=0 0 0 0, clip]{img/feed_down/Comparison_JetPtSpectrum_DPt_20_DetectorLevel_JetPtSpectrum_bEfficiencyMultiply_cEfficiencyDivide_BFeedDown_1505317519_1399_BFeedDown_1505317519_1399_oldcuts}
\end{overpic}
\column{0.5\textwidth}
\begin{overpic}[width=\textwidth, trim=0 0 0 0, clip]{img/feed_down/Comparison_JetPtSpectrum_DPt_20_DetectorLevel_JetPtSpectrum_bEfficiencyMultiply_cEfficiencyDivide_BFeedDown_1505317519_1399_BFeedDown_1505317519_1399_oldcuts_Ratio}
\end{overpic}
\end{columns}
\footnotesize
\begin{itemize}
\item Smaller non-prompt with the new cuts...
\item ...but similar fraction because efficiency is also lower for prompt
\end{itemize}
\end{frame}

\section{Unfolding}

\begin{frame}{Kinematic cuts and efficiency}
\begin{columns}
\column{0.5\textwidth}
\begin{overpic}[width=\textwidth, trim=0 0 0 0, clip]{img/efficiency/D0toKpiCuts_D0JetOptimLowJetPtv4_Comparison_KineCuts_KinCuts}
\end{overpic}
\column{0.5\textwidth}
\begin{overpic}[width=\textwidth, trim=0 0 0 0, clip]{img/efficiency/D0toKpiCuts_D0JetOptimLowJetPtv4_Comparison_KineCuts_KinCuts_Ratio}
\end{overpic}
\end{columns}
\tiny
\begin{itemize}
\item Denominator of the efficiency is the same for both cases: all generated \Dzero-jets within the accepted kinematic ranges
\item Numerator for white points: generated \Dzero-jets within the accepted kinematic ranges matched to reconstructed \Dzero-jets (which not necessarily within the accepted kinematic ranges)
\item Numerator for blue points: generated \Dzero-jets (not necessarily within the accepted kinematic ranges) matched to reconstructed \Dzero-jets within the accepted kinematic ranges
\end{itemize}
In other words the difference is whether in the numerator the cuts are applied at generated level (white) or reconstructed level (blue). In the first case we get the simple efficiency, 
i.e. the probability of reconstructing a \Dzero-jet with certain kinematic properties; in the second case we fold in the acceptance factor (feed-in and feed-out of the accepted range).
\end{frame}

\begin{frame}{Detector Response: Prompt}
\centering
$\ptd>2$~\GeVc\ and $5<\ptchjet<15$~\GeVc\\
\vspace{10pt}
\begin{overpic}[width=.9\textwidth, trim=0 0 0 0, clip]{img/unfolding/Prompt_D0_D0toKpiCuts_D0JetOptimLowJetPtv4_Jet_AKTChargedR040_pt_scheme_JetZSpectrum_DPt_20_JetPt_5_15_DetectorResponse}
\end{overpic}
\end{frame}

\begin{frame}{Detector Response: Prompt}
\centering
$\ptd>6$~\GeVc\ and $15<\ptchjet<30$~\GeVc\\
\vspace{10pt}
\begin{overpic}[width=.9\textwidth, trim=0 0 0 0, clip]{img/unfolding/Prompt_D0_D0toKpiCuts_D0JetOptimHighJetPtv4_Jet_AKTChargedR040_pt_scheme_JetZSpectrum_DPt_60_JetPt_15_30_DetectorResponse}
\end{overpic}
\end{frame}

\begin{frame}{Resolution of the momentum fraction}
\begin{columns}
\column{0.5\textwidth}
\centering
\tiny
$\ptd>2$~\GeVc\ and $5<\ptchjet<15$~\GeVc\\
\begin{overpic}[width=\textwidth, trim=0 0 0 0, clip]{img/unfolding/Prompt_D0_D0toKpiCuts_D0JetOptimLowJetPtv4_Jet_AKTChargedR040_pt_scheme_JetZSpectrum_DPt_20_JetPt_5_15_DetectorResponse_Resolution_canvas}
\end{overpic}
\column{0.5\textwidth}
\centering
\tiny
$\ptd>6$~\GeVc\ and $15<\ptchjet<30$~\GeVc\\
\begin{overpic}[width=\textwidth, trim=0 0 0 0, clip]{img/unfolding/Prompt_D0_D0toKpiCuts_D0JetOptimHighJetPtv4_Jet_AKTChargedR040_pt_scheme_JetZSpectrum_DPt_60_JetPt_15_30_DetectorResponse_Resolution_canvas}
\end{overpic}
\end{columns}
Resolution is smaller than the bin size $\rightarrow$ expect negligible correction from unfolding
\end{frame}

\begin{frame}{Bayesian Unfolding}
\begin{columns}
\column{0.5\textwidth}
\centering
\begin{overpic}[width=.71\textwidth, trim=0 240 290 0, clip]{img/unfolding/JetZ_SideBand_DPt_20_JetPt_5_15_Response_PriorResponseTruth}
\end{overpic}\\
\begin{overpic}[width=.88\textwidth, trim=0 0 0 0, clip]{img/unfolding/JetZ_SideBand_DPt_20_JetPt_5_15_UnfoldingSummary_Bayes_RefoldedOverMeasured}
\end{overpic}
\column{0.5\textwidth}
\centering
\begin{overpic}[width=.88\textwidth, trim=0 0 0 0, clip]{img/unfolding/JetZ_SideBand_DPt_20_JetPt_5_15_UnfoldingSummary_Bayes}
\end{overpic}\\
\begin{overpic}[width=.88\textwidth, trim=0 0 0 0, clip]{img/unfolding/JetZ_SideBand_DPt_20_JetPt_5_15_UnfoldingSummary_Bayes_UnfoldedOverMeasured}
\end{overpic}
\end{columns}
\end{frame}

\begin{frame}{Unfolding Stability}
\begin{columns}
\column{0.5\textwidth}
\centering
\tiny 
Number of iterations\\
\begin{overpic}[width=.8\textwidth, trim=0 0 0 0, clip]{img/unfolding/JetZ_SideBand_DPt_20_JetPt_5_15_UnfoldingRegularization_Bayes_PriorResponseTruth_Ratio}
\end{overpic}
\column{0.5\textwidth}
\centering
\tiny
Unfolding method\\
\begin{overpic}[width=.8\textwidth, trim=0 0 0 0, clip]{img/unfolding/JetZ_SideBand_DPt_20_JetPt_5_15_UnfoldingMethod_Ratio}
\end{overpic}
\end{columns}
\centering
\tiny
Pearsons' coefficients\\
\begin{overpic}[width=.5\textwidth, trim=0 0 0 0, clip]{img/unfolding/JetZ_SideBand_DPt_20_JetPt_5_15_Pearson_Bayes_PriorResponseTruth}
\end{overpic}
\end{frame}

\begin{frame}{FF in PYTHIA6+GEANT3}
\begin{columns}
\column{0.5\textwidth}
\centering
\tiny
$\ptd>2$~\GeVc\ and $5<\ptchjet<15$~\GeVc\\
\begin{overpic}[width=\textwidth, trim=0 0 0 0, clip]{img/unfolding/z_rec_gen_5_15}
\end{overpic}
\column{0.5\textwidth}
\centering
\tiny
$\ptd>6$~\GeVc\ and $15<\ptchjet<30$~\GeVc\\
\begin{overpic}[width=\textwidth, trim=0 0 0 0, clip]{img/unfolding/z_rec_gen_15_30}
\end{overpic}
\end{columns}
PYTHIA captures the main features of the shape of the distribution as observed in data \\
Important since we use it to estimate the corrections (efficiency, resolution)
\end{frame}

\section{Conclusions}

\begin{frame}{Conclusions: topological cuts}
\begin{itemize}
\item No significant improvement with the new cuts
\item Check is whether the systematic uncertainty from the raw yield multi-trial improves with the new cuts (because of a more stable fit)
\item For the moment, stay with the old cuts?
\item On the side I will try to adopt a multi-variate study of the topological variable to see whether a better set of cuts can be found in this way (instead of looking at each variable one by one)
\item Check whether the wider \ptd\ bins lead to an improvement: for sure they are needed for the \zpar\ spectrum in the high \ptchjet\ bin, but not obvious for the other cases
\end{itemize}
\end{frame}

\begin{frame}{Conclusions: unfolding}
\begin{itemize}
\item Unfolding of the \zpar\ spectrum is stable for $5<\ptchjet<15$~\GeVc
\item Small correction as for the jet \pt\ spectrum
\item Need to vary the prior as a final check
\item Working on the unfolding of the \zpar\ spectrum for $15<\ptchjet<30$~\GeVc, which seems more unstable (probably due to the larger statistical uncertainties)
\end{itemize}
\end{frame}

\end{document}
