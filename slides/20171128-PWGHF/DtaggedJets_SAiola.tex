% $Header: /Users/joseph/Documents/LaTeX/beamer/solutions/conference-talks/conference-ornate-20min.en.tex,v 90e850259b8b 2007/01/28 20:48:30 tantau $

\documentclass[xcolor={usenames,dvipsnames}, aspectratio=169]{beamer}

% This file is a solution template for:

% - Talk at a conference/colloquium.
% - Talk length is about 20min.
% - Style is ornate.



% Copyright 2004 by Till Tantau <tantau@users.sourceforge.net>.
%
% In principle, this file can be redistributed and/or modified under
% the terms of the GNU Public License, version 2.
%
% However, this file is supposed to be a template to be modified
% for your own needs. For this reason, if you use this file as a
% template and not specifically distribute it as part of a another
% package/program, I grant the extra permission to freely copy and
% modify this file as you see fit and even to delete this copyright
% notice. 


\mode<presentation>
{
  \usetheme{AnnArbor}

  \setbeamercovered{transparent}
  % or whatever (possibly just delete it)
 }

\usepackage[percent]{overpic}
\usepackage[english]{babel}
\usepackage{multirow}
\usepackage[at]{easylist}
% or whatever

\usepackage[latin1]{inputenc}
% or whatever

\usepackage{times}
\usepackage[T1]{fontenc}
% Or whatever. Note that the encoding and the font should match. If T1
% does not look nice, try deleting the line with the fontenc.
%particles
\newcommand{\jpsi}{\rm J/$\psi$}
\newcommand{\psip}{$\psi^\prime$}
\newcommand{\jpsiDY}{\rm J/$\psi$\,/\,DY}
\newcommand{\chic}{$\chi_{\rm c}$}
\newcommand{\pip}{$\pi^{+}$}
\newcommand{\pim}{$\pi^{-}$}
\newcommand{\pizero}{$\pi^{0}$}
\newcommand{\kap}{K$^{+}$}
\newcommand{\kam}{K$^{-}$}
\newcommand{\pbar}{$\rm\overline{p}$}
\newcommand{\ccbar}{\ensuremath{\mathrm{c\overline{c}}}}
\newcommand{\bbbar}{\ensuremath{\mathrm{b\overline{b}}}}
\newcommand{\Dzero}{\ensuremath{\mathrm{D^{0}}}}
\newcommand{\Dzerobar}{\ensuremath{\mathrm{\overline{D}^{0}}}}
\newcommand{\Dpm}{\ensuremath{\mathrm{D^{\pm}}}}
\newcommand{\Ds}{\ensuremath{\mathrm{D_{s}^{\pm}}}}
\newcommand{\Dstar}{\ensuremath{\mathrm{D^{*\pm}}}}

%collision systems
\newcommand{\pp}{pp}
\newcommand{\pPb}{p--Pb}
\newcommand{\PbPb}{Pb--Pb}

%detectors
\newcommand{\ezdc}{$E_{\rm ZDC}$}

%units
\newcommand{\GeVc}{GeV/$c$}
\newcommand{\GeVcsq}{GeV/$c^2$}

%others
\newcommand{\degree}{$^{\rm o}$}
\newcommand{\s}{\ensuremath{\sqrt{s}}}
\newcommand{\snn}{\ensuremath{\sqrt{s_{\rm NN}}}}
\newcommand{\y}{\ensuremath{y}}
\newcommand{\pt}{\ensuremath{p_{\rm T}}}
\newcommand{\dedx}{d$E$/d$x$}
\newcommand{\dndy}{d$N$/d$y$}
\newcommand{\dndydpt}{${\rm d}^2N/({\rm d}y {\rm d}p_{\rm t})$}
\newcommand{\zpar}{\ensuremath{z_{||}}}
\newcommand{\zpargen}{\ensuremath{z_{||}^{\mathrm{part}}}}
\newcommand{\zpardet}{\ensuremath{z_{||}^{\mathrm{det}}}}
\newcommand{\ptchjet}{\ensuremath{p_{\mathrm{T,ch\, jet}}}}
\newcommand{\ptjet}{\ensuremath{p_{\mathrm{T,jet}}}}
\newcommand{\ptchjetgen}{\ensuremath{p_{\mathrm{T,ch\,jet}}^{\mathrm{truth}}}}
\newcommand{\ptchjetdet}{\ensuremath{p_{\mathrm{T,ch\,jet}}^{\mathrm{reco}}}}
\newcommand{\ptd}{\ensuremath{p_{\mathrm{T,D}}}}
\newcommand{\ptdgen}{\ensuremath{p_{\mathrm{T,D}}^{\mathrm{truth}}}}
\newcommand{\ptddet}{\ensuremath{p_{\mathrm{T,D}}^{\mathrm{reco}}}}
\newcommand{\antikt}{anti-\ensuremath{k_{\mathrm{T}}}}
\newcommand{\kt}{\ensuremath{k_{\mathrm{T}}}}
\newcommand{\pthard}{\ensuremath{p_{\mathrm{T,hard}}}}

\AtBeginSection[]{
  \begin{frame}
  \vfill
  \centering
  \begin{beamercolorbox}[sep=8pt,center,shadow=true,rounded=true]{title}
    \usebeamerfont{title}\insertsectionhead\par%
  \end{beamercolorbox}
  \vfill
  \end{frame}
}

\title[D-Tagged Jets in \pp, \pPb\ and \PbPb] % (optional, use only with long paper titles)
{D-Tagged Jets in \pp, \pPb\ and \PbPb}

\author[Salvatore Aiola]% (optional, use only with lots of authors)
{Salvatore Aiola \\
for PAG-HFCJ}
% - Give the names in the same order as the appear in the paper.
% - Use the \inst{?} command only if the authors have different
%   affiliation.

\institute[Yale University] % (optional, but mostly needed)
{Yale University}

\date[PWG-HF - Nov. 28th, 2017] % (optional, should be abbreviation of conference name)
{PWG-HF \\
November 28th, 2017}
% - Either use conference name or its abbreviation.
% - Not really informative to the audience, more for people (including
%   yourself) who are reading the slides online

\subject{High-Energy Physics}
% This is only inserted into the PDF information catalog. Can be left
% out. 



% If you have a file called "university-logo-filename.xxx", where xxx
% is a graphic format that can be processed by latex or pdflatex,
% resp., then you can add a logo as follows:

% \pgfdeclareimage[height=0.5cm]{university-logo}{university-logo-filename}
% \logo{\pgfuseimage{university-logo}}


% If you wish to uncover everything in a step-wise fashion, uncomment
% the following command: 

%\beamerdefaultoverlayspecification{<+->}

\begin{document}

\begin{frame}
  \titlepage
\end{frame}

%\begin{frame}{Outline}
  %  \tableofcontents
%\end{frame}


% Structuring a talk is a difficult task and the following structure
% may not be suitable. Here are some rules that apply for this
% solution: 

% - Exactly two or three sections (other than the summary).
% - At *most* three subsections per section.
% - Talk about 30s to 2min per frame. So there should be between about
%   15 and 30 frames, all told.

% - A conference audience is likely to know very little of what you
%   are going to talk about. So *simplify*!
% - In a 20min talk, getting the main ideas across is hard
%   enough. Leave out details, even if it means being less precise than
%   you think necessary.
% - If you omit details that are vital to the proof/implementation,
%   just say so once. Everybody will be happy with that.

%\begin{overpic}[width=.85\textwidth, trim=0 0 0 0, clip]{img/ReflectionTemplates_DPt_NoJet_DoubleGaus_1010}
%\put(0,61){{\tiny No jet requirement}}
%\put(60,61){{\tiny \textcolor{ForestGreen}{\textbf{Used for QM17 preliminary}}}}
%\end{overpic}

%\begin{columns}
%\column{0.5\textwidth}
%\column{0.5\textwidth}
%\end{columns}

\section{\pp\ collisions at $\s=7$~TeV}

\subsection{\Dzero-Jet Cross Section}

\begin{frame}{Preliminary: \Dzero-Jet Cross Section}
\begin{columns}
\column{0.40\textwidth}
\begin{overpic}[width=\textwidth, trim=0 0 0 0, clip]{img/approved_figures/D0JetCrossSection_pp7TeV}
\end{overpic}\\
\small
Preliminary shown at QM17 (\href{https://indico.cern.ch/event/433345/contributions/2358064/}{poster})
\column{0.60\textwidth}
\footnotesize
\begin{itemize}
\item Jets reconstructed out of the \Dzero\ candidate (K$\pi$ channel) and all other tracks
\item Invariant mass analysis: in jet \pt\ bins (used as cross-check) or side-band method in D \pt\ bins (default)
\item MB \pp\ at $\s=7$~TeV, 2010 (LHC10b,c,d,e)
\end{itemize}
\vspace{10pt}
\large
$\rightarrow$ New
\small
\begin{itemize}
\item Attempt to improve statistical precision by optimizing the topological cuts 
\item Variation of the width of the \ptd\ bins in the side-band method
\item Reduce minimum \ptd\ cut to $2$~\GeVc\ (to be decided)
\end{itemize}
\end{columns}
\end{frame}

\begin{frame}{Topological Cuts}
\begin{columns}
\column{0.4\textwidth}
\begin{itemize}
\item Strategy: plot $S/B$ and $S/\sqrt{S+B}$ (significance) as a function of the cut value
\item Focus on 4 variables: $\cos(\theta_{\rm p})$, $\cos(\theta^{*})$, $d_{0}d_{0}$, DCA
\item Cut value on each variable optmized independently $\rightarrow$ limitation, for the future will try to use a multi-variate approach
\item Cuts optmized in two bins of \ptchjet\ [5, 15, 30] and 5 bins of \ptd\ [2, 4, 6, 9, 15, 30]
\end{itemize}
\column{0.6\textwidth}
\scriptsize
Example: $\cos(\theta^{*})$ for $6<\ptd<9$~\GeVc\ and $5<\ptchjet<15$~\GeVc
\begin{columns}
\column{0.5\textwidth}
\begin{overpic}[width=\textwidth, trim=0 0 0 0, clip]{img/topological_cuts/CosThetaStar_CutFraction_MB_Bkg_JetPt5_15_DPt6_9}
\end{overpic}\\
\column{0.5\textwidth}
\begin{overpic}[width=\textwidth, trim=0 0 0 0, clip]{img/topological_cuts/CosThetaStar_CutSignificance_MB_Bkg_JetPt5_15_DPt6_9}
\end{overpic}
\end{columns}
\centering
\footnotesize
\textcolor{ForestGreen}{green} (\textcolor{magenta}{magenta}) line\\
 $x$ such that $|\cos(\theta^{*})|<x$ maximizes \textcolor{ForestGreen}{$S/B$} (\textcolor{magenta}{$S/\sqrt{S+B}$})
\end{columns}
\end{frame}

\begin{frame}{Invariant Mass Fits: comparison new/old cuts}
\begin{columns}
\column{0.45\textwidth}
\centering
\scriptsize
Side Band Method in \ptd\ bins
\begin{columns}
\column{0.5\textwidth}
\centering
\footnotesize
Old cuts\\
\vspace{5pt}
\begin{overpic}[width=\textwidth, trim=380 170 10 0, clip]{img/topological_cuts/AnyINT_D0_D0toKpiCuts_Charged_R040_DPtBins_JetPt_5_30}
\end{overpic}
\column{0.5\textwidth}
\centering
\footnotesize
New cuts\\
\vspace{5pt}
\begin{overpic}[width=\textwidth, trim=380 170 10 0, clip]{img/topological_cuts/AnyINT_D0_D0toKpiCuts_D0JetOptimLowJetPtv4_Charged_R040_DPtBins_JetPt_5_30}
\end{overpic}
\end{columns}
\centering
Improved $S/B$ ($1.6\rightarrow3.1$) at the cost of a slightly smaller significance ($19.8\rightarrow17.8$)
\column{0.05\textwidth}
\column{0.45\textwidth}
\centering
\scriptsize
Invariant Mass Fits in \ptchjet\ bins
\begin{columns}
\column{0.5\textwidth}
\centering
\footnotesize
Old cuts\\
\vspace{5pt}
\begin{overpic}[width=\textwidth, trim=190 0 190 180, clip]{img/topological_cuts/AnyINT_D0_D0toKpiCuts_Charged_R040_JetPtBins_DPt_30}
\end{overpic}
\column{0.5\textwidth}
\centering
\footnotesize
New cuts\\
\vspace{5pt}
\begin{overpic}[width=\textwidth, trim=190 0 190 180, clip]{img/topological_cuts/AnyINT_D0_D0toKpiCuts_D0JetOptimLowJetPtv4_Charged_R040_JetPtBins_DPt_30}
\end{overpic}
\end{columns}
\centering
Improved $S/B$ ($0.4\rightarrow0.8$) with similar significance
\end{columns}
\end{frame}

\begin{frame}{Raw Yields: comparison new/old cuts}
\begin{columns}
\column{0.34\textwidth}
\centering
\begin{overpic}[width=\textwidth, trim=0 0 0 0, clip]{img/topological_cuts/Comparison_Prompt_D0toKpiCuts_D0toKpiCuts_D0JetOptimLowJetPtv4_D0toKpiCuts_D0JetOptimHighJetPtv4}
\end{overpic}\\
\footnotesize
\begin{itemize}
\item Lower efficiency at high \ptd
\item Raw yields compatible after efficiency correction with old/new cuts
\end{itemize}
\column{0.31\textwidth}
\centering
\tiny
Before efficiency correction (SB method)\\
\begin{overpic}[width=\textwidth, trim=0 0 0 0, clip]{img/topological_cuts/Charged_R040_JetPtSpectrum_DPt_30_SideBand_SpectraComparison}
\end{overpic}
\begin{overpic}[width=\textwidth, trim=0 0 00 0, clip]{img/topological_cuts/Charged_R040_JetPtSpectrum_DPt_30_SideBand_SpectraComparison_Ratio}
\end{overpic}
\column{0.31\textwidth}
\centering
\tiny
After efficiency correction (SB method)\\
\begin{overpic}[width=\textwidth, trim=0 0 0 0, clip]{img/topological_cuts/Charged_R040_JetPtSpectrum_DPt_30_SideBand_SpectraComparison_eff}
\end{overpic}
\begin{overpic}[width=\textwidth, trim=0 0 00 0, clip]{img/topological_cuts/Charged_R040_JetPtSpectrum_DPt_30_SideBand_SpectraComparison_Ratio_eff}
\end{overpic}
\end{columns}
\end{frame}

\begin{frame}{Uncertainty comparison (after efficiency correction)}
\begin{columns}[t]
\column{0.45\textwidth}
\centering
\small
Inv. Mass Fit in \ptchjet\ bins
\begin{columns}[t]
\column{0.6\textwidth}
\begin{overpic}[width=\textwidth, trim=0 0 0 0, clip]{img/topological_cuts/Charged_R040_JetPtSpectrum_DPt_30_InvMassFit_SpectraUncertaintyComparison_eff}
\end{overpic}\\
\begin{overpic}[width=\textwidth, trim=0 0 0 0, clip]{img/topological_cuts/ComparisonSystematic_DMesonCuts_JetPtSpectrum_DPt_30_InvMassFit_eff}
\end{overpic}
\column{0.4\textwidth}
\centering
\scriptsize
\vspace{-50pt}\\
Statistical uncertainty: improved with new cuts in the last bin\\
\vspace{30pt}
Systematic uncertainty on raw yield extr.: improved ($\rightarrow$ fits more stable), unable to calculate for the last bin with the old cuts (many trials failed)
\end{columns}
\column{0.05\textwidth}
\column{0.45\textwidth}
\centering
\small
Side-Band Method in \ptd\ bins
\begin{columns}[t]
\column{0.6\textwidth}
\begin{overpic}[width=\textwidth, trim=0 0 0 0, clip]{img/topological_cuts/Charged_R040_JetPtSpectrum_DPt_30_SideBand_SpectraUncertaintyComparison_eff}
\end{overpic}\\
\begin{overpic}[width=\textwidth, trim=0 0 0 0, clip]{img/topological_cuts/ComparisonSystematic_DMesonCuts_JetPtSpectrum_DPt_30_SideBand_eff}
\end{overpic}
\column{0.4\textwidth}
\centering
\small
Less obvious whether there is any improvement
\end{columns}
\end{columns}
\end{frame}

\begin{frame}{Width of the \ptd\ bins in the SB method}
\begin{columns}
\column{0.31\textwidth}
\centering
\tiny
Raw Yields\\
\begin{overpic}[width=\textwidth, trim=0 0 0 0, clip]{img/ptd_bin_width/AnyINT_D0_D0toKpiCuts_Charged_R040_DPtBinWidth_jet_pt_50_300_SpectraComparison}
\end{overpic}
\begin{overpic}[width=\textwidth, trim=0 0 0 0, clip]{img/ptd_bin_width/AnyINT_D0_D0toKpiCuts_Charged_R040_DPtBinWidth_jet_pt_50_300_SpectraComparison_Ratio}
\end{overpic}
\column{0.31\textwidth}
\centering
\tiny
Statistical Uncertainty\\
\begin{overpic}[width=\textwidth, trim=0 0 0 0, clip]{img/ptd_bin_width/AnyINT_D0_D0toKpiCuts_Charged_R040_DPtBinWidth_jet_pt_50_300_SpectraComparison_Uncertainty}
\end{overpic}\\
Systematic Uncertainty on Raw Yield Extr.
\begin{overpic}[width=\textwidth, trim=0 0 0 0, clip]{img/ptd_bin_width/D0_D0toKpiCuts_ComparisonSystematic_JetPtSpectrum_DPt_20_DPtBinWidth}
\end{overpic}
\column{0.34\textwidth}
\scriptsize
\begin{itemize}
\item Small bins: [2, 3, 4, 5, 6, 7, 8, 10, 12, 15, 20, 30] $\rightarrow$ efficiency correction applied before summing the \ptchjet\ spectra from each \ptd\ bin
\item Wide bins: [2, 4, 6, 9, 15, 30] $\rightarrow$ efficiency correction (binned in smaller bins) applied as a weight to each candidate in the invariant mass distribution
\item Raw yields identical in the two cases $\rightarrow$ confirms robustness of the side-band method
\item No significant change in the statistical/systematic uncertainties
\end{itemize}
\end{columns}
\end{frame}

\begin{frame}{$\ptd>2$ vs. $\ptd>3$~\GeVc}
\begin{columns}
\column{0.31\textwidth}
\centering
\tiny
Raw Yields (Side-Band Method)\\
\begin{overpic}[width=\textwidth, trim=0 0 0 0, clip]{img/min_ptd_cut/AnyINT_D0_D0toKpiCuts_D0JetOptimLowJetPtv4_Charged_R040_DPtCutSideBand_jet_pt_50_300_SpectraComparison}
\end{overpic}
\begin{overpic}[width=\textwidth, trim=0 0 0 0, clip]{img/min_ptd_cut/AnyINT_D0_D0toKpiCuts_D0JetOptimLowJetPtv4_Charged_R040_DPtCutSideBand_jet_pt_50_300_SpectraComparison_Ratio}
\end{overpic}
\column{0.31\textwidth}
\centering
\tiny
Statistical Uncertainty\\
\begin{overpic}[width=\textwidth, trim=0 0 0 0, clip]{img/min_ptd_cut/AnyINT_D0_D0toKpiCuts_D0JetOptimLowJetPtv4_Charged_R040_DPtCutSideBand_jet_pt_50_300_SpectraComparison_Uncertainty}
\end{overpic}\\
Systematic Uncertainty on Raw Yield Extr.
\begin{overpic}[width=\textwidth, trim=0 0 0 0, clip]{img/min_ptd_cut/D0_D0toKpiCuts_D0JetOptimLowJetPtv4_ComparisonSystematic_JetPtSpectrum_DPtCutSideBand}
\end{overpic}
\column{0.34\textwidth}
\small
Reducing the cut from $3$ to $2$~\GeVc\ 
\begin{itemize}
\item Smaller bias on jet fragmentation at low \pt
\item Increases the relative statistical uncertainties by $\sim +10\%$ for $\ptchjet>14$~\GeVc\
\item Systematic uncertainties on raw yield extraction also a bit higher
\end{itemize}
\end{columns}
\end{frame}

\subsection{Jet Momentum Fraction Carried by the \Dzero}

\begin{frame}{Raw Yield Extraction: $5<\ptchjet<15$~\GeVc}
\begin{columns}
\column{0.5\textwidth}
\centering
\begin{overpic}[width=.85\textwidth, trim=0 0 0 0, clip]{img/raw_yield_z/AnyINT_D0_D0toKpiCuts_D0JetOptimLowJetPtv4_Charged_R040_DPtBins_JetPt_5_15_SideBand_D0_D0toKpiCuts_D0JetOptimLowJetPtv4_Charged_R040_JetZSpectrum_DPt_20_JetPt_5_15_SideBand}
\end{overpic}
\column{0.5\textwidth}
\centering
\begin{overpic}[width=.85\textwidth, trim=0 0 0 0, clip]{img/raw_yield_z/AnyINT_D0_D0toKpiCuts_D0JetOptimLowJetPtv4_Charged_R040_JetZSpectrum_DPt_20_JetPt_5_15_SideBand_BkgVsSig}
\end{overpic}
\end{columns}
\end{frame}

\begin{frame}{Raw Yield Extraction: $15<\ptchjet<30$~\GeVc}
\begin{columns}
\column{0.5\textwidth}
\centering
\begin{overpic}[width=.85\textwidth, trim=0 0 0 0, clip]{img/raw_yield_z/AnyINT_D0_D0toKpiCuts_D0JetOptimHighJetPtv4_Charged_R040_DPtBins_JetPt_15_30_SideBand_D0_D0toKpiCuts_D0JetOptimHighJetPtv4_Charged_R040_JetZSpectrum_DPt_60_JetPt_15_30_SideBand}
\end{overpic}
\column{0.5\textwidth}
\centering
\begin{overpic}[width=.85\textwidth, trim=0 0 0 0, clip]{img/raw_yield_z/AnyINT_D0_D0toKpiCuts_D0JetOptimHighJetPtv4_Charged_R040_JetZSpectrum_DPt_60_JetPt_15_30_SideBand_BkgVsSig}
\end{overpic}
\end{columns}
\end{frame}

\begin{frame}{Raw Yield Extraction: Method Comparison}
\begin{columns}
\column{0.32\textwidth}
\centering
\footnotesize
$5<\ptchjet<15$~\GeVc\\
\begin{overpic}[width=\textwidth, trim=0 0 0 0, clip]{img/raw_yield_z/AnyINT_D0_D0toKpiCuts_D0JetOptimHighJetPtv4_Charged_R040_MethodHigh_d_z_2_10_SpectraComparison}
\end{overpic}\\
\begin{overpic}[width=\textwidth, trim=0 0 0 0, clip]{img/raw_yield_z/AnyINT_D0_D0toKpiCuts_D0JetOptimHighJetPtv4_Charged_R040_MethodHigh_d_z_2_10_SpectraComparison_Ratio}
\end{overpic}
\column{0.32\textwidth}
\centering
\footnotesize
$15<\ptchjet<30$~\GeVc\\
\begin{overpic}[width=\textwidth, trim=0 0 0 0, clip]{img/raw_yield_z/AnyINT_D0_D0toKpiCuts_D0JetOptimLowJetPtv4_Charged_R040_MethodLow_d_z_2_10_SpectraComparison}
\end{overpic}\\
\begin{overpic}[width=\textwidth, trim=0 0 0 0, clip]{img/raw_yield_z/AnyINT_D0_D0toKpiCuts_D0JetOptimLowJetPtv4_Charged_R040_MethodLow_d_z_2_10_SpectraComparison_Ratio}
\end{overpic}
\column{0.34\textwidth}
\small
\begin{itemize}
\item Good agreement except in the first bin
\item The bin $0.2<z<0.4$ has very low statistics therefore the fit of the invariant mass distribution is not reliable (but should be ok with the side band method)
\end{itemize}
\end{columns}
\end{frame}

\begin{frame}{Raw Yield Extraction: Systematic Uncertainty}
\begin{columns}
\column{0.32\textwidth}
\centering
\footnotesize
$5<\ptchjet<15$~\GeVc\\
\begin{overpic}[width=\textwidth, trim=0 0 0 0, clip]{img/raw_yield_z_syst/AverageRawYieldVsDefault_low}
\end{overpic}\\
\begin{overpic}[width=\textwidth, trim=0 0 0 0, clip]{img/raw_yield_z_syst/AverageRawYieldVsDefault_Ratio_low}
\end{overpic}
\column{0.32\textwidth}
\centering
\footnotesize
$15<\ptchjet<30$~\GeVc\\
\begin{overpic}[width=\textwidth, trim=0 0 0 0, clip]{img/raw_yield_z_syst/AverageRawYieldVsDefault_high}
\end{overpic}\\
\begin{overpic}[width=\textwidth, trim=0 0 0 0, clip]{img/raw_yield_z_syst/AverageRawYieldVsDefault_Ratio_high}
\end{overpic}
\column{0.34\textwidth}
\small
\begin{itemize}
\item Comparing the default fit with the average of many trials with different fit settings (fixed parameters, shape of the background function, etc.)
\item The band is the root-mean-square of the variations
\end{itemize}
\end{columns}
\end{frame}

\begin{frame}{B Feed-Down Subtraction}
\begin{columns}
\column{0.64\textwidth}
\begin{columns}
\column{0.5\textwidth}
\centering
\footnotesize
$5<\ptchjet<15$~\GeVc\\
\begin{overpic}[width=\textwidth, trim=0 0 0 0, clip]{img/b_feed_down/JetZSpectrum_DPt_20_JetPt_5_15_DetectorLevel_JetZSpectrum_bEfficiencyMultiply_cEfficiencyDivide_canvas}
\end{overpic}
\column{0.5\textwidth}
\centering
\footnotesize
$15<\ptchjet<30$~\GeVc\\
\begin{overpic}[width=\textwidth, trim=0 0 0 0, clip]{img/b_feed_down/JetZSpectrum_DPt_60_JetPt_15_30_DetectorLevel_JetZSpectrum_bEfficiencyMultiply_cEfficiencyDivide_canvas}
\end{overpic}
\end{columns}
\footnotesize
Theory systematic uncertainty: variation of beauty mass, PDF, factorization and renormalization scales...
\column{0.36\textwidth}
\small
\begin{itemize}
\item Folded with detector response (jet momentum resolution)
\item Multiplied by the ratio of the prompt / non-prompt efficiency
\item Multiply by the luminosity and subtract from efficiency-corrected raw yields from data
\end{itemize}
\end{columns}
\end{frame}

\begin{frame}{Bayesian Unfolding: $5<\ptchjet<15$~\GeVc}
\begin{columns}[t]
\column{0.33\textwidth}
\centering
\begin{overpic}[width=.81\textwidth, trim=0 240 290 0, clip]{img/unfolding/JetZ_SideBand_DPt_20_JetPt_5_15_Response_PriorResponseTruth}
\end{overpic}\\
\begin{overpic}[width=\textwidth, trim=0 0 0 0, clip]{img/unfolding/JetZ_SideBand_DPt_20_JetPt_5_15_UnfoldingSummary_Bayes_RefoldedOverMeasured}
\end{overpic}
\column{0.33\textwidth}
\centering
\begin{overpic}[width=\textwidth, trim=0 0 0 0, clip]{img/unfolding/JetZ_SideBand_DPt_20_JetPt_5_15_UnfoldingSummary_Bayes}
\end{overpic}\\
\begin{overpic}[width=\textwidth, trim=0 0 0 0, clip]{img/unfolding/JetZ_SideBand_DPt_20_JetPt_5_15_UnfoldingSummary_Bayes_UnfoldedOverMeasured}
\end{overpic}
\column{0.34\textwidth}
\begin{overpic}[width=\textwidth, trim=0 0 0 0, clip]{img/unfolding/JetZ_SideBand_DPt_20_JetPt_5_15_Priors}
\end{overpic}\\
\footnotesize
\begin{itemize}
\item Good performance of the unfolding
\item Very small correction, up to $\sim4$\% (see left)
\item Unfolded distribution very similar to PYTHIA (see above)
\end{itemize}
\end{columns}
\end{frame}

\begin{frame}{Unfolding Stability: $5<\ptchjet<15$~\GeVc}
\begin{columns}
\column{0.3\textwidth}
\centering
\tiny 
Number of iterations (Bayes)\\
\begin{overpic}[width=\textwidth, trim=0 0 0 0, clip]{img/unfolding/JetZ_SideBand_DPt_20_JetPt_5_15_UnfoldingRegularization_Bayes_PriorResponseTruth_Ratio}
\end{overpic}\\
\tiny
Unfolding method\\
\begin{overpic}[width=\textwidth, trim=0 0 0 0, clip]{img/unfolding/JetZ_SideBand_DPt_20_JetPt_5_15_UnfoldingMethod_Ratio}
\end{overpic}
\column{0.7\textwidth}
\centering
\tiny
Pearsons' coefficients (Bayesian Method)\\
\vspace{10pt}
\begin{overpic}[width=.9\textwidth, trim=0 0 0 0, clip]{img/unfolding/JetZ_SideBand_DPt_20_JetPt_5_15_Pearson_Bayes_PriorResponseTruth}
\end{overpic}
\end{columns}
\end{frame}

\begin{frame}{Bayesian Unfolding: $15<\ptchjet<30$~\GeVc}
\begin{columns}[t]
\column{0.33\textwidth}
\centering
\begin{overpic}[width=.81\textwidth, trim=0 240 290 0, clip]{img/unfolding/JetZ_SideBand_DPt_60_JetPt_15_30_Response_PriorResponseTruth}
\end{overpic}\\
\begin{overpic}[width=\textwidth, trim=0 0 0 0, clip]{img/unfolding/JetZ_SideBand_DPt_60_JetPt_15_30_UnfoldingSummary_Bayes_RefoldedOverMeasured}
\end{overpic}
\column{0.33\textwidth}
\centering
\begin{overpic}[width=\textwidth, trim=0 0 0 0, clip]{img/unfolding/JetZ_SideBand_DPt_60_JetPt_15_30_UnfoldingSummary_Bayes}
\end{overpic}\\
\begin{overpic}[width=\textwidth, trim=0 0 0 0, clip]{img/unfolding/JetZ_SideBand_DPt_60_JetPt_15_30_UnfoldingSummary_Bayes_UnfoldedOverMeasured}
\end{overpic}
\column{0.34\textwidth}
\begin{overpic}[width=\textwidth, trim=0 0 0 0, clip]{img/unfolding/JetZ_SideBand_DPt_60_JetPt_15_30_Priors}
\end{overpic}\\
\footnotesize
\begin{itemize}
\item First bin $0.2<z<0.4$ is unstable (not enough statistics)
\item Unfolding correction is $\sim20-15$\% (see left)
\item Unfolded distribution quite different from PYTHIA (see above)
\end{itemize}
\end{columns}
\end{frame}

\begin{frame}{Unfolding Stability: $15<\ptchjet<30$~\GeVc}
\begin{columns}
\column{0.3\textwidth}
\centering
\tiny 
Number of iterations (Bayes)\\
\begin{overpic}[width=\textwidth, trim=0 0 0 0, clip]{img/unfolding/JetZ_SideBand_DPt_60_JetPt_15_30_UnfoldingRegularization_Bayes_PriorResponseTruth_Ratio}
\end{overpic}\\
\centering
\tiny
Unfolding method\\
\begin{overpic}[width=\textwidth, trim=0 0 0 0, clip]{img/unfolding/JetZ_SideBand_DPt_60_JetPt_15_30_UnfoldingMethod_Ratio}
\end{overpic}
\column{0.7\textwidth}
\centering
\tiny
Pearsons' coefficients (Bayesian Method)\\
\vspace{10pt}
\begin{overpic}[width=.9\textwidth, trim=0 0 0 0, clip]{img/unfolding/JetZ_SideBand_DPt_60_JetPt_15_30_Pearson_Bayes_PriorResponseTruth}
\end{overpic}
\end{columns}
\end{frame}

\begin{frame}{Final Spectra}
\begin{columns}
\column{0.32\textwidth}
\centering
\footnotesize
$5<\ptchjet<15$~\GeVc\\
\begin{overpic}[width=\textwidth, trim=0 0 0 0, clip]{img/final_spectra/JetZSpectrum_JetPt_5_15_Systematics}
\end{overpic}\\
\begin{overpic}[width=\textwidth, trim=0 0 0 0, clip]{img/final_spectra/CompareUncertainties_JetZSpectrum_JetPt_5_15_Systematics}
\end{overpic}
\column{0.32\textwidth}
\centering
\footnotesize
$15<\ptchjet<30$~\GeVc\\
\begin{overpic}[width=\textwidth, trim=0 0 0 0, clip]{img/final_spectra/JetZSpectrum_JetPt_15_30_Systematics}
\end{overpic}\\
\begin{overpic}[width=\textwidth, trim=0 0 0 0, clip]{img/final_spectra/CompareUncertainties_JetZSpectrum_JetPt_15_30_Systematics}
\end{overpic}
\column{0.34\textwidth}
\scriptsize
\begin{itemize}
\item The fragmentation looks very different for the two \ptchjet\ ranges
\item A similar difference is present also in PYTHIA: would be interesting to try to dig into this (difference in the dominant production mechanism?)
\item The data does not match PYTHIA, especially in the range $15<\ptchjet<30$~\GeVc
\item Very interesting results despite large uncertainties
\item Theory curves in preparation (POWHEG + PYTHIA6, etc.)
\end{itemize}
\end{columns}
\end{frame}

\section{\Dzero Jets in \pPb\ Collisions at $\s=5.02$~TeV}

\section{\Dzero-Jets in \PbPb\ Collisions at $\s=5.02$~TeV}

\section{Conclusions}

\begin{frame}[fragile]{Conclusions: \pp\ collisions at $\s=7$~TeV / 1}
\footnotesize
\begin{easylist}[itemize]
@ Minor updates for the jet cross section (preliminary approved for QM17)
@@ Bug fix in configuration of the PID response task (shown at PWG-HF 26/9)
@@ Switch to new cuts used in the re-analysis of the D mesons (shown at PWG-HF 26/9)
@@ Optmizations of the topological cuts $\rightarrow$ no improvements so far, multi-variate approach in the to-do list
@@ Minimum \ptd\ cut: $2$ instead of $3$~\GeVc\ could be an option, to be discussed
@ Distribution of the jet momentum fraction carried by the \Dzero
@@ Two ranges of jet \pt: $5<\ptchjet<15$~\GeVc\ and $15<\ptchjet<30$~\GeVc\
@@ Raw yield extraction with the side-band method in \ptd\ bins (also inv. mass fit in $z$ bins as a cross check
@@ B feed-down with POWHEG+PYTHIA6 (with systematic uncertainty)
@@ Bayesian unfolding (also SVD and bin-by-bin)
@@ Interesting result, despite large uncertainties (dominated by statistics)
\end{easylist}
\end{frame}

\begin{frame}[fragile]{Conclusions: \pp\ collisions at $\s=7$~TeV / 2}
\begin{easylist}[itemize]
@ Paper proposal: ``Measurement of the \pt-differential cross-section and fragmentation function of \Dzero-tagged charged jets in \pp\ collisions at $\s=7$~TeV with ALICE''
@ Preference for two separate papers dedicated for \pp\ and \pPb\
@@ The \pp\ measurement has a strong physics case by itself and a different message
@@ Different timelines
@@ Different collision energy
@@ Potentially different audience (HEP community probably more interested in \pp)
\end{easylist}
\end{frame}

\end{document}
