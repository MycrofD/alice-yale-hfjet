% $Header: /Users/joseph/Documents/LaTeX/beamer/solutions/conference-talks/conference-ornate-20min.en.tex,v 90e850259b8b 2007/01/28 20:48:30 tantau $

\documentclass[xcolor={usenames,dvipsnames}, aspectratio=169]{beamer}

% This file is a solution template for:

% - Talk at a conference/colloquium.
% - Talk length is about 20min.
% - Style is ornate.



% Copyright 2004 by Till Tantau <tantau@users.sourceforge.net>.
%
% In principle, this file can be redistributed and/or modified under
% the terms of the GNU Public License, version 2.
%
% However, this file is supposed to be a template to be modified
% for your own needs. For this reason, if you use this file as a
% template and not specifically distribute it as part of a another
% package/program, I grant the extra permission to freely copy and
% modify this file as you see fit and even to delete this copyright
% notice. 


\mode<presentation>
{
  \usetheme{AnnArbor}

  \setbeamercovered{transparent}
  % or whatever (possibly just delete it)
 }

\usepackage[percent]{overpic}
\usepackage[english]{babel}
\usepackage{multirow}
\usepackage[at]{easylist}
\usepackage{appendixnumberbeamer}
% or whatever

\usepackage[latin1]{inputenc}
% or whatever

\usepackage{times}
\usepackage[T1]{fontenc}
% Or whatever. Note that the encoding and the font should match. If T1
% does not look nice, try deleting the line with the fontenc.
%particles
\newcommand{\jpsi}{\rm J/$\psi$}
\newcommand{\psip}{$\psi^\prime$}
\newcommand{\jpsiDY}{\rm J/$\psi$\,/\,DY}
\newcommand{\chic}{$\chi_{\rm c}$}
\newcommand{\pip}{$\pi^{+}$}
\newcommand{\pim}{$\pi^{-}$}
\newcommand{\pizero}{$\pi^{0}$}
\newcommand{\kap}{K$^{+}$}
\newcommand{\kam}{K$^{-}$}
\newcommand{\pbar}{$\rm\overline{p}$}
\newcommand{\ccbar}{\ensuremath{\mathrm{c\overline{c}}}}
\newcommand{\bbbar}{\ensuremath{\mathrm{b\overline{b}}}}
\newcommand{\Dzero}{\ensuremath{\mathrm{D^{0}}}}
\newcommand{\Dzerobar}{\ensuremath{\mathrm{\overline{D}^{0}}}}
\newcommand{\Dpm}{\ensuremath{\mathrm{D^{\pm}}}}
\newcommand{\Ds}{\ensuremath{\mathrm{D_{s}^{\pm}}}}
\newcommand{\Dstar}{\ensuremath{\mathrm{D^{*\pm}}}}

%collision systems
\newcommand{\pp}{pp}
\newcommand{\pPb}{p--Pb}
\newcommand{\PbPb}{Pb--Pb}

%detectors
\newcommand{\ezdc}{$E_{\rm ZDC}$}

%units
\newcommand{\GeVc}{GeV/$c$}
\newcommand{\GeVcsq}{GeV/$c^2$}

%others
\newcommand{\degree}{$^{\rm o}$}
\newcommand{\s}{\ensuremath{\sqrt{s}}}
\newcommand{\snn}{\ensuremath{\sqrt{s_{\rm NN}}}}
\newcommand{\y}{\ensuremath{y}}
\newcommand{\pt}{\ensuremath{p_{\rm T}}}
\newcommand{\dedx}{d$E$/d$x$}
\newcommand{\dndy}{d$N$/d$y$}
\newcommand{\dndydpt}{${\rm d}^2N/({\rm d}y {\rm d}p_{\rm t})$}
\newcommand{\zpar}{\ensuremath{z_{||}}}
\newcommand{\zpargen}{\ensuremath{z_{||}^{\mathrm{part}}}}
\newcommand{\zpardet}{\ensuremath{z_{||}^{\mathrm{det}}}}
\newcommand{\ptchjet}{\ensuremath{p_{\mathrm{T,ch\, jet}}}}
\newcommand{\ptjet}{\ensuremath{p_{\mathrm{T,jet}}}}
\newcommand{\ptchjetgen}{\ensuremath{p_{\mathrm{T,ch\,jet}}^{\mathrm{part}}}}
\newcommand{\ptchjetdet}{\ensuremath{p_{\mathrm{T,ch\,jet}}^{\mathrm{det}}}}
\newcommand{\ptd}{\ensuremath{p_{\mathrm{T,D}}}}
\newcommand{\ptdgen}{\ensuremath{p_{\mathrm{T,D}}^{\mathrm{part}}}}
\newcommand{\ptddet}{\ensuremath{p_{\mathrm{T,D}}^{\mathrm{det}}}}
\newcommand{\antikt}{anti-\ensuremath{k_{\mathrm{T}}}}
\newcommand{\Antikt}{Anti-\ensuremath{k_{\mathrm{T}}}}
\newcommand{\kt}{\ensuremath{k_{\mathrm{T}}}}
\newcommand{\pthard}{\ensuremath{p_{\mathrm{T,hard}}}}

\AtBeginSection[]{
  \begin{frame}
  \vfill
  \centering
  \begin{beamercolorbox}[sep=8pt,center,shadow=true,rounded=true]{title}
    \usebeamerfont{title}\insertsectionhead\par%
  \end{beamercolorbox}
  \vfill
  \end{frame}
}

\title[D-Tagged Jets in \pp, \pPb\ and \PbPb] % (optional, use only with long paper titles)
{D-Tagged Jets in \pp, \pPb\ and \PbPb}

\author[Salvatore Aiola]% (optional, use only with lots of authors)
{Salvatore Aiola \\
for PAG-HFCJ}
% - Give the names in the same order as the appear in the paper.
% - Use the \inst{?} command only if the authors have different
%   affiliation.

\institute[Yale University] % (optional, but mostly needed)
{Yale University}

\date[PWG-HF - Nov. 28th, 2017] % (optional, should be abbreviation of conference name)
{PWG-HF \\
November 28th, 2017}
% - Either use conference name or its abbreviation.
% - Not really informative to the audience, more for people (including
%   yourself) who are reading the slides online

\subject{High-Energy Physics}
% This is only inserted into the PDF information catalog. Can be left
% out. 



% If you have a file called "university-logo-filename.xxx", where xxx
% is a graphic format that can be processed by latex or pdflatex,
% resp., then you can add a logo as follows:

% \pgfdeclareimage[height=0.5cm]{university-logo}{university-logo-filename}
% \logo{\pgfuseimage{university-logo}}


% If you wish to uncover everything in a step-wise fashion, uncomment
% the following command: 

%\beamerdefaultoverlayspecification{<+->}

\begin{document}

\begin{frame}
  \titlepage
\end{frame}

%\begin{frame}{Outline}
  %  \tableofcontents
%\end{frame}


% Structuring a talk is a difficult task and the following structure
% may not be suitable. Here are some rules that apply for this
% solution: 

% - Exactly two or three sections (other than the summary).
% - At *most* three subsections per section.
% - Talk about 30s to 2min per frame. So there should be between about
%   15 and 30 frames, all told.

% - A conference audience is likely to know very little of what you
%   are going to talk about. So *simplify*!
% - In a 20min talk, getting the main ideas across is hard
%   enough. Leave out details, even if it means being less precise than
%   you think necessary.
% - If you omit details that are vital to the proof/implementation,
%   just say so once. Everybody will be happy with that.

%\begin{overpic}[width=.85\textwidth, trim=0 0 0 0, clip]{img/ReflectionTemplates_DPt_NoJet_DoubleGaus_1010}
%\put(0,61){{\tiny No jet requirement}}
%\put(60,61){{\tiny \textcolor{ForestGreen}{\textbf{Used for QM17 preliminary}}}}
%\end{overpic}

%\begin{columns}
%\column{0.5\textwidth}
%\column{0.5\textwidth}
%\end{columns}

\section{\pp\ collisions at $\s=7$~TeV: \Dzero-Jet Cross Section}

\subsection{Preliminary}

\begin{frame}{Preliminary: \Dzero-Jet Cross Section}
\begin{columns}
\column{0.40\textwidth}
\begin{overpic}[width=\textwidth, trim=0 0 0 0, clip]{img/approved_figures/D0JetCrossSection_pp7TeV}
\end{overpic}\\
\small
Preliminary shown at QM17 (\href{https://indico.cern.ch/event/433345/contributions/2358064/}{poster})
\column{0.60\textwidth}
\footnotesize
\begin{itemize}
\item Jets reconstructed out of the \Dzero\ candidate (K$\pi$ channel) and all other tracks
\item Invariant mass analysis: in jet \pt\ bins (used as cross-check) or side-band method in D \pt\ bins (default)
\item MB \pp\ at $\s=7$~TeV, 2010 (LHC10b,c,d,e)
\end{itemize}
\vspace{10pt}
\large
$\rightarrow$ New
\small
\begin{itemize}
\item Attempt to improve statistical precision by optimizing the topological cuts 
\item Variation of the width of the \ptd\ bins in the side-band method
\item Reduce minimum \ptd\ cut to $2$~\GeVc\ (to be decided)
\end{itemize}
\end{columns}
\end{frame}

\subsection{Topological Cuts}

\begin{frame}{Topological Cuts}
\begin{columns}
\column{0.4\textwidth}
\small
\begin{itemize}
\item Strategy: plot $S/B$ and $S/\sqrt{S+B}$ (significance) as a function of the cut value
\item Compromise between $S/B$ and significance
\item Focus on 4 variables: $\cos(\theta_{\rm p})$, $\cos(\theta^{*})$, $d_{0}d_{0}$, DCA
\item Cut value on each variable optmized independently $\rightarrow$ limitation, for the future will try to use a multi-variate approach
\item Cuts optmized in two bins of \ptchjet\ [5, 15, 30] and 5 bins of \ptd\ [2, 4, 6, 9, 15, 30]
\end{itemize}
\column{0.6\textwidth}
\scriptsize
Example: $\cos(\theta^{*})$ for $6<\ptd<9$~\GeVc\ and $5<\ptchjet<15$~\GeVc
\begin{columns}
\column{0.5\textwidth}
\centering
$S/B$\\
\begin{overpic}[width=\textwidth, trim=0 0 0 0, clip]{img/topological_cuts/CosThetaStar_CutFraction_MB_Bkg_JetPt5_15_DPt6_9}
\end{overpic}\\
\column{0.5\textwidth}
\centering
Significance\\
\begin{overpic}[width=\textwidth, trim=0 0 0 0, clip]{img/topological_cuts/CosThetaStar_CutSignificance_MB_Bkg_JetPt5_15_DPt6_9}
\end{overpic}
\end{columns}
\centering
\tiny
Signal from charm-enhanced production in \pt-hard bins LHC15i2\{b,c,d,e\}\\
Background from minimum-bias production LHC14j4\{b,c,d,e\}\\
\vspace{5pt}
\scriptsize
Absolute normalisation of the signal is not correct, but the shape should not be affected\\
\vspace{5pt}
\footnotesize
\textcolor{ForestGreen}{green} (\textcolor{magenta}{magenta}) line\\
 $x$ such that $|\cos(\theta^{*})|<x$ maximizes \textcolor{ForestGreen}{$S/B$} (\textcolor{magenta}{$S/\sqrt{S+B}$})
\end{columns}
\end{frame}

\begin{frame}{Invariant Mass Fits: comparison new/old cuts}
\begin{columns}
\column{0.45\textwidth}
\centering
\scriptsize
Side Band Method in \ptd\ bins
\begin{columns}
\column{0.5\textwidth}
\centering
\footnotesize
Old cuts\\
\vspace{5pt}
\begin{overpic}[width=\textwidth, trim=380 170 10 0, clip]{img/topological_cuts/AnyINT_D0_D0toKpiCuts_Charged_R040_DPtBins_JetPt_5_30}
\end{overpic}
\column{0.5\textwidth}
\centering
\footnotesize
New cuts\\
\vspace{5pt}
\begin{overpic}[width=\textwidth, trim=380 170 10 0, clip]{img/topological_cuts/AnyINT_D0_D0toKpiCuts_D0JetOptimLowJetPtv4_Charged_R040_DPtBins_JetPt_5_30}
\end{overpic}
\end{columns}
\centering
Improved $S/B$ ($1.6\rightarrow3.1$) at the cost of a slightly smaller significance ($19.8\rightarrow17.8$)
\column{0.05\textwidth}
\column{0.45\textwidth}
\centering
\scriptsize
Invariant Mass Fits in \ptchjet\ bins
\begin{columns}
\column{0.5\textwidth}
\centering
\footnotesize
Old cuts\\
\vspace{5pt}
\begin{overpic}[width=\textwidth, trim=190 0 190 180, clip]{img/topological_cuts/AnyINT_D0_D0toKpiCuts_Charged_R040_JetPtBins_DPt_30}
\end{overpic}
\column{0.5\textwidth}
\centering
\footnotesize
New cuts\\
\vspace{5pt}
\begin{overpic}[width=\textwidth, trim=190 0 190 180, clip]{img/topological_cuts/AnyINT_D0_D0toKpiCuts_D0JetOptimLowJetPtv4_Charged_R040_JetPtBins_DPt_30}
\end{overpic}
\end{columns}
\centering
Improved $S/B$ ($0.4\rightarrow0.8$) with similar significance
\end{columns}
\end{frame}

\begin{frame}{Raw Yields: comparison new/old cuts}
\begin{columns}
\column{0.34\textwidth}
\begin{overpic}[width=\textwidth, trim=0 0 0 0, clip]{img/topological_cuts/Comparison_Prompt_D0toKpiCuts_D0toKpiCuts_D0JetOptimLowJetPtv4_D0toKpiCuts_D0JetOptimHighJetPtv4}
\end{overpic}\\
\vspace{-5pt}
\tiny
\textcolor{NavyBlue}{Blue: new cuts \\  FF: $5<\ptchjet<15$~\GeVc\ \\ jet \pt\ spectrum: $5<\ptchjet<30$~\GeVc}\\
\textcolor{BrickRed}{Red: new cuts \\  FF only: $15<\ptchjet<30$~\GeVc}\\
White: Old cut set\\
\scriptsize
\begin{itemize}
\item Lower efficiency at high \ptd
\item Raw yields compatible after efficiency correction with old/new cuts
\end{itemize}
\column{0.31\textwidth}
\centering
\tiny
Before efficiency correction (SB method)\\
\begin{overpic}[width=\textwidth, trim=0 0 0 0, clip]{img/topological_cuts/Charged_R040_JetPtSpectrum_DPt_30_SideBand_SpectraComparison}
\put(50,35){{\tiny \textcolor{NavyBlue}{Blue: New cut set}}}
\put(50,30){\tiny White: Old cut set}
\end{overpic}
\begin{overpic}[width=\textwidth, trim=0 0 00 0, clip]{img/topological_cuts/Charged_R040_JetPtSpectrum_DPt_30_SideBand_SpectraComparison_Ratio}
\end{overpic}
\column{0.31\textwidth}
\centering
\tiny
After efficiency correction (SB method)\\
\begin{overpic}[width=\textwidth, trim=0 0 0 0, clip]{img/topological_cuts/Charged_R040_JetPtSpectrum_DPt_30_SideBand_SpectraComparison_eff}
\end{overpic}
\begin{overpic}[width=\textwidth, trim=0 0 00 0, clip]{img/topological_cuts/Charged_R040_JetPtSpectrum_DPt_30_SideBand_SpectraComparison_Ratio_eff}
\end{overpic}
\end{columns}
\end{frame}

\begin{frame}{Uncertainty comparison (after efficiency correction)}
\begin{columns}[t]
\column{0.45\textwidth}
\centering
\small
Inv. Mass Fit in \ptchjet\ bins
\begin{columns}[t]
\column{0.6\textwidth}
\begin{overpic}[width=\textwidth, trim=0 0 0 0, clip]{img/topological_cuts/Charged_R040_JetPtSpectrum_DPt_30_InvMassFit_SpectraUncertaintyComparison_eff}
\end{overpic}\\
\begin{overpic}[width=\textwidth, trim=0 0 0 0, clip]{img/topological_cuts/ComparisonSystematic_DMesonCuts_JetPtSpectrum_DPt_30_InvMassFit_eff}
\end{overpic}
\column{0.4\textwidth}
\centering
\scriptsize
\vspace{-50pt}\\
Statistical uncertainty: improved with new cuts in the last bin\\
\vspace{30pt}
Systematic uncertainty on raw yield extr.: improved ($\rightarrow$ fits more stable), unable to calculate for the last bin with the old cuts (many trials failed)
\end{columns}
\column{0.05\textwidth}
\column{0.45\textwidth}
\centering
\small
Side-Band Method in \ptd\ bins
\begin{columns}[t]
\column{0.6\textwidth}
\begin{overpic}[width=\textwidth, trim=0 0 0 0, clip]{img/topological_cuts/Charged_R040_JetPtSpectrum_DPt_30_SideBand_SpectraUncertaintyComparison_eff}
\end{overpic}\\
\begin{overpic}[width=\textwidth, trim=0 0 0 0, clip]{img/topological_cuts/ComparisonSystematic_DMesonCuts_JetPtSpectrum_DPt_30_SideBand_eff}
\end{overpic}
\column{0.4\textwidth}
\centering
\small
\\
\vspace{-30pt}
Default method\\
\vspace{30pt}
Less obvious whether there is any improvement
\end{columns}
\end{columns}
\end{frame}

\subsection{Width of the \ptd\ bins in the SB method}

\begin{frame}{Width of the \ptd\ bins in the SB method}
\begin{columns}
\column{0.31\textwidth}
\centering
\tiny
Raw Yields\\
\begin{overpic}[width=\textwidth, trim=0 0 0 0, clip]{img/ptd_bin_width/AnyINT_D0_D0toKpiCuts_Charged_R040_DPtBinWidth_jet_pt_50_300_SpectraComparison}
\end{overpic}
\begin{overpic}[width=\textwidth, trim=0 0 0 0, clip]{img/ptd_bin_width/AnyINT_D0_D0toKpiCuts_Charged_R040_DPtBinWidth_jet_pt_50_300_SpectraComparison_Ratio}
\end{overpic}
\column{0.31\textwidth}
\centering
\tiny
Statistical Uncertainty\\
\begin{overpic}[width=\textwidth, trim=0 0 0 0, clip]{img/ptd_bin_width/AnyINT_D0_D0toKpiCuts_Charged_R040_DPtBinWidth_jet_pt_50_300_SpectraComparison_Uncertainty}
\end{overpic}\\
Systematic Uncertainty on Raw Yield Extr.
\begin{overpic}[width=\textwidth, trim=0 0 0 0, clip]{img/ptd_bin_width/D0_D0toKpiCuts_ComparisonSystematic_JetPtSpectrum_DPt_20_DPtBinWidth}
\end{overpic}
\column{0.34\textwidth}
\scriptsize
\begin{itemize}
\item Small bins: [2, 3, 4, 5, 6, 7, 8, 10, 12, 15, 20, 30] $\rightarrow$ efficiency correction applied before summing the \ptchjet\ spectra from each \ptd\ bin
\item Wide bins: [2, 4, 6, 9, 15, 30] $\rightarrow$ efficiency correction (binned in smaller bins) applied as a weight to each candidate in the invariant mass distribution
\item Raw yields identical in the two cases $\rightarrow$ confirms robustness of the side-band method
\item No significant change in the statistical/systematic uncertainties
\end{itemize}
\end{columns}
\end{frame}

\subsection{$\ptd>2$ vs. $\ptd>3$~\GeVc}

\begin{frame}{$\ptd>2$ vs. $\ptd>3$~\GeVc}
\begin{columns}
\column{0.31\textwidth}
\centering
\tiny
Raw Yields (Side-Band Method)\\
\begin{overpic}[width=\textwidth, trim=0 0 0 0, clip]{img/min_ptd_cut/AnyINT_D0_D0toKpiCuts_D0JetOptimLowJetPtv4_Charged_R040_DPtCutSideBand_jet_pt_50_300_SpectraComparison}
\end{overpic}
\begin{overpic}[width=\textwidth, trim=0 0 0 0, clip]{img/min_ptd_cut/AnyINT_D0_D0toKpiCuts_D0JetOptimLowJetPtv4_Charged_R040_DPtCutSideBand_jet_pt_50_300_SpectraComparison_Ratio}
\end{overpic}
\column{0.31\textwidth}
\centering
\tiny
Statistical Uncertainty\\
\begin{overpic}[width=\textwidth, trim=0 0 0 0, clip]{img/min_ptd_cut/AnyINT_D0_D0toKpiCuts_D0JetOptimLowJetPtv4_Charged_R040_DPtCutSideBand_jet_pt_50_300_SpectraComparison_Uncertainty}
\end{overpic}\\
Systematic Uncertainty on Raw Yield Extr.
\begin{overpic}[width=\textwidth, trim=0 0 0 0, clip]{img/min_ptd_cut/D0_D0toKpiCuts_D0JetOptimLowJetPtv4_ComparisonSystematic_JetPtSpectrum_DPtCutSideBand}
\end{overpic}
\column{0.34\textwidth}
\small
Reducing the cut from $3$ to $2$~\GeVc\ 
\begin{itemize}
\item Smaller bias on jet fragmentation at low \pt
\item Increases the relative statistical uncertainties by $\sim +10\%$ for $\ptchjet>14$~\GeVc\
\item Systematic uncertainties on raw yield extraction also a bit higher
\end{itemize}
\end{columns}
\end{frame}

\section{\pp\ collisions at $\s=7$~TeV: Fragmentation Function}

\subsection{Raw Yield Extraction}

\begin{frame}{Raw Yield Extraction: $5<\ptchjet<15$~\GeVc}
\begin{columns}
\column{0.5\textwidth}
\centering
\begin{overpic}[width=.85\textwidth, trim=0 0 0 0, clip]{img/raw_yield_z/AnyINT_D0_D0toKpiCuts_D0JetOptimLowJetPtv4_Charged_R040_DPtBins_JetPt_5_15_SideBand_D0_D0toKpiCuts_D0JetOptimLowJetPtv4_Charged_R040_JetZSpectrum_DPt_20_JetPt_5_15_SideBand}
\end{overpic}
\column{0.5\textwidth}
\centering
\begin{overpic}[width=.85\textwidth, trim=0 0 0 0, clip]{img/raw_yield_z/AnyINT_D0_D0toKpiCuts_D0JetOptimLowJetPtv4_Charged_R040_JetZSpectrum_DPt_20_JetPt_5_15_SideBand_BkgVsSig}
\end{overpic}
\end{columns}
\centering
Optmized cuts (but very similar results obtained with the old cuts in smaller bins)
\end{frame}

\begin{frame}{Raw Yield Extraction: $15<\ptchjet<30$~\GeVc}
\begin{columns}
\column{0.5\textwidth}
\centering
\begin{overpic}[width=.85\textwidth, trim=0 0 0 0, clip]{img/raw_yield_z/AnyINT_D0_D0toKpiCuts_D0JetOptimHighJetPtv4_Charged_R040_DPtBins_JetPt_15_30_SideBand_D0_D0toKpiCuts_D0JetOptimHighJetPtv4_Charged_R040_JetZSpectrum_DPt_60_JetPt_15_30_SideBand}
\end{overpic}\\
\begin{overpic}[width=.85\textwidth, trim=0 0 0 0, clip]{img/raw_yield_z/AnyINT_D0_D0toKpiCuts_D0JetOptimHighJetPtv4_Charged_R040_JetZSpectrum_DPt_60_JetPt_15_30_SideBand_BkgVsSig}
\end{overpic}
\column{0.5\textwidth}
\centering
\scriptsize
Smaller \ptd\ bins
\begin{overpic}[width=\textwidth, trim=0 0 0 0, clip]{img/raw_yield_z/AnyINT_D0_D0toKpiCuts_Charged_R040_DPtBins_JetPt_15_30_FineBins}
\end{overpic}
\end{columns}
\centering
Optmized cuts (high jet \pt\ version)\\
Too little statistics to use smaller \ptd\ bins for this \ptchjet\ range
\end{frame}

\begin{frame}{Raw Yield Extraction: Method Comparison}
\begin{columns}
\column{0.32\textwidth}
\centering
\footnotesize
$5<\ptchjet<15$~\GeVc\\
\begin{overpic}[width=\textwidth, trim=0 0 0 0, clip]{img/raw_yield_z/AnyINT_D0_D0toKpiCuts_D0JetOptimHighJetPtv4_Charged_R040_MethodHigh_d_z_2_10_SpectraComparison}
\end{overpic}\\
\begin{overpic}[width=\textwidth, trim=0 0 0 0, clip]{img/raw_yield_z/AnyINT_D0_D0toKpiCuts_D0JetOptimHighJetPtv4_Charged_R040_MethodHigh_d_z_2_10_SpectraComparison_Ratio}
\end{overpic}
\column{0.32\textwidth}
\centering
\footnotesize
$15<\ptchjet<30$~\GeVc\\
\begin{overpic}[width=\textwidth, trim=0 0 0 0, clip]{img/raw_yield_z/AnyINT_D0_D0toKpiCuts_D0JetOptimLowJetPtv4_Charged_R040_MethodLow_d_z_2_10_SpectraComparison}
\end{overpic}\\
\begin{overpic}[width=\textwidth, trim=0 0 0 0, clip]{img/raw_yield_z/AnyINT_D0_D0toKpiCuts_D0JetOptimLowJetPtv4_Charged_R040_MethodLow_d_z_2_10_SpectraComparison_Ratio}
\end{overpic}
\column{0.34\textwidth}
\small
\begin{itemize}
\item Good agreement except in the first bin
\item The bin $0.2<z<0.4$ has very low $S/B$ therefore the fit of the invariant mass distribution is not very reliable, see backup slides (but should be ok with the side band method)
\end{itemize}
\end{columns}
\end{frame}

\begin{frame}{Raw Yield Extraction: Systematic Uncertainty (Side-Band Method)}
\begin{columns}
\column{0.32\textwidth}
\centering
\footnotesize
$5<\ptchjet<15$~\GeVc\\
\begin{overpic}[width=\textwidth, trim=0 0 0 0, clip]{img/raw_yield_z_syst/AverageRawYieldVsDefault_low}
\end{overpic}\\
\begin{overpic}[width=\textwidth, trim=0 0 0 0, clip]{img/raw_yield_z_syst/AverageRawYieldVsDefault_Ratio_low}
\end{overpic}
\column{0.32\textwidth}
\centering
\footnotesize
$15<\ptchjet<30$~\GeVc\\
\begin{overpic}[width=\textwidth, trim=0 0 0 0, clip]{img/raw_yield_z_syst/AverageRawYieldVsDefault_high}
\end{overpic}\\
\begin{overpic}[width=\textwidth, trim=0 0 0 0, clip]{img/raw_yield_z_syst/AverageRawYieldVsDefault_Ratio_high}
\end{overpic}
\column{0.34\textwidth}
\small
\begin{itemize}
\item Comparing the default fit with the average of many trials with different fit settings (fixed parameters, shape of the background function, etc.)
\item The band is the root-mean-square of the variations
\end{itemize}
\end{columns}
\end{frame}

\subsection{B Feed-Down}

\begin{frame}{B Feed-Down Subtraction}
\begin{columns}
\column{0.64\textwidth}
\begin{columns}
\column{0.5\textwidth}
\centering
\footnotesize
$5<\ptchjet<15$~\GeVc\\
\begin{overpic}[width=\textwidth, trim=0 0 0 0, clip]{img/b_feed_down/JetZSpectrum_DPt_20_JetPt_5_15_DetectorLevel_JetZSpectrum_bEfficiencyMultiply_cEfficiencyDivide_canvas}
\end{overpic}
\column{0.5\textwidth}
\centering
\footnotesize
$15<\ptchjet<30$~\GeVc\\
\begin{overpic}[width=\textwidth, trim=0 0 0 0, clip]{img/b_feed_down/JetZSpectrum_DPt_60_JetPt_15_30_DetectorLevel_JetZSpectrum_bEfficiencyMultiply_cEfficiencyDivide_canvas}
\end{overpic}
\end{columns}
\footnotesize
Theory systematic uncertainty: variation of beauty mass, PDF, factorization and renormalization scales...
\column{0.36\textwidth}
\small
\begin{itemize}
\item Folded with detector response (jet momentum resolution)
\item Multiplied by the ratio of the prompt / non-prompt efficiency
\item Multiply by the luminosity and subtract from efficiency-corrected raw yields from data
\end{itemize}
\end{columns}
\end{frame}

\subsection{Unfolding}

\begin{frame}{Bayesian Unfolding: $5<\ptchjet<15$~\GeVc}
\begin{columns}[t]
\column{0.33\textwidth}
\centering
\begin{overpic}[width=.81\textwidth, trim=0 240 290 0, clip]{img/unfolding/JetZ_SideBand_DPt_20_JetPt_5_15_Response_PriorResponseTruth}
\end{overpic}\\
\begin{overpic}[width=\textwidth, trim=0 0 0 0, clip]{img/unfolding/JetZ_SideBand_DPt_20_JetPt_5_15_UnfoldingSummary_Bayes_RefoldedOverMeasured}
\end{overpic}
\column{0.33\textwidth}
\centering
\begin{overpic}[width=\textwidth, trim=0 0 0 0, clip]{img/unfolding/JetZ_SideBand_DPt_20_JetPt_5_15_UnfoldingSummary_Bayes}
\end{overpic}\\
\begin{overpic}[width=\textwidth, trim=0 0 0 0, clip]{img/unfolding/JetZ_SideBand_DPt_20_JetPt_5_15_UnfoldingSummary_Bayes_UnfoldedOverMeasured}
\end{overpic}
\column{0.34\textwidth}
\begin{overpic}[width=\textwidth, trim=0 0 0 0, clip]{img/unfolding/JetZ_SideBand_DPt_20_JetPt_5_15_Priors}
\end{overpic}\\
\footnotesize
\begin{itemize}
\item Good performance of the unfolding
\item Very small correction, up to $\sim4$\% (see left)
\item Unfolded distribution very similar to PYTHIA (see above)
\end{itemize}
\end{columns}
\end{frame}

\begin{frame}{Unfolding Stability: $5<\ptchjet<15$~\GeVc}
\begin{columns}
\column{0.3\textwidth}
\centering
\tiny 
Number of iterations (Bayes)\\
\begin{overpic}[width=\textwidth, trim=0 0 0 0, clip]{img/unfolding/JetZ_SideBand_DPt_20_JetPt_5_15_UnfoldingRegularization_Bayes_PriorResponseTruth_Ratio}
\end{overpic}\\
\tiny
Unfolding method\\
\begin{overpic}[width=\textwidth, trim=0 0 0 0, clip]{img/unfolding/JetZ_SideBand_DPt_20_JetPt_5_15_UnfoldingMethod_Ratio}
\end{overpic}
\column{0.7\textwidth}
\centering
\tiny
Pearsons' coefficients (Bayesian Method)\\
\vspace{10pt}
\begin{overpic}[width=.9\textwidth, trim=0 0 0 0, clip]{img/unfolding/JetZ_SideBand_DPt_20_JetPt_5_15_Pearson_Bayes_PriorResponseTruth}
\end{overpic}
\end{columns}
\end{frame}

\begin{frame}{Bayesian Unfolding: $15<\ptchjet<30$~\GeVc}
\begin{columns}[t]
\column{0.33\textwidth}
\centering
\begin{overpic}[width=.81\textwidth, trim=0 240 290 0, clip]{img/unfolding/JetZ_SideBand_DPt_60_JetPt_15_30_Response_PriorResponseTruth}
\end{overpic}\\
\begin{overpic}[width=\textwidth, trim=0 0 0 0, clip]{img/unfolding/JetZ_SideBand_DPt_60_JetPt_15_30_UnfoldingSummary_Bayes_RefoldedOverMeasured}
\end{overpic}
\column{0.33\textwidth}
\centering
\begin{overpic}[width=\textwidth, trim=0 0 0 0, clip]{img/unfolding/JetZ_SideBand_DPt_60_JetPt_15_30_UnfoldingSummary_Bayes}
\end{overpic}\\
\begin{overpic}[width=\textwidth, trim=0 0 0 0, clip]{img/unfolding/JetZ_SideBand_DPt_60_JetPt_15_30_UnfoldingSummary_Bayes_UnfoldedOverMeasured}
\end{overpic}
\column{0.34\textwidth}
\begin{overpic}[width=\textwidth, trim=0 0 0 0, clip]{img/unfolding/JetZ_SideBand_DPt_60_JetPt_15_30_Priors}
\end{overpic}\\
\footnotesize
\begin{itemize}
\item First bin $0.2<z<0.4$ is unstable (not enough statistics)
\item Unfolding correction is $\sim20-15$\% (see left)
\item Unfolded distribution quite different from PYTHIA (see above)
\end{itemize}
\end{columns}
\end{frame}

\begin{frame}{Unfolding Stability: $15<\ptchjet<30$~\GeVc}
\begin{columns}
\column{0.3\textwidth}
\centering
\tiny 
Number of iterations (Bayes)\\
\begin{overpic}[width=\textwidth, trim=0 0 0 0, clip]{img/unfolding/JetZ_SideBand_DPt_60_JetPt_15_30_UnfoldingRegularization_Bayes_PriorResponseTruth_Ratio}
\end{overpic}\\
\centering
\tiny
Unfolding method\\
\begin{overpic}[width=\textwidth, trim=0 0 0 0, clip]{img/unfolding/JetZ_SideBand_DPt_60_JetPt_15_30_UnfoldingMethod_Ratio}
\end{overpic}
\column{0.7\textwidth}
\centering
\tiny
Pearsons' coefficients (Bayesian Method)\\
\vspace{10pt}
\begin{overpic}[width=.9\textwidth, trim=0 0 0 0, clip]{img/unfolding/JetZ_SideBand_DPt_60_JetPt_15_30_Pearson_Bayes_PriorResponseTruth}
\end{overpic}
\end{columns}
\end{frame}

\subsection{Final Spectra}

\begin{frame}{Final Spectra}
\begin{columns}
\column{0.32\textwidth}
\centering
\footnotesize
$5<\ptchjet<15$~\GeVc\\
\begin{overpic}[width=\textwidth, trim=0 0 0 0, clip]{img/final_spectra/JetZSpectrum_JetPt_5_15_Systematics}
\end{overpic}\\
\begin{overpic}[width=\textwidth, trim=0 0 0 0, clip]{img/final_spectra/CompareUncertainties_JetZSpectrum_JetPt_5_15_Systematics}
\end{overpic}
\column{0.32\textwidth}
\centering
\footnotesize
$15<\ptchjet<30$~\GeVc\\
\begin{overpic}[width=\textwidth, trim=0 0 0 0, clip]{img/final_spectra/JetZSpectrum_JetPt_15_30_Systematics}
\end{overpic}\\
\begin{overpic}[width=\textwidth, trim=0 0 0 0, clip]{img/final_spectra/CompareUncertainties_JetZSpectrum_JetPt_15_30_Systematics}
\end{overpic}
\column{0.34\textwidth}
\scriptsize
\begin{itemize}
\item The fragmentation looks very different for the two \ptchjet\ ranges
\item A similar difference is present also in PYTHIA: would be interesting to try to dig into this (difference in the dominant production mechanism?)
\item The data does not match PYTHIA, especially in the range $15<\ptchjet<30$~\GeVc
\item Very interesting results despite large uncertainties
\item Theory curves in preparation (POWHEG + PYTHIA6, etc.)
\end{itemize}
\end{columns}
\end{frame}

\section{\Dzero-Jets in \pPb\ Collisions at $\s=5.02$~TeV}

\begin{frame}{\Dzero-Jets in \pPb\ Collisions at $\s=5.02$~TeV}
\footnotesize
\alert{$\rightarrow$ The same analysis strategy as for the preliminary D*-jet analysis in  \pPb}\\
\vspace{5pt}
\scriptsize
Data: LHC16q,t;  FAST + CENT\_woSDD: $\sim600$ M selected events, kINT7 $0-100$\% centrality\\
\vspace{10pt}
\Dzero\ candidate as a jet constituent\\
Signal charged jets are reconstructed with the \antikt\ algorithm, $R = 0.4$, with hybrid tracks\\
Tracks with $\pt > 0.15$~\GeVc and $|\eta| < 0.9$ included in the jet finding; jets axis satisfies: $|\eta_{\rm jet}| < 0.9 ? R$\\
\vspace{10pt}
CMS method for the average background calculation, 2 leading jets excluded: $\rho_{\rm CMS}= {\rm median}\left(\frac{p_{{\rm T,}j}}{A_i}\right)\cdot C, C = \frac{\rm Covered Area}{\rm Total Area}, \ptchjet^{\rm sub}=\ptchjet^{\rm raw} - \rho_{\rm CMS}\cdot A_{\rm jet}$   \\
\vspace{10pt}
PID and topological selection criteria for D mesons, with quality cuts on the D daughters\\
The same cuts are used for \Dzero-h correlations measurement - good $S/B$\\
\vspace{5pt}
\Dzero-meson \pt\ ranges tested: $1,2,3 < \ptd < 36$~\GeVc\\
\vspace{5pt}
MC: LHC17d2a\_fast\_new, anchored to 2016 \pPb\ LHC16q and LHC16t FAST productions, with enhanced charm and beauty\\
\begin{itemize}
\item Prompt and non-prompt \Dzero\ efficiencies
\item Detector response matrix using PYTHIA part of the simulation
\item Background fluctuation matrix from the data using the Random Cone method
\end{itemize}
\end{frame}

\begin{frame}{Raw Signal Extraction with Side-Band Method in \ptd\ bins}
\begin{columns}
\column{.55\textwidth}
\begin{overpic}[width=\textwidth, trim=0 0 0 0, clip]{img/pPb/invmassfits}
\end{overpic}
\column{.45\textwidth}
\alert{CENT\_woSDD+FAST}\\
\vspace{10pt}
\footnotesize
\textcolor{BrickRed}{Signal region}: $3\sigma$\\
\textcolor{NavyBlue}{Side-Band regions}: $(4\sigma,8\sigma)$ away from the peak \\
\vspace{10pt}
\scriptsize
\Dzero\ signal can be reconstruction in \pt\ range: $1- 36$~\GeVc \\
For \Dstar-jet analysis $3< \ptd <36$~\GeVc\ range used
\end{columns}
\end{frame}

\begin{frame}{Efficiency-Corrected \Dzero-jet \pt\ distribution}
\begin{columns}
\column{0.5\textwidth}
\scriptsize
\alert{Efficiencies}\\
\tiny
\begin{itemize}
\item Prompt and non-prompt \Dzero-jet efficiency (with the new Improver)
\item Jet: $|\eta| < 0.5$ ($R=0.4$) ? at gen and reco level
\item No cut on the D meson rapidity
\end{itemize}
\scriptsize
\alert{\Dzero-jet \pt} \\
\tiny
\begin{itemize}
\item Comparison of efficiency corrected \Dzero-jet pT distributions with different cuts on \Dzero: $1,2,3 < \ptd < 36$~\GeVc
\item Better relative statistical uncertainties with \Dzero\ $\pt�> 3$~\GeVc, but more bias on the jet spectrum
\item Good agreement between the two methods (backup)
\end{itemize}
\centering
\begin{overpic}[width=.6\textwidth, trim=0 0 0 0, clip]{img/pPb/efficiency_corr_spectra}
\end{overpic}
\column{.5\textwidth}
\centering
\begin{overpic}[width=.65\textwidth, trim=0 0 0 0, clip]{img/pPb/efficiency}
\end{overpic}\\
\begin{overpic}[width=.65\textwidth, trim=0 0 0 0, clip]{img/pPb/uncertainties}
\end{overpic}
\end{columns}
\end{frame}

\begin{frame}{B Feed-Down Subtraction, $\ptd>3$~\GeVc}
\begin{columns}
\column{0.34\textwidth}
\footnotesize
\begin{itemize}
\item \textcolor{ForestGreen}{Efficiency-corrected \Dzero-jet \pt\ spectrum}, before FD subtraction and unfolding
\item \textcolor{NavyBlue}{B$\rightarrow$\Dzero} from the simulation (PYTHIA+POWHEG) after folding, central value and min/max of variations
\item \textcolor{BrickRed}{Prompt jet \pt\ spectrum} with the FD uncertainty envelope, before unfolding
\end{itemize}
\vspace{10pt}
\centering
$N^{c\rightarrow\Dzero}=N^{c,b\rightarrow\Dzero}-R^{b\rightarrow\Dzero}\otimes\sum_{\ptd}\frac{\epsilon^{b\rightarrow\Dzero}(\ptd)}{\epsilon^{c\rightarrow\Dzero}(\ptd)}N^{b\rightarrow\Dzero}_{\rm POWHEG}(\ptd)$
\column{0.33\textwidth}
\centering
\begin{overpic}[width=.75\textwidth, trim=0 0 0 0, clip]{img/pPb/feed_down_corr}
\end{overpic}\\
\tiny
\alert{Background-fluctuation and \\ detector response RM for B$\rightarrow$\Dzero}\\
\begin{overpic}[width=.75\textwidth, trim=0 0 0 0, clip]{img/pPb/feed_down_resp}
\end{overpic}
\column{0.33\textwidth}
\centering
\tiny
\alert{Feed-down fraction}\\
\begin{overpic}[width=.75\textwidth, trim=0 0 0 0, clip]{img/pPb/feed_down_fract}
\end{overpic}\\
\tiny
\alert{Combined RM for B$\rightarrow$\Dzero}\\
\begin{overpic}[width=.75\textwidth, trim=0 0 0 0, clip]{img/pPb/feed_down_comb_resp}
\end{overpic}
\end{columns}
\end{frame}

\section{\Dzero-Jets in \PbPb\ Collisions at $\s=5.02$~TeV}

\begin{frame}[fragile]{B Feed-Down Subtraction}
\footnotesize
\centering
POWHEG simulation for feed-down in bins of jet \pt: good agreement with FONLL calculations
\begin{columns}
\column{0.4\textwidth}
\footnotesize
\begin{easylist}[itemize]
@ D from b RM built with LHC16k4 simulation
@ D from b reconstruction efficiency in Pb-Pb evaluated with LHC16i2a
@ \RAA\ hypothesis for beauty: $\RAA^{\rm beauty} = 2\RAA^{\rm charm}$
@@ Based on CMS measurement of non-prompt \jpsi\
@@ Uncertainties consider the variation
@@ $1 < (\RAA^{\rm beauty} / \RAA^{\rm charm}) < 3$
\end{easylist}
\column{0.3\textwidth}
\begin{overpic}[width=\textwidth, trim=0 0 0 0, clip]{img/PbPb/non_prompt_eff}
\end{overpic}
\column{0.3\textwidth}
\begin{overpic}[width=\textwidth, trim=0 0 0 0, clip]{img/PbPb/NormRM}
\end{overpic}
\end{columns}
\centering
$N^{c\rightarrow\Dzero}=N^{c,b\rightarrow\Dzero}-R^{b\rightarrow\Dzero}\otimes\sum_{\ptd}\frac{\epsilon^{b\rightarrow\Dzero}(\ptd)}{\epsilon^{c\rightarrow\Dzero}(\ptd)}N^{b\rightarrow\Dzero}_{\rm POWHEG}(\ptd)$
\end{frame}

\begin{frame}[fragile]{Feed-Down Corrected Yield vs. \ptchjet}
\begin{columns}
\column{0.5\textwidth}
\begin{overpic}[width=\textwidth, trim=0 0 0 0, clip]{img/PbPb/raw_yield}
\end{overpic}
\column{0.5\textwidth}
{\centering
\begin{overpic}[width=.4\textwidth, trim=0 0 0 0, clip]{img/PbPb/feed_down_frac}
\end{overpic}}\\
\footnotesize
\begin{easylist}[itemize]
@ \Dzero-tagged jet corrected by
@@ D-meson reconstruction efficiency
@@ Feed-down from beauty decays
@ Bins are divided by bin width
@ \Dzero\ \pt\ interval: $3 < \ptd < 20$~\GeVc
@ D-tagged jets pseudorapidity: $|\eta|<0.6$
\end{easylist}
\end{columns}
\end{frame}

\begin{frame}[fragile]{Unfolding}
\begin{columns}
\column{0.4\textwidth}
\begin{overpic}[width=\textwidth, trim=0 0 0 0, clip]{img/PbPb/unfolded_spectrum}
\end{overpic}
\column{0.6\textwidth}
\footnotesize
\begin{easylist}[itemize]
@ Bayesian Unfolding
@@ Regularization: 3 iterations
@@ Prior variation is a $1-2$\% effect
@ Unfolding corrects from
@@ Detectors finite resolution
@@ Jet background fluctuation
\end{easylist}
\begin{columns}
\column{0.5\textwidth}
\begin{overpic}[width=.7\textwidth, trim=0 0 0 0, clip]{img/PbPb/RelDiffIterations}
\end{overpic}\\
\centering
\tiny Iterations
\column{0.5\textwidth}
\begin{overpic}[width=.7\textwidth, trim=0 0 0 0, clip]{img/PbPb/refold_check}
\end{overpic}\\
\centering
\tiny Refold Check
\end{columns}
\end{columns}
\end{frame}

\begin{frame}[fragile]{Systematic Uncertainties}
\begin{columns}
\column{.4\textwidth}
\begin{overpic}[width=\textwidth, trim=0 0 0 0, clip]{img/PbPb/syst_unc}
\end{overpic}
\column{.6\textwidth}
\footnotesize
\begin{easylist}[itemize]
@ Systematic uncertainties are evaluated in measured bins ($-10$ to $35$~\GeVc), then two uncertainty bands are calculated and 
unfolded in order to obtain the uncertainties in the unfolded range. At the end these uncertainties are summed in quadrature with the systematic uncertainties from unfolding.
@@ Signal Extraction
@@ Feed-Down
@@@ POWHEG simulation
@@@ RAA hypothesis
@@ D-meson cuts*
@@ D-meson track resolution*
@@ Unfolding
\end{easylist}
\vspace{5pt}
$^*$From run 2 D-meson \RAA\ analysis
\end{columns}
\end{frame}

\begin{frame}{Systematic Uncertainties}
\centering
\begin{overpic}[width=.8\textwidth, trim=0 0 0 0, clip]{img/PbPb/syst_unc_table}
\end{overpic}
\end{frame}

\begin{frame}[fragile]{Comparison with POWHEG}
\begin{columns}
\column{.5\textwidth}
\begin{overpic}[width=.9\textwidth, trim=0 0 0 0, clip]{img/PbPb/unfolded_spectrum_comp_powheg}
\end{overpic}
\column{.5\textwidth}
\begin{easylist}[itemize]
@ Invariant yield of D-tagged jets
@@ Normalized by number of events, $\Delta\pt, \Delta\eta_{\rm jet}$ and branching ratio
@ Data is presented in red points. The red lines are the statistical uncertainties and the red boxes are the systematic uncertainties
@ Theory is presented in blue points, which are POWHEG \pp\ simulations at $5.02$~TeV scaled by \TAA\ ($0-20$\%). Blue boxes are the uncertainties from theory
\end{easylist}
\end{columns}
\end{frame}

\section{Conclusions}

\begin{frame}[fragile]{Conclusions: \pp\ collisions at $\s=7$~TeV / 1}
\footnotesize
\begin{easylist}[itemize]
@ Minor updates for the jet cross section (preliminary approved for QM17)
@@ Bug fix in configuration of the PID response task (shown at PWG-HF 26/9)
@@ Switch to new cuts used in the re-analysis of the D mesons (shown at PWG-HF 26/9)
@@ Optmizations of the topological cuts $\rightarrow$ no improvements so far, multi-variate approach in the to-do list
@@ Minimum \ptd\ cut: $2$ instead of $3$~\GeVc\ could be an option, to be discussed
@ Distribution of the jet momentum fraction carried by the \Dzero
@@ Two ranges of jet \pt: $5<\ptchjet<15$~\GeVc\ and $15<\ptchjet<30$~\GeVc\
@@ Raw yield extraction with the side-band method in \ptd\ bins (inv. mass fit in $z$ bins as a cross check)
@@ B feed-down with POWHEG+PYTHIA6 (with systematic uncertainty)
@@ Bayesian unfolding (also SVD and bin-by-bin)
@@ Main open point: vary the $z$ distribution and the prior in the unfolding (reweigh events to alter the distribution in MC)
\end{easylist}
\end{frame}

\begin{frame}[fragile]{Conclusions: \pp\ collisions at $\s=7$~TeV / 2}
\footnotesize
\begin{easylist}[itemize]
@ Paper proposal: ``Measurement of the \pt-differential cross-section and fragmentation function of \Dzero-tagged charged jets in \pp\ collisions at $\s=7$~TeV with ALICE''
@ Preference for two separate papers dedicated for \pp\ and \pPb\
@@ The \pp\ measurement has a strong physics case by itself and a different message
@@ First such measurement from ALICE, so the paper would also serve to show the techniques developed for this analysis
@@ Different timelines
@@ Different collision energy
@@ Different strategies to deal with the UE (not subtracted in \pp)
@@@ In the \pPb\ paper we will report also the \pp\ spectrum with UE subtracted in a consistent way and scaled for \s\
\end{easylist}
\end{frame}

\begin{frame}[fragile]{Conclusions: \pPb\ collisions at $\s=5.02$~TeV}
\begin{columns}
\column{0.7\textwidth}
\begin{easylist}[itemize]
@ \Dzero-jet efficiency corrected \pt\ spectrum, \Dzero\ \pt\ ranges: $2,3 - 36$~\GeVc
@ Side-Band and efficiency-scaled methods agree very well between each other
@ B feed-down is subtracted
@ Ratio of \Dstar-jet / \Dzero-jet \pt\ distributions as expected
@ Next steps:
@@ Include reflections
@@ Unfolding and comparison to POWHEG+PYTHIA
@@ Systematic uncertainties
\end{easylist}
\column{0.3\textwidth}
\begin{overpic}[width=\textwidth, trim=0 0 0 0, clip]{img/pPb/d_star_comparison}
\end{overpic}\\
\begin{overpic}[width=\textwidth, trim=0 0 0 0, clip]{img/pPb/d_star_comp_d2h}
\end{overpic}
\end{columns}
\end{frame}

\begin{frame}{Conclusions: \PbPb\ collisions at $\s=5.02$~TeV}
\begin{columns}
\column{.5\textwidth}
\begin{overpic}[width=\textwidth, trim=0 0 0 0, clip]{img/PbPb/raa}
\end{overpic}
\column{.5\textwidth}
The simulation is used as baseline for the calculation of the nuclear modification factor for jets tagged by \Dzero
\end{columns}
\end{frame}

\appendix

\section{Extra Slides}

\subsection{\Dzero-Jets in \pp\ Collisions at $\s=7$~TeV}

\begin{frame}{New Cuts}
\footnotesize
\begin{table}
\begin{tabular}{llrrrr}
\ptchjet\ (\GeVc) & \ptd\ (\GeVc) & DCA ($\mu$m) & $\cos(\theta^{*})$ & $d_{0,\pi}d_{\rm 0,K}$ ($\mu$m$^2$) & $\cos(\theta_{\rm p})$ \\
\hline \hline
\multirow{2}{*}{$5 < \ptchjet < 30$}		& $2 < \ptd < 4$ & 250 & 0.70 & -15000 & 0.84 \\
								& $4 < \ptd < 6$ & 250 & 0.70 & -10000 & 0.94 \\ 
								& $6 < \ptd < 9$ & 200 & 0.65 & -8000 & 0.97 \\ 
								& $9 < \ptd < 15$ & 150 & 0.60 & -5000 & 0.98 \\ 
								& $15 < \ptd < 30$ & 150 & 0.60 & -2000 & 0.98 \\
\hline
\multirow{2}{*}{$15 < \ptchjet < 30$}		& $6 < \ptd < 12$ & 150 & 0.50 & -8000 & 0.90 \\
								& $12 < \ptd < 30$ & 150 & 0.60 & -2000 & 0.98 \\
\hline
\end{tabular}
\end{table}
\centering
Topomatic cut: $2\sigma$ for all bins
\end{frame}

\begin{frame}{Old Cuts}
\footnotesize
\begin{table}
\begin{tabular}{llrrrr}
\ptd\ (\GeVc) & DCA ($\mu$m) & $\cos(\theta^{*})$ & $d_{0,\pi}d_{\rm 0,K}$ ($\mu$m$^2$) & $\cos(\theta_{\rm p})$ & Topomatic ($n\sigma$) \\
\hline \hline
$2 < \ptd < 3$ 		& 300 & 0.80 	& -20000 	& 0.90 	& 2.0\\
$3 < \ptd < 4$ 		& 300 & 0.80 	& -12000 	& 0.90  	& 2.0\\ 
$4 < \ptd < 5$ 		& 300 & 0.80 	& -8000 	& 0.85  	& 2.0\\ 
$5 < \ptd < 6$ 		& 300 & 0.80 	& -8000 	& 0.85  	& 3.0\\ 
$6 < \ptd < 7$ 		& 300 & 0.80 	& -8000 	& 0.85  	& 3.0\\
$7 < \ptd < 8$ 		& 300 & 0.80 	& -7000 	& 0.85  	& 3.0\\ 
$8 < \ptd < 10$ 		& 300 & 0.90 	& -5000 	& 0.85  	& 3.0\\ 
$10 < \ptd < 12$ 	& 300 & 0.90 	& -5000 	& 0.85  	& 3.0\\
$12 < \ptd < 16$ 	& 300 & -		& 10000 	& 0.85  	& 4.0\\ 
$16 < \ptd <  20$ 	& 300 & - 		& -	 	& 0.85  	& 3.0\\
$20 < \ptd <  36$ 	& 300 & - 		& -	 	& 0.85  	& 2.5\\
\hline
\end{tabular}
\end{table}
\end{frame}

\begin{frame}{Kinematic cuts and efficiency}
\begin{columns}
\column{0.5\textwidth}
\centering
\begin{overpic}[width=.9\textwidth, trim=0 0 0 0, clip]{img/unfolding/backup/D0toKpiCuts_D0JetOptimLowJetPtv4_Comparison_KineCuts_KinCuts}
\end{overpic}
\column{0.5\textwidth}
\centering
\begin{overpic}[width=.9\textwidth, trim=0 0 0 0, clip]{img/unfolding/backup/D0toKpiCuts_D0JetOptimLowJetPtv4_Comparison_KineCuts_KinCuts_Ratio}
\end{overpic}
\end{columns}
\tiny
\begin{itemize}
\item Denominator of the efficiency is the same for both cases: all generated \Dzero-jets within the accepted kinematic ranges
\item Numerator for white points: generated \Dzero-jets within the accepted kinematic ranges matched to reconstructed \Dzero-jets (which not necessarily within the accepted kinematic ranges)
\item Numerator for blue points: generated \Dzero-jets (not necessarily within the accepted kinematic ranges) matched to reconstructed \Dzero-jets within the accepted kinematic ranges
\end{itemize}
In other words the difference is whether in the numerator the cuts are applied at generated level (white) or reconstructed level (blue). In the first case we get the simple efficiency, 
i.e. the probability of reconstructing a \Dzero-jet with certain kinematic properties; in the second case we fold in the acceptance factor (feed-in and feed-out of the accepted range).
\end{frame}

\begin{frame}{Invariant Mass Fits in $z$ bins: $5<\ptchjet<15$~\GeVc}
\centering
\begin{overpic}[width=.75\textwidth, trim=0 0 0 0, clip]{img/raw_yield_z/backup/AnyINT_D0_D0toKpiCuts_D0JetOptimLowJetPtv4_Charged_R040_JetZBins_DPt_20_JetPt_5_15}
\end{overpic}
\end{frame}

\begin{frame}{Invariant Mass Fits in $z$ bins (eff. corrected): $5<\ptchjet<15$~\GeVc}
\centering
\begin{overpic}[width=.75\textwidth, trim=0 0 0 0, clip]{img/raw_yield_z/backup/AnyINT_D0_D0toKpiCuts_D0JetOptimLowJetPtv4_Charged_R040_JetZBins_DPt_20_JetPt_5_15_eff}
\end{overpic}
\end{frame}

\begin{frame}{Invariant Mass Fits in $z$ bins: $15<\ptchjet<30$~\GeVc}
\centering
\begin{overpic}[width=.75\textwidth, trim=0 0 0 0, clip]{img/raw_yield_z/backup/AnyINT_D0_D0toKpiCuts_D0JetOptimHighJetPtv4_Charged_R040_JetZBins_DPt_60_JetPt_15_30}
\end{overpic}
\end{frame}

\begin{frame}{Invariant Mass Fits in $z$ bins (eff. corrected): $15<\ptchjet<30$~\GeVc}
\centering
\begin{overpic}[width=.75\textwidth, trim=0 0 0 0, clip]{img/raw_yield_z/backup/AnyINT_D0_D0toKpiCuts_D0JetOptimHighJetPtv4_Charged_R040_JetZBins_DPt_60_JetPt_15_30_eff}
\end{overpic}
\end{frame}

\subsection{\Dzero-Jets in \pPb\ Collisions at $\s=5.02$~TeV}

\begin{frame}{Raw \Dzero-jet \pt\ distribution (Side-Band Method)}
\begin{columns}
\column{0.7\textwidth}
\begin{overpic}[width=\textwidth, trim=0 0 0 0, clip]{img/pPb/side_band_sub}
\end{overpic}
\column{0.3\textwidth}
\footnotesize
CENT\_woSDD+FAST\\
\begin{itemize}
\item \textcolor{ForestGreen}{Efficiency-corrected \Dzero-jet \pt\ spectrum}, before FD subtraction and unfolding
\item \textcolor{NavyBlue}{B$\rightarrow$\Dzero} from the simulation (PYTHIA+POWHEG) after folding, central value and min/max of variations
\item \textcolor{BrickRed}{Prompt jet \pt\ spectrum} with the FD uncertainty envelope, before unfolding
\end{itemize}
\end{columns}
\end{frame}

\begin{frame}{Comparison of \Dzero-jet \pt\ spectra with two methods}
\begin{columns}
\column{0.33\textwidth}
\footnotesize
\begin{itemize}
\item Comparison of the side-band and efficiency-corrected methods, with $1,2,3 < \ptd < 36$~\GeVc
\item Good agreement with $\ptd > 2,3$~\GeVc
\item More background influence when going down to $1$~\GeVc\ in \Dzero\ \pt
\end{itemize}
\column{0.34\textwidth}
\begin{overpic}[width=\textwidth, trim=0 0 0 0, clip]{img/pPb/comp_method}
\end{overpic}
\column{0.33\textwidth}
\begin{overpic}[width=\textwidth, trim=0 0 0 0, clip]{img/pPb/comp_method_ratio}
\end{overpic}
\end{columns}
\end{frame}

\begin{frame}{\Dzero-jet signal in jet \pt\ bins, $\ptd > 1$~\GeVc}
\centering
\scriptsize
Efficiency corrected\\
\vspace{5pt}
\begin{overpic}[width=.65\textwidth, trim=0 0 0 0, clip]{img/pPb/inv_mass_fit_jetpt_1}
\end{overpic}
\end{frame}

\begin{frame}{\Dzero-jet signal in jet \pt\ bins, $\ptd > 2$~\GeVc}
\centering
\scriptsize
Efficiency corrected\\
\vspace{5pt}
\begin{overpic}[width=.65\textwidth, trim=0 0 0 0, clip]{img/pPb/inv_mass_fit_jetpt_2}
\end{overpic}
\end{frame}

\begin{frame}{\Dzero-jet signal in jet \pt\ bins, $\ptd > 3$~\GeVc}
\centering
\scriptsize
Efficiency corrected\\
\vspace{5pt}
\begin{overpic}[width=.65\textwidth, trim=0 0 0 0, clip]{img/pPb/inv_mass_fit_jetpt_3}
\end{overpic}
\end{frame}

\end{document}
