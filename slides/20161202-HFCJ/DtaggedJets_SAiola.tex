% $Header: /Users/joseph/Documents/LaTeX/beamer/solutions/conference-talks/conference-ornate-20min.en.tex,v 90e850259b8b 2007/01/28 20:48:30 tantau $

\documentclass[xcolor={usenames,dvipsnames}]{beamer}

% This file is a solution template for:

% - Talk at a conference/colloquium.
% - Talk length is about 20min.
% - Style is ornate.



% Copyright 2004 by Till Tantau <tantau@users.sourceforge.net>.
%
% In principle, this file can be redistributed and/or modified under
% the terms of the GNU Public License, version 2.
%
% However, this file is supposed to be a template to be modified
% for your own needs. For this reason, if you use this file as a
% template and not specifically distribute it as part of a another
% package/program, I grant the extra permission to freely copy and
% modify this file as you see fit and even to delete this copyright
% notice. 


\mode<presentation>
{
  \usetheme{AnnArbor}
  % or ...

  \setbeamercovered{transparent}
  % or whatever (possibly just delete it)
 }

\usepackage[percent]{overpic}

\usepackage[english]{babel}
% or whatever

\usepackage[latin1]{inputenc}
% or whatever

\usepackage{times}
\usepackage[T1]{fontenc}
% Or whatever. Note that the encoding and the font should match. If T1
% does not look nice, try deleting the line with the fontenc.
%particles
\newcommand{\jpsi}{\rm J/$\psi$}
\newcommand{\psip}{$\psi^\prime$}
\newcommand{\jpsiDY}{\rm J/$\psi$\,/\,DY}
\newcommand{\chic}{$\chi_{\rm c}$}
\newcommand{\pip}{$\pi^{+}$}
\newcommand{\pim}{$\pi^{-}$}
\newcommand{\pizero}{$\pi^{0}$}
\newcommand{\kap}{K$^{+}$}
\newcommand{\kam}{K$^{-}$}
\newcommand{\pbar}{$\rm\overline{p}$}
\newcommand{\ccbar}{\ensuremath{\mathrm{c\overline{c}}}}
\newcommand{\bbbar}{\ensuremath{\mathrm{b\overline{b}}}}
\newcommand{\Dzero}{\ensuremath{\mathrm{D^{0}}}}
\newcommand{\Dzerobar}{\ensuremath{\mathrm{\overline{D}^{0}}}}
\newcommand{\Dpm}{\ensuremath{\mathrm{D^{\pm}}}}
\newcommand{\Ds}{\ensuremath{\mathrm{D_{s}^{\pm}}}}
\newcommand{\Dstar}{\ensuremath{\mathrm{D^{*\pm}}}}

%collision systems
\newcommand{\pp}{pp}
\newcommand{\pPb}{p--Pb}
\newcommand{\PbPb}{Pb--Pb}

%detectors
\newcommand{\ezdc}{$E_{\rm ZDC}$}

%units
\newcommand{\GeVc}{GeV/$c$}
\newcommand{\GeVcsq}{GeV/$c^2$}

%others
\newcommand{\degree}{$^{\rm o}$}
\newcommand{\s}{\ensuremath{\sqrt{s}}}
\newcommand{\snn}{\ensuremath{\sqrt{s_{\rm NN}}}}
\newcommand{\y}{\ensuremath{y}}
\newcommand{\pt}{\ensuremath{p_{\rm T}}}
\newcommand{\dedx}{d$E$/d$x$}
\newcommand{\dndy}{d$N$/d$y$}
\newcommand{\dndydpt}{${\rm d}^2N/({\rm d}y {\rm d}p_{\rm t})$}
\newcommand{\zpar}{\ensuremath{z_{||}}}
\newcommand{\zpargen}{\ensuremath{z_{||}^{\mathrm{part}}}}
\newcommand{\zpardet}{\ensuremath{z_{||}^{\mathrm{det}}}}
\newcommand{\ptchjet}{\ensuremath{p_{\mathrm{T,ch\, jet}}}}
\newcommand{\ptjet}{\ensuremath{p_{\mathrm{T,jet}}}}
\newcommand{\ptchjetgen}{\ensuremath{p_{\mathrm{T,ch\,jet}}^{\mathrm{truth}}}}
\newcommand{\ptchjetdet}{\ensuremath{p_{\mathrm{T,ch\,jet}}^{\mathrm{reco}}}}
\newcommand{\ptd}{\ensuremath{p_{\mathrm{T,D}}}}
\newcommand{\ptdgen}{\ensuremath{p_{\mathrm{T,D}}^{\mathrm{truth}}}}
\newcommand{\ptddet}{\ensuremath{p_{\mathrm{T,D}}^{\mathrm{reco}}}}
\newcommand{\antikt}{anti-\ensuremath{k_{\mathrm{T}}}}
\newcommand{\kt}{\ensuremath{k_{\mathrm{T}}}}
\newcommand{\pthard}{\ensuremath{p_{\mathrm{T,hard}}}}

\title[B feed-down correction] % (optional, use only with long paper titles)
{Some thoughts on the B feed-down correction}

\author[Salvatore Aiola]% (optional, use only with lots of authors)
{Salvatore Aiola}
% - Give the names in the same order as the appear in the paper.
% - Use the \inst{?} command only if the authors have different
%   affiliation.

\institute[Yale University] % (optional, but mostly needed)
{Yale University}

\date[PAG-HFCJ - Dec. 2nd, 2016] % (optional, should be abbreviation of conference name)
{PAG-HFCJ \\
December 2nd, 2016}
% - Either use conference name or its abbreviation.
% - Not really informative to the audience, more for people (including
%   yourself) who are reading the slides online

\subject{High-Energy Physics}
% This is only inserted into the PDF information catalog. Can be left
% out. 



% If you have a file called "university-logo-filename.xxx", where xxx
% is a graphic format that can be processed by latex or pdflatex,
% resp., then you can add a logo as follows:

% \pgfdeclareimage[height=0.5cm]{university-logo}{university-logo-filename}
% \logo{\pgfuseimage{university-logo}}


% If you wish to uncover everything in a step-wise fashion, uncomment
% the following command: 

%\beamerdefaultoverlayspecification{<+->}


\begin{document}

\begin{frame}
  \titlepage
\end{frame}

%\begin{frame}{Outline}
  %  \tableofcontents
  %\end{frame}


% Structuring a talk is a difficult task and the following structure
% may not be suitable. Here are some rules that apply for this
% solution: 

% - Exactly two or three sections (other than the summary).
% - At *most* three subsections per section.
% - Talk about 30s to 2min per frame. So there should be between about
%   15 and 30 frames, all told.

% - A conference audience is likely to know very little of what you
%   are going to talk about. So *simplify*!
% - In a 20min talk, getting the main ideas across is hard
%   enough. Leave out details, even if it means being less precise than
%   you think necessary.
% - If you omit details that are vital to the proof/implementation,
%   just say so once. Everybody will be happy with that.

\section{Introduction}

\begin{frame}{Introduction}
\begin{itemize}
\item Method 1: apply a weight $\omega(\ptchjet,\ptd)=\frac{N^{\rm c\rightarrow\Dzero}_{\rm POWHEG}(\ptchjet,\ptd)}{N^{\rm c,b\rightarrow\Dzero}_{\rm POWHEG}(\ptchjet,\ptd)}$
\item Method 2: subtract $N^{\rm b\rightarrow\Dzero}_{\rm POWHEG}(\ptchjet)$ from the measured spectrum (before or after unfolding)
\end{itemize}
\end{frame}

\section{Method 1}

\begin{frame}{Method 1}
\begin{center}
Apply a weight $\omega(\ptchjet,\ptd)=\frac{N^{\rm c\rightarrow\Dzero}_{\rm POWHEG}(\ptchjet,\ptd)}{N^{\rm c,b\rightarrow\Dzero}_{\rm POWHEG}(\ptchjet,\ptd)}$
\end{center}
\begin{itemize}
\item More precisely: \\
$\omega(\ptjet,\ptd)=\frac{R_{\rm det}^{\rm c\rightarrow\Dzero}(\ptjet) \otimes N^{\rm c\rightarrow\Dzero}_{\rm POWHEG}(\ptjet,\ptd)}
{\frac{\epsilon^{\rm b\rightarrow\Dzero}(\ptd)}{\epsilon^{\rm c\rightarrow\Dzero}(\ptd)} R_{\rm det}^{\rm b\rightarrow\Dzero}(\ptjet) \otimes N^{\rm b\rightarrow\Dzero}_{\rm POWHEG}(\ptjet,\ptd) + 
R_{\rm det}^{\rm c\rightarrow\Dzero}(\ptjet) \otimes N^{\rm c\rightarrow\Dzero}_{\rm POWHEG}(\ptjet,\ptd)}$,
where $\epsilon^{\rm b\rightarrow\Dzero}(\ptd)$ and $\epsilon^{\rm c\rightarrow\Dzero}(\ptd)$ are reconstruction efficiencies
\item Applied as a weight before the signal extraction (same as done for the efficiency correction)
\item Pros: small(er) dependence on the jet fragmentation
\item Cons: depends both on the charm and beauty simulations
\end{itemize}
\end{frame}

\begin{frame}{Method 2}
\begin{center}
Subtract $N^{\rm b\rightarrow\Dzero}_{\rm POWHEG}(\ptchjet)$ from the measured spectrum
\end{center}
\footnotesize
\begin{itemize}
\item More precisely:
\begin{itemize}
\item Before unfolding: \\
$N^{\rm c\rightarrow\Dzero}_{\rm corr}(\ptchjet) = R_{\rm det}^{\rm c\rightarrow\Dzero}(\ptchjet)^{-1} \otimes \Big[N^{\rm c,b\rightarrow\Dzero}_{\rm raw}(\ptchjet) - $\\
$R_{\rm det}^{\rm b\rightarrow\Dzero}(\ptchjet) \otimes \sum_{\ptd} \frac{\epsilon^{\rm b\rightarrow\Dzero}(\ptd)}{\epsilon^{\rm c\rightarrow\Dzero}(\ptd)} N^{\rm b\rightarrow\Dzero}_{\rm POWHEG}(\ptd,\ptchjet)\Big]$
\item After unfolding: \\
$N^{\rm c\rightarrow\Dzero}_{\rm corr}(\ptchjet) = R_{\rm det}^{\rm c\rightarrow\Dzero}(\ptchjet)^{-1} \otimes N^{\rm c,b\rightarrow\Dzero}_{\rm raw}(\ptchjet) - $ \\ 
$R_{\rm det}^{\rm c\rightarrow\Dzero}(\ptchjet)^{-1} \otimes R_{\rm det}^{\rm b\rightarrow\Dzero}(\ptchjet) \otimes \sum_{\ptd} \frac{\epsilon^{\rm b\rightarrow\Dzero}(\ptd)}{\epsilon^{\rm c\rightarrow\Dzero}(\ptd)} N^{\rm b\rightarrow\Dzero}_{\rm POWHEG}(\ptd,\ptchjet)$, \\
where $N^{\rm c,b\rightarrow\Dzero}_{\rm raw}(\ptchjet)$ is the (efficiency-corrected) raw yield extracted using the inv. mass fit or the SB method.
\end{itemize}
\item Pros: depends only on the beauty simulation
\item Cons: we need to integrate out the \ptd\ dependence of the B feed-down
\end{itemize}
\end{frame}

\section*{Summary}

\begin{frame}{Conclusions}
\begin{itemize}
\item We may want to explore both methods, unless there is something wrong that I overlooked in one of them
\item Do we have an \emph{a priori} preference for one of the two methods?
\item For method 2 (which was discussed in the last PAG meeting by Andrea): in principle unfolding should be a linear operator, 
however the regularization may indeed bring some "non-linear" effects so better to apply the correction before unfolding
\item However the most compelling reason against method 2 is that it requires to unfold separately also the POWHEG jet spectrum
(as seen from the last piece of the equation in the previous slide)
\end{itemize}
\end{frame}

\end{document}
