% $Header: /Users/joseph/Documents/LaTeX/beamer/solutions/conference-talks/conference-ornate-20min.en.tex,v 90e850259b8b 2007/01/28 20:48:30 tantau $

\documentclass[xcolor={usenames,dvipsnames}]{beamer}

% This file is a solution template for:

% - Talk at a conference/colloquium.
% - Talk length is about 20min.
% - Style is ornate.



% Copyright 2004 by Till Tantau <tantau@users.sourceforge.net>.
%
% In principle, this file can be redistributed and/or modified under
% the terms of the GNU Public License, version 2.
%
% However, this file is supposed to be a template to be modified
% for your own needs. For this reason, if you use this file as a
% template and not specifically distribute it as part of a another
% package/program, I grant the extra permission to freely copy and
% modify this file as you see fit and even to delete this copyright
% notice. 


\mode<presentation>
{
  \usetheme{AnnArbor}
  % or ...

  \setbeamercovered{transparent}
  % or whatever (possibly just delete it)
 }

\usepackage[percent]{overpic}
\usepackage[english]{babel}
% or whatever

\usepackage[latin1]{inputenc}
% or whatever

\usepackage{times}
\usepackage[T1]{fontenc}
% Or whatever. Note that the encoding and the font should match. If T1
% does not look nice, try deleting the line with the fontenc.
%particles
\newcommand{\jpsi}{\rm J/$\psi$}
\newcommand{\psip}{$\psi^\prime$}
\newcommand{\jpsiDY}{\rm J/$\psi$\,/\,DY}
\newcommand{\chic}{$\chi_{\rm c}$}
\newcommand{\pip}{$\pi^{+}$}
\newcommand{\pim}{$\pi^{-}$}
\newcommand{\pizero}{$\pi^{0}$}
\newcommand{\kap}{K$^{+}$}
\newcommand{\kam}{K$^{-}$}
\newcommand{\pbar}{$\rm\overline{p}$}
\newcommand{\ccbar}{\ensuremath{\mathrm{c\overline{c}}}}
\newcommand{\bbbar}{\ensuremath{\mathrm{b\overline{b}}}}
\newcommand{\Dzero}{\ensuremath{\mathrm{D^{0}}}}
\newcommand{\Dzerobar}{\ensuremath{\mathrm{\overline{D}^{0}}}}
\newcommand{\Dpm}{\ensuremath{\mathrm{D^{\pm}}}}
\newcommand{\Ds}{\ensuremath{\mathrm{D_{s}^{\pm}}}}
\newcommand{\Dstar}{\ensuremath{\mathrm{D^{*\pm}}}}

%collision systems
\newcommand{\pp}{pp}
\newcommand{\pPb}{p--Pb}
\newcommand{\PbPb}{Pb--Pb}

%detectors
\newcommand{\ezdc}{$E_{\rm ZDC}$}

%units
\newcommand{\GeVc}{GeV/$c$}
\newcommand{\GeVcsq}{GeV/$c^2$}

%others
\newcommand{\degree}{$^{\rm o}$}
\newcommand{\s}{\ensuremath{\sqrt{s}}}
\newcommand{\snn}{\ensuremath{\sqrt{s_{\rm NN}}}}
\newcommand{\y}{\ensuremath{y}}
\newcommand{\pt}{\ensuremath{p_{\rm T}}}
\newcommand{\dedx}{d$E$/d$x$}
\newcommand{\dndy}{d$N$/d$y$}
\newcommand{\dndydpt}{${\rm d}^2N/({\rm d}y {\rm d}p_{\rm t})$}
\newcommand{\zpar}{\ensuremath{z_{||}}}
\newcommand{\zpargen}{\ensuremath{z_{||}^{\mathrm{part}}}}
\newcommand{\zpardet}{\ensuremath{z_{||}^{\mathrm{det}}}}
\newcommand{\ptchjet}{\ensuremath{p_{\mathrm{T,ch\, jet}}}}
\newcommand{\ptjet}{\ensuremath{p_{\mathrm{T,jet}}}}
\newcommand{\ptchjetgen}{\ensuremath{p_{\mathrm{T,ch\,jet}}^{\mathrm{truth}}}}
\newcommand{\ptchjetdet}{\ensuremath{p_{\mathrm{T,ch\,jet}}^{\mathrm{reco}}}}
\newcommand{\ptd}{\ensuremath{p_{\mathrm{T,D}}}}
\newcommand{\ptdgen}{\ensuremath{p_{\mathrm{T,D}}^{\mathrm{truth}}}}
\newcommand{\ptddet}{\ensuremath{p_{\mathrm{T,D}}^{\mathrm{reco}}}}
\newcommand{\antikt}{anti-\ensuremath{k_{\mathrm{T}}}}
\newcommand{\kt}{\ensuremath{k_{\mathrm{T}}}}
\newcommand{\pthard}{\ensuremath{p_{\mathrm{T,hard}}}}

\AtBeginSection[]{
  \begin{frame}
  \vfill
  \centering
  \begin{beamercolorbox}[sep=8pt,center,shadow=true,rounded=true]{title}
    \usebeamerfont{title}\insertsectionhead\par%
  \end{beamercolorbox}
  \vfill
  \end{frame}
}

\title[D-Tagged Jets in \pp] % (optional, use only with long paper titles)
{D-Tagged Jets in \pp\ Collisions}

\author[Salvatore Aiola]% (optional, use only with lots of authors)
{Salvatore Aiola}
% - Give the names in the same order as the appear in the paper.
% - Use the \inst{?} command only if the authors have different
%   affiliation.

\institute[Yale University] % (optional, but mostly needed)
{Yale University}

\date[PAG-HFCJ - Nov. 2nd, 2017] % (optional, should be abbreviation of conference name)
{PAG-HFCJ \\
November 2nd, 2017}
% - Either use conference name or its abbreviation.
% - Not really informative to the audience, more for people (including
%   yourself) who are reading the slides online

\subject{High-Energy Physics}
% This is only inserted into the PDF information catalog. Can be left
% out. 



% If you have a file called "university-logo-filename.xxx", where xxx
% is a graphic format that can be processed by latex or pdflatex,
% resp., then you can add a logo as follows:

% \pgfdeclareimage[height=0.5cm]{university-logo}{university-logo-filename}
% \logo{\pgfuseimage{university-logo}}


% If you wish to uncover everything in a step-wise fashion, uncomment
% the following command: 

%\beamerdefaultoverlayspecification{<+->}


\begin{document}

\begin{frame}
  \titlepage
\end{frame}

%\begin{frame}{Outline}
 %   \tableofcontents
 %\end{frame}


% Structuring a talk is a difficult task and the following structure
% may not be suitable. Here are some rules that apply for this
% solution: 

% - Exactly two or three sections (other than the summary).
% - At *most* three subsections per section.
% - Talk about 30s to 2min per frame. So there should be between about
%   15 and 30 frames, all told.

% - A conference audience is likely to know very little of what you
%   are going to talk about. So *simplify*!
% - In a 20min talk, getting the main ideas across is hard
%   enough. Leave out details, even if it means being less precise than
%   you think necessary.
% - If you omit details that are vital to the proof/implementation,
%   just say so once. Everybody will be happy with that.

%\begin{overpic}[width=.85\textwidth, trim=0 0 0 0, clip]{img/ReflectionTemplates_DPt_NoJet_DoubleGaus_1010}
%\put(0,61){{\tiny No jet requirement}}
%\put(60,61){{\tiny \textcolor{ForestGreen}{\textbf{Used for QM17 preliminary}}}}
%\end{overpic}

%\begin{columns}
%\column{0.5\textwidth}
%\column{0.5\textwidth}
%\end{columns}

\section{Introduction}

\subsection{Raw Yield}

\begin{frame}{Raw Yield Extraction}
\textcolor{red}{From HFCJ September 20th}
\begin{columns}
\column{0.5\textwidth}
\begin{overpic}[width=\textwidth, trim=0 0 0 0, clip]{img/raw_yield/AnyINT_D0_Charged_R040_d_z_SpectraComparison}
\end{overpic}
\column{0.5\textwidth}
\begin{overpic}[width=\textwidth, trim=0 0 0 0, clip]{img/raw_yield/AnyINT_D0_Charged_R040_d_z_SpectraComparison_Ratio}
\end{overpic}
\end{columns}
\vspace{-5pt}
\begin{itemize}
\item The two methods agree quite well except in the bin $0.4<\zpar<0.6$
\item Invariant mass for the bin $0.4<\zpar<0.6$ has a larger width (see backup slide)
\end{itemize}
\end{frame}

\begin{frame}{Multi-Trial Result}
\textcolor{red}{From HFCJ September 20th}
\begin{columns}
\column{0.5\textwidth}
\begin{overpic}[width=\textwidth, trim=0 0 0 0, clip]{img/raw_yield_unc/CompareRawYieldUncVariations_AfterDbinSum}
\end{overpic}
\column{0.5\textwidth}
\begin{overpic}[width=\textwidth, trim=0 0 0 0, clip]{img/raw_yield_unc/CompareRawYieldUncVariations_AfterDbinSum_Ratio}
\end{overpic}
\end{columns}
\vspace{-5pt}
\begin{itemize}
\item Code updated to work with \zpar
\item Same strategy used for the jet \pt\ spectrum
\end{itemize}
\end{frame}

\begin{frame}{Reflection Template Variations}
\textcolor{red}{From HFCJ September 20th}
\begin{columns}
\column{0.5\textwidth}
\begin{overpic}[width=\textwidth, trim=0 0 0 0, clip]{img/raw_yield_unc/SideBandReflectionVariationComparison}
\end{overpic}
\column{0.5\textwidth}
\begin{overpic}[width=\textwidth, trim=0 0 0 0, clip]{img/raw_yield_unc/SideBandReflectionVariationComparison_Ratio}
\end{overpic}
\end{columns}
\vspace{-5pt}
\begin{itemize}
\item Code updated to work with \zpar
\item Same strategy used for the jet \pt\ spectrum
\end{itemize}
\end{frame}

\subsection{B Feed-Down}

\begin{frame}{Feed-Down Cross-Section with Systematics}
\textcolor{red}{From HFCJ September 20th}
\begin{columns}
\column{0.5\textwidth}
\begin{overpic}[width=\textwidth, trim=0 0 0 0, clip]{img/feed-down/BFeedDown_1505317519_1242_JetZSpectrum_DPt_30_JetPt_5_30_GeneratorLevel_JetZSpectrum_canvas}
\end{overpic}
\column{0.5\textwidth}
\begin{overpic}[width=\textwidth, trim=0 0 0 0, clip]{img/feed-down/BFeedDown_1505317519_1242_JetZSpectrum_DPt_30_JetPt_5_30_DetectorLevel_JetZSpectrum_bEfficiencyMultiply_cEfficiencyDivide_canvas}
\end{overpic}
\end{columns}
\vspace{-5pt}
\begin{itemize}
\item Left: generator level
\item Right: detector level (divided by prompt efficiency)
\end{itemize}
\end{frame}

\subsection{Conclusions (September 20th)}

\begin{frame}{Conclusions (September 20th)}
\begin{itemize}
\item \textcolor<2>{gray}{The full analysis chain has been adapted to work with \zpar}
\item \textcolor<2>{gray}{Two main open points:} \alert{\textbf<2>{optimisation of kinematic/topological cuts}} \textcolor<2>{gray}{and \alert<1>{unfolding}}
\end{itemize}
\end{frame}

\section{Topological Cuts}

\subsection{Strategy}

\begin{frame}{Sig/Bkg Efficiency vs. Cut Variations: $\cos(\theta_{\rm p})$}
\begin{columns}
\column{0.7\textwidth}
\begin{overpic}[width=1.1\textwidth, trim=0 0 0 0, clip]{img/topo_cuts/CosPointing_CutEfficiency_copy_JetPt5_15_DPt3_6}
\put(12,10){\tiny $5 < \ptchjet < 15$~\GeVc\ and  $3 < \ptd < 6$~\GeVc}
\end{overpic}
\column{0.3\textwidth}
\footnotesize
\textbf{Signal} and \textbf{\textcolor{NavyBlue}{Non-Prompt}} from LHC15i2\{b,c,d,e\} (charm-enhanced with \pt-hard bins)\\
\vspace{10pt}
\textbf{\textcolor{BrickRed}{Background}} from LHC14j4\{b,c,d,e\} (minimum-bias) \\
\vspace{10pt}
\textbf{For each bin of \ptchjet\ and \ptd}
\end{columns}
\end{frame}

\begin{frame}{Sig/Bkg Efficiency vs. Cut Variations: $\cos(\theta^{*})$}
\begin{columns}
\column{0.7\textwidth}
\begin{overpic}[width=1.1\textwidth, trim=0 0 0 0, clip]{img/topo_cuts/CosThetaStar_CutEfficiency_copy_JetPt15_30_DPt6_9}
\put(12,10){\tiny $5 < \ptchjet < 15$~\GeVc\ and  $6 < \ptd < 9$~\GeVc}
\end{overpic}
\column{0.3\textwidth}
\footnotesize
\textbf{Signal} and \textbf{\textcolor{NavyBlue}{Non-Prompt}} from LHC15i2\{b,c,d,e\} (charm-enhanced with \pt-hard bins)\\
\vspace{10pt}
\textbf{\textcolor{BrickRed}{Background}} from LHC14j4\{b,c,d,e\} (minimum-bias) \\
\vspace{10pt}
\textbf{For each bin of \ptchjet\ and \ptd}
\end{columns}
\end{frame}

\begin{frame}{Sig/Bkg Efficiency vs. Cut Variations: $d_{0,\rm K}d_{0,\pi}$}
\begin{columns}
\column{0.7\textwidth}
\begin{overpic}[width=1.1\textwidth, trim=0 0 0 0, clip]{img/topo_cuts/d0d0_CutEfficiency_copy_JetPt5_15_DPt3_6}
\put(12,10){\tiny $5 < \ptchjet < 15$~\GeVc\ and  $3 < \ptd < 6$~\GeVc}
\end{overpic}
\column{0.3\textwidth}
\footnotesize
\textbf{Signal} and \textbf{\textcolor{NavyBlue}{Non-Prompt}} from LHC15i2\{b,c,d,e\} (charm-enhanced with \pt-hard bins)\\
\vspace{10pt}
\textbf{\textcolor{BrickRed}{Background}} from LHC14j4\{b,c,d,e\} (minimum-bias) \\
\vspace{10pt}
\textbf{For each bin of \ptchjet\ and \ptd}
\end{columns}
\end{frame}

\begin{frame}{Sig/Bkg Efficiency vs. Cut Variations: topomatic}
\begin{columns}
\column{0.7\textwidth}
\begin{overpic}[width=1.1\textwidth, trim=0 0 0 0, clip]{img/topo_cuts/MaxNormd0_CutEfficiency_copy_JetPt15_30_DPt6_9}
\put(12,10){\tiny $15 < \ptchjet < 30$~\GeVc\ and  $6 < \ptd < 9$~\GeVc}
\end{overpic}
\column{0.3\textwidth}
\footnotesize
\textbf{Signal} and \textbf{\textcolor{NavyBlue}{Non-Prompt}} from LHC15i2\{b,c,d,e\} (charm-enhanced with \pt-hard bins)\\
\vspace{10pt}
\textbf{\textcolor{BrickRed}{Background}} from LHC14j4\{b,c,d,e\} (minimum-bias) \\
\vspace{10pt}
\textbf{For each bin of \ptchjet\ and \ptd}
\end{columns}
\end{frame}

\begin{frame}{Startegy}
\begin{itemize}
\item Use the previous plots as a guidance to select the best cuts in each \ptchjet\ and \ptd\ bin
\item Try to maximise the significance
\item \textbf{Use wider \ptd\ bins} $\rightarrow$ this is crucial for $15 < \ptchjet < 30$~\GeVc
\end{itemize}
\end{frame}

\subsection{Results}

\begin{frame}{Summary of the Results}
\begin{itemize}
\item Small improvements of the significance of the fit
\item Not particularly crucial for $5 < \ptchjet < 15$~\GeVc\ (significance already in the range [8,18])
\item For $15 < \ptchjet < 30$~\GeVc\ use two large bins in \ptd: [6, 12] and [12, 30]
\end{itemize}
\end{frame}

\begin{frame}{Summary of the Results: $5 < \ptchjet < 15$~\GeVc}
\begin{columns}
\column{0.5\textwidth}
\begin{overpic}[width=\textwidth, trim=0 0 0 0, clip]{img/topo_cuts/AnyINT_D0_D0toKpiCuts_loosest_pid_Charged_R040_DPtBins_JetPt_5_15_DPt4_Cut0_SideBand_D0_D0toKpiCuts_loosest_pid_Charged_R040_DPtSpectrum_JetPt_5_15_DPt4_SideBand_Cut0}
\put(16,50){\scriptsize \textcolor{red}{Old cuts}}
\put(60,3){\footnotesize $m({\rm K\pi})$ (\GeVc$^2$)}
\end{overpic}
\column{0.5\textwidth}
\begin{overpic}[width=\textwidth, trim=0 0 0 0, clip]{img/topo_cuts/AnyINT_D0_D0toKpiCuts_loosest_pid_Charged_R040_DPtBins_JetPt_5_15_DPt4_Cut1_SideBand_D0_D0toKpiCuts_loosest_pid_Charged_R040_DPtSpectrum_JetPt_5_15_DPt4_SideBand_Cut1}
\put(16,50){\scriptsize \textcolor{blue}{Optimized cuts}}
\put(60,3){\footnotesize $m({\rm K\pi})$ (\GeVc$^2$)}
\end{overpic}
\end{columns}
\end{frame}

\begin{frame}{Summary of the Results: $5 < \ptchjet < 15$~\GeVc}
\begin{columns}
\column{0.5\textwidth}
\begin{overpic}[width=\textwidth, trim=0 0 0 0, clip]{img/topo_cuts/AnyINT_D0_D0toKpiCuts_loosest_pid_Charged_R040_DPtSpectrum_JetPt_5_15_DPt4_SideBand_Cut0_BkgVsSig}
\put(16,50){\scriptsize \textcolor{red}{Old cuts}}
\put(60,3){\footnotesize \ptd\ (\GeVc)}
\end{overpic}
\column{0.5\textwidth}
\begin{overpic}[width=\textwidth, trim=0 0 0 0, clip]{img/topo_cuts/AnyINT_D0_D0toKpiCuts_loosest_pid_Charged_R040_DPtSpectrum_JetPt_5_15_DPt4_SideBand_Cut1_BkgVsSig}
\put(16,50){\scriptsize \textcolor{blue}{Optimized cuts}}
\put(60,3){\footnotesize \ptd\ (\GeVc)}
\end{overpic}
\end{columns}
\end{frame}

\begin{frame}{Summary of the Results: $15 < \ptchjet < 30$~\GeVc}
\begin{columns}
\column{0.5\textwidth}
\begin{overpic}[width=\textwidth, trim=0 0 0 0, clip]{img/topo_cuts/AnyINT_D0_D0toKpiCuts_loosest_pid_Charged_R040_DPtBins_JetPt_15_30_DPt1_Cut2_SideBand_D0_D0toKpiCuts_loosest_pid_Charged_R040_DPtSpectrum_JetPt_15_30_DPt1_SideBand_Cut2}
\put(60,3){\footnotesize $m({\rm K\pi})$ (\GeVc$^2$)}
\end{overpic}
\column{0.5\textwidth}
\begin{overpic}[width=\textwidth, trim=0 0 0 0, clip]{img/topo_cuts/AnyINT_D0_D0toKpiCuts_loosest_pid_Charged_R040_DPtSpectrum_JetPt_15_30_DPt1_SideBand_Cut2_BkgVsSig}
\put(60,3){\footnotesize \ptd\ (\GeVc)}
\end{overpic}
\end{columns}
\Dzero\ meson signal comes from the full \ptd\ range
\end{frame}

\begin{frame}{Summary of the Results: $15 < \ptchjet < 30$~\GeVc}
\begin{columns}
\column{0.5\textwidth}
\begin{overpic}[width=\textwidth, trim=0 0 0 0, clip]{img/topo_cuts/AnyINT_D0_D0toKpiCuts_loosest_pid_Charged_R040_DPtBins_JetPt_15_30_DPt2_Cut0_SideBand_D0_D0toKpiCuts_loosest_pid_Charged_R040_DPtSpectrum_JetPt_15_30_DPt2_SideBand_Cut0}
\put(16,50){\scriptsize \textcolor{red}{Old cuts}}
\put(60,3){\footnotesize $m({\rm K\pi})$ (\GeVc$^2$)}
\end{overpic}
\column{0.5\textwidth}
\begin{overpic}[width=\textwidth, trim=0 0 0 0, clip]{img/topo_cuts/AnyINT_D0_D0toKpiCuts_loosest_pid_Charged_R040_DPtBins_JetPt_15_30_DPt2_Cut4_SideBand_D0_D0toKpiCuts_loosest_pid_Charged_R040_DPtSpectrum_JetPt_15_30_DPt2_SideBand_Cut4}
\put(16,50){\scriptsize \textcolor{blue}{Optimized cuts}}
\put(60,3){\footnotesize $m({\rm K\pi})$ (\GeVc$^2$)}
\end{overpic}
\end{columns}
\end{frame}

\begin{frame}{Summary of the Results: $15 < \ptchjet < 30$~\GeVc}
\begin{columns}
\column{0.5\textwidth}
\begin{overpic}[width=\textwidth, trim=0 0 0 0, clip]{img/topo_cuts/AnyINT_D0_D0toKpiCuts_loosest_pid_Charged_R040_DPtSpectrum_JetPt_15_30_DPt2_SideBand_Cut0_BkgVsSig}
\put(16,50){\scriptsize \textcolor{red}{Old cuts}}
\put(60,3){\footnotesize \ptd\ (\GeVc)}
\end{overpic}
\column{0.5\textwidth}
\begin{overpic}[width=\textwidth, trim=0 0 0 0, clip]{img/topo_cuts/AnyINT_D0_D0toKpiCuts_loosest_pid_Charged_R040_DPtSpectrum_JetPt_15_30_DPt2_SideBand_Cut4_BkgVsSig}
\put(16,50){\scriptsize \textcolor{blue}{Optimized cuts}}
\put(60,3){\footnotesize \ptd\ (\GeVc)}
\end{overpic}
\end{columns}
\end{frame}

\section{Raw Yields with New Cuts and \ptd\ Ranges}

\begin{frame}{Inv.Mass Fits w/ New Cuts: $5 < \ptchjet < 15$~\GeVc}
\begin{columns}
\column{0.65\textwidth}
\begin{overpic}[width=\textwidth, trim=0 0 0 0, clip]{img/new/AnyINT_D0_D0toKpiCuts_D0JetOptimLowJetPtv1_Charged_R040_DPtBins_JetPt_5_15_SideBand_D0_D0toKpiCuts_D0JetOptimLowJetPtv1_Charged_R040_JetZSpectrum_DPt_20_JetPt_5_15_SideBand}
\end{overpic}
\column{0.35\textwidth}
4 wide \ptd\ bins\\
\vspace{10pt}
 \textbf{starting at $\ptd>2$~\GeVc}\\
 \vspace{10pt}
 Not much room for improvement, but \textbf{significance already good}
\end{columns}
\end{frame}

\begin{frame}{\ptd\ Raw Yields w/ New Cuts: $5 < \ptchjet < 15$~\GeVc}
\begin{columns}
\column{0.65\textwidth}
\begin{overpic}[width=\textwidth, trim=0 0 0 0, clip]{img/new/AnyINT_D0_D0toKpiCuts_D0JetOptimLowJetPtv1_Charged_R040_DPtSpectrum_JetPt_5_15_SideBand_BkgVsSig}
\end{overpic}
\column{0.35\textwidth}
4 wide \ptd\ bins\\
\vspace{10pt}
 \textbf{starting at $\ptd>2$~\GeVc}
\end{columns}
\end{frame}

\begin{frame}{\zpar\ Raw Yields w/ New Cuts: $5 < \ptchjet < 15$~\GeVc}
\begin{columns}
\column{0.65\textwidth}
\begin{overpic}[width=\textwidth, trim=0 0 0 0, clip]{img/new/AnyINT_D0_D0toKpiCuts_D0JetOptimLowJetPtv1_Charged_R040_JetZSpectrum_DPt_20_JetPt_5_15_SideBand_BkgVsSig}
\end{overpic}
\column{0.35\textwidth}
4 wide \ptd\ bins\\
\vspace{10pt}
 \textbf{starting at $\ptd>2$~\GeVc}\\
 The \zpar\ distribution is unbiased for $\zpar>0.4$.
\end{columns}
\end{frame}

\begin{frame}{Inv.Mass Fits w/ New Cuts: $15 < \ptchjet < 30$~\GeVc}
\begin{center}
\begin{overpic}[width=0.9\textwidth, trim=0 0 0 0, clip]{img/new/AnyINT_D0_D0toKpiCuts_D0JetOptimHighJetPtv1_Charged_R040_DPtBins_JetPt_15_30_SideBand_D0_D0toKpiCuts_D0JetOptimHighJetPtv1_Charged_R040_JetZSpectrum_DPt_60_JetPt_15_30_SideBand}
\end{overpic}\\
2 wide \ptd\ bins\\
\vspace{10pt}
\textbf{starting at $\ptd>6$~\GeVc}\\
\vspace{6pt}
Not much room for improving the significance
\end{center}
\end{frame}

\begin{frame}{\ptd\ Raw Yields w/ New Cuts: $15 < \ptchjet < 30$~\GeVc}
\begin{center}
\begin{overpic}[width=0.9\textwidth, trim=0 0 0 0, clip]{img/new/AnyINT_D0_D0toKpiCuts_D0JetOptimHighJetPtv1_Charged_R040_DPtSpectrum_JetPt_15_30_SideBand_BkgVsSig}
\end{overpic}\\
The bin [6,12] shows a steep \ptd\ distribution of \Dzero\ signal across the \ptd\ bin $\rightarrow$ cuts need to be tuned also to avoid large variations in the efficiency within one \ptd\ bin (this could impact negatively the systematics)
 \end{center}
\end{frame}

\begin{frame}{\zpar\ Raw Yields w/ New Cuts: $15 < \ptchjet < 30$~\GeVc}
\begin{center}
\begin{overpic}[width=0.9\textwidth, trim=0 0 0 0, clip]{img/new/AnyINT_D0_D0toKpiCuts_D0JetOptimHighJetPtv1_Charged_R040_JetZSpectrum_DPt_60_JetPt_15_30_SideBand_BkgVsSig}
\end{overpic}\\
2 wide \ptd\ bins,
 \textbf{starting at $\ptd>6$~\GeVc}\\
 The \zpar\ distribution is unbiased for $\zpar>0.4$.
 \end{center}
\end{frame}

\section{Conclusions}

\begin{frame}{Conclusions}
\begin{itemize}
\item Statistics are sufficient to measure fragmentation function in two jet \pt\ bins (it won't be a precision measurement, though!): [5, 15] and [15, 30]
\item Change w.r.t. the previous results: wider \ptd\ bins used for the SB subtraction
\item In the [5, 15] jet \pt\ bin can include $\ptd>2$~\GeVc (was $3$~\GeVc\ in the approved preliminary jet cross section)
\item The optimisation of the topological cuts need to be done in a blind test using MC, however I do not expect a significant improvement w.r.t. the old cuts
\item Next: finalise cuts using a blind MC optimisation, calculate efficiency vs. \ptd\ and \ptchjet\ with the new cuts
\end{itemize}
\end{frame}

\section{Extra Slides}

\subsection{Raw Yield}

\begin{frame}{Invariant Mass Fits in \zpar\ bins}
\textcolor{red}{From HFCJ September 20th}
\begin{center}
\begin{overpic}[width=.8\textwidth, trim=0 0 0 0, clip]{img/raw_yield/AnyINT_D0_Charged_R040_JetZBins_DPt_30_JetPt_5_30}
\put(70,20){\tiny $\ptd > 3$~\GeVc}
\put(70,16){\tiny $5 < \ptchjet < 30$~\GeVc}
\end{overpic}
\end{center}
\vspace{-5pt}
\small
\begin{itemize}
\item No signal in $0 < \zpar < 0.2$ (as expected due to the kinematic cuts)
\item Good S/B in all bins
\end{itemize}
\end{frame}

\begin{frame}{Side Band Method}
\textcolor{red}{From HFCJ September 20th}
\begin{center}
\begin{overpic}[width=.8\textwidth, trim=0 0 0 0, clip]{img/raw_yield/AnyINT_D0_Charged_R040_DPtBins_JetPt_5_30_SideBand_D0_Charged_R040_JetPtSpectrum_DPt_30_SideBand}
\end{overpic}
\end{center}
\vspace{-5pt}
\small
\begin{itemize}
\item Same as for the jet \pt\ spectrum
\item $\ptd > 3$~\GeVc\ and $5 < \ptchjet < 30$~\GeVc
\end{itemize}
\end{frame}

\begin{frame}{Side Band Method in \zpar\ bins}
\textcolor{red}{From HFCJ September 20th}
\begin{columns}
\column{0.65\textwidth}
\begin{overpic}[width=\textwidth, trim=0 0 0 0, clip]{img/raw_yield/AnyINT_D0_Charged_R040_JetZSpectrum_DPt_30_JetPt_5_30_SideBand_BkgVsSig}
\end{overpic}
\column{0.35\textwidth}
\begin{overpic}[width=\textwidth, trim=0 0 0 0, clip]{img/raw_yield/AnyINT_D0_Charged_R040_JetZSpectrum_DPt_30_JetPt_5_30_SideBand_TotalBkgVsSig}
\end{overpic}
\vspace{-5pt}
\begin{itemize}
\item Subtraction of the \zpar\ spectra in bins of \ptd
\end{itemize}
\end{columns}
\end{frame}

\begin{frame}{Invariant Mass Fit Width vs. \zpar}
\textcolor{red}{From HFCJ September 20th}
\begin{overpic}[width=.8\textwidth, trim=0 0 0 0, clip]{img/raw_yield/AnyINT_D0_Charged_R040_JetZSpectrum_DPt_30_JetPt_5_30_InvMassFit_MassWidth_canvas}
\end{overpic}
\end{frame}

\subsection{B Feed-Down}

\begin{frame}{Simulation Variations (Generator Level)}
\textcolor{red}{From HFCJ September 20th}
\begin{columns}
\column{0.5\textwidth}
\begin{overpic}[width=\textwidth, trim=0 0 0 0, clip]{img/feed-down/BFeedDown_1505317519_1242_JetZSpectrum_DPt_30_JetPt_5_30_GeneratorLevel_JetZSpectrum}
\end{overpic}
\column{0.5\textwidth}
\begin{overpic}[width=\textwidth, trim=0 0 0 0, clip]{img/feed-down/BFeedDown_1505317519_1242_JetZSpectrum_DPt_30_JetPt_5_30_GeneratorLevel_JetZSpectrum_Ratio}
\end{overpic}
\end{columns}
\vspace{-5pt}
\begin{itemize}
\item Code updated to work with \zpar
\item Same strategy used for the jet \pt\ spectrum
\end{itemize}
\end{frame}

\begin{frame}{Simulation Variations (Detector Level)}
\textcolor{red}{From HFCJ September 20th}
\begin{columns}
\column{0.5\textwidth}
\begin{overpic}[width=\textwidth, trim=0 0 0 0, clip]{img/feed-down/BFeedDown_1505317519_1242_JetZSpectrum_DPt_30_JetPt_5_30_DetectorLevel_JetZSpectrum_bEfficiencyMultiply_cEfficiencyDivide}
\end{overpic}
\column{0.5\textwidth}
\begin{overpic}[width=\textwidth, trim=0 0 0 0, clip]{img/feed-down/BFeedDown_1505317519_1242_JetZSpectrum_DPt_30_JetPt_5_30_DetectorLevel_JetZSpectrum_bEfficiencyMultiply_cEfficiencyDivide_Ratio}
\end{overpic}
\end{columns}
\vspace{-5pt}
\begin{itemize}
\item Code updated to work with \zpar
\item Same strategy used for the jet \pt\ spectrum
\item It includes ratio of prompt/non-prompt efficiency and detector resolution (non-prompt)
\end{itemize}
\end{frame}

\end{document}
