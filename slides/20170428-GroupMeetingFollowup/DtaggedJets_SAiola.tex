% $Header: /Users/joseph/Documents/LaTeX/beamer/solutions/conference-talks/conference-ornate-20min.en.tex,v 90e850259b8b 2007/01/28 20:48:30 tantau $

\documentclass[xcolor={usenames,dvipsnames}]{beamer}

% This file is a solution template for:

% - Talk at a conference/colloquium.
% - Talk length is about 20min.
% - Style is ornate.



% Copyright 2004 by Till Tantau <tantau@users.sourceforge.net>.
%
% In principle, this file can be redistributed and/or modified under
% the terms of the GNU Public License, version 2.
%
% However, this file is supposed to be a template to be modified
% for your own needs. For this reason, if you use this file as a
% template and not specifically distribute it as part of a another
% package/program, I grant the extra permission to freely copy and
% modify this file as you see fit and even to delete this copyright
% notice. 


\mode<presentation>
{
  \usetheme{AnnArbor}
  % or ...

  \setbeamercovered{transparent}
  % or whatever (possibly just delete it)
 }

\usepackage[percent]{overpic}
\usepackage[english]{babel}
\usepackage{setspace}
% or whatever

\usepackage[latin1]{inputenc}
% or whatever

\usepackage[at]{easylist}
\usepackage{times}
\usepackage[T1]{fontenc}
\usepackage{xcolor}
% Or whatever. Note that the encoding and the font should match. If T1
% does not look nice, try deleting the line with the fontenc.

\definecolor{darkestblue}{RGB}{1,8,100}
\definecolor{darkerblue}{RGB}{3,17,150}
\definecolor{darkblue}{RGB}{7,26,200}
\definecolor{lightred}{RGB}{202,103,104}
\definecolor{lightgreen}{RGB}{106,202,107}

%particles
\newcommand{\jpsi}{\rm J/$\psi$}
\newcommand{\psip}{$\psi^\prime$}
\newcommand{\jpsiDY}{\rm J/$\psi$\,/\,DY}
\newcommand{\chic}{$\chi_{\rm c}$}
\newcommand{\pip}{$\pi^{+}$}
\newcommand{\pim}{$\pi^{-}$}
\newcommand{\pizero}{$\pi^{0}$}
\newcommand{\kap}{K$^{+}$}
\newcommand{\kam}{K$^{-}$}
\newcommand{\pbar}{$\rm\overline{p}$}
\newcommand{\ccbar}{\ensuremath{\mathrm{c\overline{c}}}}
\newcommand{\bbbar}{\ensuremath{\mathrm{b\overline{b}}}}
\newcommand{\Dzero}{\ensuremath{\mathrm{D^{0}}}}
\newcommand{\Dzerobar}{\ensuremath{\mathrm{\overline{D}^{0}}}}
\newcommand{\Dpm}{\ensuremath{\mathrm{D^{\pm}}}}
\newcommand{\Ds}{\ensuremath{\mathrm{D_{s}^{\pm}}}}
\newcommand{\Dstar}{\ensuremath{\mathrm{D^{*\pm}}}}

%collision systems
\newcommand{\pp}{pp}
\newcommand{\pPb}{p--Pb}
\newcommand{\PbPb}{Pb--Pb}

%detectors
\newcommand{\ezdc}{$E_{\rm ZDC}$}

%units
\newcommand{\GeVc}{GeV/$c$}
\newcommand{\GeVcsq}{GeV/$c^2$}

%others
\newcommand{\degree}{$^{\rm o}$}
\newcommand{\s}{\ensuremath{\sqrt{s}}}
\newcommand{\snn}{\ensuremath{\sqrt{s_{\rm NN}}}}
\newcommand{\y}{\ensuremath{y}}
\newcommand{\pt}{\ensuremath{p_{\rm T}}}
\newcommand{\dedx}{d$E$/d$x$}
\newcommand{\dndy}{d$N$/d$y$}
\newcommand{\dndydpt}{${\rm d}^2N/({\rm d}y {\rm d}p_{\rm t})$}
\newcommand{\zpar}{\ensuremath{z_{||}}}
\newcommand{\zpargen}{\ensuremath{z_{||}^{\mathrm{part}}}}
\newcommand{\zpardet}{\ensuremath{z_{||}^{\mathrm{det}}}}
\newcommand{\ptchjet}{\ensuremath{p_{\mathrm{T,ch\, jet}}}}
\newcommand{\ptjet}{\ensuremath{p_{\mathrm{T,jet}}}}
\newcommand{\ptchjetgen}{\ensuremath{p_{\mathrm{T,ch\,jet}}^{\mathrm{truth}}}}
\newcommand{\ptchjetdet}{\ensuremath{p_{\mathrm{T,ch\,jet}}^{\mathrm{reco}}}}
\newcommand{\ptd}{\ensuremath{p_{\mathrm{T,D}}}}
\newcommand{\ptdgen}{\ensuremath{p_{\mathrm{T,D}}^{\mathrm{truth}}}}
\newcommand{\ptddet}{\ensuremath{p_{\mathrm{T,D}}^{\mathrm{reco}}}}
\newcommand{\antikt}{anti-\ensuremath{k_{\mathrm{T}}}}
\newcommand{\kt}{\ensuremath{k_{\mathrm{T}}}}
\newcommand{\pthard}{\ensuremath{p_{\mathrm{T,hard}}}}

\AtBeginSection[]{
  \begin{frame}
  \vfill
  \centering
  \begin{beamercolorbox}[sep=8pt,center,shadow=true,rounded=true]{title}
    \usebeamerfont{title}\insertsectionhead\par%
  \end{beamercolorbox}
  \vfill
  \end{frame}
}

\title[D-meson jets reconstruction with ALICE] % (optional, use only with long paper titles)
{D-meson jets reconstruction with ALICE}

\author[Salvatore Aiola]% (optional, use only with lots of authors)
{Salvatore Aiola}
% - Give the names in the same order as the appear in the paper.
% - Use the \inst{?} command only if the authors have different
%   affiliation.

\institute[Yale University] % (optional, but mostly needed)
{Yale University}

\date[Apr. 29th, 2017] % (optional, should be abbreviation of conference name)
{Yale Relativistic Heavy-Ion Group Meeting \\
April 29th, 2017 \\
\textcolor{red}{Follow-up after 4/26/2017 meeting}}
% - Either use conference name or its abbreviation.
% - Not really informative to the audience, more for people (including
%   yourself) who are reading the slides online

\subject{High-Energy Physics}
% This is only inserted into the PDF information catalog. Can be left
% out. 



% If you have a file called "university-logo-filename.xxx", where xxx
% is a graphic format that can be processed by latex or pdflatex,
% resp., then you can add a logo as follows:

% \pgfdeclareimage[height=0.5cm]{university-logo}{university-logo-filename}
% \logo{\pgfuseimage{university-logo}}


% If you wish to uncover everything in a step-wise fashion, uncomment
% the following command: 

%\beamerdefaultoverlayspecification{<+->}


\begin{document}

\begin{frame}
  \titlepage
\end{frame}

\begin{frame}{Outline}
   \tableofcontents
\end{frame}


% Structuring a talk is a difficult task and the following structure
% may not be suitable. Here are some rules that apply for this
% solution: 

% - Exactly two or three sections (other than the summary).
% - At *most* three subsections per section.
% - Talk about 30s to 2min per frame. So there should be between about
%   15 and 30 frames, all told.

% - A conference audience is likely to know very little of what you
%   are going to talk about. So *simplify*!
% - In a 20min talk, getting the main ideas across is hard
%   enough. Leave out details, even if it means being less precise than
%   you think necessary.
% - If you omit details that are vital to the proof/implementation,
%   just say so once. Everybody will be happy with that.


%\begin{overpic}[width=\textwidth, trim=0 0 0 0, clip]{img/823_D0_Charged_R040_JetPtBins_DPt_30}
%\end{overpic}

%\begin{columns}
%\column{0.5\textwidth}
%\column{0.5\textwidth}
%\end{columns}

\section{\Dzero\ reflections}

\begin{frame}{Some special relativity}
The invariant mass of a pion-kaon pair can be written:
\begin{equation}
\begin{split}
m_{\rm inv}^{2} = m_{\pi}^{2} + m_{\rm K}^{2} + 2p_{\pi}p_{\rm K}\cos\theta + 2\sqrt{(p_{\pi}^{2}+m_{\pi}^{2})(p_{\rm K}^{2}+m_{\rm K}^{2})} =  \\
= m_{\pi}^{2} + m_{\rm K}^{2} + 2p_{\pi}p_{\rm K}\left(\sqrt{1+\frac{m_{\pi}^{2}}{p_{\pi}^{2}}}\sqrt{1+\frac{m_{\rm K}^{2}}{p_{\rm K}^{2}}} - \cos\theta\right),
\end{split}
\label{eq:minv}
\end{equation}
where $m_{\pi}$, $m_{\rm K}$ are respectively the pion and kaon masses, $p_{\pi}$, $p_{\rm K}$ are the magnitudes of the 3-momenta of respectively the pion and the kaon,
and $\theta$ is the angle between the pion and the kaon 3-momenta.
\end{frame}

\begin{frame}{\Dzero\ momentum $p_{\Dzero}\gg m_{\Dzero}$}
For large \Dzero\ momentum one may be tempted to neglect the terms $\frac{m_{\pi,\rm K}^{2}}{p_{\pi,\rm K}^{2}}$ in Eq.~\ref{eq:minv}. This is wrong for two reasons:
\begin{enumerate}
\item large \Dzero\ momentum does \textbf{not} imply \emph{both} $p_{\pi}\gg m_{\pi}$ and $p_{\rm K}\gg m_{\rm K}$ (see slide~\ref{pion_kaon_mom});
\item even in those cases where both $p_{\pi}\gg m_{\pi}$ and $p_{\rm K}\gg m_{\rm K}$, the error made by neglecting the terms $\frac{m_{\pi,\rm K}^{2}}{p_{\pi,\rm K}^{2}}$ is amplified
by the large value of the factor $p_{\pi}p_{\rm K}$.
\end{enumerate}
\vspace{5pt}
\footnotesize
Note also that the term $m_{\pi}^{2} + m_{\rm K}^{2}$ in Eq.~\ref{eq:minv} cannot be neglected in any case because it is never much smaller than 
$2p_{\pi}p_{\rm K}\left(\sqrt{(1+m_{\pi}^{2}/p_{\pi}^{2})(1+m_{\rm K}^{2}/p_{\rm K}^{2})} - \cos\theta\right)$.
In fact, $\cos\theta$ tends to $1$ for large momenta.
\end{frame}

\begin{frame}{Reflected mass vs. $p_{\rm D}$}
\footnotesize
A simple numerical simulation using a boosted decay simulator shows that indeed the reflected mass does not converge
to the correct mass for large $p_{\rm D}$.
\begin{center}
\begin{overpic}[width=.65\textwidth, trim=0 0 0 20, clip]{img/ReflDiffVsDPt}
\end{overpic}
\end{center}
\vspace{-10pt}
\footnotesize
The red lines represent the range of the invariant mass fit used in my analysis.\\
If you are satisfied with this explanation you can skip the next three slides. \\
Otherwise, go ahead...
\end{frame}

\begin{frame}{Approximation}
\footnotesize
Let's take a closer look at the approximation $2\sqrt{(p_{\pi}^{2}+m_{\pi}^{2})(p_{\rm K}^{2}+m_{\rm K}^{2})}\rightarrow 2p_{\pi}p_{\rm K}$.\\
The function:
\begin{equation}
f(p_{\pi},p_{\rm K})=2\left(\sqrt{(p_{\pi}^{2}+m_{\pi}^{2})(p_{\rm K}^{2}+m_{\rm K}^{2})} - p_{\pi}p_{\rm K}\right)
\label{eq:approx}
\end{equation}
represents the error that we make when we apply this approximation. The plot below shows this expression as a function of $p_{\rm D}$.
\begin{center}
\begin{overpic}[width=.65\textwidth, trim=0 0 0 20, clip]{img/ApproxErrorInvMassVsDP}
\end{overpic}
\end{center}
\end{frame}

\begin{frame}{Approximation (cont'd)}
\footnotesize
With the help of Mathematica we find that the minimum of $f$ in Eq.~\ref{eq:approx} is obtained for: $p_{\rm K} = \frac{m_{\rm K}}{m_{\pi}}p_{\pi}$.
The value of the function at its minimum is: 2$m_{\rm K}m_{\pi}=0.1378\,(\mathrm{GeV}/c^{2})^2$. This analytical result is confirmed by a numerical calculation with the decay simulator.
\begin{center}
\begin{overpic}[width=.65\textwidth, trim=0 0 0 20, clip]{img/ApproxErrorInvMassVsPtRatio}
\end{overpic}
\end{center}
\end{frame}

\begin{frame}[label={pion_kaon_mom}]{Pion and kaon momenta}
\small
Simple decay simulator: decay in the \Dzero\ rest frame and boost in a random direction in the 3D space.\\
\Dzero\ momenta: $0.5<\ptd<100$~\GeVc\ (flat distribution)
\begin{columns}
\column{0.75\textwidth}
\begin{overpic}[width=\textwidth, trim=0 0 0 20, clip]{img/DaughtersPt}
\end{overpic}
\column{0.25\textwidth}
\footnotesize
One can observe that there is no tight correlation between $p_{\pi}$ and $p_{\rm K}$.
In fact for a large boost (i.e. large D meson momentum) there is a larger phase space available
for both $p_{\pi}$ and $p_{\rm K}$.\\
\vspace{10pt}
The red line is the curve: $p_{\rm K} = \frac{m_{\rm K}}{m_{\pi}}p_{\pi}$
\end{columns}
\end{frame}

\section{Left vs. Right Side Bands}

\begin{frame}{Left vs. Right Side Bands}
\begin{columns}
\column{0.65\textwidth}
\begin{overpic}[width=\textwidth, trim=0 0 0 0, clip]{img/SideBandLeftVsRight}
\end{overpic}
\column{0.35\textwidth}
\small
Jet \pt\ distributions of \Dzero-jet candidates from the left (blue) and right (red) side bands of the invariant mass distributions.\\
\vspace{15pt}
\tiny
Left SB: $m_{\rm D,fit} - 8\sigma_{\rm fit} < m_{\rm inv} < m_{\rm D,fit} - 4\sigma_{\rm fit}$ \\
\vspace{10pt}
Right SB: $m_{\rm D,fit} + 4\sigma_{\rm fit} < m_{\rm inv} < m_{\rm D,fit} + 8\sigma_{\rm fit}$ 
\end{columns}
\end{frame}

\end{document}
