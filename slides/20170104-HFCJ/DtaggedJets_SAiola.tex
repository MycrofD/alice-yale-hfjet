% $Header: /Users/joseph/Documents/LaTeX/beamer/solutions/conference-talks/conference-ornate-20min.en.tex,v 90e850259b8b 2007/01/28 20:48:30 tantau $

\documentclass[xcolor={usenames,dvipsnames}]{beamer}

% This file is a solution template for:

% - Talk at a conference/colloquium.
% - Talk length is about 20min.
% - Style is ornate.



% Copyright 2004 by Till Tantau <tantau@users.sourceforge.net>.
%
% In principle, this file can be redistributed and/or modified under
% the terms of the GNU Public License, version 2.
%
% However, this file is supposed to be a template to be modified
% for your own needs. For this reason, if you use this file as a
% template and not specifically distribute it as part of a another
% package/program, I grant the extra permission to freely copy and
% modify this file as you see fit and even to delete this copyright
% notice. 


\mode<presentation>
{
  \usetheme{AnnArbor}
  % or ...

  \setbeamercovered{transparent}
  % or whatever (possibly just delete it)
 }

\usepackage[percent]{overpic}
\usepackage[english]{babel}
% or whatever

\usepackage[latin1]{inputenc}
% or whatever

\usepackage{times}
\usepackage[T1]{fontenc}
% Or whatever. Note that the encoding and the font should match. If T1
% does not look nice, try deleting the line with the fontenc.
%particles
\newcommand{\jpsi}{\rm J/$\psi$}
\newcommand{\psip}{$\psi^\prime$}
\newcommand{\jpsiDY}{\rm J/$\psi$\,/\,DY}
\newcommand{\chic}{$\chi_{\rm c}$}
\newcommand{\pip}{$\pi^{+}$}
\newcommand{\pim}{$\pi^{-}$}
\newcommand{\pizero}{$\pi^{0}$}
\newcommand{\kap}{K$^{+}$}
\newcommand{\kam}{K$^{-}$}
\newcommand{\pbar}{$\rm\overline{p}$}
\newcommand{\ccbar}{\ensuremath{\mathrm{c\overline{c}}}}
\newcommand{\bbbar}{\ensuremath{\mathrm{b\overline{b}}}}
\newcommand{\Dzero}{\ensuremath{\mathrm{D^{0}}}}
\newcommand{\Dzerobar}{\ensuremath{\mathrm{\overline{D}^{0}}}}
\newcommand{\Dpm}{\ensuremath{\mathrm{D^{\pm}}}}
\newcommand{\Ds}{\ensuremath{\mathrm{D_{s}^{\pm}}}}
\newcommand{\Dstar}{\ensuremath{\mathrm{D^{*\pm}}}}

%collision systems
\newcommand{\pp}{pp}
\newcommand{\pPb}{p--Pb}
\newcommand{\PbPb}{Pb--Pb}

%detectors
\newcommand{\ezdc}{$E_{\rm ZDC}$}

%units
\newcommand{\GeVc}{GeV/$c$}
\newcommand{\GeVcsq}{GeV/$c^2$}

%others
\newcommand{\degree}{$^{\rm o}$}
\newcommand{\s}{\ensuremath{\sqrt{s}}}
\newcommand{\snn}{\ensuremath{\sqrt{s_{\rm NN}}}}
\newcommand{\y}{\ensuremath{y}}
\newcommand{\pt}{\ensuremath{p_{\rm T}}}
\newcommand{\dedx}{d$E$/d$x$}
\newcommand{\dndy}{d$N$/d$y$}
\newcommand{\dndydpt}{${\rm d}^2N/({\rm d}y {\rm d}p_{\rm t})$}
\newcommand{\zpar}{\ensuremath{z_{||}}}
\newcommand{\zpargen}{\ensuremath{z_{||}^{\mathrm{part}}}}
\newcommand{\zpardet}{\ensuremath{z_{||}^{\mathrm{det}}}}
\newcommand{\ptchjet}{\ensuremath{p_{\mathrm{T,ch\, jet}}}}
\newcommand{\ptjet}{\ensuremath{p_{\mathrm{T,jet}}}}
\newcommand{\ptchjetgen}{\ensuremath{p_{\mathrm{T,ch\,jet}}^{\mathrm{truth}}}}
\newcommand{\ptchjetdet}{\ensuremath{p_{\mathrm{T,ch\,jet}}^{\mathrm{reco}}}}
\newcommand{\ptd}{\ensuremath{p_{\mathrm{T,D}}}}
\newcommand{\ptdgen}{\ensuremath{p_{\mathrm{T,D}}^{\mathrm{truth}}}}
\newcommand{\ptddet}{\ensuremath{p_{\mathrm{T,D}}^{\mathrm{reco}}}}
\newcommand{\antikt}{anti-\ensuremath{k_{\mathrm{T}}}}
\newcommand{\kt}{\ensuremath{k_{\mathrm{T}}}}
\newcommand{\pthard}{\ensuremath{p_{\mathrm{T,hard}}}}

\AtBeginSection[]{
  \begin{frame}
  \vfill
  \centering
  \begin{beamercolorbox}[sep=8pt,center,shadow=true,rounded=true]{title}
    \usebeamerfont{title}\insertsectionhead\par%
  \end{beamercolorbox}
  \vfill
  \end{frame}
}

\title[D-tagged jets in \pp] % (optional, use only with long paper titles)
{D-tagged jets in \pp\ collisions: systematic studies}

\author[Salvatore Aiola]% (optional, use only with lots of authors)
{Salvatore Aiola}
% - Give the names in the same order as the appear in the paper.
% - Use the \inst{?} command only if the authors have different
%   affiliation.

\institute[Yale University] % (optional, but mostly needed)
{Yale University}

\date[PAG-HFCJ - Jan. 4th, 2017] % (optional, should be abbreviation of conference name)
{PAG-HFCJ \\
January 4th, 2017}
% - Either use conference name or its abbreviation.
% - Not really informative to the audience, more for people (including
%   yourself) who are reading the slides online

\subject{High-Energy Physics}
% This is only inserted into the PDF information catalog. Can be left
% out. 



% If you have a file called "university-logo-filename.xxx", where xxx
% is a graphic format that can be processed by latex or pdflatex,
% resp., then you can add a logo as follows:

% \pgfdeclareimage[height=0.5cm]{university-logo}{university-logo-filename}
% \logo{\pgfuseimage{university-logo}}


% If you wish to uncover everything in a step-wise fashion, uncomment
% the following command: 

%\beamerdefaultoverlayspecification{<+->}


\begin{document}

\begin{frame}
  \titlepage
\end{frame}

%\begin{frame}{Outline}
 %   \tableofcontents
 %\end{frame}


% Structuring a talk is a difficult task and the following structure
% may not be suitable. Here are some rules that apply for this
% solution: 

% - Exactly two or three sections (other than the summary).
% - At *most* three subsections per section.
% - Talk about 30s to 2min per frame. So there should be between about
%   15 and 30 frames, all told.

% - A conference audience is likely to know very little of what you
%   are going to talk about. So *simplify*!
% - In a 20min talk, getting the main ideas across is hard
%   enough. Leave out details, even if it means being less precise than
%   you think necessary.
% - If you omit details that are vital to the proof/implementation,
%   just say so once. Everybody will be happy with that.

\section{B Feed-Down}

\begin{frame}{B Feed-Down Correction}

\begin{columns}
\column{.50\textwidth}
\begin{overpic}[width=\textwidth, trim=0 0 0 0, clip]{img/BFeedDown/D0_Charged_R040_JetPtSpectrum_DPt_30_SideBand_FDCorrection}
\end{overpic}
\column{.50\textwidth}
\begin{overpic}[width=\textwidth, trim=0 0 0 0, clip]{img/BFeedDown/D0_Charged_R040_JetPtSpectrum_DPt_30_SideBand_FDCorrection_Ratio}
\end{overpic}
\end{columns}
{\tiny
$\textcolor{NavyBlue}{N^{\rm c\rightarrow\Dzero}_{\rm det}(\ptchjet)} = N^{\rm c,b\rightarrow\Dzero}_{\rm raw}(\ptchjet) - 
\textcolor{BrickRed}{R_{\rm det}^{\rm b\rightarrow\Dzero}(\ptchjet) \otimes \sum_{\ptd} \frac{\epsilon^{\rm b\rightarrow\Dzero}(\ptd)}{\epsilon^{\rm c\rightarrow\Dzero}(\ptd)} N^{\rm b\rightarrow\Dzero}_{\rm POWHEG}(\ptd,\ptchjet)}$
}
\small
\begin{itemize}
\item B Feed-Down (FD) spectrum is estimated with a POWHEG+PYTHIA6 simulation (scaled by the luminosity)
\item Weighted by the ratio of the prompt and non-prompt \Dzero\ reconstruction efficiencies and smeared using detector response
\item Subtracted from the measured yield before unfolding
\end{itemize}
\vspace{-5pt}
{\tiny
Note: no normalization is applied and not divided by the bin width, these are counts weighted by the reconstruction efficiency
}

\end{frame}


\begin{frame}{Variations vs. \ptd}

\begin{columns}
\column{.50\textwidth}
\begin{overpic}[width=\textwidth, trim=0 0 0 0, clip]{img/BFeedDown/variations/BFeedDown_GeneratorLevel_DPtSpectrum_Comparison}
\end{overpic}
\column{.50\textwidth}
\begin{overpic}[width=\textwidth, trim=0 0 0 0, clip]{img/BFeedDown/variations/BFeedDown_GeneratorLevel_DPtSpectrum_Comparison_Ratio}
\end{overpic}
\end{columns}
{\footnotesize
\begin{itemize}
\item Central values: $m_{\rm b}=4.75$~\GeVcsq, $\mu_{\rm F} = \mu_{\rm R} = 1$, PDF = CTEQ10nlo (11000)
\item $m_{\rm b}$: 5-15\%, \ptd-dependent
\item PDF: CT10 (10800) shows no difference, cteq6l1 (10042) is problematic (see next slide)
\item $\mu_{\rm F}$, $\mu_{\rm R}$: 8-18\% (some of the less extreme variations are missing)
\end{itemize}
}
\end{frame}

\begin{frame}{PDF: cteq6l1}

\begin{columns}
\column{.50\textwidth}
\begin{overpic}[width=\textwidth, trim=0 0 0 0, clip]{img/BFeedDown/cteq6l1/BFeedDown_GeneratorLevel_DPtSpectrum_Comparison}
\end{overpic}
\column{.50\textwidth}
\begin{overpic}[width=\textwidth, trim=0 0 0 0, clip]{img/BFeedDown/cteq6l1/BFeedDown_GeneratorLevel_DPtSpectrum_Comparison_Ratio}
\end{overpic}
\end{columns}
{\footnotesize
\begin{itemize}
\item cteq6l1 way off
\item Need to find a more reasonable substitute?
\end{itemize}
}
\end{frame}

\begin{frame}{Variations vs. \ptchjet}

\begin{columns}
\column{.50\textwidth}
\begin{overpic}[width=\textwidth, trim=0 0 0 0, clip]{img/BFeedDown/variations/BFeedDown_GeneratorLevel_JetPtSpectrum_DPt_30_bEfficiencyMultiply_cEfficiencyDivide_Comparison}
\end{overpic}
\column{.50\textwidth}
\begin{overpic}[width=\textwidth, trim=0 0 0 0, clip]{img/BFeedDown/variations/BFeedDown_GeneratorLevel_JetPtSpectrum_DPt_30_bEfficiencyMultiply_cEfficiencyDivide_Comparison_Ratio}
\end{overpic}
\end{columns}
{\footnotesize
\begin{itemize}
\item Central values: $m_{\rm b}=4.75$~\GeVcsq, $\mu_{\rm F} = \mu_{\rm R} = 1$, PDF = CTEQ10nlo (11000)
\item $m_{\rm b}$: 5-10\%, \ptchjet-dependent
\item PDF: CT10 (10800) shows no difference
\item $\mu_{\rm F}$, $\mu_{\rm R}$: 10-18\% (some of the less extreme variations are missing)
\end{itemize}
}
\end{frame}

\begin{frame}{Systematic Uncertainty}
\begin{itemize}
\item Variations seem to be symmetric (need to verify this for PDF)
\item Extract an absolute uncertainty for each of the 3 variations ($m_{\rm b}$, PDF, $\mu_{\rm F}$/$\mu_{\rm R}$) by picking the most extreme
\item Since the FD spectrum is subtracted from the measured spectrum the absolute uncertainties are summed in quadrature
\end{itemize}
\end{frame}

\section{Raw Yield Extraction}

\section{Summary \& Outlook}

\begin{frame}{List Of Preliminary Figures (\pp\ analysis)}

{\footnotesize
    \begin{enumerate}
        \item \pt-spectrum of \Dzero\ jets: comparison with POWHEG+PYTHIA (Phys. Prel.)
        \item Invariant Mass Distribution in bins of \ptchjet\ and in bins of \ptd\ (2-3 selected bins, Tech. Prel.)
        \item \pt-spectrum of \Dzero-jet candidates in the peak area and in the side-bands (2-3 selected bins, Tech. Prel.)
        \item Relative statistical uncertainty vs. \ptchjet\ (Tech. Prel.)
        \item Relative systematic uncertainty vs. \ptchjet\ (Tech. Prel.)
        \item Raw \pt-spectrum of \Dzero\ jets before unfolding (Tech. Prel.)
        \item Raw \pt-spectrum of \Dzero\ jets before FD subtraction (Tech. Prel.)
        \item Response Matrix (Sim.)
        \item \ptchjet\ resolution for prompt and non-prompt \Dzero\ jets (Sim.)
        \item Reconstruction efficiency for prompt and non-prompt \Dzero\ jets (Sim.)
    \end{enumerate}
    }
\end{frame}

\begin{frame}{Conclusions}
    \begin{itemize}
        \item Systematics:
        \begin{itemize}
            \item Unfolding: uncertainty negligible (much smaller than statistical uncertainty)
            \item B feed-down: 12-20\%
            \item Raw yield extraction: in progress
            \item Cut variations?
        \end{itemize}
        \item Other missing points?
    \end{itemize}
\end{frame}

\end{document}
