% $Header: /Users/joseph/Documents/LaTeX/beamer/solutions/conference-talks/conference-ornate-20min.en.tex,v 90e850259b8b 2007/01/28 20:48:30 tantau $

\documentclass[xcolor={usenames,dvipsnames}]{beamer}

% This file is a solution template for:

% - Talk at a conference/colloquium.
% - Talk length is about 20min.
% - Style is ornate.



% Copyright 2004 by Till Tantau <tantau@users.sourceforge.net>.
%
% In principle, this file can be redistributed and/or modified under
% the terms of the GNU Public License, version 2.
%
% However, this file is supposed to be a template to be modified
% for your own needs. For this reason, if you use this file as a
% template and not specifically distribute it as part of a another
% package/program, I grant the extra permission to freely copy and
% modify this file as you see fit and even to delete this copyright
% notice. 


\mode<presentation>
{
  \usetheme{AnnArbor}
  % or ...

  \setbeamercovered{transparent}
  % or whatever (possibly just delete it)
 }

\usepackage[percent]{overpic}
\usepackage[english]{babel}
\usepackage{multirow}
\usepackage{url}
% or whatever

\usepackage[latin1]{inputenc}
% or whatever

\usepackage{times}
\usepackage[T1]{fontenc}
% Or whatever. Note that the encoding and the font should match. If T1
% does not look nice, try deleting the line with the fontenc.
%particles
\newcommand{\jpsi}{\rm J/$\psi$}
\newcommand{\psip}{$\psi^\prime$}
\newcommand{\jpsiDY}{\rm J/$\psi$\,/\,DY}
\newcommand{\chic}{$\chi_{\rm c}$}
\newcommand{\pip}{$\pi^{+}$}
\newcommand{\pim}{$\pi^{-}$}
\newcommand{\pizero}{$\pi^{0}$}
\newcommand{\kap}{K$^{+}$}
\newcommand{\kam}{K$^{-}$}
\newcommand{\pbar}{$\rm\overline{p}$}
\newcommand{\ccbar}{\ensuremath{\mathrm{c\overline{c}}}}
\newcommand{\bbbar}{\ensuremath{\mathrm{b\overline{b}}}}
\newcommand{\Dzero}{\ensuremath{\mathrm{D^{0}}}}
\newcommand{\Dzerobar}{\ensuremath{\mathrm{\overline{D}^{0}}}}
\newcommand{\Dpm}{\ensuremath{\mathrm{D^{\pm}}}}
\newcommand{\Ds}{\ensuremath{\mathrm{D_{s}^{\pm}}}}
\newcommand{\Dstar}{\ensuremath{\mathrm{D^{*\pm}}}}

%collision systems
\newcommand{\pp}{pp}
\newcommand{\pPb}{p--Pb}
\newcommand{\PbPb}{Pb--Pb}

%detectors
\newcommand{\ezdc}{$E_{\rm ZDC}$}

%units
\newcommand{\GeVc}{GeV/$c$}
\newcommand{\GeVcsq}{GeV/$c^2$}

%others
\newcommand{\degree}{$^{\rm o}$}
\newcommand{\s}{\ensuremath{\sqrt{s}}}
\newcommand{\snn}{\ensuremath{\sqrt{s_{\rm NN}}}}
\newcommand{\y}{\ensuremath{y}}
\newcommand{\pt}{\ensuremath{p_{\rm T}}}
\newcommand{\dedx}{d$E$/d$x$}
\newcommand{\dndy}{d$N$/d$y$}
\newcommand{\dndydpt}{${\rm d}^2N/({\rm d}y {\rm d}p_{\rm t})$}
\newcommand{\zpar}{\ensuremath{z_{||}}}
\newcommand{\zpargen}{\ensuremath{z_{||}^{\mathrm{part}}}}
\newcommand{\zpardet}{\ensuremath{z_{||}^{\mathrm{det}}}}
\newcommand{\ptchjet}{\ensuremath{p_{\mathrm{T,ch\, jet}}}}
\newcommand{\ptjet}{\ensuremath{p_{\mathrm{T,jet}}}}
\newcommand{\ptchjetgen}{\ensuremath{p_{\mathrm{T,ch\,jet}}^{\mathrm{part}}}}
\newcommand{\ptchjetdet}{\ensuremath{p_{\mathrm{T,ch\,jet}}^{\mathrm{det}}}}
\newcommand{\ptd}{\ensuremath{p_{\mathrm{T,D}}}}
\newcommand{\ptdgen}{\ensuremath{p_{\mathrm{T,D}}^{\mathrm{part}}}}
\newcommand{\ptddet}{\ensuremath{p_{\mathrm{T,D}}^{\mathrm{det}}}}
\newcommand{\antikt}{anti-\ensuremath{k_{\mathrm{T}}}}
\newcommand{\Antikt}{Anti-\ensuremath{k_{\mathrm{T}}}}
\newcommand{\kt}{\ensuremath{k_{\mathrm{T}}}}
\newcommand{\pthard}{\ensuremath{p_{\mathrm{T,hard}}}}

\AtBeginSection[]{
  \begin{frame}
  \vfill
  \centering
  \begin{beamercolorbox}[sep=8pt,center,shadow=true,rounded=true]{title}
    \usebeamerfont{title}\insertsectionhead\par%
  \end{beamercolorbox}
  \vfill
  \end{frame}
}

\title[D-Tagged Jets in \pp] % (optional, use only with long paper titles)
{D-Tagged Jets in \pp\ Collisions}

\author[Salvatore Aiola]% (optional, use only with lots of authors)
{Salvatore Aiola}
% - Give the names in the same order as the appear in the paper.
% - Use the \inst{?} command only if the authors have different
%   affiliation.

\institute[Yale University] % (optional, but mostly needed)
{Yale University}

\date[Updates - Nov. 24th, 2017] % (optional, should be abbreviation of conference name)
{Updates \\
November 24th, 2017}
% - Either use conference name or its abbreviation.
% - Not really informative to the audience, more for people (including
%   yourself) who are reading the slides online

\subject{High-Energy Physics}
% This is only inserted into the PDF information catalog. Can be left
% out. 



% If you have a file called "university-logo-filename.xxx", where xxx
% is a graphic format that can be processed by latex or pdflatex,
% resp., then you can add a logo as follows:

% \pgfdeclareimage[height=0.5cm]{university-logo}{university-logo-filename}
% \logo{\pgfuseimage{university-logo}}


% If you wish to uncover everything in a step-wise fashion, uncomment
% the following command: 

%\beamerdefaultoverlayspecification{<+->}


\begin{document}

\begin{frame}
  \titlepage
\end{frame}

%\begin{frame}{Outline}
 %   \tableofcontents
 %\end{frame}


% Structuring a talk is a difficult task and the following structure
% may not be suitable. Here are some rules that apply for this
% solution: 

% - Exactly two or three sections (other than the summary).
% - At *most* three subsections per section.
% - Talk about 30s to 2min per frame. So there should be between about
%   15 and 30 frames, all told.

% - A conference audience is likely to know very little of what you
%   are going to talk about. So *simplify*!
% - In a 20min talk, getting the main ideas across is hard
%   enough. Leave out details, even if it means being less precise than
%   you think necessary.
% - If you omit details that are vital to the proof/implementation,
%   just say so once. Everybody will be happy with that.

%\begin{overpic}[width=.85\textwidth, trim=0 0 0 0, clip]{img/ReflectionTemplates_DPt_NoJet_DoubleGaus_1010}
%\put(0,61){{\tiny No jet requirement}}
%\put(60,61){{\tiny \textcolor{ForestGreen}{\textbf{Used for QM17 preliminary}}}}
%\end{overpic}

%\begin{columns}
%\column{0.5\textwidth}
%\column{0.5\textwidth}
%\end{columns}

\section{Topological Cuts}

\begin{frame}{Optimization of the Topological Cuts}
\begin{itemize}
\item Attempt to optmize the topological cuts to improve the precision of the measurement
\item Efficiency, raw yields and statistical precision shown in this presentation: \textcolor{blue}{\underline{\href{https://indico.cern.ch/event/670521/contributions/2800244/attachments/1563677/2463018/DtaggedJets_SAiola.pdf}{HFCJ 22 November}}}
\item In the following slides: comparison of the systematic uncertainties
\end{itemize}
\end{frame}

\subsection{Raw Yield Extr. Systematic Uncertainty}

\begin{frame}{Jet \pt\ spectrum, $\ptd>2$~\GeVc}
\begin{columns}
\column{0.5\textwidth}
%\begin{overpic}[width=\textwidth, trim=0 0 0 0, clip]{img/topo_cuts/Charged_R040_JetPtSpectrum_DPt_20_SideBand_SpectraComparison}
%\end{overpic}
\column{0.5\textwidth}
%\begin{overpic}[width=\textwidth, trim=0 0 0 0, clip]{img/topo_cuts/Charged_R040_JetPtSpectrum_DPt_20_SideBand_SpectraComparison_Ratio}
%\end{overpic}
\end{columns}
\end{frame}

\begin{frame}{FF, $\ptd>2$~\GeVc\ and $5<\ptchjet<15$~\GeVc}
\begin{columns}
\column{0.5\textwidth}
%\begin{overpic}[width=\textwidth, trim=0 0 0 0, clip]{img/topo_cuts/Charged_R040_JetZSpectrum_DPt_20_JetPt_5_15_SideBand_SpectraComparison}
%\end{overpic}
\column{0.5\textwidth}
%\begin{overpic}[width=\textwidth, trim=0 0 0 0, clip]{img/topo_cuts/Charged_R040_JetZSpectrum_DPt_20_JetPt_5_15_SideBand_SpectraComparison_Ratio}
%\end{overpic}
\end{columns}
\end{frame}

\begin{frame}{FF, $\ptd>6$~\GeVc\ and $15<\ptchjet<30$~\GeVc}
\begin{columns}
\column{0.5\textwidth}
%\begin{overpic}[width=\textwidth, trim=0 0 0 0, clip]{img/topo_cuts/Charged_R040_JetZSpectrum_DPt_60_JetPt_15_30_SideBand_SpectraComparison}
%\end{overpic}
\column{0.5\textwidth}
%\begin{overpic}[width=\textwidth, trim=0 0 0 0, clip]{img/topo_cuts/Charged_R040_JetZSpectrum_DPt_60_JetPt_15_30_SideBand_SpectraComparison_Ratio}
%\end{overpic}
\end{columns}
\end{frame}

\section{$\ptd>2$~\GeVc\ vs. $\ptd>3$~\GeVc}

\begin{frame}{$\ptd>2$~\GeVc\ vs. $\ptd>3$~\GeVc}
\begin{itemize}
\item Reducing the minimum cut on \ptd\ from $3$~\GeVc\ to $2$~\GeVc\ would reduce the fragmentation bias on the jet \pt\ cross section
\item In the following slides the comparisons of the yields and uncertainties are shown
\end{itemize}
\end{frame}

\begin{frame}{Raw Yields}
\begin{columns}
\column{0.5\textwidth}
\begin{overpic}[width=\textwidth, trim=0 0 0 0, clip]{img/raw_yield/AnyINT_D0_D0toKpiCuts_D0JetOptimLowJetPtv4_Charged_R040_DPtCutSideBand_jet_pt_50_300_SpectraComparison}
\end{overpic}
\column{0.5\textwidth}
\begin{overpic}[width=\textwidth, trim=0 0 0 0, clip]{img/raw_yield/AnyINT_D0_D0toKpiCuts_D0JetOptimLowJetPtv4_Charged_R040_DPtCutSideBand_jet_pt_50_300_SpectraComparison_Ratio}
\end{overpic}
\end{columns}
\end{frame}

\begin{frame}{Efficiency-Corrected Raw Yields}
\begin{columns}
\column{0.5\textwidth}
\begin{overpic}[width=\textwidth, trim=0 0 0 0, clip]{img/raw_yield_eff/AnyINT_D0_D0toKpiCuts_D0JetOptimLowJetPtv4_Charged_R040_DPtCutSideBand_jet_pt_50_300_SpectraComparison}
\end{overpic}
\column{0.5\textwidth}
\begin{overpic}[width=\textwidth, trim=0 0 0 0, clip]{img/raw_yield_eff/AnyINT_D0_D0toKpiCuts_D0JetOptimLowJetPtv4_Charged_R040_DPtCutSideBand_jet_pt_50_300_SpectraComparison_Ratio}
\end{overpic}
\end{columns}
\end{frame}

\begin{frame}{Statistical Uncertainties}
\begin{columns}
\column{0.5\textwidth}
\centering
\small
No efficiency correction\\
\begin{overpic}[width=\textwidth, trim=0 0 0 0, clip]{img/raw_yield/AnyINT_D0_D0toKpiCuts_D0JetOptimLowJetPtv4_Charged_R040_DPtCutSideBand_jet_pt_50_300_SpectraComparison_Uncertainty}
\end{overpic}
\column{0.5\textwidth}
\centering
\small
Efficiency Corrected\\
\begin{overpic}[width=\textwidth, trim=0 0 0 0, clip]{img/raw_yield_eff/AnyINT_D0_D0toKpiCuts_D0JetOptimLowJetPtv4_Charged_R040_DPtCutSideBand_jet_pt_50_300_SpectraComparison_Uncertainty}
\end{overpic}
\end{columns}
\end{frame}

\begin{frame}{Systematic Uncertainties}
\begin{columns}
\column{0.5\textwidth}
\centering
\small
No efficiency correction\\
%\begin{overpic}[width=\textwidth, trim=0 0 0 0, clip]{img/syst_unc/Charged_R040_JetPtSpectrum_DPt_20_SideBand_SpectraUncertaintyComparison}
%\end{overpic}
\column{0.5\textwidth}
\centering
\small
Efficiency Corrected\\
%\begin{overpic}[width=\textwidth, trim=0 0 0 0, clip]{img/syst_unc_eff/Charged_R040_JetPtSpectrum_DPt_20_InvMassFit_SpectraUncertaintyComparison}
%\end{overpic}
\end{columns}
\end{frame}

\section{\ptd\ bin width in SB method}

\begin{frame}{\ptd\ bin width in SB method}
\begin{itemize}
\item For the preliminary jet \pt\ cross section the D-meson candidates were divided in the following bins for the side-band method: [(2), 3, 4, 5, 6, 7, 8, 10, 12, 15, 20, 30]
\item Wider \ptd\ bins were tested to check whether this would lead to any improvement: [2, 4, 6, 9, 15, 30]
\item For the FF at high jet \pt\ the statistics is very scarce so only two bins are used: [6, 12, 30]
\item In the following slides the comparisons of the yields and uncertainties are shown
\end{itemize}
\end{frame}

\subsection{Jet \pt\ spectrum, $\ptd>2$~\GeVc}

\begin{frame}{Jet \pt\ spectrum: Raw Yields}
\begin{columns}
\column{0.5\textwidth}
\begin{overpic}[width=\textwidth, trim=0 0 0 0, clip]{img/raw_yield/AnyINT_D0_D0toKpiCuts_Charged_R040_DPtBinWidth_jet_pt_50_300_SpectraComparison}
\end{overpic}
\column{0.5\textwidth}
\begin{overpic}[width=\textwidth, trim=0 0 0 0, clip]{img/raw_yield/AnyINT_D0_D0toKpiCuts_Charged_R040_DPtBinWidth_jet_pt_50_300_SpectraComparison_Ratio}
\end{overpic}
\end{columns}
\end{frame}

\begin{frame}{Jet \pt\ spectrum: Raw Yields (eff. corrected)}
\begin{columns}
\column{0.5\textwidth}
\begin{overpic}[width=\textwidth, trim=0 0 0 0, clip]{img/raw_yield_eff/AnyINT_D0_D0toKpiCuts_Charged_R040_DPtBinWidth_jet_pt_50_300_SpectraComparison}
\end{overpic}
\column{0.5\textwidth}
\begin{overpic}[width=\textwidth, trim=0 0 0 0, clip]{img/raw_yield_eff/AnyINT_D0_D0toKpiCuts_Charged_R040_DPtBinWidth_jet_pt_50_300_SpectraComparison_Ratio}
\end{overpic}
\end{columns}
\end{frame}

\begin{frame}{Jet \pt\ spectrum: Statistical Uncertainties}
\begin{columns}
\column{0.5\textwidth}
\centering
\small
No efficiency correction\\
\begin{overpic}[width=\textwidth, trim=0 0 0 0, clip]{img/raw_yield/AnyINT_D0_D0toKpiCuts_Charged_R040_DPtBinWidth_jet_pt_50_300_SpectraComparison_Uncertainty}
\end{overpic}
\column{0.5\textwidth}
\centering
\small
Efficiency Corrected\\
\begin{overpic}[width=\textwidth, trim=0 0 0 0, clip]{img/raw_yield_eff/AnyINT_D0_D0toKpiCuts_Charged_R040_DPtBinWidth_jet_pt_50_300_SpectraComparison_Uncertainty}
\end{overpic}
\end{columns}
\end{frame}

\begin{frame}{Jet \pt\ spectrum: Systematic Uncertainties}
\begin{columns}
\column{0.5\textwidth}
\centering
\small
No efficiency correction\\
%\begin{overpic}[width=\textwidth, trim=0 0 0 0, clip]{img/syst_unc/Charged_R040_JetPtSpectrum_DPt_20_SideBand_SpectraUncertaintyComparison}
%\end{overpic}
\column{0.5\textwidth}
\centering
\small
Efficiency Corrected\\
%\begin{overpic}[width=\textwidth, trim=0 0 0 0, clip]{img/syst_unc_eff/Charged_R040_JetPtSpectrum_DPt_20_InvMassFit_SpectraUncertaintyComparison}
%\end{overpic}
\end{columns}
\end{frame}

\subsection{FF, $\ptd>2$~\GeVc\ and $5<\ptchjet<15$~\GeVc}

\begin{frame}{FF (low \ptchjet): Raw Yields}
\begin{columns}
\column{0.5\textwidth}
\begin{overpic}[width=\textwidth, trim=0 0 0 0, clip]{img/raw_yield/AnyINT_D0_D0toKpiCuts_Charged_R040_DPtBinWidth_d_z_2_10_SpectraComparison}
\end{overpic}
\column{0.5\textwidth}
\begin{overpic}[width=\textwidth, trim=0 0 0 0, clip]{img/raw_yield/AnyINT_D0_D0toKpiCuts_Charged_R040_DPtBinWidth_jet_pt_50_300_SpectraComparison_Ratio}
\end{overpic}
\end{columns}
\end{frame}

\begin{frame}{FF (low \ptchjet): Raw Yields (eff. corrected)}
\begin{columns}
\column{0.5\textwidth}
\begin{overpic}[width=\textwidth, trim=0 0 0 0, clip]{img/raw_yield_eff/AnyINT_D0_D0toKpiCuts_Charged_R040_DPtBinWidth_d_z_2_10_SpectraComparison}
\end{overpic}
\column{0.5\textwidth}
\begin{overpic}[width=\textwidth, trim=0 0 0 0, clip]{img/raw_yield_eff/AnyINT_D0_D0toKpiCuts_Charged_R040_DPtBinWidth_jet_pt_50_300_SpectraComparison_Ratio}
\end{overpic}
\end{columns}
\end{frame}

\begin{frame}{FF (low \ptchjet): Statistical Uncertainties}
\begin{columns}
\column{0.5\textwidth}
\centering
\small
No efficiency correction\\
\begin{overpic}[width=\textwidth, trim=0 0 0 0, clip]{img/raw_yield/AnyINT_D0_D0toKpiCuts_Charged_R040_DPtBinWidth_jet_pt_50_300_SpectraComparison_Uncertainty}
\end{overpic}
\column{0.5\textwidth}
\centering
\small
Efficiency Corrected\\
\begin{overpic}[width=\textwidth, trim=0 0 0 0, clip]{img/raw_yield_eff/AnyINT_D0_D0toKpiCuts_Charged_R040_DPtBinWidth_jet_pt_50_300_SpectraComparison_Uncertainty}
\end{overpic}
\end{columns}
\end{frame}

\begin{frame}{FF (low \ptchjet): Systematic Uncertainties}
\begin{columns}
\column{0.5\textwidth}
\centering
\small
No efficiency correction\\
%\begin{overpic}[width=\textwidth, trim=0 0 0 0, clip]{img/syst_unc/Charged_R040_JetPtSpectrum_DPt_20_SideBand_SpectraUncertaintyComparison}
%\end{overpic}
\column{0.5\textwidth}
\centering
\small
Efficiency Corrected\\
%\begin{overpic}[width=\textwidth, trim=0 0 0 0, clip]{img/syst_unc_eff/Charged_R040_JetPtSpectrum_DPt_20_InvMassFit_SpectraUncertaintyComparison}
%\end{overpic}
\end{columns}
\end{frame}

\subsection{FF, $\ptd>6$~\GeVc\ and $15<\ptchjet<30$~\GeVc}

\begin{frame}{FF (high \ptchjet): Invariant Mass Fits}
\begin{columns}
\column{.6\textwidth}
\begin{overpic}[width=\textwidth, trim=0 0 0 0, clip]{img/raw_yield/AnyINT_D0_D0toKpiCuts_Charged_R040_DPtBins_JetPt_15_30_FineBins}
\end{overpic}
\column{.4\textwidth}
Statistics is very scarce
\end{columns}
\end{frame}

\section{Unfolding of $z$ spectra}

\subsection{$5<\ptchjet<15$~\GeVc}

\begin{frame}{Bayesian Unfolding}
\begin{columns}
\column{0.5\textwidth}
\centering
%\begin{overpic}[width=.71\textwidth, trim=0 240 290 0, clip]{img/unfolding/JetZ_SideBand_DPt_20_JetPt_5_15_Response_PriorResponseTruth}
%\end{overpic}\\
%\begin{overpic}[width=.88\textwidth, trim=0 0 0 0, clip]{img/unfolding/JetZ_SideBand_DPt_20_JetPt_5_15_UnfoldingSummary_Bayes_RefoldedOverMeasured}
%\end{overpic}
\column{0.5\textwidth}
\centering
%\begin{overpic}[width=.88\textwidth, trim=0 0 0 0, clip]{img/unfolding/JetZ_SideBand_DPt_20_JetPt_5_15_UnfoldingSummary_Bayes}
%\end{overpic}\\
%\begin{overpic}[width=.88\textwidth, trim=0 0 0 0, clip]{img/unfolding/JetZ_SideBand_DPt_20_JetPt_5_15_UnfoldingSummary_Bayes_UnfoldedOverMeasured}
%\end{overpic}
\end{columns}
\end{frame}

\begin{frame}{Unfolding Stability}
\begin{columns}
\column{0.5\textwidth}
\centering
\tiny 
Number of iterations\\
%\begin{overpic}[width=.8\textwidth, trim=0 0 0 0, clip]{img/unfolding/JetZ_SideBand_DPt_20_JetPt_5_15_UnfoldingRegularization_Bayes_PriorResponseTruth_Ratio}
%\end{overpic}
\column{0.5\textwidth}
\centering
\tiny
Unfolding method\\
%\begin{overpic}[width=.8\textwidth, trim=0 0 0 0, clip]{img/unfolding/JetZ_SideBand_DPt_20_JetPt_5_15_UnfoldingMethod_Ratio}
%\end{overpic}
\end{columns}
\centering
\tiny
Pearsons' coefficients\\
%\begin{overpic}[width=.5\textwidth, trim=0 0 0 0, clip]{img/unfolding/JetZ_SideBand_DPt_20_JetPt_5_15_Pearson_Bayes_PriorResponseTruth}
%\end{overpic}
\end{frame}

\subsection{$15<\ptchjet<30$~\GeVc}

\begin{frame}{Bayesian Unfolding}
\begin{columns}
\column{0.5\textwidth}
\centering
%\begin{overpic}[width=.71\textwidth, trim=0 240 290 0, clip]{img/unfolding/JetZ_SideBand_DPt_20_JetPt_5_15_Response_PriorResponseTruth}
%\end{overpic}\\
%\begin{overpic}[width=.88\textwidth, trim=0 0 0 0, clip]{img/unfolding/JetZ_SideBand_DPt_20_JetPt_5_15_UnfoldingSummary_Bayes_RefoldedOverMeasured}
%\end{overpic}
\column{0.5\textwidth}
\centering
%\begin{overpic}[width=.88\textwidth, trim=0 0 0 0, clip]{img/unfolding/JetZ_SideBand_DPt_20_JetPt_5_15_UnfoldingSummary_Bayes}
%\end{overpic}\\
%\begin{overpic}[width=.88\textwidth, trim=0 0 0 0, clip]{img/unfolding/JetZ_SideBand_DPt_20_JetPt_5_15_UnfoldingSummary_Bayes_UnfoldedOverMeasured}
%\end{overpic}
\end{columns}
\end{frame}

\begin{frame}{Unfolding Stability}
\begin{columns}
\column{0.5\textwidth}
\centering
\tiny 
Number of iterations\\
%\begin{overpic}[width=.8\textwidth, trim=0 0 0 0, clip]{img/unfolding/JetZ_SideBand_DPt_20_JetPt_5_15_UnfoldingRegularization_Bayes_PriorResponseTruth_Ratio}
%\end{overpic}
\column{0.5\textwidth}
\centering
\tiny
Unfolding method\\
%\begin{overpic}[width=.8\textwidth, trim=0 0 0 0, clip]{img/unfolding/JetZ_SideBand_DPt_20_JetPt_5_15_UnfoldingMethod_Ratio}
%\end{overpic}
\end{columns}
\centering
\tiny
Pearsons' coefficients\\
%\begin{overpic}[width=.5\textwidth, trim=0 0 0 0, clip]{img/unfolding/JetZ_SideBand_DPt_20_JetPt_5_15_Pearson_Bayes_PriorResponseTruth}
%\end{overpic}
\end{frame}

\section{Conclusions}

\begin{frame}{Conclusions: topological cuts}
\begin{itemize}
\item No significant improvement with the new cuts
\item Check is whether the systematic uncertainty from the raw yield multi-trial improves with the new cuts (because of a more stable fit)
\item For the moment, stay with the old cuts?
\item On the side I will try to adopt a multi-variate study of the topological variable to see whether a better set of cuts can be found in this way (instead of looking at each variable one by one)
\item Check whether the wider \ptd\ bins lead to an improvement: for sure they are needed for the \zpar\ spectrum in the high \ptchjet\ bin, but not obvious for the other cases
\end{itemize}
\end{frame}

\begin{frame}{Conclusions: unfolding}
\begin{itemize}
\item Unfolding of the \zpar\ spectrum is stable for $5<\ptchjet<15$~\GeVc
\item Small correction as for the jet \pt\ spectrum
\item Need to vary the prior as a final check
\item Working on the unfolding of the \zpar\ spectrum for $15<\ptchjet<30$~\GeVc, which seems more unstable (probably due to the larger statistical uncertainties)
\end{itemize}
\end{frame}

\end{document}
