% $Header: /Users/joseph/Documents/LaTeX/beamer/solutions/conference-talks/conference-ornate-20min.en.tex,v 90e850259b8b 2007/01/28 20:48:30 tantau $

\documentclass[xcolor={usenames,dvipsnames}]{beamer}

% This file is a solution template for:

% - Talk at a conference/colloquium.
% - Talk length is about 20min.
% - Style is ornate.



% Copyright 2004 by Till Tantau <tantau@users.sourceforge.net>.
%
% In principle, this file can be redistributed and/or modified under
% the terms of the GNU Public License, version 2.
%
% However, this file is supposed to be a template to be modified
% for your own needs. For this reason, if you use this file as a
% template and not specifically distribute it as part of a another
% package/program, I grant the extra permission to freely copy and
% modify this file as you see fit and even to delete this copyright
% notice. 


\mode<presentation>
{
  \usetheme{AnnArbor}
  % or ...

  \setbeamercovered{transparent}
  % or whatever (possibly just delete it)
 }

\usepackage[percent]{overpic}

\usepackage[english]{babel}
% or whatever

\usepackage[latin1]{inputenc}
% or whatever

\usepackage{times}
\usepackage[T1]{fontenc}
% Or whatever. Note that the encoding and the font should match. If T1
% does not look nice, try deleting the line with the fontenc.
%particles
\newcommand{\jpsi}{\rm J/$\psi$}
\newcommand{\psip}{$\psi^\prime$}
\newcommand{\jpsiDY}{\rm J/$\psi$\,/\,DY}
\newcommand{\chic}{$\chi_{\rm c}$}
\newcommand{\pip}{$\pi^{+}$}
\newcommand{\pim}{$\pi^{-}$}
\newcommand{\pizero}{$\pi^{0}$}
\newcommand{\kap}{K$^{+}$}
\newcommand{\kam}{K$^{-}$}
\newcommand{\pbar}{$\rm\overline{p}$}
\newcommand{\ccbar}{\ensuremath{\mathrm{c\overline{c}}}}
\newcommand{\bbbar}{\ensuremath{\mathrm{b\overline{b}}}}
\newcommand{\Dzero}{\ensuremath{\mathrm{D^{0}}}}
\newcommand{\Dzerobar}{\ensuremath{\mathrm{\overline{D}^{0}}}}
\newcommand{\Dpm}{\ensuremath{\mathrm{D^{\pm}}}}
\newcommand{\Ds}{\ensuremath{\mathrm{D_{s}^{\pm}}}}
\newcommand{\Dstar}{\ensuremath{\mathrm{D^{*\pm}}}}

%collision systems
\newcommand{\pp}{pp}
\newcommand{\pPb}{p--Pb}
\newcommand{\PbPb}{Pb--Pb}

%detectors
\newcommand{\ezdc}{$E_{\rm ZDC}$}

%units
\newcommand{\GeVc}{GeV/$c$}
\newcommand{\GeVcsq}{GeV/$c^2$}

%others
\newcommand{\degree}{$^{\rm o}$}
\newcommand{\s}{\ensuremath{\sqrt{s}}}
\newcommand{\snn}{\ensuremath{\sqrt{s_{\rm NN}}}}
\newcommand{\y}{\ensuremath{y}}
\newcommand{\pt}{\ensuremath{p_{\rm T}}}
\newcommand{\dedx}{d$E$/d$x$}
\newcommand{\dndy}{d$N$/d$y$}
\newcommand{\dndydpt}{${\rm d}^2N/({\rm d}y {\rm d}p_{\rm t})$}
\newcommand{\zpar}{\ensuremath{z_{||}}}
\newcommand{\zpargen}{\ensuremath{z_{||}^{\mathrm{part}}}}
\newcommand{\zpardet}{\ensuremath{z_{||}^{\mathrm{det}}}}
\newcommand{\ptchjet}{\ensuremath{p_{\mathrm{T,ch\, jet}}}}
\newcommand{\ptjet}{\ensuremath{p_{\mathrm{T,jet}}}}
\newcommand{\ptchjetgen}{\ensuremath{p_{\mathrm{T,ch\,jet}}^{\mathrm{part}}}}
\newcommand{\ptchjetdet}{\ensuremath{p_{\mathrm{T,ch\,jet}}^{\mathrm{det}}}}
\newcommand{\ptd}{\ensuremath{p_{\mathrm{T,D}}}}
\newcommand{\ptdgen}{\ensuremath{p_{\mathrm{T,D}}^{\mathrm{part}}}}
\newcommand{\ptddet}{\ensuremath{p_{\mathrm{T,D}}^{\mathrm{det}}}}
\newcommand{\antikt}{anti-\ensuremath{k_{\mathrm{T}}}}
\newcommand{\Antikt}{Anti-\ensuremath{k_{\mathrm{T}}}}
\newcommand{\kt}{\ensuremath{k_{\mathrm{T}}}}
\newcommand{\pthard}{\ensuremath{p_{\mathrm{T,hard}}}}

\title[D-tagged jets in pp collisions at 7 TeV] % (optional, use only with long paper titles)
{D-meson tagged jets in pp collisions at 7 TeV}

\author[Salvatore Aiola]% (optional, use only with lots of authors)
{Salvatore Aiola}
% - Give the names in the same order as the appear in the paper.
% - Use the \inst{?} command only if the authors have different
%   affiliation.

\institute[Yale University] % (optional, but mostly needed)
{Yale University}

\date[PAG-HFCJ - Oct. 10th, 2016] % (optional, should be abbreviation of conference name)
{PAG-HFCJ \\
October 10th, 2016}
% - Either use conference name or its abbreviation.
% - Not really informative to the audience, more for people (including
%   yourself) who are reading the slides online

\subject{High-Energy Physics}
% This is only inserted into the PDF information catalog. Can be left
% out. 



% If you have a file called "university-logo-filename.xxx", where xxx
% is a graphic format that can be processed by latex or pdflatex,
% resp., then you can add a logo as follows:

% \pgfdeclareimage[height=0.5cm]{university-logo}{university-logo-filename}
% \logo{\pgfuseimage{university-logo}}


% If you wish to uncover everything in a step-wise fashion, uncomment
% the following command: 

%\beamerdefaultoverlayspecification{<+->}


\begin{document}

\begin{frame}
  \titlepage
\end{frame}

\begin{frame}{Outline}
    \tableofcontents
  \end{frame}


% Structuring a talk is a difficult task and the following structure
% may not be suitable. Here are some rules that apply for this
% solution: 

% - Exactly two or three sections (other than the summary).
% - At *most* three subsections per section.
% - Talk about 30s to 2min per frame. So there should be between about
%   15 and 30 frames, all told.

% - A conference audience is likely to know very little of what you
%   are going to talk about. So *simplify*!
% - In a 20min talk, getting the main ideas across is hard
%   enough. Leave out details, even if it means being less precise than
%   you think necessary.
% - If you omit details that are vital to the proof/implementation,
%   just say so once. Everybody will be happy with that.

\section{Follow-up from PWG-JE+HF meeting (Oct. 5th)}
\subsection{Unfolding statistical uncertainties}

\begin{frame}{SVD method}
\begin{columns}
\column{.50\textwidth}
\begin{overpic}[width=\textwidth, trim=0 0 50 0, clip]{img/SignalOnly_UnfoldingStatisticalUncertaintyStrategy_Svd_Reg3_PriorResponseTruth}
\end{overpic}
\column{.50\textwidth}
\begin{overpic}[width=\textwidth, trim=0 0 50 0, clip]{img/SignalOnly_UnfoldingStatisticalUncertainty_Svd_PriorResponseTruth}
\end{overpic}
\end{columns}
\begin{itemize}
\item \textbf{Left}: 3 options for error propagation provided by RooUnfold
\begin{itemize}
\item \textbf{kError} and \textbf{\textcolor{NavyBlue}{kCovariance}} are the same: square root of the diagonal elements of the covariance matrix
\item \textbf{\textcolor{BrickRed}{kCovToy}:} Guassian smearing of the central points of the input spectrum, repeat unfolding $n$ times ($n=50$)
\end{itemize}
\item \textbf{Right}: statistical uncertainty should increase as a function of the regularization parameter
\end{itemize}
\end{frame}

\begin{frame}{Bayesian method}
\begin{columns}
\column{.50\textwidth}
\begin{overpic}[width=\textwidth, trim=0 0 50 0, clip]{img/SignalOnly_UnfoldingStatisticalUncertaintyStrategy_Bayes_Reg3_PriorResponseTruth}
\end{overpic}
\column{.50\textwidth}
\begin{overpic}[width=\textwidth, trim=0 0 50 0, clip]{img/SignalOnly_UnfoldingStatisticalUncertainty_Bayes_PriorResponseTruth}
\end{overpic}
\end{columns}
\begin{itemize}
\item \textbf{Left}: 3 options for error propagation provided by RooUnfold
\begin{itemize}
\item \textbf{kError} and \textbf{\textcolor{NavyBlue}{kCovariance}} are the same: square root of the diagonal elements of the covariance matrix
\item \textbf{\textcolor{BrickRed}{kCovToy}:} Guassian smearing of the central points of the input spectrum, repeat unfolding $n$ times ($n=50$)
\end{itemize}
\item \textbf{Right}: statistical uncertainty should increase as a function of the regularization parameter
\end{itemize}
\end{frame}

\subsection{Prior choice}

\begin{frame}{Prior \ptchjet\ spectra}
\begin{columns}
\column{.50\textwidth}
\begin{overpic}[width=.95\textwidth, trim=0 0 50 0, clip]{img/PriorPowerLaw8vs2SmallBins}
\put(50,34){\textbf{\textcolor{BrickRed}{$a=2$}}}
\put(50,40){\textbf{\textcolor{NavyBlue}{$a=8$}}}
\end{overpic}
\column{.50\textwidth}
\begin{overpic}[width=.95\textwidth, trim=0 0 50 0, clip]{img/SignalOnly_Priors}
\end{overpic}
\end{columns}
\begin{itemize}
\item Prior spectra: "modified" power law functions to mimic qualitatively the shape of the spectrum ($\rightarrow 0$ when $\pt \rightarrow 0$):
$\pt^{-a}e^{\frac{-ab}{\pt}}$
\begin{itemize}
\item $a$: power law index
\item $b=3$~\GeVc\ is the position of the local maximum
\end{itemize}
\item ResponseTruth: truth from the charm-enhanced production
\item MC Truth: truth from the minimum-bias production
\item Right plot: not divided by the bin width!
\end{itemize}
\end{frame}

\begin{frame}{Prior choice: SVD and Bayesian}
\begin{columns}
\column{.50\textwidth}
\begin{overpic}[width=.95\textwidth, trim=0 0 50 0, clip]{img/SignalOnly_UnfoldingPrior_Svd_Ratio}
\end{overpic}
\column{.50\textwidth}
\begin{overpic}[width=.95\textwidth, trim=0 0 50 0, clip]{img/SignalOnly_UnfoldingPrior_Bayes_Ratio}
\end{overpic}
\end{columns}
\begin{itemize}
\item Good performance with quite extreme choice of priors
\end{itemize}
\end{frame}

\subsection{Over-/Under-flow bins}
\begin{frame}{Overflow bins in SVD unfolded spectrum}
\begin{columns}
\column{.53\textwidth}
\begin{overpic}[width=\textwidth, trim=0 0 50 0, clip]{img/SignalOnlySvdReg3PriorResponseTruthOverflowBins}
\end{overpic}
\column{.47\textwidth}
\begin{itemize}
\item A concern was raised about leaving enough freedom to the unfolding procedure to move entries outside of defined \pt\ range
\item In fact the overflow bin option was not enabled
\end{itemize}
\end{columns}
\begin{itemize}
\item After enabling the overflow option in RooUnfold the overflow bins (\ptchjet~$<5$~\GeVc\ and \ptchjet~$>24$~\GeVc) are populated
\item Negligible effect for $5>\ptchjet>24$~\GeVc
\end{itemize}
\end{frame}

\section{\zpar\ cut vs \ptd\ cut}
\begin{frame}{\zpar\ cut vs \ptd\ cut}
\begin{itemize}
\item Before HQ2016 we discussed whether we should use $\ptd>2$~\GeVc\ or $\zpar>0.1$
\item It turns out that in the minimum-bias MC there are only 15 real D mesons with $1<\ptd<2$~\GeVc\ and 0 for $\ptd<1$~\GeVc\, for $\ptchjet > 5$~\GeVc\ (after applying all the PID and topological cuts).
\item We introduce also a bit more background
\item After signal extraction the spectra with $\ptd>2$~\GeVc or $\zpar>0.1$ are the same within statistical uncertainties (both in data and MC)
\item For the moment we better stick with $\ptd>2$
\item Switching to $\zpar>0.1$ only makes sense if we try to do the analysis w/o the topological cuts (might be interesting trying since the cut $\ptchjet>5$~\GeVc\ may reduce the background)
\end{itemize}
\end{frame}

\section{Status of the data analysis}

\subsection{Signal extraction}
\begin{frame}{Signal extraction}
\begin{columns}
\column{.50\textwidth}
\begin{overpic}[width=.95\textwidth, trim=0 0 50 0, clip]{img/Data_D0_SpectraComparison}
\end{overpic}
\column{.50\textwidth}
\begin{overpic}[width=.95\textwidth, trim=0 0 50 0, clip]{img/Data_D0_SpectraComparison_Ratio}
\end{overpic}
\end{columns}
\begin{itemize}
\item Like-Sign method has largest uncertainties
\item Inv.Mass Fit and Side-Band agree within uncertainties
\item $\ptd>2$~\GeVc\ and $\zpar>0.1$ agree within uncertainties
\end{itemize}
\end{frame}

\subsection{Unfolding}
\begin{frame}{Unfolding method}
\begin{columns}
\column{.50\textwidth}
\begin{overpic}[width=.95\textwidth, trim=0 0 50 0, clip]{img/Data_InvMassFit_UnfoldingMethod}
\end{overpic}
\column{.50\textwidth}
\begin{overpic}[width=.95\textwidth, trim=0 0 50 0, clip]{img/Data_InvMassFit_UnfoldingMethod_Ratio}
\end{overpic}
\end{columns}
\begin{itemize}
\item The 3 methods agree with each other
\item Large uncertainty in the last bin: it will require some care to extract the systematic uncertainty without "double counting" the statistical uncertainty
\end{itemize}
\end{frame}

\begin{frame}{Prior choice: SVD}
\begin{columns}
\column{.50\textwidth}
\begin{overpic}[width=.95\textwidth, trim=0 0 50 0, clip]{img/Data_InvMassFit_UnfoldingPrior_Svd}
\end{overpic}
\column{.50\textwidth}
\begin{overpic}[width=.95\textwidth, trim=0 0 50 0, clip]{img/Data_InvMassFit_UnfoldingPrior_Svd_Ratio}
\end{overpic}
\end{columns}
\begin{itemize}
\item Reasonably stable with different prior choices
\end{itemize}
\end{frame}

\begin{frame}{Regularization: SVD}
\begin{columns}
\column{.50\textwidth}
\begin{overpic}[width=.95\textwidth, trim=0 0 50 0, clip]{img/Data_InvMassFit_UnfoldingRegularization_Svd_PriorResponseTruth}
\end{overpic}
\column{.50\textwidth}
\begin{overpic}[width=.95\textwidth, trim=0 0 50 0, clip]{img/Data_InvMassFit_UnfoldingRegularization_Svd_PriorResponseTruth_Ratio}
\end{overpic}
\end{columns}
\begin{itemize}
\item $k=4$ used as baseline
\item $k=3,5$ will be used to assess the systematic uncertainty
\end{itemize}
\end{frame}

\section*{Summary}

\subsection*{Conclusions}
\begin{frame}{Conclusions}
\begin{itemize}
\item Unfolding error propagation verified using RooUnfold
\item Prior spectra with "modified" power law with 2 tunable parameters
\item Overflow bin option enabled in RooUnfold
\item Unfolding closure test on MC and data unfolding 
\item Next: B feed-down correction
\item In the meantime: reflections in signal extraction
\end{itemize}
\end{frame}

\subsection*{Further Readings}
\begin{frame}{More...}
\begin{itemize}
\item My Hot Quarks 2016 talk: \url{https://aliceinfo.cern.ch/node/27729}
\bigskip
\item Brief Analysis Note for the HQ16 simulation figures: \url{https://alice.its.cern.ch/jira/secure/attachment/34159/HQ16_SimulationFigures_v4.pdf}
\bigskip
\item JIRA ticket: \url{https://alice.its.cern.ch/jira/browse/PWGHF-108}
\end{itemize}
\end{frame}

% All of the following is optional and typically not needed. 
\appendix
\section<presentation>*{\appendixname}
\subsection<presentation>*{Backup}

\begin{frame}{Unfolding closure test}
\begin{itemize}
\item \textbf{\textcolor{BrickRed}{"Mock" data}}: Minimum Bias (MB) production
\begin{itemize}
\item LHC14j4\string{b,c,d,e\string} anchored to LHC10\string{b,c,d,e\string} pass4
\item Similar statistics as in data $\sim 300$~M events
\end{itemize}
\item \textbf{\textcolor{ForestGreen}{Response matrix}}: charm-enhanced, \pthard-binned production
\begin{itemize}
\item LHC15i2\string{b,c,d,e\string} anchored to LHC10\string{b,c,d,e\string} pass4
\item 10 \pthard\ bins
\item all events required to have a \ccbar\ or \bbbar\ pair and all D mesons decay hadronically
\end{itemize}
\item Perform \textbf{\textcolor{Fuchsia}{signal extraction}} in MB production at \textbf{\textcolor{Fuchsia}{detector level}}
\begin{enumerate}
\item \textbf{Invariant mass fit}, \textbf{Like-Sign} (LS) or \textbf{Side-Band} (SB) methods
\item \textbf{MC truth} information
\end{enumerate}
\item \textbf{\textcolor{NavyBlue}{Unfold}} using response matrix from charm-enhanced production
\item Compare with \textbf{\textcolor{NavyBlue}{particle level}} truth
\end{itemize}
\end{frame}

\begin{frame}{Response Matrix and Kinematic Efficiency}
\begin{overpic}[width=\textwidth, trim=0 240 0 20, clip]{img/ResponseMatrix}
\end{overpic}
\begin{itemize}
\item Reconstruction efficiency already applied during signal extraction (see backup slides for details)
\item Kinematic efficiency due to limited \pt-range in the input spectrum
\end{itemize}
\end{frame}

\begin{frame}{Truth / Measured (MC truth signal-only)}
\center
\begin{overpic}[width=.7\textwidth, trim=0 0 50 20, clip]{img/Unfolding_TruthOverMeasured_SignalOnly}
\end{overpic}
\newline
Relatively small correction $\sim 10-20$~\%
\end{frame}

\begin{frame}{Unfolding Methods}
\begin{columns}
\column{.50\textwidth}
\begin{overpic}[width=\textwidth, trim=0 0 0 0, clip]{img/UnfoldingMethod_SignalOnly}
\end{overpic}
\column{.50\textwidth}
\begin{overpic}[width=\textwidth, trim=0 0 0 0, clip]{img/UnfoldingMethod_SignalOnly_Ratio}
\end{overpic}
\end{columns}
\begin{itemize}
\item \textbf{\textcolor{NavyBlue}{Singular Value Decomposition (SVD)}}: Hocker and Kartvelishvili, NIM A372 (1996) 469
\item \textbf{\textcolor{BrickRed}{Bin-By-Bin}}: "generalized" efficiency correction
\item \textbf{\textcolor{ForestGreen}{Bayesian}}: D'Agostini, NIM A362 (1995) 487
\item All 3 unfolding methods work well
\item Spectrum from the charm-enhanced production used as "prior"
\end{itemize}
\end{frame}

\begin{frame}{Prior Choice: Bin-By-Bin}
\begin{columns}
\column{.50\textwidth}
\begin{overpic}[width=\textwidth, trim=0 0 0 0, clip]{img/UnfoldingPrior_SignalOnly_BinByBin}
\end{overpic}
\column{.50\textwidth}
\begin{overpic}[width=\textwidth, trim=0 0 0 0, clip]{img/UnfoldingPrior_SignalOnly_BinByBin_Ratio}
\end{overpic}
\end{columns}
\begin{itemize}
\item Very extreme choice of prior: \textbf{\textcolor{BrickRed}{flat spectrum}}
\item Strong dependence on prior for the bin-by-bin correction (as expected)
\item Bin-By-Bin correction is a "generalized" efficiency correction (not technically unfolding)
\end{itemize}
\end{frame}

\begin{frame}{Prior Choice: SVD}
\begin{columns}
\column{.50\textwidth}
\begin{overpic}[width=\textwidth, trim=0 0 0 0, clip]{img/UnfoldingPrior_SignalOnly_Svd}
\end{overpic}
\column{.50\textwidth}
\begin{overpic}[width=\textwidth, trim=0 0 0 0, clip]{img/UnfoldingPrior_SignalOnly_Svd_Ratio}
\end{overpic}
\end{columns}
\begin{itemize}
\item Very extreme choice of prior: \textbf{\textcolor{BrickRed}{flat spectrum}}
\item SVD is still able to recover the truth with larger $k$ (regularization parameter)
\end{itemize}
\end{frame}

\begin{frame}{Prior Choice: Bayesian}
\begin{columns}
\column{.50\textwidth}
\begin{overpic}[width=\textwidth, trim=0 0 0 0, clip]{img/UnfoldingPrior_SignalOnly_Bayes}
\end{overpic}
\column{.50\textwidth}
\begin{overpic}[width=\textwidth, trim=0 0 0 0, clip]{img/UnfoldingPrior_SignalOnly_Bayes_Ratio}
\end{overpic}
\end{columns}
\begin{itemize}
\item Very extreme choice of prior: \textbf{\textcolor{BrickRed}{flat spectrum}}
\item Bayesian method is still able to recover the truth with a larger number of iterations (6 vs. 3)
\end{itemize}
\end{frame}

\begin{frame}{Regularization: SVD with "reasonable" prior}
\begin{columns}
\column{.50\textwidth}
\begin{overpic}[width=\textwidth, trim=0 0 0 0, clip]{img/UnfoldingRegularization_Svd_train_truth_SignalOnly}
\end{overpic}
\column{.50\textwidth}
\begin{overpic}[width=\textwidth, trim=0 0 0 0, clip]{img/UnfoldingRegularization_Svd_train_truth_SignalOnly_Ratio}
\end{overpic}
\end{columns}
\begin{itemize}
\item SVD is regularized through the $k$ parameter, with $0\leq k\leq n$ and $n=$~number of bins
\item Very stable for $k>1$
\end{itemize}
\end{frame}

\begin{frame}{Regularization: SVD with "extreme" prior}
\begin{columns}
\column{.50\textwidth}
\begin{overpic}[width=\textwidth, trim=0 0 0 0, clip]{img/UnfoldingRegularization_Svd_flat_SignalOnly}
\end{overpic}
\column{.50\textwidth}
\begin{overpic}[width=\textwidth, trim=0 0 0 0, clip]{img/UnfoldingRegularization_Svd_flat_SignalOnly_Ratio}
\end{overpic}
\end{columns}
\begin{itemize}
\item Very unstable, need $k=6$ to get close to the truth
\item Easy to spot if such an extreme scenario is observed in data
\item Data will probably exhibit something in between these two extremes
\end{itemize}
\end{frame}

\begin{frame}{Regularization: SVD $d$-vectors}
\begin{columns}
\column{.50\textwidth}
\begin{overpic}[width=\textwidth, trim=0 0 0 0, clip]{img/SvdDvector_SignalOnly_train_truth}
\put(30,50){"reasonable" prior}
\end{overpic}
\column{.50\textwidth}
\begin{overpic}[width=\textwidth, trim=0 0 0 0, clip]{img/SvdDvector_SignalOnly_flat}
\put(50,50){flat prior}
\end{overpic}
\end{columns}
\begin{itemize}
\item "Unphysical" modulations in the unfolded result are suppressed using $k<n=$~number of bins
\item The $d$-vector should show a sudden drop to unity for a certain value of $k$, which corresponds to the "best" choice
\item Left: "healthy" $d$-vector which suggests $k \approx 2-3$
\item Right: the $d$-vector values do not show a plateau at unity
\end{itemize}
\end{frame}

\begin{frame}{Pearsons' Coefficients: SVD with "reasonable" prior }
\center
\begin{overpic}[width=.65\textwidth, trim=0 0 0 0, clip]{img/Pearson_SignalOnly_Svd_train_truth}
\end{overpic}
\begin{itemize}
\item Pearsons' coefficients measure bin-bin correlations
\item $-1=$~max anti-correlation, $0=$~no correlation, $1=$~max correlation
\item Strong correlation between far-away bins, e.g. $k=1,2$
\item Strong anti-correlation between close-by bins, e.g. $k=5,6$
\item Pearsons' coefficients suggest $k=3,4$
\end{itemize}
\end{frame}

\begin{frame}{Pearsons' Coefficients: SVD with "extreme" prior }
\center
\begin{overpic}[width=.75\textwidth, trim=0 0 0 0, clip]{img/Pearson_SignalOnly_Svd_flat}
\end{overpic}
\begin{itemize}
\item Pearsons' coefficients would suggest $k=5,6$
\end{itemize}
\end{frame}

\begin{frame}{Regularization: Bayesian}
\begin{columns}
\column{.50\textwidth}
\center
"Reasonable" prior
\begin{overpic}[width=\textwidth, trim=0 0 0 0, clip]{img/UnfoldingRegularization_Bayes_train_truth_SignalOnly_Ratio}
\end{overpic}
\column{.50\textwidth}
\center
Flat prior
\begin{overpic}[width=\textwidth, trim=0 0 0 0, clip]{img/UnfoldingRegularization_Bayes_flat_SignalOnly_Ratio}
\end{overpic}
\end{columns}
\begin{itemize}
\item Bayesian method seems to perform well even with an extreme choice of prior
\item However maybe a small bias in some "nodes"
\end{itemize}
\end{frame}

\begin{frame}{Pearsons: Bayesian with "reasonable" prior }
\begin{overpic}[width=\textwidth, trim=0 0 0 0, clip]{img/Pearson_SignalOnly_Bayes_train_truth}
\end{overpic}
\end{frame}

\begin{frame}{Pearsons: Bayesian with "extreme" prior }
\begin{overpic}[width=\textwidth, trim=0 0 0 0, clip]{img/Pearson_SignalOnly_Bayes_flat}
\end{overpic}
\end{frame}

\begin{frame}{Full Closure Test: Invariant Mass Fit Method}
\begin{columns}
\column{.50\textwidth}
\begin{overpic}[width=\textwidth, trim=0 0 0 0, clip]{img/UnfoldingMethod_InvMassFit}
\end{overpic}
\column{.50\textwidth}
\begin{overpic}[width=\textwidth, trim=0 0 0 0, clip]{img/UnfoldingMethod_InvMassFit_Ratio}
\end{overpic}
\end{columns}
\begin{itemize}
\item All 3 unfolding methods work well
\item Spectrum from the charm-enhanced production used as "prior"
\end{itemize}
\end{frame}

\begin{frame}{Full Closure Test: Like-Sign Method}
\begin{columns}
\column{.50\textwidth}
\begin{overpic}[width=\textwidth, trim=0 0 0 0, clip]{img/UnfoldingMethod_LikeSign}
\end{overpic}
\column{.50\textwidth}
\begin{overpic}[width=\textwidth, trim=0 0 0 0, clip]{img/UnfoldingMethod_LikeSign_Ratio}
\end{overpic}
\end{columns}
\begin{itemize}
\item All 3 unfolding methods work well
\item Spectrum from the charm-enhanced production used as "prior"
\end{itemize}
\end{frame}




\end{document}
